\documentclass[12pt,a4paper]{article}
\usepackage[utf8]{inputenc}
\usepackage{amsmath, amssymb, amsthm}
\usepackage{geometry}
\geometry{a4paper, margin=1in}
\usepackage{hyperref}
\usepackage[backend=bibtex,style=numeric]{biblatex}
\usepackage{csquotes}
\usepackage[brazil]{babel}
\usepackage{microtype}
\usepackage{capt-of} % permitir legenda/label sem float
\usepackage{tikz}
\usetikzlibrary{positioning,arrows.meta,fit,calc}

\addbibresource{referencias.bib}

\title{Análise e Implementação de Algoritmos de Busca de uma r-Arborescência Inversa de Custo Mínimo em Grafos Dirigidos com Aplicação Didática Interativa}
\author{Orientador: Mário Leston 
\and Discentes: Lorena Silva Sampaio, Samira Haddad}
\date{\today}

\begin{document}

\maketitle

\section{Introdução}

\paragraph{}
Encontrar uma \textit{r-arborescência inversa de custo mínimo} em grafos dirigidos é um problema estudado em ciência da computação desde os anos 1960, com formulações fundamentais apresentadas por Jack Edmonds em 1967 \cite{edmonds1967optimum}.

\paragraph{}
Essa busca dialoga com um princípio formulado na Idade Média por Guilherme de Ockham: a navalha de Occam (princípio da parcimônia), uma heurística filosófica segundo a qual, entre explicações concorrentes para um fenômeno, devemos preferir a mais simples ou a que faz menos suposições.

\paragraph{}
Podemos pensar na navalha de Occam como critério de escolha entre explicações por meio de uma \textit{teia explicativa mínima}: uma estrutura que conecta fatos ou observações com o mínimo de relações explicativas necessárias.
Quando tais relações envolvem dependência ou causalidade, podemos representá-las pictograficamente como setas direcionadas entre os fatos (as hipóteses aparecem como rótulos dessas setas).
Para refinar o modelo, associamos um custo a cada relação (por exemplo, o esforço para validar a relação ou a complexidade da explicação).

\paragraph{}
Encontrar a \textit{teia explicativa mínima} equivale, nessa metáfora, a encontrar uma \textit{r-arborescência de custo mínimo}: fixamos um vértice raiz \(r\) (a explicação inicial) e escolhemos um conjunto mínimo de relações explicativas de modo que todos os fatos tenham um caminho dirigido que leve a \(r\), minimizando o custo total das arestas.

\paragraph{}
A \textit{r-arborescência} (também chamada \textit{out-arborescência}) orienta as arestas para fora de \(r\): cada vértice \(v\neq r\) tem exatamente uma aresta de entrada, e há um caminho dirigido único de \(r\) até \(v\). Já a \textit{r-arborescência inversa} (\textit{in-arborescência}) orienta as arestas em direção a \(r\): cada \(v\neq r\) tem exatamente uma aresta de saída, e de cada vértice parte um caminho dirigido único até \(r\) \cite{edmonds1967optimum,frank2014}.

\paragraph{}
Com essa distinção em mente, este trabalho concentra-se na variante inversa. Formalmente, dado um grafo dirigido \(G=(V,E)\) com custos \(c:E\to\mathbb{R}^+\) nas arestas e um vértice raiz \(r\in V\), procura-se uma \textit{r-arborescência inversa} — isto é, uma árvore direcionada que atinja todos os vértices por caminhos dirigidos até \(r\) — que minimize o custo total das arestas selecionadas (cf. \cite{edmonds1967optimum,frank2014}).

\paragraph{}
Nosso interesse, porém, não é apenas encontrar a arborescência mínima: o percurso até ela também importa, pois revela propriedades estruturais dos dígrafos e ilumina técnicas distintas de otimização. Por isso, investigamos duas rotas clássicas e complementares: (i) o algoritmo de Chu--Liu/Edmonds, que opera por normalização dos custos das arestas de entrada, seleção sistemática de arestas de custo zero e contração de ciclos até obter um grafo reduzido, seguida pela reexpansão para reconstrução da solução \cite{chu1965,edmonds1967optimum}; e (ii) a abordagem dual, em duas fases, de András Frank, fundamentada em cortes dirigidos, na qual se maximiza uma função de cortes c-viável para induzir arestas de custo zero e, em seguida, extrai-se a arborescência apenas a partir dessas arestas \cite{frank2014}. Embora assentados em princípios distintos — contração de ciclos no plano primal versus empacotamento/dualidade por cortes —, ambos os paradigmas produzem soluções ótimas e tornam explícitas a variedade de abordagens matemáticas que podem ser empregadas para resolver o mesmo problema.

\paragraph{}
Assim sendo, analisamos e implementamos, em Python, essas duas abordagens, e apresentaremos detalhes das implementações, desafios enfrentados e soluções adotadas. Realizamos testes de volume com milhares de instâncias geradas aleatoriamente, registrando resultados em arquivos CSV e de log; os custos obtidos pelo Chu--Liu/Edmonds e pelas duas variantes de András Frank coincidem, corroborando a correção das implementações.

\paragraph{}
Adicionalmente, desenvolvemos uma aplicação web com fins didáticos, utilizando o framework PyScript e as bibliotecas NetworkX e Matplotlib, que permitem construir grafos dirigidos interativamente, escolher o vértice-raiz, executar o algoritmo de Chu--Liu/Edmonds e acompanhar, passo a passo, a evolução do grafo e o registro detalhado da execução (log). A interface inclui operações de adicionar arestas com pesos, carregar um grafo de teste e exportar a instância em formato JSON, facilitando a experimentação por estudantes e educadores.

\subsection{Justificativa}

\paragraph{}
A busca por uma \textit{r-arborescência inversa de custo mínimo} em grafos dirigidos é um problema clássico com aplicações em diversas áreas, como redes de comunicação, planejamento de rotas, análise de dependências e modelagem de processos. Mas, não precisamos dessa justificação prática para nos interessarmos pelo problema: a riqueza estrutural dos dígrafos e a variedade de técnicas algorítmicas disponíveis o tornam um excelente caso de estudo em otimização combinatória.

\paragraph{}
Do ponto de vista didático, a metáfora da “teia explicativa mínima” torna concreto o porquê de estudarmos arborescências enraizadas: ela mapeia perguntas sobre explicação, alcance e economia de recursos para estruturas dirigidas, servindo de fio condutor nas implementações e nos experimentos que apresentamos.

\subsection{Objetivos}

\paragraph{}
O objetivo principal deste trabalho é analisar, implementar e comparar duas abordagens clássicas para o problema de \textit{r-arborescência de custo mínimo} em grafos dirigidos oferecendo uma aplicação web interativa que facilite o entendimento e a experimentação com o algoritmo de Chu--Liu/Edmonds e o método de András Frank, tornando-o acessível para estudantes e educadores.

\subsection{Estrutura do Trabalho}
\paragraph{}
Resumidamente, o trabalho abrange as seguintes frentes:  

\begin{enumerate}
    \item \textbf{Fundamentação teórica}: revisão da literatura sobre arborescências em grafos dirigidos, incluindo definições, propriedades e resultados relevantes.
    \item \textbf{Análise teórica}: consolidação dos conceitos de dígrafos e arborescências, compondo as formulações primal (normalização de custos e contração/reexpansão de ciclos no algoritmo de Chu--Liu/Edmonds) e dual (cortes dirigidos e função c-viável no método de András Frank), destacando resultados e intuições estruturais.
    \item \textbf{Implementação computacional}: implementação em Python das rotinas de normalização dos custos de entrada, construção de \(F^\ast\), detecção e contração de ciclos e reconstrução da solução (Chu--Liu/Edmonds), bem como das duas fases do método de András Frank; além de uma suíte de testes automatizados em larga escala sobre instâncias aleatórias com até centenas de vértices, verificando a coincidência dos custos entre os métodos e registrando resultados em CSV e log.
    \item \textbf{Aplicação pedagógica}: desenvolvimento de uma aplicação web interativa (PyScript + NetworkX + Matplotlib) que permite montar instâncias, escolher o vértice-raiz e acompanhar, passo a passo, a execução do algoritmo com visualização do grafo e dos pesos das arestas, log textual e importação/exportação em JSON para facilitar a reprodução de experimentos.
\end{enumerate}

Deste modo, o trabalho entrega implementações verificadas de Chu--Liu/Edmonds e András Frank, um visualizador web interativo e testes de volume que confirmam a equivalência de custos, úteis ao estudo e ao ensino de arborescências.

\section{Definições Preliminares}
\paragraph{}
Neste capítulo, reunimos as noções matemáticas básicas necessárias para compreensão completa do texto. 

Fixaremos notações e conceitos (conjuntos, relações, funções, dígrafos, propriedade em dígrafos, dígrafos ponderados, ramificações geradoras, arborescências, funções de custo, dualidade, problemas duais e algoritmos), até chegar à formulação do problema da r-arborescência inversa de custo mínimo e adiamos descrições algorítmicas para capítulos posteriores.

\subsection{Conjuntos}
\paragraph{}
Este trabalho depende profundamente da teoria dos conjuntos, podemos dizer que todos os objetos matemáticos que iremos utilizar nessa dissertação se reduzem a conjuntos e operações entre eles.

\paragraph{}
Um \textbf{conjunto} é uma agregação de objetos distintos com características bem definidas, chamados elementos ou membros do conjunto. Os conjuntos são geralmente representados por letras maiúsculas (por exemplo, \(A\), \(B\), \(C\)) e seus elementos são listados entre chaves (por exemplo, \(A = \{1, 2, 3\}\)). Dois conjuntos são iguais se contêm exatamente os mesmos elementos.

\paragraph{}
Podemos ter conjuntos de qualquer tipo de objeto, incluindo números, letras e elementos da natureza. Para motivar as definições ao longo do texto, usaremos dois exemplos complementares que vamos chamar de exemplos-mestres:
\begin{itemize}
    \item (i) Considere um universo \(U\) composto por três conjuntos: árvores \(T\), plantas \(P\) e fungos \(F\) — para praticar pertinência, inclusão e operações; e
    
    \begin{figure}[htbp]
    \centering
    \begin{tikzpicture}[scale=1]
        % Conjuntos P, T e F
        % P: plantas (círculo maior)
        \draw[fill=blue!8, draw=blue!60] (0,0) circle (2);
        \node[blue!60] at (0,2.25) {$P$};
        % T: árvores (círculo menor dentro de P)
        \draw[fill=blue!25, draw=blue!60] (-0.6,0) circle (0.9);
        \node[blue!60] at (-0.6,0) {$T$};
        % F: fungos (círculo disjunto)
        \draw[fill=green!20, draw=green!50!black] (3.8,0) circle (1.2);
        \node[green!50!black] at (3.8,0) {$F$};
    \end{tikzpicture}
    \caption{Relações entre os conjuntos de organismos: $T\subseteq P$ (toda árvore é planta) e $F$ é disjunto de $P$.}
    \label{fig:organismos}
    \end{figure}

    \item (ii) exemplo inspirado na navalha de Occam, com três famílias: evidências \(E\), hipóteses \(H\) e explicações \(\mathcal{M} \subseteq 2^{H}\). Nesse segundo exemplo, privilegiaremos explicações parcimoniosas: entre as que cobrem \(E\), preferimos as minimais por inclusão.
\end{itemize}

\paragraph{}
No segundo exemplo, consideremos \(E=\{\text{queda de temperatura},\, \text{céu nublado}\}\) e \(H=\{H_A, H_B\}\), em que \(H_A\) significa “frente fria” e \(H_B\), “ilha de calor”. Tanto \(\{H_A\}\) quanto \(\{H_A,H_B\}\) explicam \(E\) (cobrem ambas as evidências), mas, por parcimônia, preferimos \(\{H_A\}\), por ser estritamente menor do que \(\{H_A,H_B\}\) (\(\{H_A\} \subset \{H_A,H_B\}\)). Ao longo do texto, recorreremos às noções de pertinência (por exemplo, \(H_A\in H\)), de inclusão e às operações usuais sobre conjuntos (união, interseção etc.) para comparar explicações.

 \begin{figure}[htbp]
        \centering
        \begin{tikzpicture}[>=Stealth, node distance=1.6cm]
            % Evidências
            \node[draw, rounded corners, fill=gray!10, minimum width=3.8cm] (E1) {queda de temperatura};
            \node[draw, rounded corners, fill=gray!10, minimum width=3.8cm, below=of E1] (E2) {céu nublado};
            % Hipóteses
            \node[draw, rounded corners, fill=orange!20, left=4.2cm of E1, minimum width=3.2cm] (HA) {$H_A$: frente fria};
            \node[draw, rounded corners, fill=orange!10, below=of HA, minimum width=3.2cm] (HB) {$H_B$: ilha de calor};
            % Setas de cobertura
            \draw[->, thick] (HA.east) -- (E1.west);
            \draw[->, thick] (HA.east) -- (E2.west);
            \draw[->, dashed] (HB.east) -- (E2.west);
            % Nota de parcimônia
            \node[align=center, font=\small, below=1.0cm of E2] (nota) {Ambas as opções \emph{H\_A} e \emph{H\_A + H\_B} cobrem $E$;\\ por parcimônia, preferimos apenas \emph{H\_A}.};
        \end{tikzpicture}
        \caption{Exemplo inspirado na navalha de Occam: $H_A$ cobre ambas as evidências ($E$), enquanto $H_B$ seria redundante; prefere-se a explicação menor.}
        \label{fig:occam-exemplo}
    \end{figure}

\paragraph{}    
\subsubsection{Subconjuntos}
\paragraph{}
Dizemos que \(A\) é um \textbf{subconjunto} de \(B\), denotado \(A \subseteq B\), quando todo elemento de \(A\) também pertence a \(B\). Se, além disso, \(A \neq B\), escrevemos \(A \subset B\) e chamamos \(A\) de \textbf{subconjunto próprio} de \(B\). Ex.: \( \{1,2\} \subseteq \{1,2,3\} \) e \( \{1,2\} \subset \{1,2,3\} \).Por convenção, o conjunto vazio \(\varnothing\) é subconjunto de qualquer conjunto \(X\) (isto é, \(\varnothing \subseteq X\)), e todo conjunto é subconjunto de si mesmo (\(X \subseteq X\)).

\paragraph{}
No primeiro exemplo-mestre, sejam \(P\) o conjunto de plantas, \(T\) o de árvores e \(F\) o de fungos; então \(T \subseteq P\) (toda árvore é planta), ao passo que \(F \not\subseteq P\).

\paragraph{}
No segundo, seja \(H=\{H_A, H_B\}\) e \(E=\{\text{queda de temperatura},\, \text{céu nublado}\}\), vale \(\{H_A\} \subset \{H_A,H_B\} \subseteq H\); ambas as opções explicam \(E\), mas, 
por parcimônia, preferimos \(\{H_A\}\).

\subsubsection{Pertinência e inclusão}
\paragraph{}

Pertinência e inclusão são os conceitos mais fundamentais da teoria dos conjuntos. 

\paragraph{}
Começando pela \textbf{noção de pertinência} denotado por \(\in\): dizemos que um elemento \(x\) pertence a um conjunto \(X\) quando \(x \in X\) e não pertence quando \(x \notin X\).

\paragraph{}
Seja o nosso universo \(U\) de organismos: \(P=\{\text{todas as plantas}\}\), \(T=\{\text{todas as árvores}\}\) e \(F=\{\text{todos os fungos}\}\). Se \(x\) é um carvalho, então \(x\in T\) e, como toda árvore é uma planta, \(x\in P\). Já se \(y\) é um cogumelo, então \(y\in F\) e, na taxonomia moderna, \(y\notin T\) e \(y\notin P\). Agora, considere \(A=\{\text{árvores com folhas verdes}\}\); a pertinência fica clara: \(x\in A\) se, e somente se, \(x\) é árvore e tem folhas verdes.

\begin{figure}[htbp]
    \centering
    \begin{tikzpicture}[scale=1]
        % P: plantas (círculo maior)
        \draw[fill=blue!8, draw=blue!60] (0,0) circle (2);
        \node[blue!60] at (0,2.25) {$P$};
        % T: árvores (círculo menor dentro de P)
        \draw[fill=blue!25, draw=blue!60] (-0.6,0) circle (0.9);
        \node[blue!60] at (-0.6,-0.45) {$T$};
        % F: fungos (círculo disjunto)
        \draw[fill=green!20, draw=green!50!black] (3.8,0) circle (1.2);
        \node[green!50!black] at (3.8,-0.55) {$F$};
        % Pontos de pertinência
        \fill[red!70] (-0.9,0.2) circle (2pt);
        \node[anchor=east, text=red!70] at (-1.05,0.2) {$x$};
        \fill[black] (3.6,0.4) circle (2pt);
        \node[anchor=west] at (3.72,0.4) {$y$};
    \end{tikzpicture}
    \caption{Pertinência: $x\in T\subseteq P$ (ponto vermelho dentro de $T$) e $y\in F$ (ponto preto); logo $y\notin P$ e $y\notin T$.}
    \label{fig:pertinencia}
\end{figure}

\paragraph{}
Continuando, vem a \textbf{relação de inclusão} entre conjuntos denotada por \(\subseteq\): escrevemos \(X \subseteq Y\) quando todo elemento de \(X\) também pertence a \(Y\) (e \(X\subset Y\) quando, além disso, \(X\neq Y\)). No nosso exemplo, \(A\subseteq T\subset P\) e \(T\cap F=\varnothing\) (árvores e fungos não se sobrepõem).

\begin{figure}[htbp]
    \centering
    \begin{tikzpicture}[scale=1]
        % P: conjunto maior
        \draw[fill=blue!8, draw=blue!60] (0,0) circle (2);
        \node[blue!60] at (0,2.25) {$P$};
        % T: subconjunto de P
        \draw[fill=blue!25, draw=blue!60] (-0.6,0) circle (0.9);
        \node[blue!60] at (-0.6,0) {$T$};
        % A: subconjunto de T
        \draw[fill=blue!45, draw=blue!70] (-0.9,0.1) circle (0.45);
        \node[blue!70] at (-0.9,0.1) {$A$};
        % F: disjunto de P (e portanto de T e A)
        \draw[fill=green!20, draw=green!50!black] (3.8,0) circle (1.2);
        \node[green!50!black] at (3.8,0) {$F$};
    \end{tikzpicture}
    \caption{Inclusão: $A\subseteq T\subset P$ (círculos aninhados) e $T\cap F=\varnothing$ (círculos disjuntos).}
    \label{fig:inclusao}
\end{figure}

\subsubsection{Operações entre conjuntos}
\paragraph{}
Com essas definições de pertinência, inclusão e subconjuntos, apresentamos as operações básicas entre conjuntos, que usaremos ao longo do texto (mantendo o exemplo com \(P,T,F,A\)).

Outras operações comuns entre conjuntos incluem:
\begin{itemize}
    \item \textbf{União} (\(A \cup B\)): o conjunto de todos os elementos que pertencem a \(A\), a \(B\), ou ambos.
    
    Exemplo: \(T \cup F\) é o conjunto de todos os organismos que são árvores ou fungos (ou ambos, se existissem tais organismos).
    \begin{center}
    \begin{tikzpicture}[scale=0.95]
        % União T ∪ F com sobreposição visível (intersecção)
        \def\r{1.2}
        \coordinate (LT) at (-0.9,0);
        \coordinate (LF) at (0.9,0);
        % Preenchimento semitransparente para evidenciar a intersecção
        \fill[blue!35, opacity=0.45] (LT) circle (\r);
        \fill[green!35, opacity=0.45] (LF) circle (\r);
        % Contornos e rótulos
        \draw[thick, blue!60] (LT) circle (\r) node[left=6pt] {$T$};
        \draw[thick, green!50!black] (LF) circle (\r) node[right=6pt] {$F$};
    \end{tikzpicture}
    \captionof{figure}{União: a área colorida representa $T\cup F$; a sobreposição evidencia a intersecção $T\cap F$ (hipotética).}
    \label{fig:op-uniao}
    \end{center}
    
    \paragraph{}
    \item \textbf{União disjunta} (\(A \uplus B\)): o conjunto de todos os elementos que pertencem a \(A\) ou a \(B\), mas não a ambos; é igual a \(A \cup B\) quando \(A\) e \(B\) são disjuntos.
    
    Exemplo: \(T \uplus F\) é o conjunto de todos os organismos que são árvores ou fungos, mas não ambos (o que é trivialmente igual a \(T \cup F\) pois \(T\) e \(F\) são disjuntos).
    \begin{center}
    \begin{tikzpicture}[scale=0.95]
        \def\r{1.1}
        \coordinate (LT) at (-1.8,0);
        \coordinate (LF) at (1.8,0);
        \fill[blue!20] (LT) circle (\r);
        \fill[green!20] (LF) circle (\r);
        \draw[thick, blue!60] (LT) circle (\r) node[left=6pt] {$T$};
        \draw[thick, green!50!black] (LF) circle (\r) node[right=6pt] {$F$};
    \end{tikzpicture}
    \captionof{figure}{União disjunta: como $T\cap F=\varnothing$, tem-se $T\uplus F = T\cup F$.}
    \label{fig:op-uniao-disjunta}
    \end{center}

    \paragraph{}
    \item \textbf{Interseção} (\(A \cap B\)): o conjunto de todos os elementos que pertencem tanto a \(A\) quanto a \(B\).
    
    Exemplo: \(T \cap P = T\), pois todas as árvores são plantas.
    \begin{center}
    \begin{tikzpicture}[scale=0.95]
        % Interseção T ∩ P = T (T ⊆ P)
        \def\rP{1.6}
        \def\rT{0.9}
        \coordinate (cP) at (0,0);
        \coordinate (cT) at (-0.4,0.1);
        % P (maior) e T (dentro de P)
        \draw[fill=blue!10, draw=blue!60, thick] (cP) circle (\rP);
        \node[blue!60] at (0,\rP+0.25) {$P$};
        % Preencher T (interseção equivale a T)
        \draw[fill=blue!35, draw=blue!60, thick] (cT) circle (\rT);
        \node[blue!60] at (cT) {$T$};
    \end{tikzpicture}
    \captionof{figure}{Interseção: como $T\subseteq P$, $T\cap P = T$ (a região escura é $T$).}
    \label{fig:op-intersecao}
    \end{center}

    \paragraph{}    
    \item \textbf{Diferença} (\(A \setminus B\)): o conjunto de todos os elementos que pertencem a \(A\) mas não a \(B\).
    
    Exemplo: \(T \setminus A\) é o conjunto de todas as árvores que não têm folhas verdes.
    \begin{center}
    \begin{tikzpicture}[scale=0.95]
        % Diferença T \ A (A ⊆ T)
        \def\rT{1.4}
        \def\rA{0.8}
        \coordinate (cT) at (0,0);
        \coordinate (cA) at (0.4,0.2);
        % Preencher T
        \draw[fill=blue!25, draw=blue!60, thick] (cT) circle (\rT);
        % Remover a parte A de dentro de T
        \begin{scope}
            \clip (cT) circle (\rT);
            \fill[white] (cA) circle (\rA);
        \end{scope}
        \draw[thick, blue!60] (cT) circle (\rT) node[above right=2pt and -2pt] {$T$};
        \draw[thick, blue!60] (cA) circle (\rA) node[right=4pt] {$A$};
    \end{tikzpicture}
    \captionof{figure}{Diferença: região azul representa $T\setminus A$ (árvores que não têm folhas verdes).}
    \label{fig:op-diferenca}
    \end{center}
    
    \paragraph{}
    \item \textbf{Complemento} de \(X\) em um universo fixo \(U\): \(X^c := U\setminus X\) (também chamado de \textit{complemento absoluto}); o \textit{complemento relativo} de \(X\) em \(Y\) é \(Y\setminus X\).
    
    Exemplo: \(T^c = U \setminus T\) é o conjunto de todos os organismos que não são árvores. Ou seja, \(T^c\) inclui plantas que não são árvores, fungos e quaisquer outros organismos no universo \(U\).
    \begin{center}
    \begin{tikzpicture}[scale=0.95]
        % Universo U e conjunto T
        \draw[fill=gray!10, draw=black] (-2.6,-1.5) rectangle (2.6,1.5);
        \draw[fill=white, draw=black, thick] (-0.3,0) circle (1.0);
        \node at (-0.3,0) {$T$};
        \node at (2.3,1.25) {$U$};
    \end{tikzpicture}
    \captionof{figure}{Complemento: a área cinza representa $T^c = U\setminus T$.}
    \label{fig:op-complemento}
    \end{center}

    \paragraph{}
    \item \textbf{Diferença simétrica} (\(A\,\Delta\, B\)): \((A\setminus B)\cup(B\setminus A)\); é igual a \(A\cup B\) quando \(A\) e \(B\) são disjuntos.
    
    Exemplo: \(P\,\Delta\, F\) é o conjunto de todos os organismos que são plantas ou fungos, mas não ambos (o que é trivialmente igual a \(P \cup F\) pois \(P\) e \(F\) são disjuntos).
    \begin{center}
    \begin{tikzpicture}[scale=0.95]
        % Diferença simétrica P Δ F com P e F disjuntos
        \def\r{1.2}
        \coordinate (LP) at (-1.6,0);
        \coordinate (LF) at (1.6,0);
        \fill[blue!20] (LP) circle (\r);
        \fill[green!20] (LF) circle (\r);
        \draw[thick, blue!60] (LP) circle (\r) node[left=6pt] {$P$};
        \draw[thick, green!50!black] (LF) circle (\r) node[right=6pt] {$F$};
    \end{tikzpicture}
    \captionof{figure}{Diferença simétrica: como $P\cap F=\varnothing$, temos $P\,\Delta\, F = P\cup F$.}
    \label{fig:op-dif-simetrica}
    \end{center}

    \paragraph{}
    \item \textbf{Produto cartesiano} (\(A\times B\)): o conjunto de pares ordenados \((a,b)\) com \(a\in A\) e \(b\in B\).

    Exemplo: \(T=\{t_1,t_2\}\) e \(F=\{f_1,f_2\}\). Então \(T\times F = \{(t_1,f_1),(t_1,f_2),(t_2,f_1),(t_2,f_2)\}\).
    \begin{center}
    \begin{tikzpicture}[scale=1]
        % Grade limpa para T × F (2×2) sem sobreposições
        \def\xone{0}
        \def\xtwo{2.5}
        \def\yone{0}
        \def\ytwo{1.6}
        % Moldura e linhas da grade
        \draw[gray!35] (\xone-0.4,\yone-0.4) rectangle (\xtwo+0.4,\ytwo+0.4);
        \draw[gray!35] (\xone,\yone) -- (\xone,\ytwo);
        \draw[gray!35] (\xtwo,\yone) -- (\xtwo,\ytwo);
        \draw[gray!35] (\xone,\yone) -- (\xtwo,\yone);
        \draw[gray!35] (\xone,\ytwo) -- (\xtwo,\ytwo);
        % Pontos de T×F com cores distintas por par
        \fill[blue!70] (\xone,\yone) circle (2.4pt);   % (t1,f1)
        \fill[purple!70] (\xone,\ytwo) circle (2.4pt); % (t1,f2)
        \fill[green!60!black] (\xtwo,\yone) circle (2.4pt); % (t2,f1)
        \fill[orange!80!black] (\xtwo,\ytwo) circle (2.4pt); % (t2,f2)
        % Rótulos dos pares (posicionados para não sobrepor)
        \node[font=\scriptsize, text=blue!70, anchor=west]  at (\xone+0.18,\yone+0.18) {$(t_1,f_1)$};
        \node[font=\scriptsize, text=purple!70, anchor=west] at (\xone+0.18,\ytwo) {$(t_1,f_2)$};
        \node[font=\scriptsize, text=green!60!black, anchor=east] at (\xtwo-0.18,\yone+0.18) {$(t_2,f_1)$};
        \node[font=\scriptsize, text=orange!80!black, anchor=east] at (\xtwo-0.18,\ytwo-0.18) {$(t_2,f_2)$};
        % Rótulos dos elementos
        \node[blue!60] at (\xone,\yone-0.45) {$t_1$};
        \node[blue!60] at (\xtwo,\yone-0.45) {$t_2$};
        \node[green!50!black, anchor=east] at (\xone-0.25,\yone) {$f_1$};
        \node[green!50!black, anchor=east] at (\xone-0.25,\ytwo) {$f_2$};
        % Rótulos dos conjuntos (eixos)
        \node at (0.5*\xtwo, -0.95) {$T=\{t_1,t_2\}$};
        \node[rotate=90] at (\xone-0.95, 0.5*\ytwo) {$F=\{f_1,f_2\}$};
    \end{tikzpicture}
    \captionof{figure}{Produto cartesiano: pontos representam os pares de $T\times F$ para $T=\{t_1,t_2\}$ e $F=\{f_1,f_2\}$.}
    \label{fig:op-produto}
    \end{center}

    \paragraph{}
    \item \textbf{Conjunto das partes} (\(2^U\)): a família de todos os subconjuntos de \(U\) (inclui \(\varnothing\) e o próprio \(U\)).
    
    Exemplo: se \(U = \{x,y\}\), então \(2^U = \{\varnothing, \{x\}, \{y\}, \{x,y\}\}\). Logo, \(|2^U|=4=2^{|U|}\).
    \begin{center}
    \begin{tikzpicture}[scale=1, node distance=0.9cm]
        % Diagrama de Hasse para U={x,y}
        \node (empty) at (0,0) {$\varnothing$};
        \node (x) [above left=of empty] {$\{x\}$};
        \node (y) [above right=of empty] {$\{y\}$};
        \node (xy) [above=of empty] {$\{x,y\}$};
        \draw (empty) -- (x) -- (xy) -- (y) -- (empty);
    \end{tikzpicture}
    \captionof{figure}{Conjunto das partes: diagrama de Hasse de $2^{U}$ para $U=\{x,y\}$.}
    \label{fig:op-partes}
    \end{center}
\end{itemize}

\paragraph{Identidades úteis.}
Usaremos livremente as propriedades clássicas de conjuntos — comutatividade e associatividade de \(\cup\) e \(\cap\), distributividade e as \textbf{leis de De Morgan} — sem prova. Quando for relevante, explicitaremos a identidade no ponto de uso. Por exemplo, no nosso universo \(U\), \((P\cup F)^c = P^c\cap F^c\).

\subsubsection{Coleção}

\paragraph{}
Entre os objetos que podem pertencer a um conjunto, estão também eles mesmos, outros conjuntos. Chamaremos tais conjuntos de \textbf{coleções} (ou \textbf{famílias}) de conjuntos. Por exemplo, \(\mathcal{C} = \{P, T, F\}\) é uma coleção formada pelos conjuntos de organismos já definidos: plantas \(P\), árvores \(T\) e fungos \(F\). Note que \(\mathcal{C}\) é um conjunto como outro qualquer; seus elementos são, cada um, um conjunto.

\paragraph{}
Coleções são úteis para agrupar subconjuntos relacionados de um mesmo universo. Por exemplo, considere \(\mathcal{D} = \{A, B\}\), onde \(A = \{\text{árvores com folhas verdes}\}\) e \(B = \{\text{árvores com folhas vermelhas}\}\). Assim, \(\mathcal{D} \subseteq 2^{T}\) é uma coleção de subconjuntos de \(T\).

\paragraph{}
Uma coleção \(\mathcal{F}\) é dita \textbf{laminar} quando, para quaisquer \(X, Y \in \mathcal{F}\), vale que \(X \subseteq Y\), \(Y \subseteq X\) ou \(X \cap Y = \varnothing\); isto é, quaisquer dois conjuntos são aninhados (um está contido no outro) ou são disjuntos. 

\paragraph{}
Por exemplo, na coleção \(\mathcal{C} = \{P, T, F\}\): \(P\) é o conjunto de todas as plantas, \(T\) o de todas as árvores (portanto \(T\subseteq P\)) e \(F\) o de todos os fungos (disjunto de plantas e, logo, de árvores). Assim, quaisquer dois conjuntos em \(\mathcal{C}\) são aninhados ou disjuntos, e \(\mathcal{C}\) é laminar. Na coleção \(\mathcal{D} = \{A, T\}\): \(A\) é o conjunto de árvores com folhas verdes e \(T\) o de todas as árvores; como toda árvore de \(A\) é árvore de \(T\), temos \(A\subseteq T\) e a coleção é laminar. Já em \(\mathcal{E} = \{A, R\}\): \(R\) é o conjunto de árvores frutíferas; há árvores que são ao mesmo tempo frutíferas e de folhas verdes (a interseção é não vazia), mas nenhuma das classes contém a outra, então \(\mathcal{E}\) não é laminar.

\begin{center}
\begin{tikzpicture}[scale=0.95]
    % (a) Coleção C = {P, T, F} é laminar: T ⊆ P e F disjunto
    \begin{scope}[xshift=-6.2cm]
        % P (maior)
        \draw[fill=blue!8, draw=blue!60, thick] (0,0) circle (1.6);
        \node[blue!60] at (0,1.9) {$P$};
        % T (dentro de P)
        \draw[fill=blue!25, draw=blue!60, thick] (-0.4,0.1) circle (0.8);
        \node[blue!70] at (-0.4,0.1) {$T$};
        % F (disjunto)
        \draw[fill=green!20, draw=green!50!black, thick] (2.8,0) circle (0.9);
        \node[green!50!black] at (2.8,0) {$F$};
        % Rótulo do subpainel
        \node[font=\small] at (0,-2.2) {(a) $\mathcal{C}=\{P,T,F\}$ — laminar};
    \end{scope}

    % (b) Coleção D = {A, T} é laminar: A ⊆ T
    \begin{scope}
        % T
        \draw[fill=blue!25, draw=blue!60, thick] (0,0) circle (1.3);
        \node[blue!60] at (0,1.55) {$T$};
        % A (dentro de T)
        \draw[fill=blue!45, draw=blue!70, thick] (-0.4,0.1) circle (0.6);
        \node[blue!70] at (-0.4,0.1) {$A$};
        % Rótulo do subpainel
        \node[font=\small] at (0,-2.0) {(b) $\mathcal{D}=\{A,T\}$ — laminar};
    \end{scope}

    % (c) Coleção E = {A, R} não é laminar: A e R se interceptam sem inclusão
    \begin{scope}[xshift=6.2cm]
        % A e R sobrepostos
        \def\ra{1.1}
        \coordinate (cA) at (-0.5,0);
        \coordinate (cR) at (0.5,0);
        \fill[blue!35, opacity=0.5] (cA) circle (\ra);
        \fill[orange!60, opacity=0.45] (cR) circle (\ra);
        \draw[blue!70, thick] (cA) circle (\ra) node[left=6pt] {$A$};
        \draw[orange!80!black, thick] (cR) circle (\ra) node[right=6pt] {$R$};
        % Rótulo do subpainel
        \node[font=\small] at (0,-2.0) {(c) $\mathcal{E}=\{A,R\}$ — não laminar};
    \end{scope}
\end{tikzpicture}
\captionof{figure}{Laminaridade em coleções: em (a) e (b), quaisquer dois conjuntos são aninhados ou disjuntos; em (c), $A$ e $R$ se interceptam sem inclusão, violando a laminaridade.}
\label{fig:laminaridade}
\end{center}

\paragraph{}
Este é um importante conceito que aparecerá no restante do trabalho. A ideia de laminaridade retornará quando tratarmos de cortes dirigidos.

\subsubsection{Comparando conjuntos: cardinalidade e maximalidade}

\paragraph{}
Podemos comparar conjuntos através de relações de tamanho (cardinalidade) ou por relações de inclusão. Essas duas formas de comparação são distintas e importantes, especialmente quando lidamos com coleções de conjuntos.

\paragraph{}
A \textbf{cardinalidade} de um conjunto \(A\), denotada por \(|A|\), é o número de elementos de \(A\). Para conjuntos finitos, é simplesmente a contagem dos elementos (por exemplo, se \(A=\{1,2,3\}\), então \(|A|=3\)). Para conjuntos infinitos, a cardinalidade pode ser mais complexa, envolvendo conceitos como infinito enumerável e não enumerável. Por exemplo, o conjunto dos números naturais \(\mathbb{N}\) é infinito enumerável, enquanto o conjunto dos números reais \(\mathbb{R}\) é infinito não enumerável.

\paragraph{}
Dizemos que \(A\in\mathcal{C}\) tem \textbf{maior cardinalidade} se \(|A|\ge |B|\) para todo \(B\in\mathcal{C}\) (podendo haver empates). Esse critério não coincide, em geral, com a comparação por relação de inclusão. Em grafos, por exemplo, distinguem-se conjuntos independentes \emph{maximais} (não ampliáveis) de conjuntos independentes \emph{máximos} (de cardinalidade máxima).

\paragraph{}
Ao compararmos uma coleção \(\mathcal{C}\) de conjuntos utilizando sua relações de inclusão \((\mathcal{C},\subseteq)\), é imprescindível distinguir \textbf{maximal} de \textbf{máximo}.

\paragraph{}
Um conjunto \(A\in\mathcal{C}\) é \textbf{maximal} se não existe \(B\in\mathcal{C}\) tal que \(A\subset B\). Em palavras: não dá para ampliar \(A\) estritamente dentro da coleção. Podem haver vários elementos maximais, e eles podem ser incomparáveis entre si. Ex.: em \(\mathcal{C}=\big\{\{1\},\{2\}\big\}\), ambos \(\{1\}\) e \(\{2\}\) são maximais, mas não existe máximo.

\paragraph{}
Um conjunto \(A\in\mathcal{C}\) é \textbf{máximo} se \(B\subseteq A\) para todo \(B\in\mathcal{C}\). Se existe, é único. Ex.: em \(\mathcal{C}=\big\{\{1\},\{2\},\{1,2\}\big\}\), o conjunto \(\{1,2\}\) é o máximo.

\paragraph{}
Um bom exemplo para ilustrar a distinção entre conjuntos maximais e máximos é a coleção \(\mathcal{C}=\big\{\{1\},\{2\},\{1,2\},\{3\}\big\}\). Aqui, \(\{1,2\}\) é o único conjunto máximo (contém todos os outros), enquanto \(\{1\}\), \(\{2\}\) e \(\{3\}\) são todos maximais (não podem ser ampliados dentro da coleção).

\paragraph{}
Esses conceitos reaparecerão ao longo do texto, especialmente na diferença entre estruturas \textbf{maximais} (saturadas por inclusão) e \textbf{máximas/ótimas} (de maior cardinalidade ou menor custo). Para fixar ideias:
\begin{itemize}
    \item Em muitos problemas, “\textbf{maximal}” quer dizer: não dá para ampliar uma escolha sem violar as regras; já “\textbf{máximo/ótimo}” quer dizer: entre todas as escolhas válidas, essa é a melhor segundo o critério (por exemplo, menor custo).
    \item No algoritmo de \textbf{Chu--Liu/Edmonds}, começamos com escolhas locais que já não podem ser ampliadas dentro das regras do problema e, a partir delas, chegamos a uma solução de menor custo.
    \item No método de \textbf{András Frank}, primeiro construímos uma estrutura organizada que garante escolhas suficientes; depois, usando apenas relações já ativadas por essa estrutura, extraímos a solução ótima.
    \item Moral: partimos da ideia de “não dá para aumentar” (maximal) e chegamos a “melhor possível” (máximo/ótimo). Os detalhes técnicos de cada método aparecerão nas seções próprias.
\end{itemize}

\subsection{Relações e Funções}

\paragraph{}
Desde a introdução, vimos a ideia filosófica de explicar como “ligar” fatos a hipóteses da forma mais parcimoniosa possível. Para tornar essa intuição precisa, precisamos de uma linguagem que descreva objetos (conjuntos) e como eles se conectam. É aqui que entram as \textbf{relações} e, de modo ainda mais disciplinado, as \textbf{funções}: regras que associam a cada elemento de um conjunto exatamente um elemento de outro. Com elas, passamos do discurso qualitativo sobre explicações para uma estrutura matemática que permite medir, comparar e, adiante, otimizar.

% \paragraph{}
% Quando lidamos com conjuntos, frequentemente precisamos estabelecer conexões ou associações entre seus elementos. Essas conexões são formalizadas através de \textbf{relações} e \textbf{funções}, que são conceitos fundamentais em matemática e ciência da computação.

\paragraph{}
Na matemática, uma \textbf{relação} entre dois conjuntos \(A\) e \(B\) é uma maneira de associar elementos de \(A\) com elementos de \(B\). Uma \textbf{função} é um tipo especial de relação que associa cada elemento de \(A\) a exatamente um elemento de \(B\).

\paragraph{}
Uma \textbf{relação} \(R\) entre dois conjuntos \(A\) e \(B\) é um subconjunto do produto cartesiano \(A \times B\). Ou seja, \(R \subseteq A \times B\). Se \((a,b) \in R\), dizemos que \(a\) está relacionado a \(b\) pela relação \(R\), denotado \(aRb\).

\begin{center}
\begin{tikzpicture}[>=Stealth, node distance=1.2cm]
    % Coluna A (domínio) — retângulos arredondados
    \node[draw, rounded corners, fill=blue!10, minimum width=8mm, minimum height=6mm] (a1) {$a_1$};
    \node[draw, rounded corners, fill=blue!10, below=of a1, minimum width=8mm, minimum height=6mm] (a2) {$a_2$};
    \node[draw, rounded corners, fill=blue!10, below=of a2, minimum width=8mm, minimum height=6mm] (a3) {$a_3$};
    % Coluna B (contradomínio) — círculos
    \node[circle, draw, fill=green!10, right=3.2cm of a1, minimum size=6mm] (b1) {$b_1$};
    \node[circle, draw, fill=green!10, below=of b1, minimum size=6mm] (b2) {$b_2$};
    \node[circle, draw, fill=green!10, below=of b2, minimum size=6mm] (b3) {$b_3$};
    % Agrupadores (caixas) para A e B com rótulos
    \node[draw=blue!50, rounded corners, fit=(a1)(a2)(a3), inner sep=5pt, label={[blue!60]above:A (domínio)}] {};
    \node[draw=green!50!black, fit=(b1)(b2)(b3), inner sep=7pt, label={[green!50!black]above:B (contradomínio)}] {};
    % Setas representando os pares de R (neutras para destacar os conjuntos)
    \draw[->, thick, draw=gray!70] (a1) -- (b2);
    \draw[->, thick, draw=gray!70] (a2) -- (b1);
    \draw[->, thick, draw=gray!70] (a2) -- (b3);
    \draw[->, thick, draw=gray!70] (a3) -- (b3);
\end{tikzpicture}
\captionof{figure}{Relação $R\subseteq A\times B$. Cada seta representa um par $(a,b)\in R$ (isto é, $a\,R\,b$). As formas/cores distinguem domínio ($A$, retângulos azuis) de contradomínio ($B$, círculos verdes). Note que $a_2$ se relaciona com $b_1$ e $b_3$; logo, este $R$ \emph{não} é uma função. Para que $R$ fosse uma função $f\!:\!A\to B$, cada $a\in A$ deveria ter \emph{exatamente uma} seta saindo para algum $b\in B$.}
\label{fig:relacao}
\end{center}

\paragraph{}
No nosso primeiro exemplo-mestre, considere \(P=\{\text{todas as plantas}\}\) e \(F=\{\text{todos os fungos}\}\). Definimos a relação \(R\) como "é um organismo que compete com". Assim, se uma planta \(p \in P\) compete com um fungo \(f \in F\), então \((p,f) \in R\).

\paragraph{}
No nosso segundo exemplo, considere \(H=\{\text{hipóteses}\}\) e \(E=\{\text{evidências}\}\). Definimos a relação \(R\) como "explica". Se uma hipótese \(h \in H\) explica uma evidência \(e \in E\), então \((h,e) \in R\).

\paragraph{}
Em teoria dos grafos, uma relação pode representar conexões entre vértices. Por exemplo, em um grafo dirigido, a relação "existe uma aresta de \(u\) para \(v\)" pode ser representada como um conjunto de pares ordenados \((u,v)\).

\paragraph{}
Uma \textbf{função} \(f\) de um conjunto \(A\) em um conjunto \(B\) é uma relação especial que associa cada elemento de \(A\) a exatamente um elemento de \(B\). Denotamos isso como \(f: A \to B\). Se \(f(a) = b\), dizemos que \(b\) é a imagem de \(a\) sob \(f\).

\begin{center}
\begin{tikzpicture}[>=Stealth, node distance=1.2cm]
    % Coluna A (domínio) — retângulos arredondados
    \node[draw, rounded corners, fill=blue!10, minimum width=8mm, minimum height=6mm] (a1f) {$a_1$};
    \node[draw, rounded corners, fill=blue!10, below=of a1f, minimum width=8mm, minimum height=6mm] (a2f) {$a_2$};
    \node[draw, rounded corners, fill=blue!10, below=of a2f, minimum width=8mm, minimum height=6mm] (a3f) {$a_3$};
    % Coluna B (contradomínio) — círculos
    \node[circle, draw, fill=green!10, right=3.2cm of a1f, minimum size=6mm] (b1f) {$b_1$};
    \node[circle, draw, fill=green!10, below=of b1f, minimum size=6mm] (b2f) {$b_2$};
    \node[circle, draw, fill=green!10, below=of b2f, minimum size=6mm] (b3f) {$b_3$};
    % Agrupadores (caixas) para A e B com rótulos
    \node[draw=blue!50, rounded corners, fit=(a1f)(a2f)(a3f), inner sep=5pt, label={[blue!60]above:A (domínio)}] {};
    \node[draw=green!50!black, fit=(b1f)(b2f)(b3f), inner sep=7pt, label={[green!50!black]above:B (contradomínio)}] {};
    % Setas de função (exatamente uma saída por elemento de A)
    \draw[->, thick, draw=gray!70] (a1f) -- (b1f);
    \draw[->, thick, draw=gray!70] (a2f) -- (b2f);
    \draw[->, thick, draw=gray!70] (a3f) -- (b2f); % muitos-para-um permitido
\end{tikzpicture}
\captionof{figure}{Função $f\!:\!A\to B$. Cada elemento de $A$ tem \emph{exatamente uma} seta saindo para um elemento de $B$ (a imagem). Elementos distintos de $A$ podem ter a mesma imagem (muitos-para-um), e nem todo elemento de $B$ precisa ser imagem (aqui, $b_3$ não é atingido). Compare com a Fig.\ \ref{fig:relacao}.}
\label{fig:funcao}
\end{center}

\paragraph{}
No nosso exemplo-mestre, considere \(P=\{\text{todas as plantas}\}\) e \(\mathbb{N}=\{0,1,2,\ldots\}\) (números naturais). Definimos a função \(f: P \to \mathbb{N}\) que associa cada planta ao seu número de folhas. Se \(p \in P\) é uma árvore com 100 folhas, então \(f(p) = 100\).

\paragraph{}
Em teoria dos grafos, funções podem ser usadas para atribuir pesos às arestas. Por exemplo, se temos um grafo \(G\) com arestas \(e_1, e_2, \ldots, e_n\), podemos definir uma função \(c: E \to \mathbb{R}^+\) que atribui um peso \(c(e_i)\) a cada aresta \(e_i\).

\paragraph{}
Na ciência da computação, relações e funções são usadas para modelar conexões entre dados, estruturas de dados e operações. Por exemplo, em bancos de dados relacionais, tabelas representam relações entre diferentes entidades. Em programação funcional, funções são tratadas como cidadãos de primeira classe, permitindo a criação de funções de ordem superior que podem receber outras funções como argumentos ou retorná-las como resultados.

\subsubsection{Conceitos em Funções}
\paragraph{}
Alguns conceitos importantes relacionados a funções incluem:
\begin{itemize}
    \item \textbf{Domínio}: o conjunto \(A\) de entrada da função \(f: A \to B\).
    \item \textbf{Contradomínio}: o conjunto \(B\) de possíveis saídas da função.
    \item \textbf{Imagem}: o conjunto de valores efetivamente atingidos pela função, \(f(A) = \{f(a) \mid a \in A\}\).
    
    \begin{center}
    \begin{tikzpicture}[>=Stealth, node distance=1.1cm]
        % Elementos do domínio A (retângulos azuis)
        \node[draw, rounded corners, fill=blue!10, minimum width=9mm, minimum height=6mm] (a1img) {$a_1$};
        \node[draw, rounded corners, fill=blue!10, below=of a1img, minimum width=9mm, minimum height=6mm] (a2img) {$a_2$};
        \node[draw, rounded corners, fill=blue!10, below=of a2img, minimum width=9mm, minimum height=6mm] (a3img) {$a_3$};
        % Elementos do contradomínio B (círculos verdes)
        \node[circle, draw, fill=green!10, right=3.2cm of a1img, minimum size=6mm] (b1img) {$b_1$};
        \node[circle, draw, fill=green!10, below=of b1img, minimum size=10mm] (b2img) {$b_2$};
        \node[circle, draw, fill=green!10, below=of b2img, minimum size=6mm] (b3img) {$b_3$};
        % Caixas de agrupamento com rótulos
        \node[draw=blue!50, rounded corners, fit=(a1img)(a2img)(a3img), inner sep=5pt, label={[blue!60]above:A (domínio)}] {};
        \node[draw=green!50!black, fit=(b1img)(b2img)(b3img), inner sep=7pt, label={[green!50!black]above:B (contradomínio)}] {};
        % Setas de f (exatamente uma por elemento de A)
        \draw[->, thick, draw=gray!70] (a1img) -- (b1img);
        \draw[->, thick, draw=gray!70] (a2img) -- (b2img);
        \draw[->, thick, draw=gray!70] (a3img) -- (b2img);
        % Destaque da imagem f(A) ⊆ B
    \node[draw=purple!70!black, thick, fit=(b1img)(b2img), inner sep=3pt, label distance=15mm, label={[purple!70!black]right:Imagem $f(A)$}] {};
    \end{tikzpicture}
    
    \captionof{figure}{Domínio, contradomínio e imagem: $A$ (retângulos azuis) mapeia via $f$ para $B$ (círculos verdes). A imagem $f(A)$ é o subconjunto de $B$ efetivamente atingido (aqui, $\{b_1,b_2\}$).}
    \label{fig:dom-contradom-imagem}
    \end{center}
    
    \item \textbf{Injetora}: uma função \(f\) é injetora se \(f(a_1) = f(a_2)\) implica \(a_1 = a_2\); ou seja, elementos distintos do domínio têm imagens distintas.
    \item \textbf{Sobrejetora}: uma função \(f\) é sobrejetora se para todo \(b \in B\), existe \(a \in A\) tal que \(f(a) = b\); ou seja, a imagem é igual ao contradomínio.
    \item \textbf{Bijetora}: uma função que é tanto injetora quanto sobrejetora; estabelece uma correspondência um-para-um entre os elementos de \(A\) e \(B\).
    
\end{itemize}

\begin{center}
\begin{tikzpicture}[>=Stealth, scale=0.9, every node/.style={transform shape}, node distance=0.8cm]
    % estilos reutilizados
    \tikzset{
        Aelem/.style={draw, rounded corners, fill=blue!10, minimum width=7mm, minimum height=5mm},
        Belem/.style={circle, draw, fill=green!10, minimum size=5mm},
        Abox/.style={draw=blue!50, rounded corners},
        Bbox/.style={draw=green!50!black}
    }

    % (a) Injetora: imagens distintas
    \begin{scope}
        \node[Aelem] (a1i) {$a_1$};
        \node[Aelem, below=of a1i] (a2i) {$a_2$};
        \node[Aelem, below=of a2i] (a3i) {$a_3$};
    \node[style=Belem, right=2.2cm of a1i] (b1i) {$b_1$};
        \node[style=Belem, below=of b1i] (b2i) {$b_2$};
        \node[style=Belem, below=of b2i] (b3i) {$b_3$};
    \node[Abox, fit=(a1i)(a2i)(a3i), inner sep=3pt, label={[blue!60]above:$A$}] {};
    \node[Bbox, fit=(b1i)(b2i)(b3i), inner sep=5pt, label={[green!50!black]above:$B$}] {};
    \draw[->, semithick, draw=gray!70] (a1i) -- (b1i);
    \draw[->, semithick, draw=gray!70] (a2i) -- (b2i);
    \draw[->, semithick, draw=gray!70] (a3i) -- (b3i);
    \node[font=\footnotesize, anchor=south] at ($(a1i)!0.5!(b1i)+(0,0.7)$) {(a) Injetora};
    \end{scope}
    
    % (b) Sobrejetora: todo B é atingido
    \begin{scope}[xshift=4.8cm]
        \node[Aelem] (a1s) {$a_1$};
        \node[Aelem, below=of a1s] (a2s) {$a_2$};
        \node[Aelem, below=of a2s] (a3s) {$a_3$};
    \node[style=Belem, right=2.2cm of a1s] (b1s) {$b_1$};
    \node[style=Belem, below=of b1s] (b2s) {$b_2$};
    \node[style=Belem, below=of b2s] (b3s) {$b_3$};
    \node[Abox, fit=(a1s)(a2s)(a3s), inner sep=3pt, label={[blue!60]above:$A$}] {};
    \node[Bbox, fit=(b1s)(b2s)(b3s), inner sep=5pt, label={[green!50!black]above:$B$}] {};
    \draw[->, semithick, draw=gray!70] (a1s) -- (b1s);
    \draw[->, semithick, draw=gray!70] (a2s) -- (b2s);
    \draw[->, semithick, draw=gray!70] (a3s) -- (b3s);
    \node[font=\footnotesize, anchor=south] at ($(a1s)!0.5!(b1s)+(0,0.7)$) {(b) Sobrejetora};
    \end{scope}
   
    % (c) Bijetora: um-para-um e sobre
    \begin{scope}[xshift=9.6cm]
        \node[Aelem] (a1b) {$a_1$};
        \node[Aelem, below=of a1b] (a2b) {$a_2$};
        \node[Aelem, below=of a2b] (a3b) {$a_3$};
    \node[style=Belem, right=2.2cm of a1b] (b1b) {$b_1$};
    \node[style=Belem, below=of b1b] (b2b) {$b_2$};
    \node[style=Belem, below=of b2b] (b3b) {$b_3$};
    \node[Abox, fit=(a1b)(a2b)(a3b), inner sep=3pt, label={[blue!60]above:$A$}] {};
    \node[Bbox, fit=(b1b)(b2b)(b3b), inner sep=5pt, label={[green!50!black]above:$B$}] {};
    \draw[->, semithick, draw=gray!70] (a1b) -- (b2b);
    \draw[->, semithick, draw=gray!70] (a2b) -- (b3b);
    \draw[->, semithick, draw=gray!70] (a3b) -- (b1b);
    \node[font=\footnotesize, anchor=south] at ($(a1b)!0.5!(b1b)+(0,0.7)$) {(c) Bijetora};
    \end{scope}
\end{tikzpicture}

\captionof{figure}{Funções especiais: (a) Injetora — elementos distintos em $A$ têm imagens distintas em $B$; (b) Sobrejetora — todo elemento de $B$ é imagem; (c) Bijetora — um-para-um e sobre $B$.}
\label{fig:inj-sobre-bij}
\end{center}

\subsubsection{Funções de agregação e somatórios}

\paragraph{}
Além de relacionar elementos de conjuntos, muitas operações familiares em matemática, envolvem \emph{funções} que recebem coleções de números (ou funções) e devolvem um número.

\paragraph{}
Uma \textbf{função de agregação} é uma função que recebe um conjunto (ou sequência) de valores e retorna um único valor que representa algum aspecto agregado desses valores. Exemplos comuns incluem:
\begin{itemize}
    \item \textbf{Média}: A média aritmética de um conjunto de números \(x_1, x_2, \ldots, x_n\) é dada por \(\frac{1}{n}\sum_{i=1}^{n} x_i\).
    \item \textbf{Máximo e Mínimo}: A função máximo retorna o maior valor em um conjunto, enquanto a função mínimo retorna o menor valor.
    \item \textbf{Produto}: O produto de um conjunto de números \(x_1, x_2, \ldots, x_n\) é dado por \(\prod_{i=1}^{n} x_i\).
    \item \textbf{Contagem}: A função contagem retorna o número de elementos em um conjunto.
\end{itemize}

O \textbf{somatório}, por exemplo, é uma função de agregação linear que mapeia uma sequência \((x_1,\dots,x_n)\) em sua soma:
\[\sum_{i=1}^{n} x_i.\]

Esse conceito é especialmente útil em otimização e em análise combinatória: somatórios aparecem o tempo todo e serão explorados ao longo deste trabalho.

\paragraph{Exemplos com grafos.}
As sessões seguintes explorarão em maiores detalhes grafos e dígrafos, mas agora, consideremos a ideia básica: um \textbf{grafo} é um conjunto de pontos (\emph{vértices}) ligados por linhas (\emph{arestas}). No caso \emph{não dirigido}, as linhas não têm seta; no caso \emph{dirigido}, cada linha tem um sentido e é chamada de \emph{arco}.

\paragraph{}
A noção de somatória aparecerá naturalmente quando lidamos com propriedades dos grafos. Por exemplo:

Seja um grafo não dirigido \(G=(V,E)\). O \textbf{grau} de um vértice \(v\in V\), escrito \(\deg(v)\), é quantas arestas tocam em \(v\). A soma dos graus de todos os vértices conta cada aresta \emph{duas vezes} (uma por extremidade), portanto:
\[\sum_{v\in V} \deg(v) = 2\,|E|.\]

Em grafos dirigidos, distinguimos \(\deg^{-}(v)\) (quantos arcos \emph{chegam} em \(v\)) e \(\deg^{+}(v)\) (quantos arcos \emph{saem} de \(v\)). Cada arco contribui com 1 para um grau de saída e 1 para um grau de entrada, logo:
\[\sum_{v\in V} \deg^{-}(v) = \sum_{v\in V} \deg^{+}(v) = |E|.\]

\paragraph{}
Agora suponha que cada aresta/arco \(e\in E\) tenha um \emph{peso} (ou \emph{custo}) \(c(e)\ge 0\). O \textbf{custo total} de um subconjunto \(F\subseteq E\) é simplesmente a soma dos pesos das arestas escolhidas:
\[C(F) = \sum_{e\in F} c(e).\]
De maneira análoga, se \(X\subseteq V\) é um conjunto de vértices, o \textbf{valor total} (ou peso total) dos arcos que \emph{saem} de \(X\) é a soma dos pesos dessas setas. Usaremos mais adiante a notação \(\delta^{+}(X)\) para o conjunto de arcos que saem de \(X\) (ver a seção de dígrafos); com essa notação,
\[\operatorname{val}^+(X) = \sum_{e\in \delta^{+}(X)} c(e).\]
Esses exemplos mostram como somatórios capturam propriedades estruturais do grafo por meio de funções simples de agregação.

\subsubsection{Funções Especiais}

\paragraph{Função de custo}
\paragraph{}Uma \textbf{função de custo} é uma função \(c: A \to \mathbb{R}^+\) que atribui um valor numérico não negativo (custo) a cada elemento de um conjunto \(A\). Essas funções são amplamente utilizadas em otimização, economia e teoria dos grafos para modelar despesas, penalidades ou recursos associados a escolhas ou ações.

\paragraph{}
Exemplo: Considere um conjunto de tarefas \(T = \{t_1, t_2, t_3\}\). Uma função de custo \(c: T \to \mathbb{R}^+\) pode ser definida como:
\[c(t_1) = 5, \quad c(t_2) = 10, \quad c(t_3) = 3.\]
Aqui, \(c(t_i)\) representa o custo de realizar a tarefa \(t_i\). 

\paragraph{}
Depende diretamente do conceito de somatório, pois frequentemente queremos minimizar o custo total de um conjunto de escolhas. Se \(S \subseteq A\) é um subconjunto de elementos escolhidos, o custo total associado a \(S\) é dado por:
\[C(S) = \sum_{a \in S} c(a).\] 

\paragraph{Função c-disjunta}
Uma \textbf{função c-disjunta} é uma função \(f: A \to B\) que, para quaisquer \(a_1, a_2 \in A\) com \(a_1 \neq a_2\), as imagens \(f(a_1)\) e \(f(a_2)\) são disjuntas, ou seja, \(f(a_1) \cap f(a_2) = \varnothing\). Em outras palavras, elementos distintos do domínio são mapeados para conjuntos disjuntos no contradomínio.

\begin{center}
\begin{tikzpicture}[>=Stealth, node distance=1.0cm]
    % Domínio A (retângulos azuis)
    \node[draw, rounded corners, fill=blue!10, minimum width=9mm, minimum height=6mm] (a1cd) {$a_1$};
    \node[draw, rounded corners, fill=blue!10, below=of a1cd, minimum width=9mm, minimum height=6mm] (a2cd) {$a_2$};
    \node[draw, rounded corners, fill=blue!10, below=of a2cd, minimum width=9mm, minimum height=6mm] (a3cd) {$a_3$};
    \node[draw=blue!50, rounded corners, fit=(a1cd)(a2cd)(a3cd), inner sep=4pt, label={[blue!60]above:$A$}] {};

    % Universo U (círculos verdes como átomos) à direita
    \node[circle, draw, fill=green!10, right=3.2cm of a1cd, minimum size=6mm] (ua) {$a$};
    \node[circle, draw, fill=green!10, right=8mm of ua, minimum size=6mm] (ub) {$b$};
    \node[circle, draw, fill=green!10, below=8mm of ua, minimum size=6mm] (uc) {$c$};
    \node[circle, draw, fill=green!10, right=8mm of uc, minimum size=6mm] (ud) {$d$};
    % Caixa do universo U
    \node[draw=gray!60, rounded corners, fit=(ua)(ub)(uc)(ud), inner sep=6pt, label={[gray!70]above:$U$}] (Ubox) {};

    % Destaque das imagens (subconjuntos disjuntos de U)
    \node[draw=purple!70!black, rounded corners, fit=(ua)(ub), inner sep=2pt, label={[purple!70!black]above:$f(a_1)$}] (S1) {};
    \node[draw=orange!80!black, rounded corners, fit=(uc), inner sep=3pt, label={[orange!80!black]below:$f(a_2)$}] (S2) {};
    \node[draw=blue!60!black, rounded corners, fit=(ud), inner sep=3pt, label={[blue!60!black]below:$f(a_3)$}] (S3) {};

    % Setas de mapeamento para os subconjuntos (apontando para as caixas de destaque)
    \draw[->, thick, draw=gray!70] (a1cd) -- (S1.west);
    \draw[->, thick, draw=gray!70] (a2cd) -- (S2.west);
    \draw[->, thick, draw=gray!70] (a3cd) -- (S3.west);
\end{tikzpicture}
\captionof{figure}{Função c-disjunta: cada \(a\in A\) mapeia para um \emph{subconjunto} de \(U\), e as imagens são dois a dois disjuntas (sem sobreposição). No exemplo, $f(a_1)=\{a,b\}$, $f(a_2)=\{c\}$ e $f(a_3)=\{d\}$.}
\label{fig:c-disjunta}
\end{center}

\paragraph{}
Exemplo: Considere \(A = \{1, 2, 3\}\) e \(B = \{\{a\}, \{b\}, \{c\}, \{d\}\}\). Definimos a função \(f: A \to B\) como:
\[f(1) = \{a, b\}, \quad f(2) = \{c\}, \quad f(3) = \{d\}.\]
Aqui, \(f\) é c-disjunta, pois \(f(1) \cap f(2) = \varnothing\), \(f(1) \cap f(3) = \varnothing\) e \(f(2) \cap f(3) = \varnothing\).

\paragraph{Função c-viável}

\paragraph{}Uma \textbf{função c-viável} é uma função \(g: A \to \mathbb{R}^+\) que satisfaz certas condições de viabilidade relacionadas a um conjunto de restrições ou critérios. Essas funções são frequentemente usadas em otimização e teoria dos grafos para garantir que as soluções propostas atendam a requisitos específicos.
\paragraph{}
Exemplo: Considere um conjunto de projetos \(P = \{p_1, p_2, p_3\}\) e uma função \(g: P \to \mathbb{R}^+\) que atribui um valor de viabilidade a cada projeto. Suponha que temos a restrição de que a soma dos valores de viabilidade deve ser menor ou igual a um certo limite \(L\). Se definirmos:
\[g(p_1) = 4, \quad g(p_2) = 6, \quad g(p_3) = 3,\]
então a função \(g\) é c-viável se \(g(p_1) + g(p_2) + g(p_3) \leq L\).
\paragraph{}Essas funções são essenciais para garantir que as soluções propostas em problemas de otimização sejam práticas e atendam aos critérios estabelecidos.

\paragraph{Funções de otimização}

\paragraph{}
Do ponto de vista semiótico, “melhor” exprime uma preferência entre interpretações: ao comparar alternativas, escolhemos aquela cuja significação é mais adequada a um critério. Para tornar isso operacional, a matemática troca “fazer mais sentido” por “ter maior (ou menor) valor” em uma escala formal: fixamos (i) um conjunto de soluções viáveis \(\mathcal{F}\) e (ii) uma função numérica sobre \(\mathcal{F}\) que induz uma ordem de comparação.

\paragraph{}
Formalmente, usamos uma \textbf{função objetivo} (ou \textbf{função de otimização})
\[h:\; \mathcal{F} \to \mathbb{R},\]
que atribui um número real a cada solução. Buscamos uma solução \(S^*\in\mathcal{F}\) que \emph{minimize} ou \emph{maximize} \(h\) (isto é, um \(\operatorname*{argmin}\) ou \(\operatorname*{argmax}\)). Quando esse número resulta da soma de contribuições elementares, obtemos o caso aditivo, em ligação direta com os somatórios apresentados antes.

\paragraph{}
Caso \emph{aditivo}. Quando cada elemento \(a\in A\) tem um custo \(c(a)\ge 0\) e as soluções são subconjuntos \(S\subseteq A\), a função objetivo mais comum é o custo total
\[C(S)=\sum_{a\in S} c(a),\]
que desejamos \emph{minimizar}. De modo análogo, se cada item tem um benefício \(p(a)\ge 0\), podemos \emph{maximizar} o benefício total \(P(S)=\sum_{a\in S} p(a)\), possivelmente sujeito a restrições (por exemplo, de orçamento ou limite).

\paragraph{}
Outro exemplo: considere produtos \(X=\{x_1,x_2,x_3\}\) com lucro \(p(x_1)=10\), \(p(x_2)=15\), \(p(x_3)=7\). Se houver um limite de custo que impede escolher todos, o objetivo típico é escolher um subconjunto \(S\subseteq X\) que maximize \(\sum_{x\in S} p(x)\) respeitando as restrições. Essa forma reflete exatamente os somatórios introduzidos antes.

\paragraph{}
Em todos os casos, a função objetivo explicita o critério de “melhor”, e as restrições determinam quais soluções são aceitáveis.

\subsubsection{Otimização}
\paragraph{}
O princípio da navalha de occam nos diz que a explicação mais simples tende a ser a correta. Do ponto de vista semiótico, isso é escolher, entre interpretações possíveis, a que melhor satisfaz um critério. A matemática também se preocupa com identificar a “melhor” solução entre várias alternativas, mas traduz essa ideia em termos quantitativos: fixamos (i) um conjunto de soluções viáveis \(\mathcal{F}\) e (ii) uma função numérica sobre \(\mathcal{F}\) que induz uma ordem de comparação. 

\paragraph{}
Assim, a otimização envolve a maximização ou minimização de uma função objetivo \(h: \mathcal{F} \to \mathbb{R}\) sobre um conjunto de soluções viáveis \(\mathcal{F}\).

\paragraph{}
Esse conceito pode aparecer em muitas necessidades do dia-a-dia: uma empresa pode querer minimizar custos de produção, um viajante pode buscar o caminho mais curto entre dois pontos, ou um investidor pode tentar maximizar o retorno de um portfólio. Em cada caso, a função objetivo quantifica o que significa ser “melhor” ou “mais eficiente”.

\paragraph{}
Pensando em modelagem de problemas em grafos, podemos pensar em exemplos clássicos de otimização:

\begin{itemize}
    \item \textbf{Caminho mais curto}: Dado um grafo com pesos nas arestas, encontrar o caminho entre dois vértices que minimize a soma dos pesos das arestas percorridas.
    \item \textbf{Árvore geradora mínima}: Encontrar uma árvore que conecte todos os vértices de um grafo com o menor custo total das arestas.
    \item \textbf{Fluxo máximo}: Em um grafo direcionado com limites nas arestas, encontrar o fluxo máximo que pode ser enviado de uma fonte a um sumidouro sem exceder esses limites.
\end{itemize}

\paragraph{A dualidade}

\paragraph{}
Um tema recorrente em otimização é a \emph{dualidade}, que conecta problemas de minimização a problemas de maximização. 

Falando de forma intuitiva, a dualidade nos remete à contrastes, yin-yang, noite e dia, matéria e anti-matéria, máximos e mínimos. Em termos matemáticos, para cada problema de otimização (o \emph{primal}), existe um problema associado (o \emph{dual}) que oferece uma perspectiva complementar. Resolver um desses problemas pode fornecer insights ou soluções para o outro.

Considere como problema primal a minimização do custo de uma estrutura (como uma árvore geradora mínima) e como dual a maximização de um conjunto de pesos ou preços que justificam esse custo mínimo. A relação entre primal e dual é formalizada por teoremas de dualidade, que garantem que o valor ótimo do primal é igual ao valor ótimo do dual sob certas condições.

\paragraph{}
Esse ponto de vista leva a “teoremas min–max” que ligam problemas de \emph{minimização} a problemas de \emph{maximização} e fornecem certificados verificáveis de otimalidade.

\begin{center}
\begin{tikzpicture}[>=Stealth, node distance=1.4cm]
    % estilos
    \tikzset{
        box/.style={draw, rounded corners, fill=blue!5, minimum width=37mm, minimum height=10mm, align=center},
        arrow/.style={->, semithick, draw=gray!70},
        badge/.style={draw=green!50!black, rounded corners, fill=green!10, inner sep=4pt, font=\scriptsize}
    }

    % caixas primal (min) e dual (max)
    \node[box] (min) {Minimização\\(cobrir)};
    \node[box, right=4.2cm of min] (max) {Maximização\\(empacotar)};

    % relação de valores dual <= primal
    \draw[arrow] (max.west) -- node[above, font=\scriptsize, black!70]{valor dual $\;\le\;$ valor primal} (min.east);

    % destaque do "min--max" (igualdade no ótimo)
    \path (min) -- (max) node[pos=0.5, yshift=12mm, badge] (mm) {teorema min--max: igualdade no ótimo};

    % certificado de otimalidade
    \path (min) -- (max) node[pos=0.5, yshift=-12mm, badge] (cert) {certificado: verificação simples do ótimo};
\end{tikzpicture}
\captionof{figure}{Intuição de min--max: um problema de cobrir (minimização) e um de empacotar (maximização) andam juntos. Sempre vale valor dual $\le$ valor primal; quando há igualdade, temos um certificado de otimalidade.}
\label{fig:min-max-cert}
\end{center}

\paragraph{}
No contexto de grafos, a otimização costuma aparecer como a busca por subestruturas (caminhos, árvores, cortes, fluxos) que minimizam ou maximizam um custo, sempre respeitando a topologia do grafo.

\paragraph{}
Nesta dissertação, essa noção de otimização é central: olhamos para o mesmo problema por dois ângulos que se completam. No lado “primal”, queremos montar diretamente a arborescência de menor custo. O algoritmo de Chu–Liu/Edmonds faz isso de forma gulosa: ajusta os custos por vértice, cria arestas de custo zero (0‑arestas), contrai ciclos quando aparecem e segue até montar a solução ótima. 
(\cite{chu1965,edmonds1967optimum}).

\paragraph{}
No lado “dual”, em vez de montar a árvore, colocamos custos em cortes do grafo com raiz $r$. A regra é simples: nenhum custo pode ultrapassar o custo das arestas que cruzam o corte. Buscamos escolher esses custos para somar o máximo possível. As arestas que “batem no limite” viram 0‑arestas, e a partir delas conseguimos reconstruir uma arborescência ótima. Essa visão, desenvolvida por Frank, leva a um teorema min–max e a um procedimento em duas etapas: primeiro ajustamos os custos, depois extraímos a solução usando apenas 0‑arestas. 
(cf. \cite{frank2014,schrijver2003comb})

\subsection{Problemas interessantes}

\paragraph{}
Qual o número mínimo de cores necessárias para colorir um mapa de países, de modo que países vizinhos tenham cores diferentes? Qual o caminho mais curto entre duas cidades em um mapa rodoviário? Como encontrar a árvore geradora mínima que conecta todas as cidades com o menor custo total? Essas perguntas são exemplos clássicos de problemas que podem ser modelados e resolvidos usando a teoria dos grafos. Vistas sob a lente da navalha de occam, todas elas buscam a solução mais parcimoniosa que atende ao requisito: usar poucas cores, percorrer um caminho curto ou conectar tudo com custo mínimo.

\begin{center}
\begin{tikzpicture}[>=Stealth]
    % estilo de vértice e arestas
    \tikzset{v/.style={circle, draw, thick, minimum size=7mm, inner sep=0pt}, e/.style={semithick, draw=gray!70}}

    % vértices de um K4 (exige 4 cores)
    \node[v, fill=red!30, label=below:$v_1$]  (v1) at (0,0) {};
    \node[v, fill=blue!30, label=below:$v_2$] (v2) at (2.1,0) {};
    \node[v, fill=green!35, label=above:$v_3$] (v3) at (1,1.9) {};
    \node[v, fill=orange!40, label=right:$v_4$] (v4) at (1,0.75) {};

    % arestas (todas as pares -> K4)
    \draw[e] (v1) -- (v2);
    \draw[e] (v2) -- (v3);
    \draw[e] (v3) -- (v1);
    \draw[e] (v1) -- (v4);
    \draw[e] (v2) -- (v4);
    \draw[e] (v3) -- (v4);
\end{tikzpicture}
\captionof{figure}{Coloração de grafos: exemplo de coloração própria do grafo completo $K_4$. Como $K_4$ é completo, precisamos de 4 cores para colorir seus vértices de modo que vértices adjacentes tenham cores diferentes. Uma coloração é uma função $\varphi:V\to C$ tal que, se $uv\in E$, então $\varphi(u)\neq\varphi(v)$.}
\label{fig:coloracao}
\end{center}

\paragraph{}
Sem a teoria dos grafos, seria difícil formalizar e resolver esses problemas de maneira eficiente. Ao representar situações do mundo real como grafos, tornamos a parcimônia da navalha de occam algo operacional: escolhemos uma medida simples (número de cores, comprimento, custo) e aplicamos algoritmos que, entre as soluções viáveis, minimizam ou maximizam esse critério — produzindo soluções ótimas ou, quando necessário, boas aproximações.

\subsubsection{Grafos}
\paragraph{}
Falamos bastante de grafos ao longo do texto, aqui fixamos a noção básica. 

\paragraph{}
Um \textbf{grafo} \(G = (V, E)\) é uma estrutura matemática composta por um conjunto \(V\) de \emph{vértices} (ou \emph{nós}) e um conjunto \(E\) de \emph{arestas} (ou \emph{ligações}) que conectam pares de vértices. 

\begin{center}
\begin{tikzpicture}[>=Stealth]
    % estilos
    \tikzset{v/.style={circle, draw, fill=blue!10, minimum size=7mm, inner sep=0pt}, e/.style={semithick, draw=gray!70}}

    % vértices (V)
    \node[v, label=below:$v_1$] (v1) at (0,0) {};
    \node[v, label=below:$v_2$] (v2) at (2.0,0) {};
    \node[v, label=above:$v_3$] (v3) at (2.0,1.6) {};
    \node[v, label=above:$v_4$] (v4) at (0,1.6) {};

    % arestas (E: pares não ordenados)
    \draw[e] (v1) -- (v2);
    \draw[e] (v2) -- (v3);
    \draw[e] (v3) -- (v4);
    \draw[e] (v4) -- (v1);

    % legenda com os conjuntos V e E
    \node[align=left, anchor=west] at (3.0,0.9) {$V=\{v_1,v_2,v_3,v_4\}$\\ $E=\{\{v_1,v_2\},\{v_2,v_3\},\{v_3,v_4\},\{v_4,v_1\}\}$};
\end{tikzpicture}
\captionof{figure}{Definição de grafo: exemplo de grafo simples \(G=(V,E)\). Pontos representam os vértices \(V\) e linhas representam as arestas \(E\), que são pares não ordenados de vértices distintos.}
\label{fig:def-grafo}
\end{center}

\paragraph{}
O conjunto de vértices \(V\) pode ser definido como \(V = \{v_1, v_2, \ldots, v_n\}\), onde cada \(v_i\) representa um ponto distinto no grafo. O conjunto de arestas \(E\) é um conjunto de pares não ordenados de vértices, ou seja, \(E \subseteq \{\{u, v\} \mid u, v \in V, u \neq v\}\). Cada aresta \(\{u, v\}\) indica uma conexão entre os vértices \(u\) e \(v\).

\paragraph{}
Esses vértices e arestas podem representar uma variedade de entidades e relações no mundo real. Por exemplo, em um grafo que modela uma rede social, os vértices podem representar pessoas, e as arestas podem representar amizades entre elas. Em um grafo que representa uma rede de transporte, os vértices podem ser cidades, e as arestas podem ser estradas ou rotas de voo conectando essas cidades.

\paragraph{}
Tendo em mente esses problemas, podemos falar de custos associados às arestas. Por exemplo, em um grafo que representa uma rede de transporte, cada aresta pode ter um custo associado, como a distância entre duas cidades ou o tempo necessário para percorrer uma estrada. Em um grafo que modela uma rede de comunicação, as arestas podem ter custos relacionados à largura de banda ou à latência.

\paragraph{}
Esses custos representam uma função \(c: E \to \mathbb{R}^+\) que atribui um valor numérico não negativo a cada aresta do grafo. Assim, para cada aresta \(\{u, v\} \in E\), \(c(\{u, v\})\) representa o custo associado a essa conexão.

\begin{center}
\begin{tikzpicture}[>=Stealth]
    % estilos
    \tikzset{v/.style={circle, draw, fill=blue!10, minimum size=7mm, inner sep=0pt}, e/.style={semithick, draw=gray!70}}

    % vértices
    \node[v, label=below:$v_1$] (v1) at (0,0) {};
    \node[v, label=below:$v_2$] (v2) at (2.2,0) {};
    \node[v, label=above:$v_3$] (v3) at (3.0,1.8) {};
    \node[v, label=above:$v_4$] (v4) at (-0.4,1.8) {};

    % arestas com custos (rótulos c(e))
    \draw[e] (v1) -- node[midway, fill=white, inner sep=2pt, font=\scriptsize] {3} (v2);
    \draw[e] (v2) -- node[midway, fill=white, inner sep=2pt, font=\scriptsize] {5} (v3);
    \draw[e] (v3) -- node[midway, fill=white, inner sep=2pt, font=\scriptsize] {2} (v4);
    \draw[e] (v4) -- node[midway, fill=white, inner sep=2pt, font=\scriptsize] {4} (v1);
    \draw[e] (v1) -- node[midway, fill=white, inner sep=2pt, font=\scriptsize] {6} (v3);

    % legenda breve
    \node[align=left, anchor=west] at (4.0,1.0) {$c:E\to\mathbb{R}^+$\\ $c(\{v_1,v_2\})=3,\; c(\{v_1,v_3\})=6,\dots$};
\end{tikzpicture}
\captionof{figure}{Grafo com custos nas arestas: a função $c:E\to\mathbb{R}^+$ atribui a cada aresta um custo $c(e)$ (mostrado junto às arestas). Por exemplo, $c(\{v_1,v_2\})=3$. O custo total de um subconjunto $F\subseteq E$ é $C(F)=\sum_{e\in F} c(e)$.}
\label{fig:grafo-custos}
\end{center}

\paragraph{}
Os problemas não envolvem a necessidade de direção nas conexões. No entanto, em muitos contextos, as conexões entre os vértices têm uma direção específica. Em problemas de fluxo de tráfego, por exemplo, as ruas podem ser de mão única, ou em redes de comunicação, os dados podem ser enviados em uma direção específica. Nesses casos, usamos \emph{grafos dirigidos} (ou grafos direcionados ou simplesmente dígrafos), onde as arestas são pares ordenados de vértices. Esse é um conceito central para o desenvolvimento deste trabalho e portanto vamos explorá-lo em detalhes na próxima seção.

\paragraph{}
Antes, de avançarmos para falar de dígrafos, vale notar que grafos apresentam diversas estruturas interessantes, como caminhos, ciclos, árvores e componentes conexas. Essas estruturas são fundamentais para entender a topologia e as propriedades dos grafos, e muitas vezes são o foco de problemas de otimização.

\paragraph{Caminhos}
\paragraph{}Um \textbf{caminho} em um grafo é uma sequência de vértices conectados por arestas. Formalmente, um caminho \(P\) de comprimento \(k\geq 1\) é uma sequência de vértices \(P = (v_1, v_2, \ldots, v_{k+1})\) tal que cada par consecutivo \((v_i, v_{i+1})\) é uma aresta em \(E\). O comprimento do caminho é o número de arestas que ele contém, que é \(k\).

\begin{center}
\begin{tikzpicture}[>=Stealth]
    % estilos
    \tikzset{v/.style={circle, draw, fill=blue!10, minimum size=7mm, inner sep=0pt}, e/.style={semithick, draw=gray!60}, p/.style={very thick, draw=blue!60}}

    % vértices do caminho
    \node[v, label=below:$v_1$] (v1) at (0,0) {};
    \node[v, label=below:$v_2$] (v2) at (1.8,0) {};
    \node[v, label=below:$v_3$] (v3) at (3.6,0) {};
    \node[v, label=below:$v_4$] (v4) at (5.4,0) {};

    % vértices extras (arestas que não fazem parte do caminho)
    \node[v, label=above:$x$] (x) at (1.8,1.2) {};
    \node[v, label=above:$y$] (y) at (3.6,1.2) {};

    % arestas não pertencentes ao caminho (cinza)
    \draw[e] (v1) -- (x);
    \draw[e] (x) -- (v3);
    \draw[e] (y) -- (v4);

    % caminho destacado (azul)
    \draw[p] (v1) -- (v2);
    \draw[p] (v2) -- (v3);
    \draw[p] (v3) -- (v4);

    % legenda
    \node[align=left, anchor=west] at (6.2,0.6) {$P=(v_1,v_2,v_3,v_4)$\\ $|P|=3$};
\end{tikzpicture}
\captionof{figure}{Caminho em grafo não dirigido: o caminho $P=(v_1,v_2,v_3,v_4)$ está destacado em azul. Seu comprimento é o número de arestas percorridas, $|P|=3$.}
\label{fig:caminho}
\end{center}

\paragraph{Ciclos}
\paragraph{}Um \textbf{ciclo} é um caminho que começa e termina no mesmo vértice, ou seja, \(v_1 = v_{k+1}\). Formalmente, um ciclo \(C\) é uma sequência de vértices \(C = (v_1, v_2, \ldots, v_k, v_1)\) tal que cada par consecutivo \((v_i, v_{i+1})\) é uma aresta em \(E\) e \(k \geq 2\). O comprimento do ciclo é o número de arestas que ele contém, que é \(k\).

\begin{center}
\begin{tikzpicture}[>=Stealth]
    % estilos
    \tikzset{v/.style={circle, draw, fill=blue!10, minimum size=7mm, inner sep=0pt}, e/.style={semithick, draw=gray!60}, c/.style={very thick, draw=blue!60}}

    % vértices do ciclo (dispostos aproximadamente em círculo)
    \node[v, label=below:$v_1$] (v1) at (0,0) {};
    \node[v, label=right:$v_2$] (v2) at (2.0,0.8) {};
    \node[v, label=above:$v_3$] (v3) at (1.2,2.1) {};
    \node[v, label=left:$v_4$] (v4) at (-0.8,1.6) {};

    % arestas não pertencentes ao ciclo (cinza)
    \node[v, label=right:$x$] (x) at (2.6,1.8) {};
    \draw[e] (x) -- (v2);
    \draw[e] (x) -- (v3);

    % ciclo destacado (azul)
    \draw[c] (v1) -- (v2);
    \draw[c] (v2) -- (v3);
    \draw[c] (v3) -- (v4);
    \draw[c] (v4) -- (v1);

    % legenda
    \node[align=left, anchor=west] at (3.2,1.0) {$C=(v_1,v_2,v_3,v_4,v_1)$\\ $|C|=4$};
\end{tikzpicture}
\captionof{figure}{Ciclo em grafo não dirigido: o ciclo $C=(v_1,v_2,v_3,v_4,v_1)$ está destacado em azul. Seu comprimento é o número de arestas, $|C|=4$.}
\label{fig:ciclo}
\end{center}

\paragraph{Árvores}
\paragraph{}Uma \textbf{árvore} é um grafo conexo e acíclico. Formalmente, uma árvore \(T\) é um grafo \(T = (V_T, E_T)\) onde \(V_T \subseteq V\) e \(E
_T \subseteq E\), que satisfaz as seguintes propriedades:
\begin{itemize}
    \item \(T\) é conexo: existe um caminho entre qualquer par de vértices em \(V_T\).
    \item \(T\) é acíclico: não contém ciclos.
\end{itemize}
Além disso, uma árvore com \(n\) vértices sempre tem exatamente \(n-1\) arestas.

\begin{center}
\begin{tikzpicture}[>=Stealth]
    % estilos
    \tikzset{v/.style={circle, draw, fill=blue!10, minimum size=7mm, inner sep=0pt}, e/.style={semithick, draw=gray!70}}

    % vértices da árvore
    \node[v, label=below:$r$] (r) at (0,0) {};
    \node[v, label=above:$a$] (a) at (-1.6,1.4) {};
    \node[v, label=above:$b$] (b) at (0.0,1.8) {};
    \node[v, label=above:$c$] (c) at (1.6,1.4) {};
    \node[v, label=above:$d$] (d) at (-2.4,2.8) {};
    \node[v, label=above:$e$] (e) at (-0.8,3.0) {};
    \node[v, label=above:$f$] (f) at (0.8,3.0) {};

    % arestas da árvore (sem ciclos)
    \draw[e] (r) -- (a);
    \draw[e] (r) -- (b);
    \draw[e] (r) -- (c);
    \draw[e] (a) -- (d);
    \draw[e] (a) -- (e);
    \draw[e] (b) -- (f);

    % indicação de |E| = |V| - 1
    \node[align=left, anchor=west] at (3.0,1.8) {$|V_T|=7,\; |E_T|=6$\\ $|E_T|=|V_T|-1$};
\end{tikzpicture}
\captionof{figure}{Árvore: grafo conexo e acíclico. No exemplo, $|V_T|=7$ e $|E_T|=6$, satisfazendo $|E_T|=|V_T|-1$. Não há ciclos e existe um único caminho simples entre quaisquer dois vértices.}
\label{fig:arvore}
\end{center}

\paragraph{Componentes conexas}
\paragraph{}Uma \textbf{componente conexa} de um grafo é um subgrafo maximal que é conexo. Formalmente, uma componente conexa \(C\) é um subgrafo \(C = (V_C, E_C)\) onde \(V_C \subseteq V\) e \(E_C \subseteq E\), que satisfaz a seguinte propriedade:
\begin{itemize}
    \item \(C\) é conexo: existe um caminho entre qualquer par de vértices em \(V_C\).
\end{itemize}
Além disso, \(C\) é maximal, o que significa que não é possível adicionar mais vértices ou arestas a \(C\) sem perder a propriedade de conexidade.

\begin{center}
\begin{tikzpicture}[>=Stealth]
    % estilos
    \tikzset{v/.style={circle, draw, fill=blue!10, minimum size=7mm, inner sep=0pt}, e/.style={semithick, draw=gray!70}, comp/.style={draw=blue!40, rounded corners}}

    % Componente C1 (à esquerda)
    \node[v, label=above:$a$] (a) at (0,1.2) {};
    \node[v, label=below:$b$] (b) at (-0.8,0) {};
    \node[v, label=below:$c$] (c) at (0.8,0) {};
    \draw[e] (a) -- (b);
    \draw[e] (a) -- (c);
    \node[comp, fit=(a)(b)(c), inner sep=20pt, label={[blue!60]above:$C_1$}] {};

    % Componente C2 (centro)
    \node[v, label=below:$d$] (d) at (4,0) {};
    \node[v, label=below:$e$] (e) at (5.6,0) {};
    \draw[e] (d) -- (e);
    \node[comp, fit=(d)(e), inner sep=15pt, label={[blue!60]above:$C_2$}] {};

    % Componente C3 (direita)
    \node[v, label=above:$f$] (f) at (8.6,1.2) {};
    \node[v, label=below:$g$] (g) at (7.8,0) {};
    \node[v, label=below:$h$] (h) at (9.4,0) {};
    \draw[e] (f) -- (g);
    \draw[e] (f) -- (h);
    \draw[e] (g) -- (h);
    \node[comp, fit=(f)(g)(h), inner sep=20pt, label={[blue!60]above:$C_3$}] {};

    % texto explicativo
    \node[align=center, anchor=north west] at (-0.2,-1.0) {Cada caixa destaca uma \emph{componente conexa}.\\ Não há arestas entre $C_1$, $C_2$ e $C_3$.};
\end{tikzpicture}
\captionof{figure}{Componentes conexas: o grafo possui três componentes $C_1$, $C_2$ e $C_3$. Cada $C_i$ é conexo e \emph{maximal}: adicionar qualquer vértice de fora quebraria a propriedade de conexidade interna ou exigiria arestas ausentes.}
\label{fig:componentes}
\end{center}

\paragraph{}
Existem muitas outras estruturas e propriedades interessantes em grafos, mas essas são algumas das mais fundamentais para esse trabalho. E para o nosso objetivo, nos interessa entendê-las em mais profundidade em grafos direcionados.

\subsubsection{Quando a direção importa: dígrafos}
\paragraph{}
Um \textbf{grafo dirigido} (grafos direcionados ou simplesmente dígrafos) é uma estrutura matemática composta por um conjunto \(V\) de \emph{vértices} e um conjunto \(A\) de \emph{arcos} (ou \emph{arestas direcionadas}) que conectam pares ordenados de vértices.

\paragraph{}
Por vértices entendemos os mesmos pontos que em grafos comuns, mas agora as conexões entre eles têm uma direção específica. Cada arco \((u, v) \in A\) indica uma conexão direcionada do vértice \(u\) para o vértice \(v\), significando que a relação ou fluxo ocorre de \(u\) para \(v\).

\paragraph{}
Em digrafos podem ocorrer laços (arcos que conectam um vértice a ele mesmo, como \((u, u)\)) e múltiplos arcos entre o mesmo par de vértices (como \((u, v)\) e \((u, v)\) distintos). Pictograficamente representamos essas condições com setas com dupla ponta ou com rótulos diferentes.

\begin{center}
\begin{tikzpicture}[>=Stealth]
    % estilos
    \tikzset{v/.style={circle, draw, fill=blue!10, minimum size=7mm, inner sep=0pt}, a/.style={->, semithick, draw=gray!70}, p/.style={->, draw=gray!60}};

    % vértices
    \node[v, label=below:$a$] (a) at (0,0) {};
    \node[v, label=below:$b$] (b) at (2,0) {};
    \node[v, label=above:$c$] (c) at (1,1.5) {};
    \node[v, label=above:$d$] (d) at (3,1.5) {};
    \node[v, label=below:$e$] (e) at (4,0) {};
    \node[v, label=below:$f$] (f) at (5,0) {};
    \node[v, label=above:$g$] (g) at (4.5,1.5) {};
    \node[v, label=above:$h$] (h) at (5.5,1.5) {};

    % arcos
    \draw[a] (a) -- (b);
    \draw[a] (c) -- (a);
    \draw[a] (b) -- (c);
    \draw[a] (d) -- (b);
    \draw[a] (e) -- (d);
    \draw[a] (f) -- (e);
    \draw[a] (g) -- (f);
    \draw[a] (h) -- (g);
    \draw[p] (b) to[bend left=30] (e);
    \draw[p] (e) to[bend left=30] (b);
    \draw[p] (f) to[bend left=40] (f); % laço
\end{tikzpicture}
\captionof{figure}{Dígrafo: exemplo de grafo dirigido \(D=(V,A)\). Pontos representam os vértices \(V\) e setas representam os arcos \(A\), que são pares ordenados de vértices. Laços (como \((f,f)\)) e múltiplos arcos (como \((b,e)\) e \((e,b)\)) são permitidos.}
\label{fig:def-digrafo-simples}
\end{center}       

\paragraph{}
Assim, temos os conceitos de cabeça e cauda de um arco: em \((u, v)\), \(u\) é a \emph{cauda} (origem) e \(v\) é a \emph{cabeça} (destino). Esses conceitos podem ser formalizados por meio de funções \(s, t: A \to V\), onde \(s((u, v)) = u\) (cauda) e \(t((u, v)) = v\) (cabeça).

\begin{center}
\begin{tikzpicture}[>=Stealth]
    % estilos
    \tikzset{v/.style={circle, draw, fill=blue!10, minimum size=7mm, inner sep=0pt}, a/.style={->, semithick, draw=gray!70}}

    % vértices
    \node[v, label=below:$u$] (u) at (0,0) {};
    \node[v, label=below:$v$] (v) at (2.6,0) {};
    \node[v, label=above:$x$] (x) at (1.3,1.5) {};

    % arcos
    \draw[a] (u) -- node[above, sloped, font=\scriptsize] {$a$} (v);
    \draw[a] (x) -- (u);
    \draw[a] (v) -- (x);

    % anotações de cauda/cabeça
    \node[font=\scriptsize, anchor=north] at ($(u)!0.5!(v)$) {$a=(u,v)$};
    \path (u) ++(0,-0.9) node[font=\scriptsize] {cauda $s(a)=u$};
    \path (v) ++(0,-0.9) node[font=\scriptsize] {cabeça $t(a)=v$};
\end{tikzpicture}
\captionof{figure}{Dígrafos: arcos têm direção. No arco $a=(u,v)$, $u$ é a \emph{cauda} (origem) e $v$ é a \emph{cabeça} (destino). As funções $s,t:A\to V$ dão a origem e o destino de cada arco.}
\label{fig:def-digrafo}
\end{center}

\paragraph{}
Tal qual os grafos comuns, os dígrafos podem ter custos associados aos arcos. A função de custo \(c: A \to \mathbb{R}^+\) atribui um valor numérico geralmente não negativo a cada arco do dígrafo. Assim, para cada arco \((u, v) \in A\), \(c((u, v))\) representa o custo associado a essa conexão direcionada.

\begin{center}
\begin{tikzpicture}[>=Stealth]
    % estilos
    \tikzset{v/.style={circle, draw, fill=blue!10, minimum size=7mm, inner sep=0pt}, a/.style={->, semithick, draw=gray!70}}

    % vértices
    \node[v, label=below:$s$] (s) at (0,0) {};
    \node[v, label=above:$u$] (u) at (2,1.0) {};
    \node[v, label=below:$v$] (v) at (2,-1.0) {};
    \node[v, label=below:$t$] (t) at (4,0) {};

    % arcos com custos
    \draw[a] (s) -- node[midway, fill=white, inner sep=2pt, font=\scriptsize] {2} (u);
    \draw[a] (s) -- node[midway, fill=white, inner sep=2pt, font=\scriptsize] {4} (v);
    \draw[a] (u) -- node[midway, fill=white, inner sep=2pt, font=\scriptsize] {1} (v);
    \draw[a] (u) -- node[midway, fill=white, inner sep=2pt, font=\scriptsize] {5} (t);
    \draw[a] (v) -- node[midway, fill=white, inner sep=2pt, font=\scriptsize] {3} (t);

    % legenda
    \node[align=left, anchor=west] at (5.1,0.9) {$c:A\to\mathbb{R}^+$\\ $c((s,u))=2,\; c((v,t))=3,\dots$\\ $C(P)=\sum\limits_{a\in P} c(a)$};
\end{tikzpicture}
\captionof{figure}{Dígrafos com custos nos arcos: a função $c:A\to\mathbb{R}^+$ atribui a cada arco $(u,v)$ um custo $c((u,v))$ (mostrado junto à seta). O custo de um caminho $P$ é a soma dos custos de seus arcos, $C(P)=\sum_{a\in P} c(a)$.}
\label{fig:digrafo-custos}
\end{center}

\paragraph{Subestruturas em dígrafos}

\paragraph{}
Assim como falamos na sessão anterior, os dígrafos também possuem as mesmas estruturas especiais que os grafos, essas estruturas chamadas sub-digrafos mudam um pouco em nomenclatura caminhos quando direcionados são chamados de trilhas, e ciclos são chamados de circuitos, árvores viram arborescências, e componentes conexas são componentes fortemente conexas. Além da nomenclatura, a direção dos arcos traz algumas nuances importantes para essas estruturas, não necessariamente exploraremos essas nuances aqui, mas é importante estar ciente delas.

\paragraph{}
Um \textbf{subdígrafo} \(D' = (V', A')\) de um dígrafo \(D = (V, A)\) é um dígrafo onde \(V' \subseteq V\) e \(A' \subseteq A\). Ou seja, \(D'\) é formado por um subconjunto dos vértices e arcos de \(D\).

\paragraph{Subdígrafos Induzidos}
Um subdígrafo pode ser \emph{induzido} por um conjunto de vértices \(V' \subseteq V\), denotado como \(D[V']\). Nesse caso, o conjunto de arcos \(A'\) inclui todos os arcos em \(A\) que têm ambas as extremidades em \(V'\), ou seja, \(A' = \{(u, v) \in A \mid u, v \in V'\}\).
\begin{center}
\begin{tikzpicture}[>=Stealth]
    % estilos
    \tikzset{v/.style={circle, draw, fill=blue!10, minimum size=7mm, inner sep=0pt}, a/.style={->, semithick, draw=gray!70}, sub/.style={->, draw=blue!60}}
    % vértices
    \node[v, label=below:$a$] (a) at (0,0) {};
    \node[v, label=below:$b$] (b) at (2,0) {};
    \node[v, label=above:$c$] (c) at (1,1.5) {};
    \node[v, label=above:$d$] (d) at (3,1.5) {};
    \node[v, label=below:$e$] (e) at (4,0) {};
    \node[v, label=below:$f$] (f) at (5,0) {};
    \node[v, label=above:$g$] (g) at (4.5,1.5) {};
    \node[v, label=above:$h$] (h) at (5.5,1.5) {};
    % arcos
    \draw[a] (a) -- (b);
    \draw[a] (c) -- (a);
    \draw[a] (b) -- (c);
    \draw[a] (d) -- (b);
    \draw[a] (e) -- (d);
    \draw[a] (f) -- (e);
    \draw[a] (g) -- (f);
    \draw[a] (h) -- (g);
    \draw[sub] (b) to[bend left=30] (e);
    \draw[sub] (e) to[bend left=30] (b);
    \draw[sub] (f) to[bend left=40] (f); % laço
    % subdígrafo destacado
    \node[draw=blue!40, rounded corners, fit=(b)(c)(d)(e), inner sep=15pt, label={[blue!60]above:$D'$}] {};
\end{tikzpicture}
\captionof{figure}{Subdígrafo: o subdígrafo $D'=(V',A')$ está destacado em azul. Aqui, $V'=\{b,c,d,e\}$ e $A'=\{(b,c),(c,b),(d,b),(e,d),(b,e),(e,b)\}$.}
\label{fig:subdigrafo}
\end{center}

\paragraph{Subdigrafo Maximal}
Um subdigrafo é tido como maximal se não é possível adicionar mais vértices ou arcos a ele sem perder alguma propriedade específica, como conexidade ou aciclicidade.

\paragraph{Subdigrafo Gerador}
Um subdigrafo é tido como gerador se inclui todos os vértices do dígrafo original, ou seja, \(V' = V\). Nesse caso, o subdigrafo é formado por um subconjunto dos arcos do dígrafo original.

Com essas definições em mente, podemos explorar as subdigrafos específicos que citaremos ao longo da dissertação, começando pelas trilhas, circuitos, arborescências e componentes fortemente conexas.

\paragraph{Trilhas}

\paragraph{}
Uma \textbf{trilha} (ou caminho direcionado) em um dígrafo é uma sequência de vértices conectados por arcos que respeitam a direção. Formalmente, uma trilha \(P\) de comprimento \(k \geq 1\) é uma sequência de vértices \(P = (v_1, v_2, \ldots, v_{k+1})\) tal que cada par consecutivo \((v_i, v_{i+1})\) é um arco em \(A\). O comprimento da trilha é o número de arcos que ela contém, que é \(k\).

\begin{center}
\begin{tikzpicture}[>=Stealth]
    % estilos
    \tikzset{v/.style={circle, draw, fill=blue!10, minimum size=7mm, inner sep=0pt}, a/.style={->, semithick, draw=gray!60}, p/.style={->, draw=blue!60}};
    % vértices do caminho
    \node[v, label=below:$v_1$] (v1) at (0,0) {};
    \node[v, label=below:$v_2$] (v2) at (1.8,0) {};
    \node[v, label=below:$v_3$] (v3) at (3.6,0) {};
    \node[v, label=below:$v_4$] (v4) at (5.4,0) {};
    % vértices extras (arcos que não fazem parte do caminho)
    \node[v, label=above:$x$] (x) at (1.8,1.2) {};
    \node[v, label=above:$y$] (y) at (3.6,1.2) {};
    % arcos não pertencentes ao caminho (cinza)
    \draw[a] (v1) -- (x);
    \draw[a] (x) -- (v3);
    \draw[a] (y) -- (v4);
    % caminho destacado (azul)
    \draw[p] (v1) -- (v2);
    \draw[p] (v2) -- (v3);
    \draw[p] (v3) -- (v4);
    % legenda
    \node[align=left, anchor=west] at (6.2,0.6) {$P=(v_1,v_2,v_3,v_4)$\\ $|P|=3$};
\end{tikzpicture}
\captionof{figure}{Trilha em dígrafo: a trilha $P=(v_1,v_2,v_3,v_4)$ está destacada em azul. Seu comprimento é o número de arcos percorridos, $|P|=3$.}
\label{fig:trilha}
\end{center}

\paragraph{Circuitos}
\paragraph{}
Um \textbf{circuito} é um caminho direcionado que começa e termina no mesmo vértice, ou seja, \(v_1 = v_{k+1}\). Formalmente, um circuito \(C\) é uma sequência de vértices \(C = (v_1, v_2, \ldots, v_k, v_1)\) tal que cada par consecutivo \((v_i, v_{i+1})\) é um arco em \(A\) e \(k \geq 2\). O comprimento do circuito é o número de arcos que ele contém, que é \(k\).

\begin{center}
\begin{tikzpicture}[>=Stealth]
    % estilos
    \tikzset{v/.style={circle, draw, fill=blue!10, minimum size=7mm, inner sep=0pt}, a/.style={->, semithick, draw=gray!60}, c/.style={->, draw=blue!60}};
    % vértices do ciclo (dispostos aproximadamente em círculo)
    \node[v, label=below:$v_1$] (v1) at (0,0) {};
    \node[v, label=right:$v_2$] (v2) at (2.0,0.8) {};
    \node[v, label=above:$v_3$] (v3) at (1.2,2.1) {};
    \node[v, label=left:$v_4$] (v4) at (-0.8,1.6) {};
    % arcos não pertencentes ao ciclo (cinza)
    \node[v, label=right:$x$] (x) at (2.6,1.8) {};
    \draw[a] (x) -- (v2);
    \draw[a] (x) -- (v3);
    % ciclo destacado (azul)
    \draw[c] (v1) -- (v2);
    \draw[c] (v2) -- (v3);
    \draw[c] (v3) -- (v4);
    \draw[c] (v4) -- (v1);
    % legenda
    \node[align=left, anchor=west] at (3.2,1.0) {$C=(v_1,v_2,v_3,v_4,v_1)$\\ $|C|=4$};
\end{tikzpicture}
\captionof{figure}{Ciclo direcionado em dígrafo: o ciclo $C=(v_1,v_2,v_3,v_4,v_1)$ está destacado em azul. Seu comprimento é o número de arcos, $|C|=4$.}
\label{fig:ciclo-direcionado}
\end{center}

\paragraph{Arborescências}
\paragraph{}Uma \textbf{arborescência} é um dígrafo acíclico e conexo, onde há um vértice especial chamado \emph{raiz} que tem um caminho direcionado para todos os outros vértices. Formalmente, uma arborescência \(T\) é um dígrafo \(T = (V_T, A_T)\) onde \(V_T \subseteq V\) e \(A_T \subseteq A\), que satisfaz as seguintes propriedades:
\begin{itemize}
    \item \(T\) é conexo: existe um caminho direcionado da raiz para qualquer vértice em \(V_T\).
    \item \(T\) é acíclico: não contém ciclos direcionados.
\end{itemize}
Além disso, uma arborescência com \(n\) vértices sempre tem exatamente \(n-1\) arcos.
\begin{center}
\begin{tikzpicture}[>=Stealth]
    % estilos
    \tikzset{v/.style={circle, draw, fill=blue!10, minimum size=7mm, inner sep=0pt}, a/.style={->, semithick, draw=blue!70}, p/.style={->, draw=gray!60}};
    % vértices da arborescência
    \node[v, label=below:$r$] (r) at (0,0) {};
    \node[v, label=above:$a$] (a) at (-1.6,1.4) {};
    \node[v, label=above:$b$] (b) at (0.0,1.8) {};
    \node[v, label=above:$c$] (c) at (1.6,1.4) {};
    \node[v, label=above:$d$] (d) at (-2.4,2.8) {};
    \node[v, label=above:$e$] (e) at (-0.8,3.0) {};
    \node[v, label=above:$f$] (f) at (0.8,3.0) {};
    \node[v, label=above:$g$] (g) at (-4.95,3.0) {};
    \node[v, label=above:$h$] (h) at (-3.50,3.0) {};
    % arcos da arborescência (sem ciclos)                      
    \draw[a] (r) -- (a);
    \draw[a] (r) -- (b);
    \draw[a] (r) -- (c);
    \draw[a] (a) -- (d);
    \draw[a] (b) -- (e);
    \draw[a] (c) -- (f);
    % arco extra (não faz parte da arborescência)
    \draw[p] (e) -- (d);
    \draw[p] (f) -- (r);
    \draw[p] (d) -- (b);
    \draw[p] (a) -- (b);
    \draw[p] (g) -- (h);
    
    \node[align=left, anchor=west] at (3.0,1.8) {$|V_T|=7,\; |A_T|=6$\\ $|A_T|=|V_T|-1$};
\end{tikzpicture}
\captionof{figure}{Arborescência: dígrafo conexo e acíclico com raiz $r$ de onde há um caminho direcionado para todos os outros vértices em azul. No exemplo, $|V_T|=7$ e $|A_T|=6$, satisfazendo $|A_T|=|V_T|-1$. Em cinza, arcos que não fazem parte da arborescência.}
\label{fig:arborescencia}
\end{center}

\paragraph{}
Uma arborescência é considerada geradora se inclui todos os vértices do dígrafo original, ou seja, \(V_T = V\). Nesse caso, a arborescência é formada por um subconjunto dos arcos do dígrafo original.

\paragraph{Componentes fortemente conexas}
\paragraph{}Uma \textbf{componente fortemente conexa} de um dígrafo é um subdígrafo maximal onde existe um caminho direcionado entre qualquer par de vértices. Formalmente, uma componente fortemente conexa \(C\) é um subdígrafo \(C = (V_C, A_C)\) onde \(V_C \subseteq V\) e \(A_C \subseteq A\), que satisfaz a seguinte propriedade:
\begin{itemize}
    \item \(C\) é fortemente conexo: existe um caminho direcionado entre qualquer par de vértices em \(V_C\).
\end{itemize}
Além disso, \(C\) é maximal, o que significa que não é possível adicionar mais vértices ou arcos a \(C\) sem perder a propriedade de forte conexidade.
\begin{center}
\begin{tikzpicture}[>=Stealth]
    % estilos
    \tikzset{v/.style={circle, draw, fill=blue!10, minimum size=7mm, inner sep=0pt}, a/.style={->, semithick, draw=gray!70}, comp/.style={draw=blue!40, rounded corners}};
    % Componente C1 (à esquerda)
    \node[v, label=above:$a$] (a) at (0,1.2) {};
    \node[v, label=below:$b$] (b) at (-0.8,0) {};
    \node[v, label=below:$c$] (c) at (0.8,0) {};
    \draw[a] (a) -- (b);
    \draw[a] (b) -- (c);
    \draw[a] (c) -- (a);
    \node[comp, fit=(a)(b)(c), inner sep=20pt, label={[blue!60]above:$C_1$}] {};
    % Componente C2 (centro)
    \node[v, label=below:$d$] (d) at (4,0) {};
    \node[v, label=below:$e$] (e) at (5.6,0) {};
    \draw[a] (d) to[bend left=20] (e);
    \draw[a] (e) to[bend left=20] (d);
    \node[comp, fit=(d)(e), inner sep=15pt, label={[blue!60]above:$C_2$}] {};
    % Componente C3 (direita)
    \node[v, label=above:$f$] (f) at (8.6,1.2) {};
    \node[v, label=below:$g$] (g) at (7.8,0) {};
    \node[v, label=below:$h$] (h) at (9.4,0) {};
    \draw[a] (f) to[bend left=20] (g);
    \draw[a] (g) to[bend left=20] (h);
    \draw[a] (h) to[bend left=20] (f);
    \draw[a] (g) to[bend left=20] (f);
    \draw[a] (h) to[bend left=20] (g);
    \draw[a] (f) to[bend left=20] (h);
    \node[comp, fit=(f)(g)(h), inner sep=20pt, label={[blue!60]above:$C_3$}] {};
    % texto explicativo
    \node[align=center, anchor=north west] at (-0.2,-1.0) {Cada caixa destaca uma \emph{componente fortemente conexa}.\\ Não há arcos entre $C_1$, $C_2$ e $C_3$.};
\end{tikzpicture}
\captionof{figure}{Componentes fortemente conexas: o dígrafo possui três componentes $C_1$, $C_2$ e $C_3$. Cada $C_i$ é fortemente conexo e \emph{maximal}: adicionar qualquer vértice de fora quebraria a propriedade de forte conexidade interna ou exigiria arcos ausentes.}   
\end{center}

\paragraph{}
Essas estruturas são importantes porque elas possuem propriedades especiais, trilhas e circuitos em dígrafos, por exemplo, respeitam a direção dos arcos, o que pode afetar a existência e o número de tais caminhos. Árvores direcionadas (arborescências) têm uma raiz única e caminhos direcionados para todos os outros vértices, o que é útil em muitas aplicações, como hierarquias e fluxos de dados. Componentes fortemente conexas ajudam a identificar subestruturas dentro de um dígrafo onde todos os vértices são mutuamente alcançáveis, o que é crucial em análise de redes e sistemas complexos.

\subsubsection{Propriedades matemáticas em dígrafos}
\paragraph{}
Para organizar ideias, fixamos uma notação leve para cortes. Dado um dígrafo \(D=(V,A)\) e um subconjunto \(X\subseteq V\), denotamos por
\[\delta^+(X)=\{\, (u,v)\in A:\; u\in X,\ v\notin X\,\}\quad\text{e}\quad \delta^-(X)=\{\, (u,v)\in A:\; u\notin X,\ v\in X\,\}\]
o conjunto de arcos que saem de \(X\) e o conjunto de arcos que entram em \(X\), respectivamente. Se houver uma função de custos/pesos \(c:A\to\mathbb R_+\), definimos o \emph{valor} de um conjunto de arcos \(F\subseteq A\) como \(c(F)=\sum_{a\in F}c(a)\).

\paragraph{}
Um corte em um dígrafo é um conjunto de arcos cuja remoção desconecta o dígrafo, ou seja, impede que haja um caminho direcionado entre certos pares de vértices. Formalmente, dado um dígrafo \(D = (V, A)\), um corte \(C\) é um subconjunto de arcos \(C \subseteq A\) tal que a remoção dos arcos em \(C\) resulta em um dígrafo \(D' = (V, A \setminus C)\) onde não existe mais um caminho direcionado entre pelo menos um par de vértices \(u, v \in V\).

\begin{center}
\begin{tikzpicture}[>=Stealth]
    % estilos
    \tikzset{v/.style={circle, draw, fill=blue!10, minimum size=7mm, inner sep=0pt}, a/.style={->, semithick, draw=gray!70}, cut/.style={->, very thick, draw=red!70}};
    % vértices
    \node[v, label=below:$a$] (a) at (0,0) {};
    \node[v, label=below:$b$] (b) at (2,0) {};
    \node[v, label=above:$c$] (c) at (1,1.5) {};
    \node[v, label=below:$d$] (d) at (4,0) {};
    \node[v, label=above:$e$] (e) at (3,1.5) {};
    \node[v, label=below:$f$] (f) at (6,0) {};
    % arcos
    \draw[a] (a) -- (b);
    \draw[a] (b) -- (c);
    \draw[a] (c) -- (a);
    \draw[a] (b) -- (d);
    \draw[a] (d) -- (e);
    \draw[a] (e) -- (b);
    \draw[a] (d) -- (f);
    \draw[a] (f) -- (e);
    % corte destacado (vermelho)
    \draw[cut] (b) -- (d);
    % legenda
    \node[align=left, anchor=west] at (6.5,1.0) {Corte $C=\{(b,d)\}$\\ Remover $C$ desconecta $D$};
\end{tikzpicture}
\captionof{figure}{Corte em dígrafo: o corte $C=\{(b,d)\}$ está destacado em vermelho. Remover esse arco desconecta o dígrafo, impedindo caminhos direcionados entre certos pares de vértices.}
\label{fig:corte}
\end{center}

\paragraph{}
Um min-corte é um corte de tamanho mínimo, ou seja, é o corte com o menor número possível de arcos cuja remoção desconecta o dígrafo. Formalmente, dado um dígrafo \(D = (V, A)\), um min-corte \(C_{min}\) é um corte tal que para qualquer outro corte \(C\), \(|C_{min}| \leq |C|\). O tamanho do min-corte é o número de arcos em \(C_{min}\).

\begin{center}
\begin{tikzpicture}[>=Stealth]
    % estilos
    \tikzset{v/.style={circle, draw, fill=blue!10, minimum size=7mm, inner sep=0pt}, a/.style={->, semithick, draw=gray!70}, cut/.style={->, very thick, draw=red!70}};
    % vértices 
    \node[v, label=below:$a$] (a) at (0,0) {};
    \node[v, label=below:$b$] (b) at (2,0) {};
    \node[v, label=above:$c$] (c) at (1,1.5) {};
    \node[v, label=below:$d$] (d) at (4,0) {};
    \node[v, label=above:$e$] (e) at (3,1.5) {};
    \node[v, label=below:$f$] (f) at (6,0) {};
    % arcos
    \draw[a] (a) -- (b);
    \draw[a] (b) -- (c);
    \draw[a] (c) -- (a);
    \draw[a] (b) -- (d);
    \draw[a] (d) -- (e);
    \draw[a] (e) -- (b);
    \draw[a] (d) -- (f);
    \draw[a] (f) -- (e);
    % min-corte destacado (vermelho)
    \draw[cut] (b) -- (d);
    % legenda
    \node[align=left, anchor=west] at (6.5,1.0) {Min-corte $C_{min}=\{(b,d)\}$\\ $|C_{min}|=1$ é o menor corte que desconecta $D$};
\end{tikzpicture}
\captionof{figure}{Min-corte em dígrafo: o min-corte $C_{min}=\{(b,d)\}$ está destacado em vermelho. Remover esse arco desconecta o dígrafo, e é o menor corte possível, com tamanho $|C_{min}|=1$.}
\label{fig:min-corte}
\end{center}

\paragraph{}
Quando trabalhamos com limites/pesos nas arestas, é comum focar em cortes \(s\text{--}t\). Um corte \(s\text{--}t\) é determinado pela escolha de um conjunto de vértices \(X\subseteq V\) tal que \(s\in X\) e \(t\notin X\); pensa-se nele como a “divisão” do grafo em dois lados: \(X\) e \(V\setminus X\).

Para tornar a notação precisa e fácil de ler:
\begin{itemize}
    \item \textbf{\(s\text{--}t\)}: lê-se “de \(s\) para \(t\)”. Aqui, \(s\) é a fonte (onde o fluxo nasce) e \(t\) é o sumidouro (onde o fluxo chega).
    \item \textbf{\(\delta^+(X)\)} (fronteira de saída de \(X\)): conjunto de todos os arcos que \emph{saem} de \(X\) para fora, isto é, para \(V\setminus X\).
    \item \textbf{\(\delta^-(X)\)} (fronteira de entrada de \(X\)): conjunto de todos os arcos que \emph{entram} em \(X\) vindos de \(V\setminus X\). Note que \(\delta^-(X)=\delta^+(V\setminus X)\).
    \item \textbf{Valor do corte}: dado um peso (ou custo) \(c:A\to\mathbb{R}_+\) para cada arco, o valor do corte induzido por \(X\) é a soma dos pesos dos arcos que cruzam de \(X\) para fora: 
    \[c(\delta^+(X))=\sum_{a\in\delta^+(X)} c(a).\]
    No caso não ponderado, esse valor coincide com a \emph{quantidade} de arcos que saem de \(X\).
\end{itemize}

Exemplo rápido: se \(\delta^+(X)=\{(u_1,v_1),(u_2,v_2)\}\) com \(c((u_1,v_1))=2\) e \(c((u_2,v_2))=3\), então \(c(\delta^+(X))=2+3=5\).

\paragraph{}
\begin{center}

\fcolorbox{blue!15}{blue!3}{%
\parbox{\textwidth}{%
\indent\colorbox{blue!15}{\parbox{\dimexpr\linewidth-2\fboxsep\relax}{\textbf{Teorema (Fluxo–Corte Máximo = Mínimo Corte).}\\[1.0em]
{\color{black!70!black}Em dígrafos com limites/pesos não negativos nos arcos, o valor de um fluxo máximo de \(s\) para \(t\) é igual ao valor de um min-corte \(s\text{--}t\). Em símbolos: \(\max\,\text{valor}(f) = \min\, c(\delta^+(X))\), onde o mínimo é sobre \(X\subseteq V\) com \(s\in X\), \(t\notin X\), e \(c(\delta^+(X))=\sum_{a\in\delta^+(X)} c(a)\).}}}

\paragraph{}
    \textbf{Prova (esboço):}
    \paragraph{} 
    \emph{(i) Desigualdade \(\le\).} Seja \(f\) um fluxo qualquer. Para um corte \((X, V\setminus X)\) com \(s\in X\), \(t\notin X\), a conservação de fluxo implica que o fluxo líquido que sai de \(X\) é exatamente \(\text{valor}(f)\). 
    
    Logo, \[\text{valor}(f)= \sum_{a\in\delta^+(X)} f(a) - \sum_{a\in\delta^-(X)} f(a) \le \sum_{a\in\delta^+(X)} f(a) \le \sum_{a\in\delta^+(X)} c(a)=c(\delta^+(X)),\]

    pois \(f(a)\le c(a)\) para todo arco (limite). Como isso vale para todo \(X\), obtemos \(\text{valor}(f)\le \min_X c(\delta^+(X))\).

    \paragraph{}
    \emph{(ii) Desigualdade \(\ge\) e igualdade.} Considere um fluxo máximo \(f\) sem caminho aumentante no \emph{grafo residual} \(R_f\) (isto é, não há como aumentar o valor do fluxo). Defina \(X\) como o conjunto de vértices alcançáveis a partir de \(s\) em \(R_f\). Então, não existe arco residual de \(X\) para \(V\setminus X\); logo, todo arco original que sai de \(X\) está saturado (\(f(a)=c(a)\)), e todo arco que entra em \(X\) carrega fluxo zero. Assim,

    \[\text{valor}(f)=\sum_{a\in\delta^+(X)} f(a)=\sum_{a\in\delta^+(X)} c(a)=c(\delta^+(X)).\]
    Portanto, \(f\) atinge exatamente o valor de um corte \(s\text{--}t\); em particular, esse corte é mínimo e \(f\) é máximo. \hfill$\square$

\paragraph{}
\smallskip
\textbf{Comentário:} Esse resultado é um protótipo de teorema \emph{min--max}: “empacotar” muito fluxo (caminhos) equivale a “cobrir” pouco com um corte. Ver, por exemplo, \cite{schrijver2003comb}.
}}
\end{center}

\paragraph{}
Empacotamento de vértices (ou empacotamento de nós) em um dígrafo é um conjunto de vértices tal que nenhum par de vértices no conjunto está conectado por um arco. Formalmente, dado um dígrafo \(D = (V, A)\), um empacotamento de vértices \(M\) é um subconjunto de vértices \(M \subseteq V\) tal que para qualquer par distinto \(u, v \in M\), não existe um arco \((u, v) \in A\) nem \((v, u) \in A\).

\paragraph{}
Empacotamento máximo de vértices é um empacotamento que contém o maior número possível de vértices. Formalmente, um empacotamento máximo \(M_{max}\) é um empacotamento tal que para qualquer outro empacotamento \(M\), \(|M_{max}| \geq |M|\). O tamanho do empacotamento máximo é o número de vértices em \(M_{max}\).

\paragraph{Empacotamento de caminhos e separadores}
\paragraph{}
Outra relação \emph{min–max} clássica em dígrafos é dada pelo teorema de Menger: o número máximo de caminhos \(s\text{--}t\) mutuamente disjuntos (em vértices, ou em arcos) é igual ao tamanho mínimo de um conjunto de vértices (ou de arcos) cuja remoção separa \(s\) de \(t\). Assim como em fluxo–corte, ele expressa o equilíbrio entre \emph{empacotar} muitas trilhas/caminhos e \emph{cobrir} todos eles com poucas remoções.

\paragraph{Cortes dirigidos e arborescências: uma visão dual}
\paragraph{}
As ideias de cortes também aparecem no problema central desta dissertação. Para custos não negativos e um vértice-raiz \(r\), o custo mínimo de uma arborescência geradora dirigida é igual ao valor máximo de uma função \(z\) definida em subconjuntos \(X\subseteq V\setminus\{r\}\) que respeita as desigualdades
\[ c(a)\ \ge\ z(X)\quad \text{para todo arco } a\text{ que entra }X. \]
Intuitivamente, \(z(X)\) é um “preço” pago para entrar em \(X\); nenhum arco pode cobrar menos do que a soma dos preços dos conjuntos que ele entra. Esse é o teorema de Fulkerson (ver, por exemplo, \cite{frank2014, schrijver2003comb}).

\paragraph{}
Essa formulação leva a um algoritmo em duas fases (Fulkerson–Frank). Na Fase 1, ajustamos os custos de modo a tornar certos arcos de custo zero (0‑arcos), aumentando gradualmente valores \(z(X)\) para conjuntos mínimos sem 0‑arcos entrando. Na Fase 2, extraímos uma arborescência apenas com 0‑arcos, escolhendo-os de forma gulosa. Quando os dois lados “se encontram” (valor dual = valor primal), obtemos ao mesmo tempo a arborescência ótima e um certificado de otimalidade. Essa visão se conecta diretamente à normalização de custos e à contração/reexpansão de ciclos no algoritmo de Chu–Liu/Edmonds \cite{chu1965, edmonds1967optimum}.

\paragraph{}     
\section{Considerações Finais}
\printbibliography

\end{document}