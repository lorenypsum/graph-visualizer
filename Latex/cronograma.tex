\documentclass[12pt,a4paper]{article}
\usepackage[utf8]{inputenc}
\usepackage{amsmath, amssymb, amsthm}
\usepackage{geometry}
\geometry{a4paper, margin=1in}
\usepackage{graphicx}
\usepackage{hyperref}
\usepackage{pgfgantt}
\usepackage[brazil]{babel}
\usepackage[utf8]{inputenc}
\usepackage{abntex2cite}

\title{Relatório I de Projeto de Graduação: Análise e Implementação de Algoritmo de Busca de uma r-Arborescência Inversa de Custo Mínimo em Grafos Dirigidos}
\author{Orientador: Mário Leston \and Lorena Silva Sampaio - 11201812025, Samira Haddad - 11201812350}
\date{}

\begin{document}

\maketitle

\section{Introdução}
Este projeto tem como objetivo analisar o artigo "A Simple Algorithm and Min–Max Formula for the Inverse Arborescence Problem", desenvolvido por András Frank e Gergely Hajdu. O foco do estudo é na compreensão do algoritmo exposto pelos autores, o qual consiste na modificação mínima de uma função de custo em um grafo dirigido de modo que uma arborescência de entrada se torne a mais econômica. O artigo introduz uma abordagem conceitualmente mais simples e uma nova fórmula min-max para essa modificação, baseada em um teorema min-max e em um algoritmo guloso de duas fases. O projeto de graduação consiste na compreensão e implementação do problema em linguagem de programação a ser definida e validação da solução por meio de testes práticos.

\section{Objetivos}
\begin{itemize}
    \item Compreender a teoria por trás das arborescências e o problema da arborescência inversa.
    \item Implementar o algoritmo proposto no artigo em linguagem de programação a ser definida.
    \item Validar a implementação através de casos de teste com diferentes grafos.
    \item Documentar o processo e os resultados obtidos.
\end{itemize}

\section{Metodologia}

\subsection{Estudo Teórico}
\begin{itemize}
    \item Revisão do artigo e análise do trabalho de Frank e Hajdu sobre o problema da arborescência inversa.
    \item Análise dos teoremas e da nova fórmula min-max proposta, assim como a conexão com o algoritmo guloso de duas fases.
\end{itemize}

\subsection{Implementação}
\begin{itemize}
    \item Desenvolvimento do algoritmo, seguindo a abordagem proposta:
    \begin{itemize}
        \item Modificação da função de custo das arestas.
        \item Aplicação da fórmula min-max para garantir a minimização dos custos.
        \item Implementação do algoritmo guloso para encontrar a arborescência mais econômica.
    \end{itemize}
\end{itemize}

\subsection{Validação}
\begin{itemize}
    \item Criação de casos de teste com grafos de diferentes topologias e custos.
    \item Comparação dos resultados da implementação com as expectativas teóricas e exemplos do artigo.
\end{itemize}

\subsection{Documentação}
\begin{itemize}
    \item Registro do processo de desenvolvimento, incluindo decisões tomadas e resultados obtidos.
\end{itemize}

\section{Divisão de Tarefas}
O presente projeto de graduação de curso em ciência da computação será realizado de forma conjunta pelos discentes Lorena Silva Sampaio e Samira Haddad, a atribuição de atividades foi realizada conforme o descrito na tabela 1 abaixo.

\begin{center}
\begin{tabular}{|l|l|l|}
\hline
\textbf{Capítulos | Atribuições} & \textbf{Início} & \textbf{Duração} \\
\hline
\multicolumn{3}{|l|}{\textbf{1. Definições | Lorena Sampaio}} \\
\hline
Dígrafo & 2024-12-07 & 1 dia \\
Função de custo & 2024-12-08 & 1 dia \\
Arcos de custo zero & 2024-12-09 & 1 dia \\
Solução ótima & 2024-12-10 & 1 dia \\
Definição do problema & 2024-12-11 & 1 dia \\
Definição do problema R-Arborescência & 2024-12-12 & 1 dia \\
Definição do problema R-Arborescência Inversa & 2024-12-13 & 1 dia \\
Revisão das definições & 2024-12-14 & 2 dias \\
\hline
\multicolumn{3}{|l|}{\textbf{Algoritmo de Chu-Liu | Lorena Sampaio}} \\
\hline
Descrição & 2024-12-16 & 1 dia \\
Correção e complexidade do algoritmo & 2024-12-17 & 1 dia \\
Descrição da implementação & 2024-12-18 & 3 dias \\
\hline
\multicolumn{3}{|l|}{\textbf{Algoritmo de Fulkerson | Samira Haddad}} \\
\hline
Descrição & 2024-12-27 & 2 dias \\
Correção e complexidade do algoritmo & 2025-01-03 & 2 dias \\
Descrição da implementação & 2025-01-05 & 5 dias \\
\hline
\multicolumn{3}{|l|}{\textbf{Algoritmo Dual Guloso de Frank | Samira Haddad}} \\
\hline
Descrição & 2025-01-13 & 2 dias \\
Correção e complexidade do algoritmo & 2025-01-15 & 2 dias \\
Descrição da implementação & 2025-01-17 & 5 dias \\
\hline
\multicolumn{3}{|l|}{\textbf{O Problema da Arborescência Inversa | Lorena Sampaio}} \\
\hline
Descrição & 2025-01-17 & 2 dias \\
Correção e complexidade do algoritmo & 2025-01-19 & 2 dias \\
Descrição da implementação & 2025-01-21 & 2 dias \\
\hline
\multicolumn{3}{|l|}{\textbf{Implementação de código | Ambas de acordo o capítulo}} \\
\hline
Definição de estruturas de dados & 2025-01-30 & 2 semanas \\
Definição de funções auxiliares & 2025-02-13 & 2 semanas \\
Algoritmo de Chu-Liu & 2025-02-27 & 2 semanas \\
Algoritmo de Fulkerson & 2025-03-12 & 2 semanas \\
Algoritmo Dual Guloso de Frank & 2025-03-27 & 2 semanas \\
O Problema da Arborescência Inversa & 2025-04-11 & 2 semanas \\
\hline
\multicolumn{3}{|l|}{\textbf{7.Corpo do artigo | Ambas}} \\
\hline
Introdução e justificativa & 2025-05-01 & 4 horas \\
Objetivo & 2025-05-01 & 4 horas \\
Metodologia & 2025-05-02 & 4 horas \\
Conclusão & 2025-05-02 & 4 horas \\
Referências & 2025-05-03 & 1 dia \\
\hline
\multicolumn{3}{|l|}{\textbf{8.Revisão}} \\
\hline
Textual & 2025-05-10 & 1 mês \\
ABNT & 2025-06-10 & 1 dia \\
\hline
\end{tabular}
\end{center}
\begin{center}
\caption{} Tabela 1: Planejamento de Atividades e Atribuições
\end{center}

\section{Cronograma}
O planejamento deste projeto foi elaborado em etapas essenciais para sua organização e execução. Abaixo, listamos as etapas do cronograma e a sequência de tarefas necessárias:
\begin{figure}[h!]
    \centering
    \includegraphics[width=1\textwidth]{images/waterfall-planning-2.png}
    \caption{Legenda da imagem.}
    \label{fig:Cronograma}
\end{figure}

%% MERMAID: FEGAkBeyFJdnOwndZGykCNN0i7c0udFFGGjlsZNLgIU03yNPsxQIFelXsE_jFdkid6FaJtTavrIM_zvzzkxw9BqnIIJgHg8Ej40zPyeOJvoU1nMxPciFB6ZOnp8Eg4SvKtU44wZFRDRdCrqkm5A8co6urUZaVzzTTBZByLKTgYiXpmpJMkHevF

\section{Considerações Finais}
Este projeto visa a compreensão profunda do problema abordado e o estudo de suas possíveis aplicações em problemas práticos. A colaboração entre os integrantes será fundamental para o sucesso do projeto.

\begin{thebibliography}{98}

\bibitem{frank2014} FRANK, A.; HAJDU, G. A Simple Algorithm and Min--Max Formula for the Inverse Arborescence Problem. \textit{Algorithms}, v. 7, n. 4, p. 637--647, 2014. DOI: 10.3390/a7040637.

\bibitem{hu1997strongly} HU, Z.; LIU, Z. A strongly polynomial algorithm for the inverse shortest arborescence problem. \textit{Discrete Applied Mathematics}, v. 82, n. 1--3, p. 135--154, 1998.

\bibitem{cormen2009} CORMEN, T. H.; LEISERSON, C. E.; RIVEST, R. L.; STEIN, C. \textit{Introduction to Algorithms}. 3. ed. MIT Press, 2009.

\bibitem{kleinberg2006} KLEINBERG, J.; TARDOS, É. \textit{Algorithm Design}. Addison-Wesley, 2006.

\bibitem{bondy2008graph} BONDY, J. A.; MURTY, U. S. R. \textit{Graph Theory with Applications}. Springer, 2008.

\bibitem{schrijver2003comb} SCHRIJVER, A. \textit{Combinatorial Optimization: Polyhedra and Efficiency}. Springer, 2003.

\bibitem{edmonds1967optimum} EDMONDS, J. Optimum Branchings. \textit{Journal of Research of the National Bureau of Standards}, v. 71B, p. 233--240, 1967.

\bibitem{west2001introduction} WEST, D. B. \textit{Introduction to Graph Theory}. 2. ed. Prentice Hall, 2001.

\bibitem{diestel2017graph} DIESTEL, R. \textit{Graph Theory}. 5. ed. Springer, 2017.

\bibitem{nemhauser1988} WOLSEY, L. A.; NEMHAUSER, G. L. \textit{Integer and Combinatorial Optimization}. 1. ed. Wiley-Interscience, 1988.

\end{thebibliography}

\end{document}







