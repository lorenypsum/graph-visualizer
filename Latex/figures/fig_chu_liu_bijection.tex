% !TEX program = latexmk
\documentclass[tikz,border=2pt]{standalone}
\usepackage[T1]{fontenc}
\usepackage[utf8]{inputenc}
\usepackage[brazil]{babel}
\usepackage{microtype}
\usepackage{tikz}
\usetikzlibrary{positioning,arrows.meta,fit,calc}

\begin{document}
\centering
    \begin{minipage}{.47\textwidth}
        \centering
        \begin{tikzpicture}[>=Latex]
            % styles
            	\tikzset{
                vertex/.style={circle,draw,fill=white,inner sep=1pt,font=\small},
                sel/.style={-Latex,very thick,green!60!black},
                zero/.style={-Latex,very thick,blue!70},
                ext/.style={-Latex,dashed,gray},
                normal/.style={-Latex,thin}
            }
            % nodes (contracted graph)
            \node[vertex] (r) at (-0.1,1.2) {$r$};
            \node[vertex] (u) at (-0.9,0.0) {$u$};
            \node[vertex,very thick] (x) at (0.8,0.0) {$x_C$};
            \node[vertex] (v) at (2.2,0.0) {$v$};
            % arcs
            \draw[sel]   (u) to[bend left=8] (x);       % chosen arc entering x_C
            \draw[normal](x) to[bend left=8] (v);       % some outgoing from x_C
            \draw[ext]   (r) to[bend left=15] (x);      % optional from r
            \draw[ext]   (r) to[bend left=12] (u);
        \end{tikzpicture}
        \\[2pt]
        \small (a) Grafo contraído: arborescência com um único arco entrando em $x_C$.
    \end{minipage}\hfill
    \begin{minipage}{.47\textwidth}
        \centering
        \begin{tikzpicture}[>=Latex]
            % styles
            	\tikzset{
                vertex/.style={circle,draw,fill=white,inner sep=1pt,font=\small},
                sel/.style={-Latex,very thick,green!60!black},
                zero/.style={-Latex,very thick,blue!70},
                ext/.style={-Latex,dashed,gray},
                normal/.style={-Latex,thin}
            }
            % nodes (expanded original graph)
            \node[vertex] (r2) at (-0.1,1.2) {$r$};
            \node[vertex] (u2) at (-0.9,0.0) {$u$};
            \node[vertex] (a)  at (0.6,0.1) {$a$};
            \node[vertex] (b)  at (1.4,0.3) {$b$};
            \node[vertex] (c)  at (1.0,-0.5) {$c$};
            \node[vertex] (v2) at (2.4,0.0) {$v$};
            % cycle C with zero reduced cost arcs
            \draw[zero] (a) -- (b);
            \draw[zero] (b) -- (c);
            \draw[zero] (c) -- (a);
            % chosen entering arc to some w in C (e.g., u -> b)
            \draw[sel] (u2) to[bend left=8] (b);
            % an outgoing arc from C
            \draw[normal] (c) to[bend left=8] (v2);
            % mark one internal arc to be removed when expanding (break cycle)
            \path (b) -- (c) node[pos=0.5] (mid) {};
            \node[red] at (mid) {$\times$};
            % optional external arcs
            \draw[ext] (r2) to[bend left=12] (a);
            \draw[ext] (r2) to[bend right=12] (u2);
        \end{tikzpicture}
        \\[2pt]
        \small (b) Grafo original: expansão de $C$; mantém-se o arco que entra em $C$ e remove-se um arco interno para quebrar o ciclo.
    \end{minipage}
    \end{document}
