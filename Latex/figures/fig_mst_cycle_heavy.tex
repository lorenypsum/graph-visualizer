% !TEX program = latexmk
\documentclass[tikz,border=2pt]{standalone}
\usepackage[T1]{fontenc}
\usepackage[utf8]{inputenc}
\usepackage[brazil]{babel}
\usepackage{microtype}
\usepackage{tikz}
\usetikzlibrary{positioning,arrows.meta,fit,calc}

\begin{document}
\centering
    % Ilustração do princípio da aresta mais pesada do ciclo
    \begin{tikzpicture}[scale=1]
        % Estilos
        	\tikzset{
            v/.style={circle, draw, fill=blue!10, minimum size=6mm, inner sep=0pt},
            edgeG/.style={line width=0.9pt, draw=gray!65},
            heavy/.style={line width=1.6pt, draw=red!70},
            note/.style={gray!70}
        }
        % Nós do ciclo
        \node[v, label=left:$u$]  (U) at (-2, 0) {};
        \node[v, label=above:$v$] (V) at ( 0, 1.7) {};
        \node[v, label=right:$x$] (X) at ( 2, 0) {};
        \node[v, label=below:$y$] (Y) at ( 0,-1.7) {};

        % Arestas do ciclo com pesos
        \draw[edgeG] (U) -- node[above left,  fill=white, inner sep=1pt] {1} (V);
        \draw[heavy] (V) -- node[above right, fill=white, inner sep=1pt] {\textbf{5}} (X);
        \draw[edgeG] (X) -- node[below right, fill=white, inner sep=1pt] {2} (Y);
        \draw[edgeG] (Y) -- node[below left,  fill=white, inner sep=1pt] {3} (U);

        % Observação
        \node[red!90!] at (0, -3.0) {A aresta mais pesada do ciclo (vermelha) não precisa aparecer em nenhuma MST};
    \end{tikzpicture}
    \end{document}
