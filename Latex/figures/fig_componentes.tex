% !TEX program = latexmk
\documentclass[tikz,border=2pt]{standalone}
\usepackage[T1]{fontenc}
\usepackage[utf8]{inputenc}
\usepackage[brazil]{babel}
\usepackage{microtype}
\usepackage{tikz}
\usetikzlibrary{positioning,arrows.meta,fit,calc}

\begin{document}
\centering
\begin{tikzpicture}[>=Stealth]
    % estilos
    \tikzset{v/.style={circle, draw, fill=blue!10, minimum size=7mm, inner sep=0pt}, e/.style={semithick, draw=gray!70}, comp/.style={draw=blue!40, rounded corners}}

    % Componente C1 (à esquerda)
    \node[v, label=above:$a$] (a) at (0,1.2) {};
    \node[v, label=below:$b$] (b) at (-0.8,0) {};
    \node[v, label=below:$c$] (c) at (0.8,0) {};
    \draw[e] (a) -- (b);
    \draw[e] (a) -- (c);
    \node[comp, fit=(a)(b)(c), inner sep=20pt, label={[blue!60]above:$C_1$}] {};

    % Componente C2 (centro)
    \node[v, label=below:$d$] (d) at (4,0) {};
    \node[v, label=below:$e$] (e) at (5.6,0) {};
    \draw[e] (d) -- (e);
    \node[comp, fit=(d)(e), inner sep=15pt, label={[blue!60]above:$C_2$}] {};

    % Componente C3 (direita)
    \node[v, label=above:$f$] (f) at (8.6,1.2) {};
    \node[v, label=below:$g$] (g) at (7.8,0) {};
    \node[v, label=below:$h$] (h) at (9.4,0) {};
    \draw[e] (f) -- (g);
    \draw[e] (f) -- (h);
    \draw[e] (g) -- (h);
    \node[comp, fit=(f)(g)(h), inner sep=20pt, label={[blue!60]above:$C_3$}] {};

    % texto explicativo
    \node[align=center, anchor=north west] at (-0.2,-1.0) {Cada caixa destaca uma \emph{componente conexa}.\\ Não há arestas entre $C_1$, $C_2$ e $C_3$.};
\end{tikzpicture}
% Texto explicativo removido para manter PDF de 1 página.
\end{document}
