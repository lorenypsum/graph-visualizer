% !TEX program = latexmk
\documentclass[tikz,border=2pt]{standalone}
\usepackage[T1]{fontenc}
\usepackage[utf8]{inputenc}
\usepackage[brazil]{babel}
\usepackage{microtype}
\usepackage{tikz}
\usetikzlibrary{positioning,arrows.meta,fit,calc}

\begin{document}
\centering
    \begin{minipage}[t]{0.52\textwidth}
        \centering
        \begin{tikzpicture}[>=Stealth, node distance=1.4cm]
            	\tikzset{v/.style={circle, draw, fill=blue!8, minimum size=7mm, inner sep=0pt},
                     cheap/.style={->, very thick, draw=blue!70},
                     e/.style={->, semithick, draw=gray!65}}
            % ciclo C e arcos internos (a_w em negrito)
            \node[v] (a) {\small $a$};
            \node[v] (b) [above right=0.9cm and 1.4cm of a] {\small $b$};
            \node[v] (c) [below right=0.9cm and 1.4cm of a] {\small $c$};
            % interno: a->b é a_b (mais barato em b), custo 5
            \draw[cheap] (a) to[bend left=12] node[midway, above, sloped] {\scriptsize $5$} (b);
            % outros do ciclo (custos quaisquer)
            \draw[e] (b) to[bend left=12] node[midway, above, sloped] {\scriptsize $4$} (c);
            \draw[e] (c) to[bend left=14] node[midway, below, sloped] {\scriptsize $6$} (a);
            % arco externo u->b com custo 7
            \node[v] (u) [left=1.9cm of b] {\small $u$};
            \draw[e] (u) -- node[midway, above, sloped] {\scriptsize $7$} (b);
            % contorno do ciclo C
            \node[draw=blue!40, rounded corners, fit=(a)(b)(c), inner sep=6pt, label={[blue!60]below:{\small $C$}}] {};
        \end{tikzpicture}
        \vspace{2pt}
        \small (a) Antes: $a_b=(a\!\to\!b)$ tem custo $5$; o arco externo $(u\!\to\!b)$ tem custo $7$.
    \end{minipage}\hfill
    \begin{minipage}[t]{0.44\textwidth}
        \centering
        \begin{tikzpicture}[>=Stealth, node distance=2.0cm]
            	\tikzset{v/.style={circle, draw, fill=blue!8, minimum size=7mm, inner sep=0pt},
                     e/.style={->, very thick, draw=blue!70}}
            \node[v] (x) {\small $x_C$};
            \node[v] (u) [left=2.1cm of x] {\small $u$};
            \draw[e] (u) -- node[midway, above, sloped] {\scriptsize $7-5=2$} (x);
        \end{tikzpicture}
        \vspace{2pt}
        \small (b) Depois: $(u\!\to\!x_C)$ recebe custo reduzido $c'(u,x_C)=7-5=2$.
    \end{minipage}
    em $C$. Para $(u,w)$ com $w\in C$, define-se $c'(u,x_C)=c(u,w)-c(a_w)$, onde $a_w$ é o arco mais barato que entra em $w$. No exemplo, $c'(u,x_C)=7-5=2$.}
    \end{document}
