% !TEX program = latexmk
\documentclass[tikz,border=2pt]{standalone}
\usepackage[T1]{fontenc}
\usepackage[utf8]{inputenc}
\usepackage[brazil]{babel}
\usepackage{microtype}
\usepackage{tikz}
\usetikzlibrary{positioning,arrows.meta,fit,calc}

\begin{document}
\centering
\begin{tikzpicture}[>=Stealth]
    % estilos
    \tikzset{v/.style={circle, draw, fill=blue!10, minimum size=7mm, inner sep=0pt}, e/.style={semithick, draw=gray!70}}

    % vértices
    \node[v, label=below:$v_1$] (v1) at (0,0) {};
    \node[v, label=below:$v_2$] (v2) at (2.2,0) {};
    \node[v, label=above:$v_3$] (v3) at (3.0,1.8) {};
    \node[v, label=above:$v_4$] (v4) at (-0.4,1.8) {};

    % arestas com custos (rótulos c(e))
    \draw[e] (v1) -- node[midway, fill=white, inner sep=2pt, font=\scriptsize] {3} (v2);
    \draw[e] (v2) -- node[midway, fill=white, inner sep=2pt, font=\scriptsize] {5} (v3);
    \draw[e] (v3) -- node[midway, fill=white, inner sep=2pt, font=\scriptsize] {2} (v4);
    \draw[e] (v4) -- node[midway, fill=white, inner sep=2pt, font=\scriptsize] {4} (v1);
    \draw[e] (v1) -- node[midway, fill=white, inner sep=2pt, font=\scriptsize] {6} (v3);

    % legenda breve
    \node[align=left, anchor=west] at (4.0,1.0) {$c:E\to\mathbb{R}^+$\\ $c(\{v_1,v_2\})=3,\; c(\{v_1,v_3\})=6,\dots$};
\end{tikzpicture}
% Texto explicativo removido para manter o PDF em uma página.
\end{document}
