% !TEX program = latexmk
\documentclass[tikz,border=2pt]{standalone}
\usepackage[T1]{fontenc}
\usepackage[utf8]{inputenc}
\usepackage[brazil]{babel}
\usepackage{microtype}
\usepackage{tikz}
\usetikzlibrary{positioning,arrows.meta,fit,calc}

\begin{document}
\centering
    % Panel (a): contracted graph just before expansion
    \begin{minipage}{.31\textwidth}
        \centering
        \begin{tikzpicture}[>=Latex]
            	\tikzset{
                vertex/.style={circle,draw,fill=white,inner sep=1pt,font=\small},
                sel/.style={-Latex,very thick,green!60!black},
                zero/.style={-Latex,very thick,blue!70},
                ext/.style={-Latex,dashed,gray},
                normal/.style={-Latex,thin}
            }
            \node[vertex] (r) at (-0.2,1.0) {$r$};
            \node[vertex] (u) at (-0.9,0.0) {$u$};
            \node[vertex,very thick] (x) at (0.6,0.0) {$x_C$};
            \node[vertex] (v) at (1.9,0.0) {$v$};
            \draw[sel]   (u) to[bend left=8] (x);   % chosen entering arc to x_C
            \draw[normal](x) to[bend left=8] (v);   % some outgoing arc
            \draw[ext]   (r) to[bend left=15] (x);
        \end{tikzpicture}
        \\[2pt]
        \small (a) Antes: grafo contraído com o arco que entra em $x_C$ selecionado.
    \end{minipage}\hfill
    % Panel (b): expand C, before removing an internal arc
    \begin{minipage}{.31\textwidth}
        \centering
        \begin{tikzpicture}[>=Latex]
            	\tikzset{
                vertex/.style={circle,draw,fill=white,inner sep=1pt,font=\small},
                sel/.style={-Latex,very thick,green!60!black},
                zero/.style={-Latex,very thick,blue!70},
                ext/.style={-Latex,dashed,gray},
                normal/.style={-Latex,thin}
            }
            \node[vertex] (r2) at (-0.2,1.0) {$r$};
            \node[vertex] (u2) at (-0.9,0.0) {$u$};
            \node[vertex] (a)  at (0.4,0.35) {$a$};
            \node[vertex] (b)  at (1.2,0.35) {$b$};
            \node[vertex] (c)  at (0.8,-0.4) {$c$};
            \node[vertex] (v2) at (2.2,0.0) {$v$};
            % internal cycle C with zero-reduced-cost arcs
            \draw[zero] (a) -- (b);
            \draw[zero] (b) -- (c);
            \draw[zero] (c) -- (a);
            % mapped entering arc to some w in C (e.g., u -> b)
            \draw[sel] (u2) to[bend left=8] (b);
            % an outgoing arc from C
            \draw[normal] (c) to[bend left=8] (v2);
            % optional external arc from r
            \draw[ext] (r2) to[bend left=12] (a);
        \end{tikzpicture}
        \\[2pt]
        \small (b) Expansão: $x_C$ é substituído por $C$, mantendo o arco que entra em $C$.
    \end{minipage}\hfill
    % Panel (c): remove one internal arc to break the cycle
    \begin{minipage}{.31\textwidth}
        \centering
        \begin{tikzpicture}[>=Latex]
            	\tikzset{
                vertex/.style={circle,draw,fill=white,inner sep=1pt,font=\small},
                sel/.style={-Latex,very thick,green!60!black},
                zero/.style={-Latex,very thick,blue!70},
                ext/.style={-Latex,dashed,gray},
                normal/.style={-Latex,thin}
            }
            \node[vertex] (u3) at (-0.9,0.0) {$u$};
            \node[vertex] (a3)  at (0.4,0.35) {$a$};
            \node[vertex] (b3)  at (1.2,0.35) {$b$};
            \node[vertex] (c3)  at (0.8,-0.4) {$c$};
            \node[vertex] (v3) at (2.2,0.0) {$v$};
            % keep entering arc and one outgoing
            \draw[sel]   (u3) to[bend left=8] (b3);
            \draw[normal](c3) to[bend left=8] (v3);
            % keep two internal zero-cost arcs and remove one (b->c removed)
            \draw[zero] (a3) -- (b3);
            % removed arc visual: faded/dashed with cross
            \draw[ext] (b3) -- (c3);
            \path (b3) -- (c3) node[pos=0.5] (mid) {};
            \node[red] at (mid) {$\times$};
            \draw[zero] (c3) -- (a3);
        \end{tikzpicture}
        \\[2pt]
        \small (c) Final: remove-se um arco interno de $C$ para quebrar o ciclo (mantendo grau de entrada 1).
    \end{minipage}
    % Texto explicativo removido para manter saída em uma única página.
    \end{document}
