% !TEX program = latexmk
\documentclass[tikz,border=2pt]{standalone}
\usepackage[T1]{fontenc}
\usepackage[utf8]{inputenc}
\usepackage[brazil]{babel}
\usepackage{microtype}
\usepackage{tikz}
\usetikzlibrary{positioning,arrows.meta,fit,calc}

\begin{document}
\centering
\begin{tikzpicture}[>=Stealth]
    % estilos
    \tikzset{v/.style={circle, draw, fill=blue!10, minimum size=7mm, inner sep=0pt}, a/.style={->, semithick, draw=gray!70}}
    % vértices
    \node[v, label=below:$a$, fill=green!18] (a) at (0,0) {};
    \node[v, label=below:$b$] (b) at (2,0) {};
    \node[v, label=above:$c$] (c) at (1,1.6) {};
    \node[v, label=above:$d$, fill=green!18] (d) at (3,1.6) {};
    \node[v, label=below:$e$] (e) at (4,0) {};
    \node[v, label=above:$f$, fill=green!18] (f) at (5.2,1.1) {};
    % arcos do dígrafo
    \draw[a] (a) -- (b);
    \draw[a] (c) -- (a);
    \draw[a] (b) -- (c);
    \draw[a] (d) -- (b);
    \draw[a] (e) -- (d);
    \draw[a] (f) -- (e);
    % anotação
    \node[align=left, anchor=west] at (6.2,1.0) {$S^{*}=\{a,d,f\}$ é máximo\\ (não há arco com extremidades em $S^{*}$)};
\end{tikzpicture}
% Texto explicativo removido para evitar geração de segunda página.
\end{document}
