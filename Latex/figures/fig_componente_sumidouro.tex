% !TEX program = latexmk
\documentclass[tikz,border=2pt]{standalone}
\usepackage[T1]{fontenc}
\usepackage[utf8]{inputenc}
\usepackage[brazil]{babel}
\usepackage{microtype}
\usepackage{tikz}
\usetikzlibrary{positioning,arrows.meta,fit,calc}

\begin{document}
\centering
\begin{tikzpicture}[>=Stealth]
    % estilos
    \tikzset{v/.style={circle, draw, fill=blue!10, minimum size=7mm, inner sep=0pt}, a/.style={->, semithick, draw=gray!70}, comp/.style={draw=blue!40, rounded corners}};
    % Componente-fonte C1 (à esquerda)
    \node[v, label=above:$a$] (a) at (0,1.2) {};
    \node[v, label=below:$b$] (b) at (-0.8,0) {};
    \node[v, label=below:$c$] (c) at (0.8,0) {};
    \draw[a] (a) -- (b);
    \draw[a] (b) -- (c);
    \draw[a] (c) -- (a);
    \node[comp, fit=(a)(b)(c), inner sep=20pt, label={[blue!60]above:$C_1$}] {};
    % Componente C2 (centro)
    \node[v, label=below:$d$] (d) at (4,0) {};
    \node[v, label=below:$e$] (e) at (5.6,0) {};
    \draw[a] (d) to[bend left=20] (e);
    \draw[a] (e) to[bend left=20] (d);
    \node[comp, fit=(d)(e), inner sep=15pt, label={[blue!60]above:$C_2$}] {};
    % Componente C3 (direita)
    \node[v, label=above:$f$] (f) at (8.6,1.2) {};
    \node[v, label=below:$g$] (g) at (7.8,0) {};
    \node[v, label=below:$h$] (h) at (9.4,0) {};
    \draw[a] (f) to[bend left=20] (g);
    \draw[a] (g) to[bend left=20] (h);
    \draw[a] (h) to[bend left=20] (f);
    \draw[a] (g) to[bend left=20] (f);
    \draw[a] (h) to[bend left=20] (g);
    \draw[a] (f) to[bend left=20] (h);
    \node[comp, fit=(f)(g)(h), inner sep=15pt, label={[blue!60]above:$C_3$}] {};
    % arcos entre componentes
    \draw[a] (c) to[bend left=15] (d);
    \draw[a] (e) to[bend left=15] (f);
    % texto explicativo
    \node[align=center, anchor=north west, text width=15.0cm] at (-1.5,-1.0) {A componente $C_3$ é uma \emph{componente-sumidouro} porque não há arcos entrando nela vindos de outras componentes. Já $C_1$ e $C_2$ não são componentes-sumidouro, pois há arcos saindo delas.};
\end{tikzpicture}
\end{document}
