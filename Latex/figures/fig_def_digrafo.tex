% !TEX program = latexmk
\documentclass[tikz,border=2pt]{standalone}
\usepackage[T1]{fontenc}
\usepackage[utf8]{inputenc}
\usepackage[brazil]{babel}
\usepackage{microtype}
\usepackage{tikz}
\usetikzlibrary{positioning,arrows.meta,fit,calc}

\begin{document}
\centering
\begin{tikzpicture}[>=Stealth]
    % estilos
    \tikzset{v/.style={circle, draw, fill=blue!10, minimum size=7mm, inner sep=0pt}, a/.style={->, semithick, draw=gray!70}}

    % vértices
    \node[v, label=below:$u$] (u) at (0,0) {};
    \node[v, label=below:$v$] (v) at (2.6,0) {};
    \node[v, label=above:$x$] (x) at (1.3,1.5) {};

    % arcos
    \draw[a] (u) -- node[above, sloped, font=\scriptsize] {$a$} (v);
    \draw[a] (x) -- (u);
    \draw[a] (v) -- (x);

    % anotações de cauda/cabeça
    \node[font=\scriptsize, anchor=north] at ($(u)!0.5!(v)$) {$a=(u,v)$};
    \path (u) ++(0,-0.9) node[font=\scriptsize] {cauda $s(a)=u$};
    \path (v) ++(0,-0.9) node[font=\scriptsize] {cabeça $t(a)=v$};
\end{tikzpicture}
% Texto explicativo removido para evitar segunda página no PDF.
\end{document}
