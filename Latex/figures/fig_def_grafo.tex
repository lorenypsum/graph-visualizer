% !TEX program = latexmk
\documentclass[tikz,border=2pt]{standalone}
\usepackage[T1]{fontenc}
\usepackage[utf8]{inputenc}
\usepackage[brazil]{babel}
\usepackage{microtype}
\usepackage{tikz}
\usetikzlibrary{positioning,arrows.meta,fit,calc}

\begin{document}
\centering
\begin{tikzpicture}[>=Stealth]
    % estilos
    \tikzset{v/.style={circle, draw, fill=blue!10, minimum size=7mm, inner sep=0pt}, e/.style={semithick, draw=gray!70}}

    % vértices (V)
    \node[v, label=below:$v_1$] (v1) at (0,0) {};
    \node[v, label=below:$v_2$] (v2) at (2.0,0) {};
    \node[v, label=above:$v_3$] (v3) at (2.0,1.6) {};
    \node[v, label=above:$v_4$] (v4) at (0,1.6) {};

    % arestas (E: pares não ordenados)
    \draw[e] (v1) -- (v2);
    \draw[e] (v2) -- (v3);
    \draw[e] (v3) -- (v4);
    \draw[e] (v4) -- (v1);

    % legenda com os conjuntos V e E
    \node[align=left, anchor=west] at (3.0,0.9) {$V=\{v_1,v_2,v_3,v_4\}$\\ $E=\{\{v_1,v_2\},\{v_2,v_3\},\{v_3,v_4\},\{v_4,v_1\}\}$};
\end{tikzpicture}
\end{document}
