% !TEX program = latexmk
\documentclass[tikz,border=2pt]{standalone}
\usepackage[T1]{fontenc}
\usepackage[utf8]{inputenc}
\usepackage[brazil]{babel}
\usepackage{microtype}
\usepackage{tikz}
\usetikzlibrary{positioning,arrows.meta,fit,calc}

\begin{document}
\centering
\begin{tikzpicture}[>=Stealth]
    % estilos locais para o condensado
    	\tikzset{
        comp/.style={circle, draw, fill=blue!8, minimum size=7mm, inner sep=0pt},
        src/.style={circle, draw=green!60!black, fill=green!15, minimum size=7mm, inner sep=0pt, very thick},
        sink/.style={circle, draw=red!60!black, fill=red!15, minimum size=7mm, inner sep=0pt, very thick},
        arc/.style={->, semithick, draw=gray!70}
    }
    % nós (CFCs do condensado)
    \node[src, label=below:$C_1$] (c1) at (0,0) {};
    \node[comp, label=below:$C_2$] (c2) at (2.2,0.9) {};
    \node[comp, label=below:$C_3$] (c3) at (2.2,-0.9) {};
    \node[sink, label=below:$C_4$] (c4) at (4.4,0) {};
    % arcos do condensado (DAG)
    \draw[arc] (c1) -- (c2);
    \draw[arc] (c1) -- (c3);
    \draw[arc] (c2) -- (c4);
    \draw[arc] (c3) -- (c4);
    % legendas curtas
    \node[align=left, anchor=west] at (5.1,0.6) {CFC-\textit{fonte}: sem entrada ($C_1$)};
    \node[align=left, anchor=west] at (5.1,-0.1) {CFC-\textit{sumidouro}: sem saída ($C_4$)};
    \node[align=left, anchor=west] at (5.1,-0.8) {$s=1,\; t=1$ neste exemplo};
\end{tikzpicture}
% Texto explicativo removido para evitar segunda página.
\end{document}
