% !TEX program = latexmk
\documentclass[tikz,border=2pt]{standalone}
\usepackage[T1]{fontenc}
\usepackage[utf8]{inputenc}
\usepackage[brazil]{babel}
\usepackage{microtype}
\usepackage{tikz}
\usetikzlibrary{positioning,arrows.meta,fit,calc}

\begin{document}
\centering
    \begin{tikzpicture}[scale=1]
        % P: conjunto maior
        \draw[fill=blue!8, draw=blue!60] (0,0) circle (2);
        \node[blue!60] at (0,2.25) {$P$};
        % T: subconjunto de P
        \draw[fill=blue!25, draw=blue!60] (-0.6,0) circle (0.9);
        \node[blue!60] at (-0.6,0) {$T$};
        % A: subconjunto de T
        \draw[fill=blue!45, draw=blue!70] (-0.9,0.1) circle (0.45);
        \node[blue!70] at (-0.9,0.1) {$A$};
        % F: disjunto de P (e portanto de T e A)
        \draw[fill=green!20, draw=green!50!black] (3.8,0) circle (1.2);
        \node[green!50!black] at (3.8,0) {$F$};
    \end{tikzpicture}
    \end{document}
