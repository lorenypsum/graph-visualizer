% !TEX program = latexmk
\documentclass[tikz,border=2pt]{standalone}
\usepackage[T1]{fontenc}
\usepackage[utf8]{inputenc}
\usepackage[brazil]{babel}
\usepackage{microtype}
\usepackage{tikz}
\usetikzlibrary{positioning,arrows.meta,fit,calc}

\begin{document}
\centering
\begin{tikzpicture}[>=Stealth, node distance=1.0cm]
    % Domínio A (retângulos azuis)
    \node[draw, rounded corners, fill=blue!10, minimum width=9mm, minimum height=6mm] (a1cd) {$a_1$};
    \node[draw, rounded corners, fill=blue!10, below=of a1cd, minimum width=9mm, minimum height=6mm] (a2cd) {$a_2$};
    \node[draw, rounded corners, fill=blue!10, below=of a2cd, minimum width=9mm, minimum height=6mm] (a3cd) {$a_3$};
    \node[draw=blue!50, rounded corners, fit=(a1cd)(a2cd)(a3cd), inner sep=4pt, label={[blue!60]above:$A$}] {};

    % Universo U (círculos verdes como átomos) à direita
    \node[circle, draw, fill=green!10, right=3.2cm of a1cd, minimum size=6mm] (ua) {$a$};
    \node[circle, draw, fill=green!10, right=8mm of ua, minimum size=6mm] (ub) {$b$};
    \node[circle, draw, fill=green!10, below=8mm of ua, minimum size=6mm] (uc) {$c$};
    \node[circle, draw, fill=green!10, right=8mm of uc, minimum size=6mm] (ud) {$d$};
    % Caixa do universo U
    \node[draw=gray!60, rounded corners, fit=(ua)(ub)(uc)(ud), inner sep=6pt, label={[gray!70]above:$U$}] (Ubox) {};

    % Destaque das imagens (subconjuntos disjuntos de U)
    \node[draw=purple!70!black, rounded corners, fit=(ua)(ub), inner sep=2pt, label={[purple!70!black]above:$f(a_1)$}] (S1) {};
    \node[draw=orange!80!black, rounded corners, fit=(uc), inner sep=3pt, label={[orange!80!black]below:$f(a_2)$}] (S2) {};
    \node[draw=blue!60!black, rounded corners, fit=(ud), inner sep=3pt, label={[blue!60!black]below:$f(a_3)$}] (S3) {};

    % Setas de mapeamento para os subconjuntos (apontando para as caixas de destaque)
    \draw[->, thick, draw=gray!70] (a1cd) -- (S1.west);
    \draw[->, thick, draw=gray!70] (a2cd) -- (S2.west);
    \draw[->, thick, draw=gray!70] (a3cd) -- (S3.west);
\end{tikzpicture}
de \(U\), e as imagens são dois a dois disjuntas (sem sobreposição). No exemplo, $f(a_1)=\{a,b\}$, $f(a_2)=\{c\}$ e $f(a_3)=\{d\}$.}
\end{document}
