% !TEX program = latexmk
\documentclass[tikz,border=2pt]{standalone}
\usepackage[T1]{fontenc}
\usepackage[utf8]{inputenc}
\usepackage[brazil]{babel}
\usepackage{microtype}
\usepackage{tikz}
\usetikzlibrary{positioning,arrows.meta,fit,calc}

\begin{document}
\centering
\begin{tikzpicture}[>=Stealth, node distance=1.4cm]
    % estilos
    \tikzset{
        box/.style={draw, rounded corners, fill=blue!5, minimum width=37mm, minimum height=10mm, align=center},
        arrow/.style={->, semithick, draw=gray!70},
        badge/.style={draw=green!50!black, rounded corners, fill=green!10, inner sep=4pt, font=\scriptsize}
    }

    % caixas primal (min) e dual (max)
    \node[box] (min) {Minimização\\(cobrir)};
    \node[box, right=4.2cm of min] (max) {Maximização\\(empacotar)};

    % relação de valores dual <= primal
    \draw[arrow] (max.west) -- node[above, font=\scriptsize, black!70]{valor dual $\;\le\;$ valor primal} (min.east);

    % destaque do "min--max" (igualdade no ótimo)
    \path (min) -- (max) node[pos=0.5, yshift=12mm, badge] (mm) {teorema min--max: igualdade no ótimo};

    % certificado de otimalidade
    \path (min) -- (max) node[pos=0.5, yshift=-12mm, badge] (cert) {certificado: verificação simples do ótimo};
\end{tikzpicture}
\end{document}
