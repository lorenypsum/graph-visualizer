% !TEX program = latexmk
\documentclass[tikz,border=2pt]{standalone}
\usepackage[T1]{fontenc}
\usepackage[utf8]{inputenc}
\usepackage[brazil]{babel}
\usepackage{microtype}
\usepackage{tikz}
\usetikzlibrary{positioning,arrows.meta,fit,calc}

\begin{document}
\centering
\begin{tikzpicture}[>=Stealth]
    % estilos
    \tikzset{v/.style={circle, draw, fill=blue!10, minimum size=7mm, inner sep=0pt}, a/.style={->, semithick, draw=gray!60}, keep/.style={->, very thick, draw=blue!60}}
    % vértices (triângulo)
    \node[v, label=below:$b$] (b) at (0,0) {};
    \node[v, label=below:$c$] (c) at (2.0,0) {};
    \node[v, label=above:$d$] (d) at (1.0,1.6) {};
    % arcos do dígrafo original (cinza)
    \draw[a] (b) -- (c);
    \draw[a] (c) -- (d);
    \draw[a] (d) -- (b);
    % subdígrafo maximal acíclico (azul): mantém dois arcos, omite o terceiro que fecharia o circuito
    \draw[keep] (b) -- (c);
    \draw[keep] (c) -- (d);
    % anotação
    \node[align=left, anchor=west] at (3.0,0.9) {$D'=\{b,c,d\},\ {(b,c),(c,d)}$ é acíclico.\\ Adicionar $(d,b)$ cria o circuito $b\to c\to d\to b$.};
\end{tikzpicture}
\end{document}
