\documentclass[aspectratio=169]{beamer}

% Tema e cores baseados no design do site ArboGraph
\usetheme{CambridgeUS}
\usecolortheme{default}

% Paleta de cores do logo e site
\definecolor{primarypurple}{RGB}{53, 43, 103}    % #352B67 - roxo escuro principal
\definecolor{secondarypurple}{RGB}{75, 66, 119}  % #4B4277 - roxo médio
\definecolor{accentpurple}{RGB}{141, 121, 229}   % #8d79e5 - roxo vibrante
\definecolor{lightpurple}{RGB}{164, 156, 255}    % #a49cc5ff - roxo claro (borda)
\definecolor{textgray}{RGB}{120, 116, 134}       % #787486 - cinza texto
\definecolor{lightgray}{RGB}{229, 229, 229}      % #E5E5E5 - fundo claro
\definecolor{divider}{RGB}{219, 219, 219}        % #DBDBDB - linhas divisórias

% Estrutura geral
\setbeamercolor{structure}{fg=primarypurple}
\setbeamercolor{palette primary}{bg=primarypurple,fg=white}
\setbeamercolor{palette secondary}{bg=secondarypurple,fg=white}
\setbeamercolor{palette tertiary}{bg=accentpurple,fg=white}
\setbeamercolor{frametitle}{bg=primarypurple,fg=white}

% Cores da capa
\setbeamercolor{title}{fg=lightgray}
\setbeamercolor{titlelike}{bg=primarypurple,fg=white}
\setbeamercolor{author}{fg=textgray}
\setbeamercolor{institute}{fg=textgray}
\setbeamercolor{date}{fg=textgray}

% Cores dos blocos (caixas de texto)
\setbeamercolor{block title}{bg=secondarypurple,fg=white}
\setbeamercolor{block body}{bg=lightgray,fg=black}
\setbeamercolor{block title alerted}{bg=secondarypurple,fg=white}
\setbeamercolor{block body alerted}{bg=lightgray,fg=black}

% Itens e enumerações
\setbeamercolor{item}{fg=accentpurple}
\setbeamercolor{enumerate item}{fg=accentpurple}
\setbeamercolor{enumerate subitem}{fg=lightpurple}

% Remover sombra dos blocos
\setbeamertemplate{blocks}[rounded][shadow=false]

% Remover "Figura" dos captions
\setbeamertemplate{caption}{\raggedright\insertcaption\par}

\usepackage[utf8]{inputenc}
\usepackage[T1]{fontenc}
\usepackage[brazil]{babel}
\usepackage{amsmath,amssymb}
\usepackage{graphicx}
\usepackage{tikz}
\usetikzlibrary{graphs,arrows.meta,positioning,fit,calc}
\usepackage{algorithm}
\usepackage{algpseudocode}
\usepackage{xcolor}
\usepackage{booktabs}
\usepackage{listings}

% Configuração do listings para código Python
\lstset{
	language=Python,
	basicstyle=\ttfamily\small,
	backgroundcolor=\color{lightgray},
	frame=single,
	rulecolor=\color{primarypurple},
	framesep=5pt,
	framexleftmargin=5pt,
	xleftmargin=10pt,
	xrightmargin=10pt,
	aboveskip=10pt,
	belowskip=10pt,
	keywordstyle=\color{primarypurple}\bfseries,
	commentstyle=\color{textgray}\itshape,
	stringstyle=\color{accentpurple},
	showstringspaces=false,
	breaklines=true,
	tabsize=4
}

% Logo da UFABC
\logo{\includegraphics[height=0.9cm]{logo-ufabc.png}}

% Configurações do título
\title[ArboGraph]{Algoritmos para r-Arborescências Geradoras Mínimas em Digrafos:}
\subtitle{Uma Aplicação Web Interativa}
\author[L. Sampaio, S. Haddad]{Lorena Sampaio, Samira Haddad \\ Orientador: Prof. Dr. Mário Leston Rey}
\institute[UFABC]{Universidade Federal do ABC \\ Centro de Matemática, Computação e Cognição}
\date{\today}

% Comandos personalizados
\newcommand{\G}{\mathcal{G}}
\newcommand{\D}{\mathcal{D}}

\begin{document}

% ==================== SLIDE 1: TÍTULO ====================
\begin{frame}
	\titlepage
\end{frame}

% ==================== SLIDE 2: SUMÁRIO ====================
\begin{frame}{Sumário}
	\tableofcontents
\end{frame}

% ==================== SEÇÃO 1: INTRODUÇÃO ====================
\section{Introdução}

\begin{frame}{O Problema}
	\begin{block}{Encontrar uma $r$-Arborescência Geradora de Custo Mínimo}
		Dado um $r$-digrafo ponderado $(D,w,r)$:
		\begin{itemize}
			\item Encontrar uma $r$-arborescência geradora de custo mínimo de $D$
		\end{itemize}
	\end{block}

	\vspace{0.5cm}

	\textbf{Algoritmos estudados:}
	\begin{enumerate}
		\item Chu-Liu-Edmonds (1965-67)
		\item András Frank (1981-2014)
	\end{enumerate}
\end{frame}

% ==================== SLIDE 3: EXEMPLO ====================
\begin{frame}{Exemplo: $r$-Arborescência Geradora Mínima}
	\begin{columns}[c]
		\column{0.3\textwidth}
		\begin{figure}[H]
			\begin{tikzpicture}[scale=0.9,
					vertex/.style={circle,draw,fill=blue!10,minimum size=0.6cm,font=\small},
					edge/.style={->,>=stealth,thick,gray!90}]

				\node[vertex,fill=red!20] (r) at (2,2.5) {$r$};
				\node[vertex] (a) at (0,1) {$a$};
				\node[vertex] (b) at (2,1) {$b$};
				\node[vertex] (c) at (4,1) {$c$};
				\node[vertex] (d) at (1,0) {$d$};
				\node[vertex] (e) at (3,0) {$e$};

				\draw[edge] (r) -- node[above left,font=\tiny] {3} (a);
				\draw[edge] (r) -- node[above left,font=\tiny] {5} (b);
				\draw[edge] (r) -- node[above right,font=\tiny] {4} (c);
				\draw[edge] (a) -- node[left,font=\tiny] {2} (d);
				\draw[edge] (b) -- node[above,font=\tiny] {1} (a);
				\draw[edge] (b) -- node[above,font=\tiny] {3} (d);
				\draw[edge] (c) -- node[right,font=\tiny] {2} (e);
				\draw[edge] (c) -- node[above,font=\tiny] {4} (b);
				\draw[edge] (d) to[bend left] node[below,font=\tiny] {1} (e);
			\end{tikzpicture}
			\caption{Digrafo Original}
			\caption{}
		\end{figure}

		\column{0.4\textwidth}
		\begin{figure}[H]
			\begin{tikzpicture}[scale=0.9,
					vertex/.style={circle,draw,fill=blue!10,minimum size=0.6cm,font=\small},
					arb/.style={->,>=stealth,very thick,accentpurple}]

				\node[vertex,fill=red!20] (r) at (2,2.5) {$r$};
				\node[vertex] (a) at (0,1) {$a$};
				\node[vertex] (b) at (2,1) {$b$};
				\node[vertex] (c) at (4,1) {$c$};
				\node[vertex] (d) at (1,0) {$d$};
				\node[vertex] (e) at (3,0) {$e$};

				\draw[arb] (r) -- node[above left,font=\tiny,black] {3} (a);
				\draw[arb] (r) -- node[above right,font=\tiny,black] {4} (c);
				\draw[arb] (b) -- node[above,font=\tiny,black] {1} (a);
				\draw[arb] (a) -- node[left,font=\tiny,black] {2} (d);
				\draw[arb] (c) -- node[right,font=\tiny,black] {2} (e);
				\draw[arb] (c) -- node[above,font=\tiny,black] {4} (b);
			\end{tikzpicture}
			\caption{r-Arborescência Geradora}
			\caption{Custo: 16}
		\end{figure}

		\column{0.3\textwidth}
		\begin{figure}[H]
			\begin{tikzpicture}[scale=0.9,
					vertex/.style={circle,draw,fill=blue!10,minimum size=0.6cm,font=\small},
					arb/.style={->,>=stealth,very thick,accentpurple}]

				\node[vertex,fill=red!20] (r) at (2,2.5) {$r$};
				\node[vertex] (a) at (0,1) {$a$};
				\node[vertex] (b) at (2,1) {$b$};
				\node[vertex] (c) at (4,1) {$c$};
				\node[vertex] (d) at (1,0) {$d$};
				\node[vertex] (e) at (3,0) {$e$};

				\draw[arb] (r) -- node[above left,font=\tiny,black] {5} (b);
				\draw[arb] (r) -- node[above right,font=\tiny,black] {4} (c);
				\draw[arb] (b) -- node[above,font=\tiny,black] {1} (a);
				\draw[arb] (a) -- node[left,font=\tiny,black] {2} (d);
				\draw[arb] (d) -- node[below,font=\tiny,black] {1} (e);
			\end{tikzpicture}
			\caption{Geradora Mínima}
			\caption{Custo: 13}
		\end{figure}
	\end{columns}
\end{frame}

% ==================== SEÇÃO 2: CHU-LIU-EDMONDS ====================
\section{Algoritmo de Chu-Liu-Edmonds}

\begin{frame}{Chu-Liu-Edmonds}
	\begin{block}{Algoritmo Recursivo: dado um r-digrafo ponderado $(D,w,r)$}
		chu--liu--edmonds($(D,w,r)$):
		\begin{enumerate}
			\item \textbf{Reduzir custos}: para cada vértice $v \neq r$, subtrair $\lambda(v) = \min\{w(a) : a \in \delta^-(v)\}$

			\item \textbf{Construir $D_0$}: escolhendo um arco $a_v$ de custo reduzido zero para cada $v \neq r$

			\item \textbf{Verificar}: se $D_0$ é uma $r$-arborescência $\Rightarrow$ \textbf{devolver} $D_0$

			      Caso contrário:

			\item \textbf{Contração}: encontrar ciclo $C$ em $D_0$ e contrair

			\item \textbf{Chamada recursiva}: Seja $D' = D/C$ e $w' = w_\lambda/C$. Calcular $T' =$ chu-liu-edmonds$(D', w', r)$

			\item \textbf{Devolver}: expandir$(T')$
		\end{enumerate}
	\end{block}
\end{frame}


% ...existing code...
\begin{frame}{Escolha Gulosa}
	\begin{columns}[c]
		\column{0.48\textwidth}
		\textbf{Definição:}

		Para cada $v \neq r$, escolher um arco $a_v$ de custo mínimo que entra em $v$:
		$$T := \{a_v : v \in V \setminus \{r\}\}$$

		\vspace{0.2cm}
		\textbf{Propriedade:}

		Se $T$ é uma $r$-arborescência, então $T$ tem custo mínimo.

		\column{0.52\textwidth}
		\begin{tikzpicture}[>=Stealth, node distance=2.0cm,
				v/.style={circle, draw, fill=blue!8, minimum size=7mm},
				cheap/.style={->, very thick, draw=blue!70},
				expensive/.style={->, dashed, draw=gray!60}]
			\node[v] (r) {\small $r$};
			\node[v] (a) [right=2.2cm of r] {\small $a$};
			\node[v] (b) [above right=1.2cm and 1.6cm of a] {\small $b$};
			\node[v] (c) [below right=1.2cm and 1.6cm of a] {\small $c$};

			\draw[cheap] (a) to[bend left=12] node[midway, above, sloped] {\scriptsize 3} (b);
			\draw[cheap] (b) to[bend left=12] node[midway, above, sloped] {\scriptsize 2} (c);
			\draw[expensive] (c) to[bend left=14] node[midway, below, sloped] {\scriptsize 3} (a);

			\draw[cheap] (r) to[bend left=8] node[midway, above, sloped] {\scriptsize 3} (a);
			\draw[expensive] (r) to[bend left=18] node[midway, above, sloped] {\scriptsize 4} (b);
			\draw[expensive] (r) to[bend right=10] node[midway, below, sloped] {\scriptsize 5} (c);
		\end{tikzpicture}
	\end{columns}

	\vspace{0.2cm}
	\begin{alertblock}{Resultado}
		$T = \{(r,a), (a,b), (b,c)\}$ é uma $r$-arborescência de custo mínimo!
	\end{alertblock}
\end{frame}

\begin{frame}{E quando a escolha gulosa falha?}
	\begin{columns}[c]
		\column{0.48\textwidth}
		\textbf{Problema:}

		A escolha gulosa pode produzir um conjunto $T$ que \emph{não} é uma $r$-arborescência.

		\vspace{0.3cm}
		\textbf{Exemplo:}

		Os arcos de custo mínimo formam um ciclo $(a, b, c, a)$ sem alcançar $r$.

		\column{0.52\textwidth}
		\begin{tikzpicture}[>=Stealth, node distance=2.0cm,
				v/.style={circle, draw, fill=blue!8, minimum size=7mm},
				cheap/.style={->, very thick, draw=blue!70},
				expensive/.style={->, dashed, draw=gray!60}]
			\node[v] (r) {\small $r$};
			\node[v] (a) [right=2.2cm of r] {\small $a$};
			\node[v] (b) [above right=1.2cm and 1.6cm of a] {\small $b$};
			\node[v] (c) [below right=1.2cm and 1.6cm of a] {\small $c$};

			\draw[cheap] (a) to[bend left=12] node[midway, above, sloped] {\scriptsize 3} (b);
			\draw[cheap] (b) to[bend left=12] node[midway, above, sloped] {\scriptsize 2} (c);
			\draw[cheap] (c) to[bend left=14] node[midway, below, sloped] {\scriptsize 3} (a);

			\draw[expensive] (r) to[bend left=8] node[midway, above, sloped] {\scriptsize 3} (a);
			\draw[expensive] (r) to[bend left=18] node[midway, above, sloped] {\scriptsize 4} (b);
			\draw[expensive] (r) to[bend right=10] node[midway, below, sloped] {\scriptsize 5} (c);
		\end{tikzpicture}

		\vspace{0.2cm}
		{\small Arcos azuis formam um \textbf{ciclo}!}
	\end{columns}
\end{frame}

% \begin{frame}{Passo 1: Redução de Custos}
% \begin{columns}[t] % alinhamento pelo topo
% 	\column{0.01\textwidth}
% 	\column{0.44\textwidth}
% 	\hspace*{6mm}
% 	\begin{alertblock}{Objetivo}
% 		Encontrar a $r$-arborescência geradora de custo mínimo através do algoritmo de Chu-Liu-Edmonds.
% 		\vspace{0.6em}

% 		Solução:
% 		\begin{itemize}
% 			\item Redução de custos por vértice;
% 			\item Construção de \(D_0\) com arestas de custo zero;
% 			\item Identificar e contrair de ciclos;
% 			\item Resolver recursivamente;
% 		\end{itemize}
% 		\vspace{4mm}
% 	\end{alertblock}

% 	\column{0.55\textwidth}
% 	\begin{center}
% 		\begin{tikzpicture}[scale=0.9, >=Stealth,
% 				vertex/.style={circle,draw,fill=blue!8,thick,inner sep=1.2pt,minimum size=7mm},
% 				edge/.style={->,semithick,gray!70}]
% 			% nós (posições ajustadas)
% 			\node[vertex] (r) at (0,0) {$r$};
% 			\node[vertex] (a) at (2,1.6) {$a$};
% 			\node[vertex] (b) at (2,-1.6) {$b$};
% 			\node[vertex] (c) at (4,2.2) {$c$};
% 			\node[vertex] (d) at (4,0) {$d$};
% 			\node[vertex] (e) at (4,-2.2) {$e$};

% 			\draw[edge] (r) -- node[above,font=\scriptsize] {1} (a);
% 			\draw[edge] (r) -- node[below,font=\scriptsize] {2} (b);
% 			\draw[edge] (r) .. controls (0.8,2.4) and (3.2,3.0) .. node[above, yshift=2pt,pos=0.53,font=\scriptsize] {3} (c);
% 			\draw[edge] (b) to[bend left=12] node[left,font=\scriptsize] {4} (a);
% 			\draw[edge] (a) to[bend left=14] node[right,font=\scriptsize] {5} (b);
% 			\draw[edge] (a) -- node[right,font=\scriptsize] {3} (d);
% 			\draw[edge] (a) -- node[above,font=\scriptsize] {1} (c);
% 			\draw[edge] (b) -- node[above,font=\scriptsize] {1} (d);
% 			\draw[edge] (b) -- node[below,font=\scriptsize] {2} (e);
% 			\draw[edge] (c) .. controls (5.6,2.6) and (5.6,-2.6) .. node[xshift=8pt,pos=0.52,font=\scriptsize] {5} (e);
% 		\end{tikzpicture}
% 	\end{center}
% \end{columns}
% \end{frame}
% ...existing code...

\begin{frame}{Passo 1: Redução de Custos}
	\begin{columns}[c]
		\column{0.48\textwidth}
		\textbf{Definição:}

		\vspace{0.2cm}
		Para cada $v \in V \setminus \{r\}$:
		$$\lambda(v) := \min\{w(a) : a \in \delta^-(v)\}$$

		Custo $\lambda$-reduzido:
		$$w_\lambda(uv) := w(uv) - \lambda(v)$$
		\vspace{0.1cm}
		\textbf{Valores de $\lambda$:}
		\begin{itemize}
			\item $\lambda(a) = 3$, $\lambda(b) = 3$, $\lambda(c) = 2$
		\end{itemize}

		\column{0.52\textwidth}
		\begin{tikzpicture}[>=Stealth, node distance=2.0cm,
				v/.style={circle, draw, fill=blue!8, minimum size=7mm},
				cycle/.style={->, very thick, draw=blue!70},
				arc/.style={->, dashed, draw=gray!60}]
			\node[v] (r) {\small $r$};
			\node[v] (a) [right=2.2cm of r] {\small $a$};
			\node[v] (b) [above right=1.2cm and 1.6cm of a] {\small $b$};
			\node[v] (c) [below right=1.2cm and 1.6cm of a] {\small $c$};

			\draw[cycle] (a) to[bend left=12] node[midway, above, sloped] {\scriptsize 0} (b);
			\draw[cycle] (b) to[bend left=12] node[midway, above, sloped] {\scriptsize 0} (c);
			\draw[cycle] (c) to[bend left=14] node[midway, below, sloped] {\scriptsize 0} (a);

			\draw[cycle] (r) to[bend left=8] node[midway, above, sloped] {\scriptsize 0} (a);
			\draw[arc] (r) to[bend left=18] node[midway, above, sloped] {\scriptsize 1} (b);
			\draw[arc] (r) to[bend right=10] node[midway, below, sloped] {\scriptsize 3} (c);
		\end{tikzpicture}

		\vspace{0.2cm}
		{\small Arcos do ciclo têm custo zero!}
	\end{columns}

	\vspace{0.1cm}
	\begin{block}{}
		Arcos com custo zero são candidatos para $D_0$
	\end{block}
\end{frame}

\begin{frame}[fragile]{Implementação: Redução de Custos}
	\textbf{Função \texttt{reduce\_weights}:}
	\begin{columns}[t]
		\column{0.5\textwidth}
		\begin{lstlisting}
def reduce_weights(D: nx.DiGraph, v: int):
    in_edges = D.in_edges(v, data=True)
    yv = min((data["w"] 
              for _, _, data in in_edges))
    for u, _, _ in in_edges:
        D[u][v]["w"] -= yv
\end{lstlisting}

		\column{0.45\textwidth}
		\textbf{Descrição:}
		\begin{itemize}
			\item Calcula $\lambda(v) = \min\{w(a) : a \in \delta^-(v)\}$
			\item Reduz o custo de cada arco que entra em $v$
			\item Complexidade: $O(k)$ onde $k$ é o número de arcos entrando em $v$
		\end{itemize}
	\end{columns}

	\vspace{0.3cm}
	\begin{block}{Resultado}
		Após executar \texttt{reduce\_weights(D, v)} para cada $v \neq r$, todos os vértices têm ao menos um arco de entrada com custo zero.
	\end{block}
\end{frame}

\begin{frame}{Passo 2: Construção de $D_0$}
	\begin{columns}[c]
		\column{0.5\textwidth}
		\textbf{Formação de $D_0$:}
		\vspace{0.1cm}
		Para cada $v \neq r$, escolher um arco $a_v \in \delta^-(v)$ com $w_\lambda(a_v) = 0$ formar:

		$$D_0 := (V, \{a_v : v \in V \setminus \{r\}\})$$

		\vspace{0.1cm}
		\textbf{Arcos escolhidos:}
		\begin{itemize}
			\item $(r,a)$
			\item $(a,b)$, $(c,a)$
		\end{itemize}

		\column{0.5\textwidth}
		\begin{tikzpicture}[>=Stealth, node distance=2.0cm,
				v/.style={circle, draw, fill=blue!8, minimum size=7mm},
				cycle/.style={->, very thick, draw=blue!70},
				arc/.style={->, dashed, draw=gray!60}]
			\node[v] (r) {\small $r$};
			\node[v] (a) [right=2.2cm of r] {\small $a$};
			\node[v] (b) [above right=1.2cm and 1.6cm of a] {\small $b$};
			\node[v] (c) [below right=1.2cm and 1.6cm of a] {\small $c$};

			\draw[cycle] (a) to[bend left=12] node[midway, above, sloped] {\scriptsize 0} (b);
			\draw[cycle] (b) to[bend left=12] node[midway, above, sloped] {\scriptsize 0} (c);
			\draw[cycle] (c) to[bend left=14] node[midway, below, sloped] {\scriptsize 0} (a);

			\draw[cycle] (r) to[bend left=8] node[midway, above, sloped] {\scriptsize 0} (a);
			\draw[arc] (r) to[bend left=18] node[midway, above, sloped] {\scriptsize 1} (b);
			\draw[arc] (r) to[bend right=10] node[midway, below, sloped] {\scriptsize 3} (c);
		\end{tikzpicture}
	\end{columns}
\end{frame}

\begin{frame}[fragile]{Passo 2: Implementação da Construção de D0 em Python}
	\textbf{Função \texttt{get\_Dzero}:}

	\vspace{0.2cm}
	\begin{columns}[t]
		\column{0.55\textwidth}
		\begin{lstlisting}
def get_Dzero(D: nx.DiGraph, r: int):
    D_zero = nx.DiGraph()
    for v in D.nodes():
        if v != r:
            in_edges = D.in_edges(v, 
                                  data=True)
            u = next((u for u, _, data 
                      in in_edges 
                      if data["w"] == 0))
            D_zero.add_edge(u, v)
    return D_zero
\end{lstlisting}

		\column{0.45\textwidth}
		\textbf{Descrição:}
		\begin{itemize}
			\item Para cada vértice $v \neq r$, seleciona um arco com custo zero
			\item Constrói subdigrafo gerador $D_0$
			\item Garantido existir arco de custo zero após redução
		\end{itemize}
	\end{columns}

	\vspace{0.3cm}
	\begin{block}{Observação}
		Se $D_0$ for uma arborescência, então $D_0$ é necessariamente uma $r$-arborescência ótima.
	\end{block}
\end{frame}

\begin{frame}{Passo 3: Verificação de $D_0$}
	\begin{columns}[c]
		\column{0.48\textwidth}
		\textbf{Verificar:}

		Se $D_0$ é uma $r$-arborescência $\Rightarrow$ \textbf{devolver} $D_0$

		\vspace{0.3cm}
		\textbf{Caso contrário:}

		$D_0$ contém algum ciclo $C$.

		$\Rightarrow$ \textbf{prosseguir} para os passos 4 e 5.

		\column{0.52\textwidth}
		\begin{tikzpicture}[>=Stealth, node distance=2.0cm,
				v/.style={circle, draw, fill=blue!8, minimum size=7mm},
				cycle/.style={->, very thick, draw=blue!70},
				arc/.style={->, dashed, draw=gray!60}]
			\node[v] (r) {\small $r$};
			\node[v] (a) [right=2.2cm of r] {\small $a$};
			\node[v] (b) [above right=1.2cm and 1.6cm of a] {\small $b$};
			\node[v] (c) [below right=1.2cm and 1.6cm of a] {\small $c$};

			\draw[cycle] (a) to[bend left=12] node[midway, above, sloped] {\scriptsize 0} (b);
			\draw[cycle] (b) to[bend left=12] node[midway, above, sloped] {\scriptsize 0} (c);
			\draw[cycle] (c) to[bend left=14] node[midway, below, sloped] {\scriptsize 0} (a);

			\draw[cycle] (r) to[bend left=8] node[midway, above, sloped] {\scriptsize 0} (a);
			\draw[arc] (r) to[bend left=18] node[midway, above, sloped] {\scriptsize 1} (b);
			\draw[arc] (r) to[bend right=10] node[midway, below, sloped] {\scriptsize 3} (c);
		\end{tikzpicture}
	\end{columns}
	\vspace{0.3cm}
	\begin{alertblock}{$D_0$ não é uma $r$-arborescência!}
		Neste exemplo, $D_0 = \{(r,a), (a,b), (c,a)\}$ não forma uma $r$-arborescência pois contém o ciclo $(a, b, c, a)$.
	\end{alertblock}
\end{frame}

\begin{frame}[fragile]{Passo 3: Implementação da Verificação de D0 em Python}
	\textbf{Verificação e detecção de ciclo:}

	\vspace{0.2cm}
	\begin{columns}[t]
		\column{0.55\textwidth}
		\begin{lstlisting}
# Verificar se D_zero eh arborescencia
if nx.is_arborescence(D_zero):
    # Restaurar pesos originais
    for u, v in D_zero.edges:
        D_zero[u][v]["w"] = D[u][v]["w"]
    return D_zero

# Encontrar ciclo
def find_cycle(D_zero: nx.DiGraph):
    nodes_in_cycle = set()
    for u, v, _ in nx.find_cycle(
            D_zero, orientation="original"):
        nodes_in_cycle.update([u, v])
    return D_zero.subgraph(
                    nodes_in_cycle)
\end{lstlisting}

		\column{0.45\textwidth}
		\textbf{Descrição:}
		\begin{itemize}
			\item Usa função de biblioteca \texttt{is\_arborescence}
			\item Se não for arborescência, encontra um ciclo
			\item \texttt{find\_cycle} retorna os arcos do ciclo
			\item Constrói subdigrafo induzido pelos vértices do ciclo
		\end{itemize}
	\end{columns}
\end{frame}

\begin{frame}{Passo 4: Contração de Ciclos}
	\begin{columns}[c]
		\column{0.48\textwidth}
		\textbf{Operação:}

		Contrair ciclo $C$ em supervértice $x_C$.

		\vspace{0.2cm}
		\textbf{Novo problema:}
		$(D', w', r)$ onde:
		\begin{itemize}
			\item $D' := D/C \mapsto x_C$
			\item $w' := w_\lambda/C \mapsto x_C$
		\end{itemize}

		\vspace{0.1cm}
		O arco de $D'$ que entra em $x_C$ deve corresponder ao arco de $D$ que entra em algum vértice de $C$

		\vspace{0.2cm}
		\emph{Podem ter arcos saindo de $x_C$ em $D'$.}

		\column{0.52\textwidth}
		\begin{tikzpicture}[>=Stealth, node distance=2cm,
				v/.style={circle, draw, fill=blue!8, minimum size=7mm},
				super/.style={circle, draw, very thick, fill=orange!10, minimum size=9mm}]
			\node[v] (r) {\small $r$};
			\node[super] (xC) [right=3.5cm of r] {\small $x_C$};

			\draw[->, very thick] (r) to[bend left=12] node[midway, above, sloped] {\scriptsize 0} (xC);
		\end{tikzpicture}

		\vspace{0.3cm}
		{\small Digrafo contraído $D'$}
	\end{columns}

	\vspace{0.2cm}
	\begin{block}{Propriedade}
		Uma solução ótima em $D'$ pode ser expandida para uma solução ótima em $D$.
	\end{block}
\end{frame}

\begin{frame}[fragile]{Passo 4: Implementação da Contração de Ciclos em Python}
	\textbf{Função \texttt{contract\_cycle} (parte 1):}

	\vspace{0.1cm}
	\begin{lstlisting}[basicstyle=\tiny\ttfamily]
def contract_cycle(D: nx.DiGraph, C: nx.DiGraph, label: int):
    cycle_nodes: set[int] = set(C.nodes())
    
    # Encontrar arcos essenciais que ENTRAM no ciclo
    in_to_cycle: dict[int, tuple[int, float]] = {}
    for u in D.nodes:
        if u not in cycle_nodes:
            min_edge = min(((v, data["w"])
                           for _, v, data in D.out_edges(u, data=True)
                           if v in cycle_nodes),
                          key=lambda x: x[1], default=None)
            if min_edge:
                in_to_cycle[u] = min_edge
    
    # Adicionar arcos de u para label (supervertice)
    for u, (v, w) in in_to_cycle.items():
        D.add_edge(u, label, w=w)
\end{lstlisting}

	\vspace{0.1cm}
	{\small Arcos \textbf{essenciais}: menor custo de cada vértice externo para o ciclo}
\end{frame}

\begin{frame}[fragile]{Passo 4: Implementação da Contração de Ciclos em Python (cont.)}
	\textbf{Função \texttt{contract\_cycle} (parte 2):}

	\vspace{0.1cm}
	\begin{lstlisting}[basicstyle=\tiny\ttfamily]
    # Encontrar arcos essenciais que SAEM do ciclo
    out_from_cycle: dict[int, tuple[int, float]] = {}
    for v in D.nodes:
        if v not in cycle_nodes:
            min_edge = min(((u, data["w"])
                           for u, _, data in D.in_edges(v, data=True)
                           if u in cycle_nodes),
                          key=lambda x: x[1], default=None)
            if min_edge:
                out_from_cycle[v] = min_edge
    
    # Adicionar arcos de label para v
    for v, (u, w) in out_from_cycle.items():
        D.add_edge(label, v, w=w)
    
    # Remover vertices do ciclo
    D.remove_nodes_from(cycle_nodes)
    return in_to_cycle, out_from_cycle
\end{lstlisting}

	\vspace{0.1cm}
	{\small Os dicionários retornados são usados na expansão. Complexidade: $O(m)$}
\end{frame}

\begin{frame}{Passo 5: Chamada Recursiva}
	\begin{columns}[c]
		\column{0.48\textwidth}
		\textbf{Novo problema:} $(D', w', r)$

		\vspace{0.2cm}
		\textbf{Chamada recursiva:}
		$$T' := \text{chu--liu--edmonds}(D', w', r)$$

		\vspace{0.2cm}
		\textbf{Resultado:} $T'$ é uma $r$-arborescência de custo mínimo em $(D', w')$

		\column{0.52\textwidth}
		\begin{tikzpicture}[>=Stealth, node distance=2cm,
				v/.style={circle, draw, fill=blue!8, minimum size=7mm},
				super/.style={circle, draw, very thick, fill=orange!10, minimum size=9mm},
				selected/.style={->, very thick, draw=green!70}]
			\node[v] (r) {\small $r$};
			\node[super] (xC) [right=3.5cm of r] {\small $x_C$};

			\draw[selected] (r) to[bend left=12] node[midway, above, sloped] {\scriptsize 0} (xC);
		\end{tikzpicture}

		\vspace{0.3cm}
		{\small $r$-arborescência ótima em $D'$}
	\end{columns}
\end{frame}

\begin{frame}[fragile]{Passo 5: Implementação da Chamada Recursiva em Python}
	\textbf{Estrutura recursiva do algoritmo:}

	\vspace{0.2cm}
	\begin{columns}[t]
		\column{0.55\textwidth}
		\begin{lstlisting}[basicstyle=\scriptsize\ttfamily]
def chuliu_edmonds(D: nx.DiGraph, 
                   r: int, label: int):
    D_copy = D.copy()
    
    # Reducao de custos
    for v in D_copy.nodes:
        if v != r:
            reduce_weights(D_copy, v)
    
    D_zero = get_Dzero(D_copy, r)
    
    if nx.is_arborescence(D_zero):
        # Restaurar pesos e devolver
        ...
        return D_zero
    
    # Contrair ciclo e chamar recursao
    C = find_cycle(D_zero)
    in_to_cycle, out_from_cycle = \
        contract_cycle(D_copy, C, label)
    
    F_prime = chuliu_edmonds(
                D_copy, r, label + 1)
\end{lstlisting}

		\column{0.45\textwidth}
		\textbf{Observações:}
		\begin{itemize}
			\item \texttt{label} identifica o supervértice
			\item Cada chamada recursiva incrementa \texttt{label}
			\item Caso base: $D_0$ é arborescência
			\item Caso recursivo: contrai ciclo e resolve subproblema
		\end{itemize}
	\end{columns}

	\vspace{0.3cm}
	\begin{block}{Garantia}
		Cada chamada reduz o número de vértices em pelo menos 1 (o ciclo contraído).
	\end{block}
\end{frame}

\begin{frame}{Passo 6: Reexpansão da Solução}
	\begin{columns}[c]
		\column{0.48\textwidth}
		\textbf{Dado:} $T'$ ótima em $(D', w')$

		\vspace{0.2cm}
		\textbf{Construir:} $T$ ótima em $(D, w)$

		\vspace{0.2cm}
		\textbf{Procedimento:}
		\begin{enumerate}
			\item Seja $uv$ o arco de $D$ correspondente ao arco $u x_C$ de $T'$
			\item Incluir $uv$ em $T$
			\item Incluir todos os arcos de $C$ exceto aquele que entra em $v$
		\end{enumerate}

		\vspace{0.2cm}
		{\small Resultado: $T$ é uma $r$-arborescência de custo mínimo}

		\column{0.52\textwidth}
		\begin{tikzpicture}[>=Stealth, node distance=2.0cm,
				v/.style={circle, draw, fill=blue!8, minimum size=7mm},
				selected/.style={->, very thick, draw=green!70}]
			\node[v] (r) {\small $r$};
			\node[v] (a) [right=2.2cm of r] {\small $a$};
			\node[v] (b) [above right=1.2cm and 1.6cm of a] {\small $b$};
			\node[v] (c) [below right=1.2cm and 1.6cm of a] {\small $c$};

			\draw[selected] (r) to[bend left=8] node[midway, above, sloped] {\scriptsize 3} (a);
			\draw[selected] (a) to[bend left=12] node[midway, above, sloped] {\scriptsize 3} (b);
			\draw[selected] (b) to[bend left=12] node[midway, above, sloped] {\scriptsize 2} (c);
		\end{tikzpicture}

		\vspace{0.3cm}
		{\small $r$-arborescência final no digrafo original}
	\end{columns}
\end{frame}

\begin{frame}[fragile]{Passo 6: Implementação da Reexpansão da Solução em Python}
	\textbf{Expansão após chamada recursiva:}

	\vspace{0.2cm}
	\begin{columns}[t]
		\column{0.55\textwidth}
		\begin{lstlisting}[basicstyle=\tiny\ttfamily]
# F_prime eh solucao em D' (com supervertice)
# Encontrar arco que entra em label
in_edge = next(iter(
    F_prime.in_edges(label, data=True)))
u, _, _ = in_edge

# Arco correspondente em D original
v, _ = in_to_cycle[u]
F_prime.add_edge(u, v)

# Adicionar arcos do ciclo, 
# exceto o que entra em v
for u_c, v_c in C.edges:
    if v != v_c:
        F_prime.add_edge(u_c, v_c)

# Arcos que saem do supervertice
for _, z, _ in list(
        F_prime.out_edges(label, data=True)):
    u_cycle, _ = out_from_cycle[z]
    F_prime.add_edge(u_cycle, z)

# Remover supervertice
F_prime.remove_node(label)

# Restaurar pesos originais
for u, v in F_prime.edges:
    F_prime[u][v]["w"] = D[u][v]["w"]

return F_prime
\end{lstlisting}

		\column{0.45\textwidth}
		\textbf{Passos da expansão:}
		\begin{enumerate}
			\item Identificar arco que entra em \texttt{label}
			\item Encontrar arco correspondente em $D$
			\item Adicionar arcos do ciclo $C$ (exceto um)
			\item Transferir arcos que saem de \texttt{label}
			\item Remover \texttt{label}
			\item Restaurar pesos originais
		\end{enumerate}
	\end{columns}

	\vspace{0.2cm}
	\begin{block}{Correção}
		A expansão garante que $T$ é uma $r$-arborescência ótima no digrafo original $D$.
	\end{block}
\end{frame}

% ==================== SEÇÃO 3: ANDRÁS FRANK ====================
\section{Algoritmo de András Frank}

\begin{frame}{András Frank: Visão Geral}
	\begin{block}{Abordagem em Duas Fases}
		\textbf{Fase I:} Construir cobertura de subconjuntos minimais via redução de custos

		\textbf{Fase II:} Extrair arborescência da cobertura
	\end{block}

	\vspace{0.5cm}

	\begin{columns}[c]
		\column{0.5\textwidth}
		\textbf{Diferencial:}
		\begin{itemize}
			\item Trabalha com múltiplos vértices simultaneamente
			\item Usa componentes fortemente conexas
			\item Redução sistemática de custos
		\end{itemize}

		\column{0.5\textwidth}
		\textbf{Complexidade:}
		\begin{itemize}
			\item Fase I: $O(nm)$
			\item Fase II v1 (lista): $O(n^2)$
			\item Fase II v2 (heap): $O(n \log n)$
		\end{itemize}
	\end{columns}
\end{frame}

\begin{frame}{Fase I: Redução de Custos}
	\textbf{Para cada vértice} $v \neq r$: subtrair o mínimo de entrada

	\begin{center}
		\begin{tikzpicture}[scale=1.1,
				vertex/.style={circle,draw,minimum size=0.7cm}]

			% Antes
			\begin{scope}
				\node[vertex,fill=red!20] (r1) at (0,2) {$r$};
				\node[vertex] (A1) at (2,3) {$A$};
				\node[vertex] (B1) at (4,3) {$B$};
				\node[vertex] (C1) at (2,1) {$C$};

				\draw[->,>=stealth] (r1) -- node[above] {2} (A1);
				\draw[->,>=stealth] (r1) to[bend left] node[above] {10} (B1);
				\draw[->,>=stealth] (B1) -- node[right] {1} (A1);
				\draw[->,>=stealth] (A1) -- node[left] {4} (C1);

				\node at (2,0) {\textbf{Original}};
			\end{scope}

			% Depois
			\begin{scope}[xshift=6cm]
				\node[vertex,fill=red!20] (r2) at (0,2) {$r$};
				\node[vertex] (A2) at (2,3) {$A$};
				\node[vertex] (B2) at (4,3) {$B$};
				\node[vertex] (C2) at (2,1) {$C$};

				\draw[->,>=stealth,blue] (r2) -- node[above] {1} (A2);
				\draw[->,>=stealth] (r2) to[bend left] node[above] {10} (B2);
				\draw[->,>=stealth,blue,very thick] (B2) -- node[right] {\textbf{0}} (A2);
				\draw[->,>=stealth,blue,very thick] (A2) -- node[left] {\textbf{0}} (C2);

				\node at (2,0) {\textbf{Após Redução}};
			\end{scope}

		\end{tikzpicture}
	\end{center}

	Arcos com custo \textbf{zero} formam o digrafo $D_0$
\end{frame}

\begin{frame}{Fase I: Componentes Fortemente Conexas}
	\begin{columns}[c]
		\column{0.5\textwidth}
		Identificar componentes fortemente conexas (CFCs) em $D_0$

		\vspace{0.3cm}
		Cada CFC forma um \textbf{subconjunto minimal}

		\vspace{0.3cm}
		Construir sequência laminar de subconjuntos

		\column{0.5\textwidth}
		\begin{tikzpicture}[scale=0.9]
			\node[draw,circle,fill=yellow!20,minimum size=1.2cm] (S1) at (0,2) {$S_1$};
			\node[draw,circle,fill=yellow!20,minimum size=1.2cm] (S2) at (2,2) {$S_2$};
			\node[draw,circle,fill=yellow!20,minimum size=1.2cm] (S3) at (1,0.5) {$S_3$};

			\draw[->,>=stealth,thick] (S1) to[bend left] (S2);
			\draw[->,>=stealth,thick] (S2) to[bend left] (S1);
			\draw[->,>=stealth,thick] (S2) -- (S3);
		\end{tikzpicture}
	\end{columns}

	\vspace{0.5cm}
	\begin{block}{Condição de Otimalidade}
		Sequência $\lambda$ satisfaz: $|\delta^-(X)| = 1$ para cada $X$ em $\lambda$
	\end{block}
\end{frame}

\begin{frame}{Fase II: Construção da Arborescência}
	\textbf{Objetivo:} Extrair arborescência de $D_0$ respeitando $\lambda$

	\vspace{0.3cm}

	\begin{enumerate}
		\item Iniciar com conjunto $R = \{r\}$
		\item Para cada $v$ fora de $R$:
		      \begin{itemize}
			      \item Selecionar arco $(u,v)$ com $u \in R$ e custo reduzido zero
			      \item Adicionar $v$ a $R$
		      \end{itemize}
		\item Repetir até incluir todos os vértices
	\end{enumerate}

	\vspace{0.5cm}

	\begin{exampleblock}{Resultado}
		Arborescência ótima com mesma solução: custo 14
	\end{exampleblock}
\end{frame}

% ==================== SEÇÃO 4: RESULTADOS ====================
\section{Resultados Experimentais}

\begin{frame}{Comparação de Desempenho}
	\textbf{Experimentos:} 2000 digrafos aleatórios, $|V| \in [101, 4996]$

	\vspace{0.3cm}

	\begin{center}
		\begin{tabular}{lcc}
			\toprule
			\textbf{Algoritmo}    & \textbf{Tempo Mediano} & \textbf{Tempo Médio} \\
			\midrule
			Chu-Liu-Edmonds       & 0,25 s                 & 0,58 s               \\
			Frank Fase I          & 8,93 s                 & 12,40 s              \\
			Frank Fase II (lista) & 0,98 s                 & 1,34 s               \\
			Frank Fase II (heap)  & \textbf{0,016 s}       & \textbf{0,020 s}     \\
			\bottomrule
		\end{tabular}
	\end{center}

	\vspace{0.5cm}

	\begin{block}{Speedup Fase II}
		Heap vs Lista: aceleração de \textbf{58,12 vezes} (mediana)
	\end{block}
\end{frame}

\begin{frame}{Características Estruturais}
	\begin{columns}[c]
		\column{0.5\textwidth}
		\textbf{Contrações (Chu-Liu):}
		\begin{itemize}
			\item Mediana: 2 contrações
			\item Média: 6,82
			\item Máximo: 406
			\item 93,8\% com $< 20$
		\end{itemize}

		\vspace{0.3cm}
		Muito abaixo do limite teórico $O(n)$

		\column{0.5\textwidth}
		\begin{center}
			\includegraphics[width=\textwidth]{figures/fig_contractions_depth.png}
		\end{center}
	\end{columns}

	\vspace{0.3cm}
	\textbf{Consumo de memória:} mediana 11,5 MB (Fase I)
\end{frame}

% ==================== SEÇÃO 5: APLICAÇÃO WEB ====================
\section{Aplicação Web}

\begin{frame}{Motivação Didática}
	\begin{block}{Desafio}
		Algoritmos de grafos são \textbf{abstratos} e \textbf{difíceis de visualizar}
	\end{block}

	\vspace{0.5cm}

	\begin{columns}[c]
		\column{0.5\textwidth}
		\textbf{Solução Proposta:}
		\begin{itemize}
			\item Visualização interativa
			\item Execução passo a passo
			\item Feedback imediato
			\item Acessível via navegador
		\end{itemize}

		\column{0.5\textwidth}
		\textbf{Tecnologias:}
		\begin{itemize}
			\item PyScript (Python no browser)
			\item JavaScript
			\item HTML5/CSS3
			\item NetworkX
		\end{itemize}
	\end{columns}
\end{frame}

\begin{frame}{Arquitetura da Aplicação}
	\begin{center}
		\begin{tikzpicture}[
				node distance=1.5cm,
				box/.style={rectangle,draw,thick,minimum width=3cm,minimum height=1cm,align=center}]

			\node[box,fill=blue!20] (ui) {Interface HTML/CSS};
			\node[box,fill=green!20,below of=ui] (pyscript) {PyScript\\(Python)};
			\node[box,fill=yellow!20,below of=pyscript] (algos) {Algoritmos\\Chu-Liu | Frank};
			\node[box,fill=orange!20,below of=algos] (nx) {NetworkX};

			\draw[->,>=stealth,thick] (ui) -- (pyscript);
			\draw[->,>=stealth,thick] (pyscript) -- (algos);
			\draw[->,>=stealth,thick] (algos) -- (nx);
			\draw[->,>=stealth,thick] (nx) to[bend right=60] (ui);

		\end{tikzpicture}
	\end{center}
\end{frame}

\begin{frame}{Interface: Página Principal}
	\begin{center}
		\includegraphics[width=0.8\textwidth]{../assets/homehtml.png}
	\end{center}

	\begin{itemize}
		\item Navegação intuitiva entre algoritmos
		\item Acesso direto aos exemplos interativos
	\end{itemize}
\end{frame}

\begin{frame}{Interface: Desenho de Grafos}
	\begin{center}
		\includegraphics[width=0.75\textwidth]{../assets/drawhtml.png}
	\end{center}

	Funcionalidades:
	\begin{itemize}
		\item Adicionar vértices e arestas
		\item Definir pesos
		\item Exportar/importar grafos em JSON
	\end{itemize}
\end{frame}

\begin{frame}{Interface: Chu-Liu-Edmonds}
	\begin{center}
		\includegraphics[width=0.75\textwidth]{../assets/chuliuhtml.png}
	\end{center}

	\begin{itemize}
		\item Visualização passo a passo
		\item Destacamento de ciclos detectados
		\item Log detalhado das operações
	\end{itemize}
\end{frame}

\begin{frame}{Interface: András Frank}
	\begin{center}
		\includegraphics[width=0.75\textwidth]{../assets/andrasfrankv1html.png}
	\end{center}

	\begin{itemize}
		\item Exibição das duas fases
		\item Visualização de CFCs
		\item Comparação entre versões (lista vs heap)
	\end{itemize}
\end{frame}

\begin{frame}{Princípios de Design}
	\begin{block}{Teoria dos Registros de Representação (Duval)}
		Transitar entre diferentes representações:
		\begin{itemize}
			\item \textbf{Visual:} diagramas do grafo
			\item \textbf{Simbólico:} código Python
			\item \textbf{Textual:} log das operações
		\end{itemize}
	\end{block}

	\vspace{0.5cm}

	\begin{alertblock}{Feedback Imediato}
		Validação em tempo real das operações do usuário
	\end{alertblock}
\end{frame}

% ==================== SEÇÃO 6: CONCLUSÕES ====================
\section{Conclusões}

\begin{frame}{Contribuições do Trabalho}
	\begin{enumerate}
		\item \textbf{Implementação completa} de dois algoritmos clássicos
		      \begin{itemize}
			      \item Chu-Liu-Edmonds: recursivo com contração
			      \item András Frank: duas fases com otimização heap
		      \end{itemize}

		      \vspace{0.3cm}

		\item \textbf{Análise experimental} detalhada
		      \begin{itemize}
			      \item 2000 instâncias aleatórias
			      \item Comparação de desempenho e características estruturais
		      \end{itemize}

		      \vspace{0.3cm}

		\item \textbf{Aplicação web interativa}
		      \begin{itemize}
			      \item Ferramenta didática para visualização
			      \item Execução passo a passo dos algoritmos
			      \item Design centrado no usuário
		      \end{itemize}
	\end{enumerate}
\end{frame}

\begin{frame}{Principais Resultados}
	\begin{itemize}
		\item \textbf{Corretude validada:} custos idênticos em todas as instâncias

		      \vspace{0.3cm}

		\item \textbf{Chu-Liu-Edmonds} mais rápido para construção direta
		      \begin{itemize}
			      \item Mediana: 0,25 s vs 8,93 s (Fase I Frank)
		      \end{itemize}

		      \vspace{0.3cm}

		\item \textbf{Otimização heap} fundamental na Fase II
		      \begin{itemize}
			      \item Speedup: 58× (mediana), 61× (média)
		      \end{itemize}

		      \vspace{0.3cm}

		\item \textbf{Comportamento prático} muito melhor que limites teóricos
		      \begin{itemize}
			      \item Contrações: mediana 2 (limite $O(n)$)
			      \item Memória modesta: 11,5 MB
		      \end{itemize}
	\end{itemize}
\end{frame}

\begin{frame}{Trabalhos Futuros}
	\begin{block}{Extensões Possíveis}
		\begin{itemize}
			\item Implementar outras variantes (Tarjan, Gabow)
			\item Análise em grafos com estruturas especiais
			\item Paralelização dos algoritmos
			\item Extensão para grafos dinâmicos
		\end{itemize}
	\end{block}

	\vspace{0.5cm}

	\begin{block}{Melhorias na Aplicação}
		\begin{itemize}
			\item Modo de edição visual de grafos
			\item Geração automática de casos de teste
			\item Exercícios interativos com correção automática
			\item Integração com plataformas de ensino (Moodle, Jupyter)
		\end{itemize}
	\end{block}
\end{frame}

% ==================== SLIDE FINAL ====================
\begin{frame}[plain]
	\centering
	\vfill

	{\Huge Obrigado!}

	\vspace{1cm}

	{\Large Perguntas?}

	\vspace{1cm}

	{\normalsize \texttt{https://github.com/lorenypsum/graph-visualizer}}

	\vfill
\end{frame}

\end{document}
