\documentclass[aspectratio=169]{beamer}

% Tema e cores baseados no design do site ArboGraph
\usetheme{CambridgeUS}
\usecolortheme{default}

% Paleta de cores do logo e site
\definecolor{primarypurple}{RGB}{53, 43, 103}    % #352B67 - roxo escuro principal
\definecolor{secondarypurple}{RGB}{75, 66, 119}  % #4B4277 - roxo médio
\definecolor{accentpurple}{RGB}{141, 121, 229}   % #8d79e5 - roxo vibrante
\definecolor{lightpurple}{RGB}{164, 156, 255}    % #a49cc5ff - roxo claro (borda)
\definecolor{textgray}{RGB}{120, 116, 134}       % #787486 - cinza texto
\definecolor{lightgray}{RGB}{229, 229, 229}      % #E5E5E5 - fundo claro
\definecolor{divider}{RGB}{219, 219, 219}        % #DBDBDB - linhas divisórias

% Estrutura geral
\setbeamercolor{structure}{fg=primarypurple}
\setbeamercolor{palette primary}{bg=primarypurple,fg=white}
\setbeamercolor{palette secondary}{bg=secondarypurple,fg=white}
\setbeamercolor{palette tertiary}{bg=accentpurple,fg=white}
\setbeamercolor{frametitle}{bg=primarypurple,fg=white}

% Cores da capa
\setbeamercolor{title}{fg=lightgray}
\setbeamercolor{titlelike}{bg=primarypurple,fg=white}
\setbeamercolor{author}{fg=textgray}
\setbeamercolor{institute}{fg=textgray}
\setbeamercolor{date}{fg=textgray}

% Cores dos blocos (caixas de texto)
\setbeamercolor{block title}{bg=secondarypurple,fg=white}
\setbeamercolor{block body}{bg=lightgray,fg=black}
\setbeamercolor{block title alerted}{bg=secondarypurple,fg=white}
\setbeamercolor{block body alerted}{bg=lightgray,fg=black}

% Itens e enumerações
\setbeamercolor{item}{fg=accentpurple}
\setbeamercolor{enumerate item}{fg=accentpurple}
\setbeamercolor{enumerate subitem}{fg=lightpurple}

% Remover sombra dos blocos
\setbeamertemplate{blocks}[rounded][shadow=false]

% Remover "Figura" dos captions
\setbeamertemplate{caption}{\raggedright\insertcaption\par}

\usepackage[utf8]{inputenc}
\usepackage[T1]{fontenc}
\usepackage[brazil]{babel}
\usepackage{amsmath,amssymb}
\usepackage{graphicx}
\usepackage{tikz}
\usetikzlibrary{graphs,arrows.meta,positioning,fit,calc}
\usepackage{algorithm}
\usepackage{algpseudocode}
\usepackage{xcolor}
\usepackage{booktabs}

% Logo da UFABC
\logo{\includegraphics[height=0.9cm]{logo-ufabc.png}}

% Configurações do título
\title[ArboGraph]{Algoritmos para r-Arborescências Geradoras Mínimas em Digrafos:}
\subtitle{Uma Aplicação Web Interativa}
\author[L. Sampaio, S. Haddad]{Lorena Sampaio, Samira Haddad \\ Orientador: Prof. Dr. Mário Leston Rey}
\institute[UFABC]{Universidade Federal do ABC \\ Centro de Matemática, Computação e Cognição}
\date{\today}

% Comandos personalizados
\newcommand{\G}{\mathcal{G}}
\newcommand{\D}{\mathcal{D}}

\begin{document}

% ==================== SLIDE 1: TÍTULO ====================
\begin{frame}
	\titlepage
\end{frame}

% ==================== SLIDE 2: SUMÁRIO ====================
\begin{frame}{Sumário}
	\tableofcontents
\end{frame}

% ==================== SEÇÃO 1: INTRODUÇÃO ====================
\section{Introdução}

\begin{frame}{O Problema}
	\begin{block}{Encontrar uma $r$-Arborescência Geradora de Custo Mínimo}
		Dado um $r$-digrafo ponderado $(D,w,r)$:
		\begin{itemize}
			\item Encontrar uma $r$-arborescência geradora de custo mínimo de $D$
		\end{itemize}
	\end{block}

	\vspace{0.5cm}

	\textbf{Algoritmos estudados:}
	\begin{enumerate}
		\item Chu-Liu-Edmonds (1965-67)
		\item András Frank (1981-2014)
	\end{enumerate}
\end{frame}

% ==================== SLIDE 3: EXEMPLO ====================
\begin{frame}{Exemplo: $r$-Arborescência Geradora Mínima}
	\begin{columns}[c]
		\column{0.3\textwidth}
		\begin{figure}[H]
			\begin{tikzpicture}[scale=0.9,
					vertex/.style={circle,draw,fill=blue!10,minimum size=0.6cm,font=\small},
					edge/.style={->,>=stealth,thick,gray!90}]

				\node[vertex,fill=red!20] (r) at (2,2.5) {$r$};
				\node[vertex] (a) at (0,1) {$a$};
				\node[vertex] (b) at (2,1) {$b$};
				\node[vertex] (c) at (4,1) {$c$};
				\node[vertex] (d) at (1,0) {$d$};
				\node[vertex] (e) at (3,0) {$e$};

				\draw[edge] (r) -- node[above left,font=\tiny] {3} (a);
				\draw[edge] (r) -- node[above left,font=\tiny] {5} (b);
				\draw[edge] (r) -- node[above right,font=\tiny] {4} (c);
				\draw[edge] (a) -- node[left,font=\tiny] {2} (d);
				\draw[edge] (b) -- node[above,font=\tiny] {1} (a);
				\draw[edge] (b) -- node[above,font=\tiny] {3} (d);
				\draw[edge] (c) -- node[right,font=\tiny] {2} (e);
				\draw[edge] (c) -- node[above,font=\tiny] {4} (b);
				\draw[edge] (d) to[bend left] node[below,font=\tiny] {1} (e);
			\end{tikzpicture}
			\caption{Digrafo Original}
			\caption{}
		\end{figure}

		\column{0.4\textwidth}
		\begin{figure}[H]
			\begin{tikzpicture}[scale=0.9,
					vertex/.style={circle,draw,fill=blue!10,minimum size=0.6cm,font=\small},
					arb/.style={->,>=stealth,very thick,accentpurple}]

				\node[vertex,fill=red!20] (r) at (2,2.5) {$r$};
				\node[vertex] (a) at (0,1) {$a$};
				\node[vertex] (b) at (2,1) {$b$};
				\node[vertex] (c) at (4,1) {$c$};
				\node[vertex] (d) at (1,0) {$d$};
				\node[vertex] (e) at (3,0) {$e$};

				\draw[arb] (r) -- node[above left,font=\tiny,black] {3} (a);
				\draw[arb] (r) -- node[above right,font=\tiny,black] {4} (c);
				\draw[arb] (b) -- node[above,font=\tiny,black] {1} (a);
				\draw[arb] (a) -- node[left,font=\tiny,black] {2} (d);
				\draw[arb] (c) -- node[right,font=\tiny,black] {2} (e);
				\draw[arb] (c) -- node[above,font=\tiny,black] {4} (b);
			\end{tikzpicture}
			\caption{r-Arborescência Geradora}
			\caption{Custo: 16}
		\end{figure}

		\column{0.3\textwidth}
		\begin{figure}[H]
			\begin{tikzpicture}[scale=0.9,
					vertex/.style={circle,draw,fill=blue!10,minimum size=0.6cm,font=\small},
					arb/.style={->,>=stealth,very thick,accentpurple}]

				\node[vertex,fill=red!20] (r) at (2,2.5) {$r$};
				\node[vertex] (a) at (0,1) {$a$};
				\node[vertex] (b) at (2,1) {$b$};
				\node[vertex] (c) at (4,1) {$c$};
				\node[vertex] (d) at (1,0) {$d$};
				\node[vertex] (e) at (3,0) {$e$};

				\draw[arb] (r) -- node[above left,font=\tiny,black] {5} (b);
				\draw[arb] (r) -- node[above right,font=\tiny,black] {4} (c);
				\draw[arb] (b) -- node[above,font=\tiny,black] {1} (a);
				\draw[arb] (a) -- node[left,font=\tiny,black] {2} (d);
				\draw[arb] (d) -- node[below,font=\tiny,black] {1} (e);
			\end{tikzpicture}
			\caption{Geradora Mínima}
			\caption{Custo: 13}
		\end{figure}
	\end{columns}
\end{frame}

% ==================== SEÇÃO 2: CHU-LIU-EDMONDS ====================
\section{Algoritmo de Chu-Liu-Edmonds}

\begin{frame}{Chu-Liu-Edmonds: Ideia Principal}
	\begin{block}{Algoritmo Recursivo: dado um r-digrafo ponderado $(D,w,r)$}
		chu--liu--edmonds($(D,w,r)$):
		\begin{enumerate}
			\item \textbf{Reduzir custos}: para cada vértice $v \neq r$, subtrair $\lambda(v) = \min\{w(a) : a \in \delta^-(v)\}$

			\item \textbf{Construir $D_0$}: escolhendo um arco $a_v$ de custo reduzido zero para cada $v \neq r$

			\item \textbf{Verificar}: se $D_0$ é uma $r$-arborescência $\Rightarrow$ \textbf{devolver} $D_0$

			      Caso contrário:

			\item \textbf{Contração}: encontrar ciclo $C$ em $D_0$ e contrair

			\item \textbf{Chamada recursiva}: Seja $D' = D/C$ e $w' = w_\lambda/C$. Calcular $T' =$ chu-liu-edmonds$(D', w', r)$

			\item \textbf{Devolver}: expandir$(T')$
		\end{enumerate}
	\end{block}
\end{frame}

% ==================== SLIDE 4: EXEMPLO DETALHADO ====================
\begin{frame}{Digrafo de Exemplo}
	\begin{center}
		\begin{figure}[H]
			\begin{tikzpicture}[scale=1.2, >=Stealth,
					vertex/.style={circle,draw,fill=blue!8,thick,inner sep=1.2pt,minimum size=8mm},
					chosen/.style={->, very thick, draw=blue},
					edge/.style={->,semithick}]

				\node[vertex] (r) at (0,0) {$r$};
				\node[vertex] (a) at (3,1.6) {$a$};
				\node[vertex] (b) at (3,-1.6) {$b$};
				\node[vertex] (c) at (6,2.2) {$c$};
				\node[vertex] (d) at (6,0.0) {$d$};
				\node[vertex] (e) at (6,-2.2) {$e$};

				\draw[chosen] (r) -- node[above,font=\scriptsize] {1} (a);
				\draw[chosen] (r) -- node[below,font=\scriptsize] {2} (b);
				\draw[edge] (r) .. controls (0.8,2.4) and (5.2,3.0) .. node[yshift=4pt,pos=0.53,font=\scriptsize] {3} (c);
				\draw[edge] (b) to[bend left=12] node[left,font=\scriptsize] {4} (a);
				\draw[edge] (a) to[bend left=14] node[right,font=\scriptsize] {5} (b);
				\draw[edge] (a) -- node[right,font=\scriptsize] {3} (d);
				\draw[chosen] (a) -- node[above,font=\scriptsize] {1} (c);
				\draw[chosen] (b) -- node[above,font=\scriptsize] {1} (d);
				\draw[chosen] (b) -- node[below,font=\scriptsize] {2} (e);
				\draw[edge] (c) .. controls (7.6,2.6) and (7.6,-2.6) .. node[xshift=8pt,pos=0.52,font=\scriptsize] {5} (e);
			\end{tikzpicture}
		\end{figure}
	\end{center}

	\begin{alertblock}{Objetivo}
		Encontrar a $r$-arborescência geradora de custo mínimo
	\end{alertblock}
\end{frame}

\begin{frame}{Passo 1: Redução de Custos e Construção de $D_0$}
	\begin{columns}[c]
		\column{0.5\textwidth}
		\textbf{Redução de custos $\lambda$:}
		\begin{itemize}
			\item $\lambda(a) = 1$ (arco $(r,a)$)
			\item $\lambda(b) = 2$ (arco $(r,b)$)
			\item $\lambda(c) = 1$ (arco $(a,c)$)
			\item $\lambda(d) = 1$ (arcos $(b,d)$)
			\item $\lambda(e) = 2$ (arco $(b,e)$)
		\end{itemize}

		\vspace{0.2cm}
		\emph{Escolhemos arcos de custo zero para formar $D_0$}

		\column{0.5\textwidth}
		\begin{tikzpicture}[scale=0.9, >=Stealth,
				vertex/.style={circle,draw,fill=blue!8,thick,inner sep=1.2pt,minimum size=7mm},
				selected/.style={->,very thick,draw=blue},
				unselected/.style={->,semithick,draw=black!35}]

			\node[vertex] (r) at (0,0) {$r$};
			\node[vertex] (a) at (2.5,1.3) {$a$};
			\node[vertex] (b) at (2.5,-1.3) {$b$};
			\node[vertex] (c) at (5,1.8) {$c$};
			\node[vertex] (d) at (5,0.0) {$d$};
			\node[vertex] (e) at (5,-1.8) {$e$};

			\draw[selected] (r) -- node[above,font=\scriptsize,text=blue] {0} (a);
			\draw[selected] (r) -- node[below,font=\scriptsize,text=blue] {0} (b);
			\draw[selected] (a) -- node[above,font=\scriptsize,text=blue] {0} (c);
			\draw[selected] (b) -- node[above,font=\scriptsize,text=blue] {0} (d);
			\draw[selected] (b) -- node[below,font=\scriptsize,text=blue] {0} (e);
			\draw[unselected] (b) to[bend left=12] node[above,font=\scriptsize,text=black] {2} (a);
			\draw[unselected] (a) to[bend left=14] node[below,font=\scriptsize,text=black] {3} (b);
			\draw[unselected] (a) -- node[right,font=\scriptsize,text=black] {2} (d);
			\draw[unselected] (r) .. controls (0.6,2.0) and (4.2,2.5) .. node[yshift=3pt,pos=0.53,font=\scriptsize,text=black] {2} (c);
			\draw[unselected] (c) .. controls (6.3,2.2) and (6.3,-2.2) .. node[xshift=7pt,pos=0.52,font=\scriptsize,text=black] {3} (e);

		\end{tikzpicture}
	\end{columns}

	\vspace{0.3cm}
	\begin{alertblock}{$D_0$ \' e uma $r$-arboresc\^encia!}
		Neste exemplo, $D_0 = \{(r,a), (r,b), (a,c), (b,d), (b,e)\}$ forma uma $r$-arboresc\^encia.

		\emph{Em outras palavras:} todos os v\'ertices s\~ao alcan\c{c}\'aveis a partir de $r$ e cada v\'ertice tem exatamente um arco entrando
	\end{alertblock}
\end{frame}

\begin{frame}{Caso com Ciclo: Outro Exemplo}
	\begin{columns}[c]
		\column{0.5\textwidth}
		\textbf{Quando $D_0$ tem ciclo:}

		\vspace{0.3cm}
		Se ap\'os redu\c{c}\~ao de custos, $D_0$ cont\'em um ciclo $C$, contra\'imos $C$ em superv\'ertice $x_C$

		\vspace{0.3cm}
		\emph{Em outras palavras:} todos os v\'ertices de $C$ viram um \'unico v\'ertice

		\vspace{0.3cm}
		Custos s\~ao ajustados usando os custos $\lambda$-reduzidos

		\column{0.5\textwidth}
		\begin{tikzpicture}[scale=1.2, >=Stealth,
				vertex/.style={circle,draw,fill=blue!8,thick,minimum size=7mm},
				super/.style={circle,draw,very thick,fill=orange!10,minimum size=1.2cm}]

			\node[vertex] (r) at (0,0) {$r$};
			\node[super] (xC) at (3,0) {$x_C$};
			\node[vertex] (v) at (5.5,0) {$v$};

			\draw[->,thick] (r) -- node[above,font=\scriptsize] {custo} (xC);
			\draw[->,thick] (xC) -- node[above,font=\scriptsize] {custo} (v);

		\end{tikzpicture}
	\end{columns}

	\vspace{0.5cm}
	\begin{block}{Chamada Recursiva}
		Resolvemos chu-liu-edmonds$(D/C \mapsto x_C, w_\lambda/C \mapsto x_C, r)$

		\emph{Em outras palavras:} aplicamos o mesmo algoritmo no digrafo menor
	\end{block}
\end{frame}

\begin{frame}{Passo 3: Solu\c{c}\~ao Recursiva (quando h\'a ciclo)}
	\begin{center}
		\begin{tikzpicture}[scale=1.2, >=Stealth,
				vertex/.style={circle,draw,fill=blue!8,thick,minimum size=7mm},
				super/.style={circle,draw,very thick,fill=orange!10,minimum size=1.2cm}]

			\node[vertex] (r) at (0,0) {$r$};
			\node[super] (xC) at (3,0) {$x_C$};
			\node[vertex] (v) at (5.5,0) {$v$};

			\draw[->,very thick,blue] (r) -- node[above,font=\scriptsize,text=blue] {} (xC);
			\draw[->,very thick,blue] (xC) -- node[above,font=\scriptsize,text=blue] {} (v);

		\end{tikzpicture}
	\end{center}

	O algoritmo devolve $T'$ (arboresc\^encia no digrafo contra\'ido)

	\vspace{0.3cm}
	\emph{Em outras palavras:} encontramos a $r$-arboresc\^encia \'otima no grafo contra\'ido

	\vspace{0.3cm}
	\textbf{Pr\'oximo passo:} expandir $T'$ de volta para obter $T$ no digrafo original
\end{frame}

\begin{frame}{Solu\c{c}\~ao Final}
	\begin{columns}[c]
		\column{0.5\textwidth}
		\textbf{Para nosso exemplo:}

		Como $D_0$ j\'a \' e uma $r$-arboresc\^encia, o algoritmo termina diretamente!

		\vspace{0.3cm}
		\emph{Em outras palavras:}
		\begin{itemize}
			\item N\~ao h\'a ciclo
			\item Todos v\'ertices alcan\c{c}\'aveis de $r$
			\item Cada v\'ertice tem grau de entrada 1
			\item Solu\c{c}\~ao \'otima encontrada!
		\end{itemize}

		\column{0.5\textwidth}
		\begin{tikzpicture}[scale=0.9, >=Stealth,
				vertex/.style={circle,draw,fill=blue!8,thick,inner sep=1.2pt,minimum size=7mm},
				selected/.style={->,very thick,draw=blue}]

			\node[vertex] (r) at (0,0) {$r$};
			\node[vertex] (a) at (2.5,1.3) {$a$};
			\node[vertex] (b) at (2.5,-1.3) {$b$};
			\node[vertex] (c) at (5,1.8) {$c$};
			\node[vertex] (d) at (5,0.0) {$d$};
			\node[vertex] (e) at (5,-1.8) {$e$};

			\draw[selected] (r) -- node[above,font=\scriptsize,text=blue] {1} (a);
			\draw[selected] (r) -- node[below,font=\scriptsize,text=blue] {2} (b);
			\draw[selected] (a) -- node[above,font=\scriptsize,text=blue] {1} (c);
			\draw[selected] (b) -- node[above,font=\scriptsize,text=blue] {1} (d);
			\draw[selected] (b) -- node[below,font=\scriptsize,text=blue] {2} (e);

		\end{tikzpicture}
	\end{columns}

	\vspace{0.3cm}
	\begin{exampleblock}{Solu\c{c}\~ao \'Otima}
		Custo total: $1 + 2 + 1 + 1 + 2 = 7$
	\end{exampleblock}
\end{frame}

% ==================== SEÇÃO 3: ANDRÁS FRANK ====================
\section{Algoritmo de András Frank}

\begin{frame}{András Frank: Visão Geral}
	\begin{block}{Abordagem em Duas Fases}
		\textbf{Fase I:} Construir cobertura de subconjuntos minimais via redução de custos

		\textbf{Fase II:} Extrair arborescência da cobertura
	\end{block}

	\vspace{0.5cm}

	\begin{columns}[c]
		\column{0.5\textwidth}
		\textbf{Diferencial:}
		\begin{itemize}
			\item Trabalha com múltiplos vértices simultaneamente
			\item Usa componentes fortemente conexas
			\item Redução sistemática de custos
		\end{itemize}

		\column{0.5\textwidth}
		\textbf{Complexidade:}
		\begin{itemize}
			\item Fase I: $O(nm)$
			\item Fase II v1 (lista): $O(n^2)$
			\item Fase II v2 (heap): $O(n \log n)$
		\end{itemize}
	\end{columns}
\end{frame}

\begin{frame}{Fase I: Redução de Custos}
	\textbf{Para cada vértice} $v \neq r$: subtrair o mínimo de entrada

	\begin{center}
		\begin{tikzpicture}[scale=1.1,
				vertex/.style={circle,draw,minimum size=0.7cm}]

			% Antes
			\begin{scope}
				\node[vertex,fill=red!20] (r1) at (0,2) {$r$};
				\node[vertex] (A1) at (2,3) {$A$};
				\node[vertex] (B1) at (4,3) {$B$};
				\node[vertex] (C1) at (2,1) {$C$};

				\draw[->,>=stealth] (r1) -- node[above] {2} (A1);
				\draw[->,>=stealth] (r1) to[bend left] node[above] {10} (B1);
				\draw[->,>=stealth] (B1) -- node[right] {1} (A1);
				\draw[->,>=stealth] (A1) -- node[left] {4} (C1);

				\node at (2,0) {\textbf{Original}};
			\end{scope}

			% Depois
			\begin{scope}[xshift=6cm]
				\node[vertex,fill=red!20] (r2) at (0,2) {$r$};
				\node[vertex] (A2) at (2,3) {$A$};
				\node[vertex] (B2) at (4,3) {$B$};
				\node[vertex] (C2) at (2,1) {$C$};

				\draw[->,>=stealth,blue] (r2) -- node[above] {1} (A2);
				\draw[->,>=stealth] (r2) to[bend left] node[above] {10} (B2);
				\draw[->,>=stealth,blue,very thick] (B2) -- node[right] {\textbf{0}} (A2);
				\draw[->,>=stealth,blue,very thick] (A2) -- node[left] {\textbf{0}} (C2);

				\node at (2,0) {\textbf{Após Redução}};
			\end{scope}

		\end{tikzpicture}
	\end{center}

	Arcos com custo \textbf{zero} formam o digrafo $D_0$
\end{frame}

\begin{frame}{Fase I: Componentes Fortemente Conexas}
	\begin{columns}[c]
		\column{0.5\textwidth}
		Identificar componentes fortemente conexas (CFCs) em $D_0$

		\vspace{0.3cm}
		Cada CFC forma um \textbf{subconjunto minimal}

		\vspace{0.3cm}
		Construir sequência laminar de subconjuntos

		\column{0.5\textwidth}
		\begin{tikzpicture}[scale=0.9]
			\node[draw,circle,fill=yellow!20,minimum size=1.2cm] (S1) at (0,2) {$S_1$};
			\node[draw,circle,fill=yellow!20,minimum size=1.2cm] (S2) at (2,2) {$S_2$};
			\node[draw,circle,fill=yellow!20,minimum size=1.2cm] (S3) at (1,0.5) {$S_3$};

			\draw[->,>=stealth,thick] (S1) to[bend left] (S2);
			\draw[->,>=stealth,thick] (S2) to[bend left] (S1);
			\draw[->,>=stealth,thick] (S2) -- (S3);
		\end{tikzpicture}
	\end{columns}

	\vspace{0.5cm}
	\begin{block}{Condição de Otimalidade}
		Sequência $\lambda$ satisfaz: $|\delta^-(X)| = 1$ para cada $X$ em $\lambda$
	\end{block}
\end{frame}

\begin{frame}{Fase II: Construção da Arborescência}
	\textbf{Objetivo:} Extrair arborescência de $D_0$ respeitando $\lambda$

	\vspace{0.3cm}

	\begin{enumerate}
		\item Iniciar com conjunto $R = \{r\}$
		\item Para cada $v$ fora de $R$:
		      \begin{itemize}
			      \item Selecionar arco $(u,v)$ com $u \in R$ e custo reduzido zero
			      \item Adicionar $v$ a $R$
		      \end{itemize}
		\item Repetir até incluir todos os vértices
	\end{enumerate}

	\vspace{0.5cm}

	\begin{exampleblock}{Resultado}
		Arborescência ótima com mesma solução: custo 14
	\end{exampleblock}
\end{frame}

% ==================== SEÇÃO 4: RESULTADOS ====================
\section{Resultados Experimentais}

\begin{frame}{Comparação de Desempenho}
	\textbf{Experimentos:} 2000 digrafos aleatórios, $|V| \in [101, 4996]$

	\vspace{0.3cm}

	\begin{center}
		\begin{tabular}{lcc}
			\toprule
			\textbf{Algoritmo}    & \textbf{Tempo Mediano} & \textbf{Tempo Médio} \\
			\midrule
			Chu-Liu-Edmonds       & 0,25 s                 & 0,58 s               \\
			Frank Fase I          & 8,93 s                 & 12,40 s              \\
			Frank Fase II (lista) & 0,98 s                 & 1,34 s               \\
			Frank Fase II (heap)  & \textbf{0,016 s}       & \textbf{0,020 s}     \\
			\bottomrule
		\end{tabular}
	\end{center}

	\vspace{0.5cm}

	\begin{block}{Speedup Fase II}
		Heap vs Lista: aceleração de \textbf{58,12 vezes} (mediana)
	\end{block}
\end{frame}

\begin{frame}{Características Estruturais}
	\begin{columns}[c]
		\column{0.5\textwidth}
		\textbf{Contrações (Chu-Liu):}
		\begin{itemize}
			\item Mediana: 2 contrações
			\item Média: 6,82
			\item Máximo: 406
			\item 93,8\% com $< 20$
		\end{itemize}

		\vspace{0.3cm}
		Muito abaixo do limite teórico $O(n)$

		\column{0.5\textwidth}
		\begin{center}
			\includegraphics[width=\textwidth]{figures/fig_contractions_depth.png}
		\end{center}
	\end{columns}

	\vspace{0.3cm}
	\textbf{Consumo de memória:} mediana 11,5 MB (Fase I)
\end{frame}

% ==================== SEÇÃO 5: APLICAÇÃO WEB ====================
\section{Aplicação Web}

\begin{frame}{Motivação Didática}
	\begin{block}{Desafio}
		Algoritmos de grafos são \textbf{abstratos} e \textbf{difíceis de visualizar}
	\end{block}

	\vspace{0.5cm}

	\begin{columns}[c]
		\column{0.5\textwidth}
		\textbf{Solução Proposta:}
		\begin{itemize}
			\item Visualização interativa
			\item Execução passo a passo
			\item Feedback imediato
			\item Acessível via navegador
		\end{itemize}

		\column{0.5\textwidth}
		\textbf{Tecnologias:}
		\begin{itemize}
			\item PyScript (Python no browser)
			\item JavaScript
			\item HTML5/CSS3
			\item NetworkX
		\end{itemize}
	\end{columns}
\end{frame}

\begin{frame}{Arquitetura da Aplicação}
	\begin{center}
		\begin{tikzpicture}[
				node distance=1.5cm,
				box/.style={rectangle,draw,thick,minimum width=3cm,minimum height=1cm,align=center}]

			\node[box,fill=blue!20] (ui) {Interface HTML/CSS};
			\node[box,fill=green!20,below of=ui] (pyscript) {PyScript\\(Python)};
			\node[box,fill=yellow!20,below of=pyscript] (algos) {Algoritmos\\Chu-Liu | Frank};
			\node[box,fill=orange!20,below of=algos] (nx) {NetworkX};

			\draw[->,>=stealth,thick] (ui) -- (pyscript);
			\draw[->,>=stealth,thick] (pyscript) -- (algos);
			\draw[->,>=stealth,thick] (algos) -- (nx);
			\draw[->,>=stealth,thick] (nx) to[bend right=60] (ui);

		\end{tikzpicture}
	\end{center}
\end{frame}

\begin{frame}{Interface: Página Principal}
	\begin{center}
		\includegraphics[width=0.8\textwidth]{../assets/homehtml.png}
	\end{center}

	\begin{itemize}
		\item Navegação intuitiva entre algoritmos
		\item Acesso direto aos exemplos interativos
	\end{itemize}
\end{frame}

\begin{frame}{Interface: Desenho de Grafos}
	\begin{center}
		\includegraphics[width=0.75\textwidth]{../assets/drawhtml.png}
	\end{center}

	Funcionalidades:
	\begin{itemize}
		\item Adicionar vértices e arestas
		\item Definir pesos
		\item Exportar/importar grafos em JSON
	\end{itemize}
\end{frame}

\begin{frame}{Interface: Chu-Liu-Edmonds}
	\begin{center}
		\includegraphics[width=0.75\textwidth]{../assets/chuliuhtml.png}
	\end{center}

	\begin{itemize}
		\item Visualização passo a passo
		\item Destacamento de ciclos detectados
		\item Log detalhado das operações
	\end{itemize}
\end{frame}

\begin{frame}{Interface: András Frank}
	\begin{center}
		\includegraphics[width=0.75\textwidth]{../assets/andrasfrankv1html.png}
	\end{center}

	\begin{itemize}
		\item Exibição das duas fases
		\item Visualização de CFCs
		\item Comparação entre versões (lista vs heap)
	\end{itemize}
\end{frame}

\begin{frame}{Princípios de Design}
	\begin{block}{Teoria dos Registros de Representação (Duval)}
		Transitar entre diferentes representações:
		\begin{itemize}
			\item \textbf{Visual:} diagramas do grafo
			\item \textbf{Simbólico:} código Python
			\item \textbf{Textual:} log das operações
		\end{itemize}
	\end{block}

	\vspace{0.5cm}

	\begin{alertblock}{Feedback Imediato}
		Validação em tempo real das operações do usuário
	\end{alertblock}
\end{frame}

% ==================== SEÇÃO 6: CONCLUSÕES ====================
\section{Conclusões}

\begin{frame}{Contribuições do Trabalho}
	\begin{enumerate}
		\item \textbf{Implementação completa} de dois algoritmos clássicos
		      \begin{itemize}
			      \item Chu-Liu-Edmonds: recursivo com contração
			      \item András Frank: duas fases com otimização heap
		      \end{itemize}

		      \vspace{0.3cm}

		\item \textbf{Análise experimental} detalhada
		      \begin{itemize}
			      \item 2000 instâncias aleatórias
			      \item Comparação de desempenho e características estruturais
		      \end{itemize}

		      \vspace{0.3cm}

		\item \textbf{Aplicação web interativa}
		      \begin{itemize}
			      \item Ferramenta didática para visualização
			      \item Execução passo a passo dos algoritmos
			      \item Design centrado no usuário
		      \end{itemize}
	\end{enumerate}
\end{frame}

\begin{frame}{Principais Resultados}
	\begin{itemize}
		\item \textbf{Corretude validada:} custos idênticos em todas as instâncias

		      \vspace{0.3cm}

		\item \textbf{Chu-Liu-Edmonds} mais rápido para construção direta
		      \begin{itemize}
			      \item Mediana: 0,25 s vs 8,93 s (Fase I Frank)
		      \end{itemize}

		      \vspace{0.3cm}

		\item \textbf{Otimização heap} fundamental na Fase II
		      \begin{itemize}
			      \item Speedup: 58× (mediana), 61× (média)
		      \end{itemize}

		      \vspace{0.3cm}

		\item \textbf{Comportamento prático} muito melhor que limites teóricos
		      \begin{itemize}
			      \item Contrações: mediana 2 (limite $O(n)$)
			      \item Memória modesta: 11,5 MB
		      \end{itemize}
	\end{itemize}
\end{frame}

\begin{frame}{Trabalhos Futuros}
	\begin{block}{Extensões Possíveis}
		\begin{itemize}
			\item Implementar outras variantes (Tarjan, Gabow)
			\item Análise em grafos com estruturas especiais
			\item Paralelização dos algoritmos
			\item Extensão para grafos dinâmicos
		\end{itemize}
	\end{block}

	\vspace{0.5cm}

	\begin{block}{Melhorias na Aplicação}
		\begin{itemize}
			\item Modo de edição visual de grafos
			\item Geração automática de casos de teste
			\item Exercícios interativos com correção automática
			\item Integração com plataformas de ensino (Moodle, Jupyter)
		\end{itemize}
	\end{block}
\end{frame}

% ==================== SLIDE FINAL ====================
\begin{frame}[plain]
	\centering
	\vfill

	{\Huge Obrigado!}

	\vspace{1cm}

	{\Large Perguntas?}

	\vspace{1cm}

	{\normalsize \texttt{https://github.com/lorenypsum/graph-visualizer}}

	\vfill
\end{frame}

\end{document}
