% Resumo em português
\setlength{\absparsep}{18pt}
\begin{resumo}
	Este trabalho aborda o problema clássico da \textit{r-arborescência geradora mínima} em digrafos, investigando duas abordagens algorítmicas fundamentais: o algoritmo de Chu--Liu--Edmonds e o procedimento em duas fases de András Frank. Implementamos ambos os métodos em Python e realizamos uma análise experimental comparativa. Adicionalmente, desenvolvemos uma aplicação web interativa baseada que permite visualização passo a passo dos algoritmos, explorando princípios de interação humano-computador. A ferramenta integra edição de grafos, execução algorítmica e feedback imediato, constituindo um recurso didático acessível via navegador.

	\textbf{Palavras-chave}: Digrafos, r-Arborescência Geradora, Chu--Liu--Edmonds, András Frank, Visualização de Algoritmos, Aplicação Web Didática.
\end{resumo}

% Resumo em inglês
\begin{resumo}[Abstract]
	\begin{otherlanguage*}{english}
		This work addresses the classic problem of finding a \textit{minimum spanning r-arborescence} in digraphs, investigating two fundamental algorithmic approaches: the Chu--Liu--Edmonds algorithm and András Frank's two-phase procedure. We implemented both methods in Python and conducted a comparative experimental analysis. Additionally, we developed an interactive web application that enables step-by-step visualization of the algorithms, exploring human-computer interaction principles. The tool integrates graph editing, algorithm execution, and immediate feedback, constituting an accessible educational resource via web browser.

		\textbf{Keywords}: Digraphs, Spanning r-Arborescence, Chu--Liu--Edmonds, András Frank, Algorithm Visualization, Educational Web Application.
	\end{otherlanguage*}
\end{resumo}
