\chapter{A interação humano--computacional em ação: uma aplicação \textit{web} interativa}


Discutimos até aqui fundamentos teóricos dos algoritmos, análises de complexidade, resultados empíricos e princípios pedagógicos que justificam o uso de ferramentas interativas. Estabelecemos \emph{o quê} ensinar (Chu–Liu/Edmonds e Frank), \emph{por quê} usar visualizações (redução de carga cognitiva, engajamento ativo) e \emph{como} estruturar o design (heurísticas de IHC, visão geral com detalhe sob demanda, feedback imediato).


Agora, traduzimos esses princípios em código e interface: a aplicação \textit{web} que desenvolvemos busca materializar cada uma das diretrizes discutidas. Isto é, cada botão, cada visualização e cada mensagem de log foi projetada intencionalmente, guiada pelos princípios de IHC e pelas necessidades pedagógicas identificadas, a fim de maximizar a compreensão e o engajamento dos usuários


A aplicação \textit{web} interativa foi desenvolvida para ilustrar os algoritmos de Chu–Liu/Edmonds e Frank, permitindo aos usuários acompanhar passo a passo o funcionamento de ambos os métodos. 

\section{Princípios de interação humano-computador}


A interação humano-computador (IHC) estuda como projetar sistemas computacionais que sejam eficientes, eficazes e agradáveis para os usuários.


Sintetizando heurísticas de usabilidade e descoberta \cite{nielsen1994heuristics,shneiderman2016designing} e quadros de interação e aprendizagem \cite{rogers2011interaction,mayer2009multimedia,sweller1988cognitive,naps2003engagement}, destacamos oito princípios orientadores: (i) usabilidade, (ii) eficiência cognitiva, (iii) feedback imediato, (iv) engajamento ativo, (v) visão geral com detalhe sob demanda (mantra \emph{overview→filter→details} de \cite{shneiderman1996eyes}), (vi) consistência semântica, (vii) múltiplos registros de representação e (viii) prevenção/recuperação de erros. A seguir descrevemos cada um e sua materialização na ferramenta.


% \subsection{Usabilidade}

Começamos pela usabilidade, que refere-se à facilidade com que os usuários podem aprender a usar um sistema, realizar tarefas e alcançar seus objetivos. Na nossa aplicação, priorizamos uma interface limpa e intuitiva, com controles claros para navegar pelos passos dos algoritmos, selecionar arestas, visualizar cortes ativos e entender a evolução dos custos reduzidos.

% \subsection{Eficiência cognitiva}

Essa clareza se conecta diretamente à eficiência cognitiva, que envolve minimizar a carga cognitiva dos usuários, facilitando a compreensão e o processamento de informações. Implementamos visualizações que destacam mudanças importantes (como contrações e expansões) e fornecem explicações textuais concisas para cada passo, ajudando os usuários a conectar ações com conceitos teóricos.

% \subsection{Feedback Imediato}

A medida que o usuário interage com o sistema, o feedback imediato garante que os usuários informados sobre o estado do sistema e as consequências de suas ações. Nossa ferramenta oferece feedback visual e textual em tempo real, mostrando como cada ação afeta o grafo e os custos associados, reforçando a compreensão causal.


% \subsection{Engajamento ativo}

Esses elementos favorecem o engajamento ativo, que refere-se à participação dos usuários no processo de aprendizagem, incentivando-os a explorar, experimentar e interagir com o sistema. Nossa aplicação promove o engajamento ativo ao permitir que os usuários manipulem o grafo, testem diferentes abordagens e visualizem os resultados de suas ações em tempo real, além de acessar descrições passo a passo da execução dos algoritmos.


% \subsection{Visão geral com detalhe sob demanda}

Para orientar essa exploração sem sobrecarregar o usuário, aplicamos o princípio de visão geral com detalhe sob demanda que permite que os usuários obtenham uma compreensão ampla do sistema, enquanto ainda têm acesso a informações detalhadas quando necessário. Implementamos essa abordagem ao fornecer uma visualização geral do grafo, com a opção de expandir informações sobre arestas e nós específicos conforme o interesse do usuário.

% \subsection{Consistência semântica}

A navegação entre essas diferentes camadas da interface é apoiada pela consistência semântica, que garante que os elementos da interface e suas interações sejam compreensíveis e previsíveis. Nossa ferramenta mantém a consistência semântica ao usar terminologia e representações visuais padronizadas em toda a aplicação, facilitando a compreensão dos usuários.

% \subsection{Múltiplos registros de representação}

Além disso, adotamos o conceito de múltiplos registros de representação, que referem-se à capacidade de apresentar informações de diferentes maneiras, atendendo às preferências e estilos de aprendizagem dos usuários. Na nossa aplicação oferecemos várias representações do grafo (visual, textual, interativa), permitindo que os usuários escolham a forma que melhor se adapta às suas necessidades.

% \subsection{Prevenção de erros}

Por fim, implementamos conceitos de prevenção de erros, que envolve projetar o sistema de forma a minimizar a probabilidade de erros dos usuários. Em nosso sistema garantimos feedback em tempo real, para ajudar os usuários a evitar ações indesejadas e compreender melhor as consequências de suas escolhas.


Esses princípios de interação humano-computador foram fundamentais para o desenvolvimento da nossa aplicação \textit{web} interativa, garantindo que ela seja não apenas funcional, mas também acessível e eficaz como ferramenta de aprendizagem. A seguir, detalhamos a implementação técnica da aplicação e como ela se posiciona no ecossistema de ferramentas didáticas para o ensino de grafos.

\begin{table}[H]
	\centering
	\renewcommand{\arraystretch}{1.18}
	\setlength{\tabcolsep}{6pt}
	\footnotesize
	% Ajuste de larguras: 1a coluna ampliada para evitar overlap; leve redução na 3a.
	\begin{tabular}{p{2.9cm} p{4.0cm} p{7.0cm}}
		\hline
		\textbf{Princípio}                               & \textbf{Exemplo Geral}                         & \textbf{Materialização na Aplicação}                                                                                                                                            \\
		\hline
		Usabilidade                                      & Botões claros para avançar/voltar etapas       & Barra de controles com rótulos diretos (Adicionar Aresta, Executar, Reset); agrupamento visual consistente via Tailwind; nenhum menu profundo aninhado.                         \\
		Eficiência\newline cognitiva                     & Reduzir elementos irrelevantes no estado atual & Layout estável entre passos; apenas arestas relevantes destacadas; eliminação de ornamentação visual; custos e rótulos legíveis sem rotação.                                    \\
		Feedback\newline imediato                        & Mostrar efeito de uma ação logo após o clique  & Cada ação dispara: (i) atualização do desenho do grafo, (ii) entrada no log textual explicando a mudança (ex.: contração, seleção de aresta).                                   \\
		Engajamento\newline ativo                        & Usuário prediz antes de revelar próximo passo  & Controles passo a passo permitem explorar sequencialmente; usuário insere/edita pesos e escolhe raiz antes de rodar o algoritmo.                                                \\
		Visão geral→\newline Detalhes                    & Visão global com acesso a informação pontual   & Visão completa do grafo em todos os passos + possibilidade de inspecionar pesos e arestas específicas no log sequencial; estados anteriores preservados para comparação mental. \\
		Consistência\newline semântica                   & Mesmo conceito, mesma cor/forma                & Raiz destacada de forma fixa; arestas selecionadas mantêm estilo; semântica cromática não muda entre passos (evita remapeamento mental).                                        \\
		Múltiplos\newline registros                      & Texto + grafo + (futuro) estrutura derivada    & Combinação de: descrição textual no log, representação visual do grafo, parâmetros simbólicos (pesos); prepara expansão futura para mostrar custos reduzidos.                   \\
		Prevenção /\newline recuperação\newline de erros & Impedir entrada inválida / ação reversível     & Validação de pesos (numéricos); bloqueio de execução sem raiz definida; botão Reset para recompor estado limpo sem recarregar página.                                           \\
		\hline
	\end{tabular}
	\caption{Síntese dos princípios de interação humano-computador aplicados e sua realização concreta na ferramenta interativa.}
	\label{tab:principios-ihc}
\end{table}

\section{Descrição da aplicação}

A aplicação \textit{web} interativa foi desenvolvida para ilustrar os algoritmos de Chu–Liu/Edmonds e Frank, permitindo aos usuários acompanhar passo a passo o funcionamento de ambos os métodos. A ferramenta foi projetada com base em princípios de interação humano-computador, visando maximizar a compreensão e o engajamento dos usuários.

\subsection{Visão geral das páginas}
A  aplicação \textit{web} é organizada em páginas HTML independentes, cada uma dedicada a uma parte específica da experiência do usuário. A página \texttt{home.html} apresenta um panorama geral da ferramenta; \texttt{chuliu.html} executa passo a passo o algoritmo de Chu-Liu e Edmonds; \texttt{andrasfrank\_v1.html} e \texttt{andrasfrank\_v2.html} oferecem a estrutura inicial para a futura visualização da abordagem primal–-dual em suas duas versões; e \texttt{draw\_graph.html} funciona como um editor livre para criação manual de grafos. Todas as páginas carregam dinamicamente o componente compartilhado de navegação lateral (\texttt{sidebar.html}) e importam apenas os scripts necessários ao seu funcionamento, garantindo modularidade e reduzindo sobrecarga desnecessária na interface.

O desenvolvimento adotou tecnologias \textit{web} modernas, combinando HTML5 e Tailwind CSS para composição responsiva, PyScript para execução de código Python diretamente no navegador e bibliotecas científicas amplamente utilizadas em grafos e visualização. NetworkX foi empregado para a manipulação de grafos dirigidos e Matplotlib para gerar snapshots estáticos dos estados intermediários dos algoritmos. Em páginas avançadas, como \texttt{chuliu.html}, \texttt{andrasfrank\_v1.html} e \texttt{andrasfrank\_v2.html}, a infraestrutura também já prevê o uso futuro de Cytoscape.js para interações mais ricas.

Para permitir exportação e reutilização do estado da aplicação, empregamos serialização JSON no formato \texttt{node\_link}. Além disso, cada página carrega exclusivamente seus componentes e dependências, evitando \emph{payload} excessivo e reduzindo a latência percebida pelo usuário — uma decisão alinhada ao objetivo de manter a eficiência cognitiva durante a navegação e exploração dos algoritmos.


\subsection{Fluxo de interação}

O fluxo de interação foi projetado para ser linear e intuitivo, guiando o usuário desde a criação do grafo até a visualização dos resultados do algoritmo. O fluxo típico é o seguinte: primeiro o usuário monta ou carrega um grafo de teste; em seguida, define (ou confirma) o vértice raiz \(r_0\); depois, executa o algoritmo, aplicando normalizações e seleção de arestas conforme implementado; então, observa os estados sequenciais gerados, onde cada snapshot reforça invariantes como arestas escolhidas, pesos e estrutura alcançada; por fim, o usuário pode optar por exportar o grafo resultante para replicação em notebooks ou comparação com a abordagem dual futura.

Além disso, o log textual funciona como uma \emph{trilha de auditoria didática}. Cada ação do usuário (adição de aresta, definição de raiz, execução de passo) atualiza o grafo e o log, permitindo rastrear a evolução do estado. A exportação em JSON facilita a reimportação e análise posterior.

\subsection{Limitações atuais}

Atualmente, a aplicação apresenta algumas limitações que podem impactar a experiência do usuário e a eficácia da visualização:
Primeiramente, a ausência de visualização explícita de contração de ciclos dificulta a compreensão completa do algoritmo, uma vez que a marcação diferenciada por cores ou aglomerados ainda não foi implementada. Além disso, a falta de comparação lado a lado (split view) entre os algoritmos de Chu–Liu e Frank limita a capacidade dos usuários de entenderem as diferenças e semelhanças entre as abordagens. O layout planar simples pode falhar em instâncias mais densas, resultando em sobreposição de rótulos, o que compromete a clareza visual (uma futura substituição por layouts adaptativos como \texttt{spring} ou \texttt{dagre}-like é recomendada). Outro ponto é a ausência de uma camada de destaque cromático para custos reduzidos e arcos “apertados” \((c' = 0)\), o que poderia facilitar a identificação de elementos críticos no grafo. Por fim, a exportação limita-se ao grafo final e ao grado original, estados intermediários não possuem tal feature implementada, o que restringe a capacidade de revisão e análise detalhada do processo algorítmico.

\subsection{Melhorias futuras}

Desse modo, entendemos que a aplicação, embora funcional, pode ser aprimorada com recursos adicionais para enriquecer a experiência didática. Entre as melhorias previstas, destacamos a inclusão de visualização animada da contração/reexpansão de ciclos com agrupamento colapsáve, geração automática de relatório (log + estados selecionados) em PDF/ZIP, criação de módulo paralelo para a abordagem primal--dual de Frank (empacotamento de cortes e duas fases), criação de “comparativo” exibindo diferenças de passos e métricas agregadas além de implementação de monitoramento de métricas de desempenho (tempo por passo, número de contrações, distribuição de pesos normalizados).

De modo geral, a aplicação serve como um protótipo funcional que demonstra o potencial de ferramentas interativas para o ensino de algoritmos complexos em teoria dos grafos. Com melhorias contínuas, pode se tornar uma plataforma robusta para aprendizagem ativa e visualização didática. Na seção seguinte, detalhamos aspectos técnicos da implementação.

\section{Detalhes de Implementação}

A aplicação foi implementada utilizando tecnologias \textit{web} modernas, com foco em simplicidade, modularidade e reprodutibilidade. A seguir, detalhamos os principais aspectos técnicos da implementação.

\subsection{Estrutura de arquivos}

A estrutura de arquivos da aplicação é organizada em diretórios e componentes bem definidos. O diretório \texttt{scripts/} reúne os scripts Python e JavaScript responsáveis pela lógica da aplicação. O diretório \texttt{assets/} armazena imagens, ícones e demais recursos estáticos utilizados pela interface. As páginas HTML encontram-se no diretório \texttt{pages/}, estruturadas de forma modular para facilitar manutenção e extensão futura. Por fim, o arquivo \texttt{pyscript.json} contém as configurações necessárias ao funcionamento do PyScript no ambiente da aplicação.

\subsection{Páginas da Aplicação \textit{web}}

A seguir, apresentamos os códigos desenvolvidos para os algoritmos implementados realizarem a interação com os usuários.


\subsection*{\texttt{Index.html}:} o arquivo \texttt{index.html} define a estrutura principal da página HTML, incluindo a integração com PyScript e Tailwind CSS para estilos responsivos.

\subsection*{\texttt{Home.html}:} o arquivo \texttt{home.html} serve como a página inicial da aplicação, oferecendo uma visão geral do projeto, incluindo um resumo do trabalho e informações sobre os integrantes. A estrutura da página é projetada para ser acolhedora e informativa, utilizando Tailwind CSS para garantir uma aparência moderna e responsiva. Abaixo, apresentamos um exemplo de captura de tela da página.

% Screenshot da página Home
\begin{figure}[H]\centering
	\includegraphics[width=0.95\textwidth]{../assets/homehtml.png}
	\caption{Captura de tela de \texttt{home.html}: visão geral com resumo e integrantes.}
	\label{fig:home_html_screenshot}
\end{figure}


\subsection*{\texttt{Draw\_graph.html}:} Editor de grafos livre com funcionalidades de criação, edição, importação e exportação. Utiliza \texttt{Cytoscape.js} para visualização interativa e \texttt{PyScript} para lógica algorítmica. Abaixo, apresentamos uma captura de tela da página.

% Screenshot da página Draw Graph
\begin{figure}[H]\centering
	\includegraphics[width=0.95\textwidth]{../assets/drawhtml.png}
	\caption{Captura de tela de \texttt{draw\_graph.html}: editor livre de grafos.}
	\label{fig:draw_html_screenshot}
\end{figure}

\subsection*{\texttt{Sidebar.html}:} componente de navegação lateral consistente em todas as páginas. Utiliza Tailwind CSS para estilo e inclui links para as principais seções do site, reforçando um modelo mental estável de navegação. 

\subsection*{\texttt{Chuliu.html}:} página dedicada ao visualizador do algoritmo de Chu-Liu/Edmonds. Inclui um passo a passo guiado para criar um grafo, selecionar o nó raiz e executar o algoritmo, com feedback visual e textual. Abaixo, apresentamos uma captura de tela da página.

% Screenshot da página Chu-Liu/Edmonds
\begin{figure}[H]\centering
	\includegraphics[width=0.95\textwidth]{../assets/chuliuhtml.png}
	\caption{Captura de tela de \texttt{chuliu.html}: criação de grafo, seleção de raiz e execução do algoritmo.}
	\label{fig:chuliu_html_screenshot}
\end{figure}

A figura a seguir destaca a tripartição funcional da página: navegação lateral, conteúdo interativo central e guia de passos à direita.
\begin{figure}[H]\centering
	\begin{tikzpicture}[node distance=1.4cm]
		\node[draw,rounded corners,fill=gray!10,minimum width=1.6cm] (sb4) {Barra Lateral};
		\node[draw,rounded corners,fill=blue!8,right=0.6cm of sb4,minimum width=5.2cm,minimum height=3.4cm] (maincl) {Principal (Botão de Passos + Editor + Logs)};
		\node[draw,rounded corners,fill=green!10,right=1.6cm of maincl,minimum width=2cm,minimum height=3.4cm] (rs) {Passos};
		\draw[->] (maincl.east) -- (rs.west);
	\end{tikzpicture}
	\caption{\texttt{chuliu.html} - tripartição funcional (navegação, conteúdo interativo, guia de passos). A presença do passo a passo auxilia na compreensão sequencial do algoritmo.}
\end{figure}

% TODO - REMOVER ESSA PARTE E COLOCAR O ANDRAS FRANK V1 E V2
\subsection{\texttt{Andrasfrank\_v1.html} e \texttt{Andrasfrank\_v2.html}:} 
Ambas as páginas são dedicadas ao visualizador do algoritmo de Andras Frank (em suas diferentes implementações elucidadas em capítulos anteriores). Inclui um passo a passo guiado para criar um grafo, selecionar o vértice raiz e executar o algoritmo, com feedback visual e textual. Abaixo, apresentamos uma captura de tela da página.

% Screenshot da página Andras Frank
\begin{figure}[H]\centering
	\includegraphics[width=0.95\textwidth]{../assets/andrasfrankv1html.png}
	\caption{Captura de tela de \texttt{andrasfrank\_v1.html}: interface para o procedimento em duas fases, a tela da página \texttt{andrasfrank\_v2.html} tem aparência similar.}
	\label{fig:andrasfrank_html_screenshot}
\end{figure}

A figura a seguir ilustra a reutilização do padrão de tripartição funcional para manter consistência cognitiva entre páginas.

\begin{figure}[H]\centering
	\begin{tikzpicture}[node distance=1.4cm]
		\node[draw,rounded corners,fill=gray!10,minimum width=1.6cm] (sb5) {Barra Lateral};
		\node[draw,rounded corners,fill=blue!8,right=0.6cm of sb5,minimum width=5.2cm,minimum height=3.4cm] (mainaf) {Principal (Passos + Editor + Logs)};
		\node[draw,rounded corners,fill=green!10,right=2.6cm of mainaf,minimum width=2cm,minimum height=3.4cm] (rsa) {Passos};
		\draw[->] (mainaf.east) -- (rsa.west);
	\end{tikzpicture}
	\caption{\texttt{andrasfrank.html} - reutilização de padrão para consistência cognitiva.}
\end{figure}

O ecossistema de páginas cria uma narrativa pedagógica: contextualização (\texttt{home}) \,$\rightarrow$\, experimentação livre (\texttt{draw\_graph}) \,$\rightarrow$\, exploração guiada (\texttt{chuliu}, \texttt{andrasfrank (v1)} e \texttt{andrasfrank (v2)}) \,$\rightarrow$\, consolidação formal (\texttt{tese}). Tal sequência ainda reduz a carga intrínseca inicial.

A figura a seguir sumariza o fluxo e a reutilização arquitetural entre páginas.

\subsection{Síntese arquitetural das páginas}
\begin{figure}[H]\centering
	\begin{tikzpicture}[node distance=2.2cm,>=Stealth]
		\node[draw,rounded corners,fill=purple!10] (home) {home};
		\node[draw,rounded corners,fill=purple!10,right=of home] (draw) {draw\_graph};
		\node[draw,rounded corners,fill=purple!10,right=of draw] (ch) {chuliu};
		\node[draw,rounded corners,fill=purple!10,right=of ch] (afp) {andrasfrank (v1 e v2)};
		\node[draw,rounded corners,fill=orange!20,above=of ch] (sb) {barra lateral};
		\node[draw,rounded corners,fill=purple!20,above=of sb] (ts) {tese};
		\draw[->] (home) -- (draw);
		\draw[->] (draw) -- (ch);
		\draw[->] (ch) -- (afp);
		\draw[->] (sb) -- (ts);
		\foreach \x in {home,draw,ch,afp,ts} {\draw[->] (sb) -- (\x);}
	\end{tikzpicture}
	\caption{A \texttt{barra lateral} injeta navegação consistente; páginas de algoritmo formam trilha exploratória.}
\end{figure}


A arquitetura modular e reutilizável das páginas \textit{web} facilita manutenção, extensão e consistência. A barra lateral comum reduz esforço cognitivo ao navegar, enquanto o padrão tripartido de conteúdo interativo reforça familiaridade. A sequência lógica de páginas guia o usuário do contexto à experimentação e formalização, alinhando-se a princípios pedagógicos. Essa estrutura coesa apoia o aprendizado eficaz des conceitos de arborescências dirigidas e algoritmos associados.

% \section{Avaliação da Interface}
% \subsection{wip}
% \textcolor{red}{A avaliação preliminar da interface \textit{web} envolveu testes com um grupo de usuários composto por estudantes de graduação em ciência da computação e profissionais da área. Os participantes foram convidados a utilizar o visualizador interativo para criar grafos, executar os algoritmos de Chu-Liu/Edmonds e Andras Frank, e explorar os passos detalhados. Feedback qualitativo foi coletado através de questionários e entrevistas, focando na usabilidade, clareza das instruções, eficácia do passo a passo e compreensão dos algoritmos.}

% \subsubsection{wip}
% \textcolor{red}{Os resultados indicaram que a maioria dos usuários considerou a interface intuitiva e fácil de navegar. O passo a passo foi particularmente elogiado por facilitar a compreensão dos algoritmos, permitindo que os usuários acompanhassem cada etapa do processo. Alguns participantes sugeriram melhorias, como a inclusão de mais exemplos de grafos e a possibilidade de salvar e carregar sessões de trabalho.}
\subsubsection{Limitações atuais e melhorias previstas}

A aplicação ainda carece de recursos que facilitariam a compreensão profunda dos algoritmos. A ausência de uma visualização explícita da contração de ciclos, por exemplo, reduz a clareza do processo; nesse ponto, planeja-se incorporar animações de contração/reexpansão com agrupamentos colapsáveis. Da mesma forma, a falta de comparação lado a lado entre as abordagens de Chu--Liu e Frank limita o entendimento de suas diferenças, algo que será endereçado com um módulo comparativo dedicado.

O layout planar simples também traz restrições, especialmente em instâncias densas com sobreposição de rótulos. Para mitigar isso, está prevista a adoção de layouts adaptativos, como \textit{spring} ou \textit{dagre-like}. Outro aspecto ausente é a sinalização cromática de custos reduzidos e arcos ``apertados'' ($c'=0$); uma camada visual desse tipo deve ajudar na identificação rápida de estruturas críticas.

Por fim, a exportação atualmente se limita ao grafo inicial e ao final, sem contemplar estados intermediários. Uma futura geração automática de relatórios (logs + estados selecionados) em PDF/ZIP pretende suprir essa lacuna, assim como o monitoramento de métricas como tempo por passo, número de contrações e distribuição de pesos normalizados. Complementarmente, está em desenvolvimento um módulo paralelo para a abordagem primal--dual de Frank.

\section{Comentários finais sobre a Interface \textit{web}}
Este capítulo detalhou a implementação técnica do visualizador interativo de arborescências dirigidas, cobrindo desde a arquitetura \textit{web} até a integração de algoritmos complexos. A escolha de tecnologias modernas como HTML5, CSS3, JavaScript e PyScript permitiu criar uma interface intuitiva e responsiva, facilitando a experimentação e compreensão dos algoritmos de Chu-Liu/Edmonds e Andras Frank. A reutilização de padrões arquiteturais entre páginas promoveu consistência cognitiva, enquanto a estrutura modular facilitou manutenção e extensão futura. No próximo capítulo, discutiremos as considerações finais e perspectivas futuras para o projeto.
