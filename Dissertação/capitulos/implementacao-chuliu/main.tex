\chapter{Algoritmo de Chu–Liu--Edmonds}
\label{cap:chuliu-edmonds}

Neste capítulo apresentaremos o algoritmo de Chu--Liu--Edmonds~\cite{chu}~\cite{edmonds}, que determina
uma $r$-arborescência geradora de custo mínimo em um $r$-digrafo ponderado.
O algoritmo baseia-se em duas operações fundamentais:
(i) a redução gulosa dos custos dos arcos e (ii) a contração de ciclos.
Essas operações permitem resolver recursivamente uma instância menor do problema
e, em seguida, estender a solução obtida para o problema original.

O propósito deste capítulo é fornecer uma descrição precisa tanto do algoritmo
quanto da implementação desenvolvida neste trabalho.


\section{O algoritmo}

O problema que nos interessa consiste em, dado um $r$-digrafo ponderado
$(D,w,r)$ (veja a página~\pageref{pag:rdig}), encontrar uma $r$-arborescência
geradora de custo mínimo de $D$.

O algoritmo de Chu--Liu--Edmonds recebe um $r$-digrafo ponderado $(D,w,r)$
e devolve uma $r$-arborescência geradora de custo mínimo de $D$.

Vamos primeiro fornecer uma visão geral do algoritmo. O algoritmo é recursivo.
Inicialmente, ele faz uma escolha gulosa de um certo conjunto de arcos.
Se esse conjunto forma uma arborescência, o algoritmo pára e devolve esse conjunto.
Caso contrário, identifica um ciclo especial no digrafo e o contrai, produzindo
um novo digrafo. Esse novo digrafo é então submetido recursivamente ao algoritmo,
que devolve uma arborescência de custo mínimo. Por fim, utilizamos o ciclo
contraído para construir uma arborescência de custo mínimo no digrafo original.
Essa construção é detalhada a seguir.


\subsection*{Escolha gulosa}
Suponha doravante que $(D, w, r)$ é um $r$-digrafo ponderado.
O algoritmo tem um caráter guloso. Note que, se \(T\) é uma \(r\)-arborescência de $D$,
então, para cada vértice \(v \neq r\), existe exatamente um arco de \(T\) que entra em \(v\). Isso sugere a seguinte escolha gulosa: para cada vértice \(v \neq r\), selecione um arco \(a_v\) de custo mínimo dentre aqueles que entram em \(v\) e forme o conjunto \(T := \{a_v : v \in V \setminus \{r\}\}\).

\begin{figure}[h]
	\centering
	\begin{tikzpicture}[>=Stealth]
		% estilos
		\tikzset{
			vtx/.style={circle, draw,fill=blue!8, thick, inner sep=1.2pt, minimum size=6mm},
			chosen/.style={->, very thick, draw=blue},
			other/.style={->, semithick, draw=black!35},
			costB/.style={font=\scriptsize, fill=white, inner sep=1pt, text=blue},
			costG/.style={font=\scriptsize, fill=white, inner sep=1pt, text=black!55}
		}

		% layout (mais espaçado)
		\node[vtx] (r) at (0,0) {$r$};
		\node[vtx] (a) at (3,1.6) {$a$};
		\node[vtx] (b) at (3,-1.6) {$b$};
		\node[vtx] (c) at (6,2.2) {$c$};
		\node[vtx] (d) at (6,0.0) {$d$};
		\node[vtx] (e) at (6,-2.2) {$e$};

		% --- arcos NÃO escolhidos (cinza), agora com curvaturas mais limpas ---
		\draw[other] (b) to[bend left=12] node[costG, above] {4} (a);   % b->a
		\draw[other] (a) to[bend left=14] node[costG, below] {5} (b);   % a->b

		% r->c: externo por cima, suave (pontos de controle explícitos)
		\draw[other]
		(r) .. controls (0.8,2.4) and (5.2,3.0) ..
		node[costG, yshift=4pt, pos=0.53] {3} (c);

		% a->d (referência)
		\draw[other] (a) -- node[costG, right] {3} (d);

		% c->e: leve curvatura pela direita (sem "barriga")
		\draw[other]
		(c) .. controls (7.6,2.6) and (7.6,-2.6) ..
		node[costG, xshift=8pt, pos=0.52] {5} (e);

		% --- arcos ESCOLHIDOS (AZUL) — formam a r-arborescência T ---
		\draw[chosen] (r) -- node[costB, above] {1} (a);   % r->a
		\draw[chosen] (r) -- node[costB, below] {2} (b);   % r->b
		\draw[chosen] (a) -- node[costB, above] {1} (c);   % a->c
		\draw[chosen] (b) -- node[costB, above] {1} (d);   % b->d
		\draw[chosen] (b) -- node[costB, below] {2} (e);   % b->e

		% legenda compacta
		\node[align=left, anchor=west] at (8.4,1.9) {\footnotesize \textbf{Escolha gulosa:}\\[-2pt]
			\footnotesize para cada \(v\neq r\), escolha\\[-1pt]
			\footnotesize um arco de custo mínimo\\[-1pt]
			\footnotesize que entra em \(v\).};
		\draw[chosen] (9.0,0.8) -- +(0.9,0) node[right, costB] {\footnotesize arcos de \(T\)};
		\draw[other]  (9.0,0.2) -- +(0.9,0) node[right, costG] {\footnotesize demais arcos};
	\end{tikzpicture}

	\caption{A figura ilustra a escolha gulosa quando esta produz uma $r$-arborescência.
		Os arcos em \textcolor{blue}{azul} são os escolhidos; os cinza são os demais arcos
		do digrafo.}
\end{figure}

Suponha que \(T\) é uma \(r\)-arborescência. Não é difícil verificar que \(T\) tem custo mínimo.
De fato, seja \(F\) uma \(r\)-arborescência de \(D\). Para cada vértice \(v \neq r\), escreva \(b_v\) para o \emph{único} arco de \(F\) que entra em \(v\). Pela escolha gulosa,
\[
	w(a_v) \leq w(b_v) \quad \text{para todo } v \neq r.
\]
Logo,
\[
	w(F) \;=\; \sum_{v \in V \setminus \{r\}} w(b_v)
	\;\;\geq\;\; \sum_{v \in V \setminus \{r\}} w(a_v)
	\;=\; w(T).
\]
Portanto, \(T\) é uma \(r\)-arborescência de custo mínimo.

A Figura~\ref{fig:esc:nfunc} ilustra que podemos não ter tanta sorte com $T$.

\begin{figure}[H]
	\centering
	\begin{tikzpicture}[>=Stealth, node distance=2.0cm]
		% estilos
		\tikzset{
			v/.style={circle, draw, fill=blue!8, minimum size=7mm, inner sep=0pt},
			cheap/.style={->, very thick, draw=blue!70},
			expensive/.style={->, dashed, draw=gray!60}
		}
		% vértices
		\node[v] (r) {\small $r$};
		\node[v] (a) [right=2.2cm of r] {\small $a$};
		\node[v] (b) [above right=1.2cm and 1.6cm of a] {\small $b$};
		\node[v] (c) [below right=1.2cm and 1.6cm of a] {\small $c$};
		% ciclo de escolhas mais baratas (custo 1)
		\draw[cheap] (a) to[bend left=12] node[midway, above, sloped] {\scriptsize 3} (b);
		\draw[cheap] (b) to[bend left=12] node[midway, above, sloped] {\scriptsize 2} (c);
		\draw[cheap] (c) to[bend left=14] node[midway, below, sloped] {\scriptsize 3} (a);
		% arcos da raiz mais caros (não escolhidos pelo critério local)
		\draw[expensive] (r) to[bend left=8] node[midway, above, sloped] {\scriptsize 3} (a);
		\draw[expensive] (r) to[bend left=18] node[midway, above, sloped] {\scriptsize 4} (b);
		\draw[expensive] (r) to[bend right=10] node[midway, below, sloped] {\scriptsize 5} (c);
	\end{tikzpicture}
	\caption{Os arcos azuis são os da escolha gulosa.}
	\label{fig:esc:nfunc}
\end{figure}
Ora, se no lugar do arco $(c,a)$ tivessemos escolhido o arco $(r, a)$, então
$r$-arborescência resultante seria de custo mínimo.
\begin{figure}[H]
	\centering
	\begin{tikzpicture}[>=Stealth, node distance=2.0cm]
		% estilos
		\tikzset{
			v/.style={circle, draw, fill=blue!8, minimum size=7mm, inner sep=0pt},
			cheap/.style={->, very thick, draw=blue!70},
			expensive/.style={->, dashed, draw=gray!60}
		}
		% vértices
		\node[v] (r) {\small $r$};
		\node[v] (a) [right=2.2cm of r] {\small $a$};
		\node[v] (b) [above right=1.2cm and 1.6cm of a] {\small $b$};
		\node[v] (c) [below right=1.2cm and 1.6cm of a] {\small $c$};
		% ciclo de escolhas mais baratas (custo 1)
		\draw[cheap] (a) to[bend left=12] node[midway, above, sloped] {\scriptsize 3} (b);
		\draw[cheap] (b) to[bend left=12] node[midway, above, sloped] {\scriptsize 2} (c);
		\draw[expensive] (c) to[bend left=14] node[midway, below, sloped] {\scriptsize 3} (a);
		% arcos da raiz mais caros (não escolhidos pelo critério local)
		\draw[cheap] (r) to[bend left=8] node[midway, above, sloped] {\scriptsize 3} (a);
		\draw[expensive] (r) to[bend left=18] node[midway, above, sloped] {\scriptsize 4} (b);
		\draw[expensive] (r) to[bend right=10] node[midway, below, sloped] {\scriptsize 5} (c);
	\end{tikzpicture}
	\caption{Os arcos azuis são os da escolha gulosa.}
\end{figure}


\section{Custos reduzidos}

Vamos introduzir agora a noção de custo reduzido. 
Essa noção permite fazer uma tranformação nos custos que preserva a otimalidade.

Seja \(q : V \setminus \{r\} \to \mathbb{R}\) uma função.
Definimos o \textbf{custo \(q\)-reduzido} \(w_q : A \to \mathbb{R}\) por%
\footnote{Recorde que nenhum arco entra em $r$; logo, a função $w_q$ está bem definida.}
\[
	w_q(uv) := w(uv) - q(v), \qquad uv \in A.
\]
Para um conjunto \(X \subseteq V\), escrevemos \(q(X) := \sum_{u \in X} q(u)\).

A próxima proposição mostra que a transformação por custo \(q\)-reduzido preserva a otimalidade.

\begin{proposicao}
\label{prop:custo:red}
	Para toda função \(q: V\setminus \{r\} \to \mathbb{R}\),
	uma \(r\)-arborescência \(T\) é de custo mínimo em $(D, w)$ se, e somente se, \(T\) é 
	de custo mínimo em \((D, w_q)\).
\end{proposicao}
\begin{proof}[Prova.]
	Seja \(F\) uma \(r\)-arborescência. Para cada \(u \in V \setminus \{r\}\), seja \(a_u\) o único arco de \(F\) que entra em \(u\). Então
	\begin{align*}
		w_q(F)
		 & = \sum_{u \in V \setminus \{r\}} w_q(a_u)                                     \\
		 & = \sum_{u \in V \setminus \{r\}} \bigl(w(a_u) - q(u)\bigr)                    \\
		 & = \sum_{u \in V \setminus \{r\}} w(a_u) - \sum_{u \in V \setminus \{r\}} q(u) \\
		 & = w(F) - q(V \setminus \{r\}).
	\end{align*}
	Assim, para quaisquer $r$-arborescências $T$ e $F$,
	\[
		w(T) \le w(F)
		\;\;\Longleftrightarrow\;\;
		w_q(T) = w(T) - q(V \setminus \{r\}) \le w(F) - q(V \setminus \{r\}) = w_q(F),
	\]
	o que prova a proposição.
\end{proof}

O custo reduzido de interesse é o dado pela função~$\lambda$ definida a seguir.
Para cada $v \in V \setminus \{r\}$, definimos
\[
  \lambda(v) := \lambda_w(v) := \min\{\, w(a) : a \in \delta^{-}(v) \,\}.
\]
Note que $\lambda$ está bem definida, uma vez que $D$ possui uma $r$-arborescência e, portanto,
existe ao menos um arco que entra em cada vértice diferente de~$r$.
Consequentemente, para todo $v \in V \setminus \{r\}$,
\[
  \min\{\, w_\lambda(a) : a \in \delta^{-}(v) \,\} = 0,
\]
isto é, precisamente os arcos de custo mínimo que entram em~$v$ passam a ter custo zero,
e os demais ficam com custo positivo.

Definimos o subdigrafo gerador $D_0$ de $D$ escolhendo, para cada
$v \neq r$, exatamente um arco $a_v$ que entra em $v$ e satisfaz
$w_\lambda(a_v) = 0$. Assim,
\[
  V(D_0) := V(D)
  \quad\text{e}\quad
  A(D_0) := \{\, a_v : v \in V(D) \setminus \{r\} \,\}.
\]

\begin{figure}[H]
	\centering
	\begin{tikzpicture}[>=Stealth, node distance=2.0cm]
		% estilos
		\tikzset{
			v/.style={circle, draw, fill=blue!8, minimum size=7mm, inner sep=0pt},
			cheap/.style={->, very thick, draw=blue!70},
			expensive/.style={->, dashed, draw=gray!60}
		}
		% vértices
		\node[v] (r) {\small $r$};
		\node[v] (a) [right=2.2cm of r] {\small $a$};
		\node[v] (b) [above right=1.2cm and 1.6cm of a] {\small $b$};
		\node[v] (c) [below right=1.2cm and 1.6cm of a] {\small $c$};
		% ciclo de escolhas mais baratas (custo 1)
		\draw[cheap] (a) to[bend left=12] node[midway, above, sloped] {\scriptsize 0} (b);
		\draw[cheap] (b) to[bend left=12] node[midway, above, sloped] {\scriptsize 0} (c);
		\draw[cheap] (c) to[bend left=14] node[midway, below, sloped] {\scriptsize 0} (a);
		% arcos da raiz mais caros (não escolhidos pelo critério local)
		\draw[expensive] (r) to[bend left=8] node[midway, above, sloped] {\scriptsize 2} (a);
		\draw[expensive] (r) to[bend left=18] node[midway, above, sloped] {\scriptsize 1} (b);
		\draw[expensive] (r) to[bend right=10] node[midway, below, sloped] {\scriptsize 3} (c);
	\end{tikzpicture}
	\caption{Os arcos azuis são os arcos de $D_0$.}
\end{figure}

Como vimos, se $D_0$ é uma $r$-arborescência, então $D_0$ tem custo mínimo.
Podemos então supor que $D_0$ não é uma $r$-arborescência. 
Vamos mostrar que, nesse caso, $D_0$ possui um ciclo.

Seja $v \neq r$ um vértice de $V$ que \emph{não} é alcançável a partir de $r$ em $D_0$;
um tal vértice existe uma vez que estamos admitindo que $D_0$ não possui uma $r$-aborescência.
Observe que, por construção, para cada vértice $s$ de $D_0$ existe exatamente um arco
de $D_0$ que entra em $s$.
Considere um caminho simples maximal%
\footnote{Maximal aqui tem o seguinte sentido. Para cada vértice $u$
	de $D_0$, as sequencias $P \cdot u$ e $u \cdot P$ não são caminhos simples.}
de $D_0$ que termina em $v$.
Seja $u$ o início de $P$. Como $v$ não é atingível a partir de $r$, temos que
$u \neq r$. Logo, existe exatamente um arco, digamos $tu$, de $D_0$ que entra em $u$.
Pela maximalidade de \(P\), o vértice \(t\) é um dos vértices de \(P\) (caso contrário,%
\footnote{Para sequências $\alpha$ e $\beta$, escrevemos $\alpha \cdot \beta$ para denotar
a concatenação de $\alpha$ e $\beta$. Para simplificar a notação, uma sequência
de comprimento $1$ é identificada com o seu único elemento.}
$t \cdot P$ é um caminho simples, o que contraria a escolha de $P$).
Como \(P\) é um caminho simples que começa em \(u\), o vértice \(t\) aparece
em \(P\) \emph{após} \(u\); portanto, \(P\) contém um subcaminho $S$ de \(u\) até \(t\).
Consequentemente, $S \cdot u$ é um ciclo de $D_0$. Isso prova que $D_0$ contém um ciclo.

\begin{figure}[h]
	\centering
	\begin{tikzpicture}[>=Stealth]
		% estilos
		\tikzset{
			vtx/.style={circle, draw, fill=blue!8, thick, inner sep=1pt, minimum size=6mm},
			Pdash/.style={->, thick, densely dashed},   % parte tracejada de P
			cycleedge/.style={->, ultra thick}          % arco que fecha o ciclo
		}

		% vértices em coordenadas (distância maior entre u e t)
		\node[vtx] (u) at (0,0) {$u$};
		\node[vtx] (t) at (4.8,0) {$t$};
		\node[vtx] (d) at (7.2,0) {$d$};
		\node[vtx] (v) at (9.6,0) {$v$};

		% raiz r (apenas contexto)
		\node[vtx] (r) at (-2,1.6) {$r$};
		\node at (-2,2.2) {\small raiz};

		% --- Caminho P ---
		% parte inicial de P (u ~~~> w) como arco ondulado azul (sem o vértice intermediário)
		\draw[blue, very thick, -{Stealth[length=2mm]}, decorate,
		decoration={snake, amplitude=1.2pt, segment length=5pt}]
		(u) .. controls (1.4,1.2) and (3.4,1.2) .. node[midway, yshift=8pt] {$S$} (t);

		% continuação de P (tracejada): w -> d -> v
		\draw[Pdash] (t) -- (d);
		\draw[Pdash] (d) -- (v);

		% rótulo P (afastado do arco w->u)
		\node at (7.2,-0.9) {\small \(P\)};

		% arco que fecha o ciclo: t -> u
		\draw[cycleedge, bend left=35] (t) to node[above, yshift=1pt] {$tu \in A(D_0)$} (u);
	\end{tikzpicture}

	\caption{O caminho simples maximal \(P\) inicia em \(u\) e termina em \(v\).
		A porção \(S\) de \(P\) entre \(u\) e \(t\) é indicada pelo arco ondulado azul; o
		caminho $S \cdot u$ é um ciclo.}
\end{figure}

Devemos agora mostrar o que fazer com um desses ciclos de $D_0$.

\subsection*{Contração de ciclos}

A próxima operação do algoritmo é a contração de ciclos (veja a Seção~\ref{sec:contracao}).
É conveniente identificar um ciclo com o conjunto de seus vértices.
Suponha que o digrafo $D_0$ possua um ciclo, digamos $C$.
Observe que $r \notin V(C)$.
Recorde que, ao \emph{contrair} o ciclo $C$ em $D$ e obter o digrafo
$D/C \mapsto x_C$, identificamos todos os vértices de $C$ em um único vértice,
denotado por $x_C$ e visto como um supervértice, e redirecionamos os arcos incidentes:
arcos da forma $uv$, em que $v \in C$ (isto é, que entravam em $C$),
passam a ser da forma $u x_C$ em $D/C \mapsto x_C$, e arcos da forma $vu$,
em que $v \in C$ (isto é, que saíam de $C$), passam a ser da forma $x_C u$
em $D/C \mapsto x_C$.
Assim, em $D/C \mapsto x_C$, o ciclo $C$ é tratado como um único vértice.

\begin{figure}[htb]
\centering
\begin{tikzpicture}[
    >=Stealth,
    node distance=2cm,
    every node/.style={circle,draw,inner sep=1pt,font=\small}
]

  % ---------------------------
  % Digrafo original D (à esquerda)
  % ---------------------------
  \begin{scope}[xshift=-4.2cm]
    \node (r)  at (0,1.6) {$r$};
    \node (u)  at (-1,0)  {$u$};
    \node (v)  at ( 1,0)  {$v$};

    % vértices do ciclo C
    \node (x1) at (-0.5,-1.4) {$v_1$};
    \node (x2) at ( 0.7,-1.2) {$v_2$};
    \node (x3) at ( 0.1,-2.4) {$v_3$};

    % arcos "externos"
    \draw[->] (r) -- (u);
    \draw[->] (r) -- (v);
    \draw[->] (u) -- (x1);
    \draw[->] (v) -- (x2);
    \draw[->] (x3) .. controls (-1.6,-2.4) .. (u);

    % ciclo C destacado
    \draw[->,thick,blue] (x1) -- (x2);
    \draw[->,thick,blue] (x2) -- (x3);
    \draw[->,thick,blue] (x3) -- (x1);

    % "moldura" em torno de C
    \node[draw=blue,rounded corners,fit=(x1) (x2) (x3),inner sep=4pt] (Cbox) {};

    % rótulo C sem círculo, em azul
    \node[draw=none,inner sep=0pt,blue,anchor=south west,font=\small]
      at (Cbox.north east) {$C$};

    % rótulo do digrafo D sem círculo, em azul
    \node[draw=none,inner sep=0pt,blue,anchor=north,font=\small]
      at (0,-3.4) {$D$};
  \end{scope}

  % ---------------------------
  % Setinha da contração
  % ---------------------------
  \node[draw=none,inner sep=0pt] at (0,0.1) {$\Longrightarrow$};
  \node[draw=none,inner sep=0pt,font=\small] at (0,-0.5) {contração de $C$};

  % ---------------------------
  % Digrafo contraído D' (à direita)
  % ---------------------------
  \begin{scope}[xshift=4.2cm]
    \node (r2)  at (0,1.6) {$r$};
    \node (u2)  at (-1,0)  {$u$};
    \node (v2)  at ( 1,0)  {$v$};

    % vértice resultante da contração
    \node[double,thick] (xC) at (0,-1.9) {$x_C$};

    % arcos "externos" preservados
    \draw[->] (r2) -- (u2);
    \draw[->] (r2) -- (v2);

    % arcos que antes entravam em C agora entram em x_C
    \draw[->] (u2) -- (xC);
    \draw[->] (v2) -- (xC);

    % arco que antes saía de C agora sai de x_C
    \draw[->] (xC) .. controls (-1.6,-2.3) .. (u2);

    % rótulo D/C → x_C sem círculo (pode ficar em preto mesmo)
    \node[draw=none,inner sep=0pt,blue,anchor=north,font=\small]
      at (0,-2.8) {$D/C \mapsto x_C$};
  \end{scope}

\end{tikzpicture}
\caption{Contração de um ciclo $C$ em um vértice $x_C$.}
\label{fig:contracao-ciclo}
\end{figure}

Precisamos agora definir um novo $r$-digrafo ponderado. Ei-lo:
$(D',w',r)$, onde $D' := D/C \mapsto x_C$ e $w' := w_\lambda/C \mapsto x_C$.
Note que $r$ é um vértice de $D'$ e que, além disso, $D'$ possui uma
$r$-arborescência.


Agora vamos ilustrar um exemplo de como essa contração é feita e os custos são ajustados.
Considere o digrafo a seguir.

\begin{figure}[H]
	\centering
	\begin{tikzpicture}[>=Stealth, node distance=2.0cm]
		% estilos
		\tikzset{
			v/.style={circle, draw, fill=blue!8, minimum size=7mm, inner sep=0pt},
			cycle/.style={->, very thick, draw=gray!60},
			arcs/.style={->, thick, draw=gray!60},
		}
		% vértices
		\node[v] (r) {\small $r$};
		\node[v] (v1) [right=2.2cm of r] {\small $v_1$};
		\node[v] (v2) [above right=1.2cm and 1.6cm of v1] {\small $v_2$};
		\node[v] (v3) [below right=1.2cm and 1.6cm of v1] {\small $v_3$};
		% vértices adicionais que saem do ciclo
		\node[v] (u) [right=2.0cm of v2] {\small $u$};
		\node[v] (w) [right=2.0cm of v3] {\small $w$};
		% ciclo de escolhas mais baratas (custo 1)
		\draw[cycle] (v1) to[bend left=12] node[midway, above, sloped] {\scriptsize 3} (v2);
		\draw[cycle] (v2) to[bend left=12] node[midway, above, sloped] {\scriptsize 2} (v3);
		\draw[cycle] (v3) to[bend left=14] node[midway, below, sloped] {\scriptsize 3} (v1);
		% arcos da raiz mais caros (não escolhidos pelo critério local)
		\draw[arcs] (r) to[bend left=8] node[midway, above, sloped] {\scriptsize 5} (v1);
		\draw[arcs] (r) to[bend left=18] node[midway, above, sloped] {\scriptsize 4} (v2);
		\draw[arcs] (r) to[bend right=10] node[midway, below, sloped] {\scriptsize 5} (v3);
		% arcos saindo do ciclo
		\draw[arcs] (v2) -- node[midway, above] {\scriptsize 1} (u);
		\draw[arcs] (v3) -- node[midway, below] {\scriptsize 2} (w);
		\draw[arcs] (v3) -- node[midway, below] {\scriptsize 4} (u);
	\end{tikzpicture}
\end{figure}

Após a redução dos custos, obtemos um ciclo $C := (v_1, v_2, v_3, v_1)$ cujos arcos têm custo reduzido igual a zero. 

\begin{figure}[H]
	\centering
	\begin{tikzpicture}[>=Stealth, node distance=2.0cm]
		% estilos
		\tikzset{
			v/.style={circle, draw, fill=blue!8, minimum size=7mm, inner sep=0pt},
			cycle/.style={->, very thick, draw=blue!70},
			arc/.style={->, dashed, draw=gray!60},
			redarc/.style={->, dashed, draw=red!60}
		}
		% vértices
		\node[v] (r) {\small $r$};
		\node[v] (v1) [right=2.2cm of r] {\small $v_1$};
		\node[v] (v2) [above right=1.2cm and 1.6cm of v1] {\small $v_2$};
		\node[v] (v3) [below right=1.2cm and 1.6cm of v1] {\small $v_3$};
		% vértices adicionais que saem do ciclo
		\node[v] (u) [right=2.0cm of v2] {\small $u$};
		\node[v] (w) [right=2.0cm of v3] {\small $w$};

		% ciclo com custo reduzido zero
		\draw[cycle] (v1) to[bend left=12] node[midway, above, sloped] {\scriptsize 0} (v2);
		\draw[cycle] (v2) to[bend left=12] node[midway, above, sloped] {\scriptsize 0} (v3);
		\draw[cycle] (v3) to[bend left=14] node[midway, below, sloped] {\scriptsize 0} (v1);
		% arcos da raiz com custos reduzidos
		\draw[arc] (r) to[bend left=8] node[midway, above, sloped] {\scriptsize 2} (v1);
		\draw[arc] (r) to[bend left=18] node[midway, above, sloped] {\scriptsize 1} (v2);
		\draw[arc] (r) to[bend right=10] node[midway, below, sloped] {\scriptsize 3} (v3);
		% arcos saindo do ciclo com custo reduzido zero
		\draw[arc] (v2) -- node[midway, above] {\scriptsize 0} (u);
		\draw[arc] (v3) -- node[midway, below] {\scriptsize 0} (w);
		\draw[arc] (v3) -- node[midway, below] {\scriptsize 3} (u);
	\end{tikzpicture}
\end	{figure}

Após a contração do ciclo \(C\), obtemos o digrafo abaixo com o supervértice \(x_C\).

\begin{figure}[H]
	\centering
	\begin{tikzpicture}[>=Stealth, node distance=2.0cm]
		% estilos
		\tikzset{
			v/.style={circle, draw, fill=blue!8, minimum size=7mm, inner sep=0pt},
			super/.style={circle, draw, very thick, fill=orange!10, minimum size=9mm, inner sep=0pt},
			cheap/.style={->, very thick, draw=gray!60},
			expensive/.style={->, dashed, draw=gray!60}
		}
		% vértices
		\node[v] (r) {\small $r$};
		\node[super] (xC) [right=3.5cm of r] {\small $x_C$};
		\node[v] (u) [above right=1.2cm and 2.0cm of xC] {\small $u$};
		\node[v] (w) [below right=1.2cm and 2.0cm of xC] {\small $w$};
		% arcos do digrafo contraído que entram em xC (mantêm arcos paralelos, escolhemos o de menor custo)
		\draw[cheap] (r) to[bend left=12] node[midway, above, sloped] {\scriptsize 1} (xC);
		% arcos que saem de xC
		\draw[cheap] (xC) to[bend left=8] node[midway, above] {\scriptsize 0} (u);
		\draw[cheap] (xC) to[bend right=8] node[midway, below] {\scriptsize 0} (w);
	\end{tikzpicture}
\end{figure}


O digrafo contraído é submetido recursivamente ao algoritmo.
Assim, o próximo passo consiste em reduzir os custos dos arcos dessa nova intância, obtendo
assim o seguinte $r$-digrafo ponderado.

\begin{figure}[H]
	\centering
	\begin{tikzpicture}[>=Stealth, node distance=2.0cm]
		% estilos
		\tikzset{
			v/.style={circle, draw, fill=blue!8, minimum size=7mm, inner sep=0pt},
			super/.style={circle, draw, very thick, fill=orange!10, minimum size=9mm, inner sep=0pt},
			cheap/.style={->, very thick, draw=gray!60},
			expensive/.style={->, dashed, draw=gray!60}
		}
		% vértices
		\node[v] (r) {\small $r$};
		\node[super] (xC) [right=3.5cm of r] {\small $x_C$};
		\node[v] (u) [above right=1.2cm and 2.0cm of xC] {\small $u$};
		\node[v] (w) [below right=1.2cm and 2.0cm of xC] {\small $w$};
		% arcos do digrafo contraído que entram em xC (mantêm arcos paralelos, escolhemos o de menor custo)
		\draw[cheap] (r) to[bend left=12] node[midway, above, sloped] {\scriptsize 0} (xC);
		% arcos que saem de xC
		\draw[cheap] (xC) to[bend left=8] node[midway, above] {\scriptsize 0} (u);
		\draw[cheap] (xC) to[bend right=8] node[midway, below] {\scriptsize 0} (w);
	\end{tikzpicture}
\end{figure}

O subdigrafo gerador obtido a partir desse digrafo, usando apenas os arcos
de custo reduzido igual a zero, possui uma $r$-arborescência.
O algoritmo devolve essa $r$-arborescência, que agora deverá passar por
um processo de expansão.




\subsection*{Reexpansão de arborescências}

Após resolver o problema no digrafo contraído \(D'\), obtemos uma \(r\)-arborescência geradora  \(T'\) de custo mínimo em \((D', w')\). Observe que tão somente um arco de $T'$ entra em $x_C$.
Para reexpandir \(T'\) em uma \(r\)-arborescência \(T\) em \(D\), substituímos o supervértice \(x_C\) pelo ciclo \(C\) e adicionamos os arcos do ciclo que formam a arborescência dentro de \(C\). Especificamente, o arco \(u x_C\) de \(T'\) corresponde a um arco \(uv\) de \(D\), onde \(v \in C\). Esse arco $uv$ é incluído em $T$. Em seguida, adicionamos os arcos do ciclo \(C\) que conectam os vértices de \(C\) de forma a manter a estrutura de arborescência. Note que, devemos escolher todos os arcos do ciclo \(C\) exceto aquele que entra em \(v\), garantindo que cada vértice de \(C\) tenha grau de entrada igual a 1 em $T$.

Retomemos o exemplo para ilustrar a expansão da \(r\)-arborescência geradora do digrafo contraído. Primeiro, adicionamos os vértices do ciclo \(C\) e, em seguida, incluímos os arcos apropriados para formar a \(r\)-arborescência geradora do digrafo original:
% Seguindo o exemplo anterior, a figura abaixo ilustra a reexpansão de \(T'\) em \(D'\) para obter a \(r\)-arborescência \(T\) em \(D\):

\begin{figure}[H]
	\centering
	\begin{tikzpicture}[>=Stealth, node distance=2.0cm]
		% estilos
		\tikzset{
			v/.style={circle, draw, fill=blue!8, minimum size=7mm, inner sep=0pt},
			cycle/.style={->, very thick, draw=blue!70},
			arcs/.style={->, dashed, draw=gray!60},
			selected/.style={->, very thick, draw=green!70}
		}
		% vértices
		\node[v] (r) {\small $r$};
		\node[v] (v1) [right=2.2cm of r] {\small $v_1$};
		\node[v] (v2) [above right=1.2cm and 1.6cm of v1] {\small $v_2$};
		\node[v] (v3) [below right=1.2cm and 1.6cm of v1] {\small $v_3$};
		% vértices adicionais que saem do ciclo
		\node[v] (u) [right=2.0cm of v2] {\small $u$};
		\node[v] (w) [right=2.0cm of v3] {\small $w$};
		% ciclo de escolhas mais baratas (custo 1)
		\draw[selected] (v1) to[bend left=12] node[midway, above, sloped] {\scriptsize 0} (v2);
		\draw[selected] (v2) to[bend left=12] node[midway, above, sloped] {\scriptsize 0} (v3);
		% \draw[arcs] (v3) to[bend left=14] node[midway, below, sloped] {\scriptsize 0} (v1);
		% arcos da raiz mais caros (não escolhidos pelo critério local)
		\draw[selected] (r) to[bend left=8] node[midway, above, sloped] {\scriptsize 0} (v1);
		% \draw[arcs] (r) to[bend left=18] node[midway, above, sloped] {\scriptsize 0} (v2);
		% \draw[arcs] (r) to[bend right=10] node[midway, below, sloped] {\scriptsize 0} (v3);
		% arcos saindo do ciclo
		\draw[selected] (v2) -- node[midway, above] {\scriptsize 0} (u);
		\draw[selected] (v3) -- node[midway, below] {\scriptsize 0} (w);
		% \draw[arcs] (v3) -- node[midway, below] {\scriptsize 0} (u);
	\end{tikzpicture}
\end{figure}

No $r$-digrafo ponderado original, temos a seguinte configuração.

\begin{figure}[H]
	\centering
	\begin{tikzpicture}[>=Stealth, node distance=2.0cm]
		% estilos
		\tikzset{
			v/.style={circle, draw, fill=blue!8, minimum size=7mm, inner sep=0pt},
			cycle/.style={->, very thick, draw=blue!70},
			arcs/.style={->, dashed, draw=gray!60},
			selected/.style={->, very thick, draw=green!70}
		}
		% vértices
		\node[v] (r) {\small $r$};
		\node[v] (v1) [right=2.2cm of r] {\small $v_1$};
		\node[v] (v2) [above right=1.2cm and 1.6cm of v1] {\small $v_2$};
		\node[v] (v3) [below right=1.2cm and 1.6cm of v1] {\small $v_3$};
		% vértices adicionais que saem do ciclo
		\node[v] (u) [right=2.0cm of v2] {\small $u$};
		\node[v] (w) [right=2.0cm of v3] {\small $w$};
		% ciclo de escolhas mais baratas (custo 1)
		\draw[selected] (v1) to[bend left=12] node[midway, above, sloped] {\scriptsize 3} (v2);
		\draw[selected] (v2) to[bend left=12] node[midway, above, sloped] {\scriptsize 2} (v3);
		% \draw[arcs] (v3) to[bend left=14] node[midway, below, sloped] {\scriptsize} (v1);
		% arcos da raiz mais caros (não escolhidos pelo critério local)
		\draw[selected] (r) to[bend left=8] node[midway, above, sloped] {\scriptsize 5} (v1);
		% \draw[arcs] (r) to[bend left=18] node[midway, above, sloped] {\scriptsize} (v2);
		% \draw[arcs] (r) to[bend right=10] node[midway, below, sloped] {\scriptsize} (v3);
		% arcos saindo do ciclo
		\draw[selected] (v2) -- node[midway, above] {\scriptsize 1} (u);
		\draw[selected] (v3) -- node[midway, below] {\scriptsize 2} (w);
		% \draw[arcs] (v3) -- node[midway, below] {\scriptsize} (u);
	\end{tikzpicture}
\end{figure}

Vamos agora verificar que a $r$-arborescência geradora $T$, obtida da construção descrita,
é de custo mínimo em $(D, w)$.
Por hipótese, $T'$ é uma $r$-arborescência geradora de custo mínimo em $(D',w')$, isto é,
\[
  w'(T') \le w'(F')
\]
para toda $r$-arborescência geradora $F'$ de $D'$.
Pela construção de $T$, temos que $T$ é uma $r$-arborescência geradora tal que
$w_\lambda(T) = w'(T')$, uma vez que todo arco $a$ de $C$ satisfaz $w_\lambda(a) = 0$.

Suponha agora que $F$ seja uma $r$-arborescência geradora de custo mínimo em
$(D,w_\lambda)$. Então $w_\lambda(F) \le w_\lambda(T)$.
Seja $F' := F/C \mapsto x_C$. Note que $F'$ pode não ser uma $r$-arborescência de $D'$.
No entanto, $F'$ \emph{contém} uma $r$-arborescência de $D'$; logo, existe
$F'' \subseteq F'$ tal que $F''$ é uma $r$-arborescência de $D'$.
Temos, então, $w'(F'') \le w'(F')$ pois $w'(a) \ge 0$ para cada $a \in A(D')$.
Por hipótese, $w'(T') \le w'(F'')$ o que, combinado com $w_\lambda(T) = w'(T')$,
permite inferir que
\[
  w_\lambda(T) \le w'(F'') \le w'(F') = w_\lambda(F).
\]
Portanto, $T$ é uma $r$-arborescência geradora de custo mínimo em $(D,w_\lambda)$.
Agora, pela Proposição~\ref{prop:custo:red}, $T$ é uma $r$-arborescência geradora
de custo mínimo em $(D, w)$, como queríamos. Isso completa a prova de que a $r$-arborescência
devolvida pelo algoritmo é de custo mínimo em $(D, w)$.


Convém agora sumarizar e exibir uma descrição em pseudocódigo do algoritmo
de Chu--Liu/Edmonds.
O mapeamento desse pseudocódigo para código \texttt{Python} será discutido
na próxima seção.


\begin{algobox}{Chu–Liu/Edmonds}{chu-liu-edmonds}
\begin{lstlisting}[mathescape=true, language=Python]
def chu-liu-edmonds($D$, $w$, $r$):
    $\lambda := \bigl\{(v, \min \{w(a) : a \in \delta^-(v)\}): v \in V\setminus \{r\}\bigr\}$
    para cada $v \in V\setminus\{r\}$, seja $a_v \in \delta^-(v)$ tal que $w_\lambda(a_v) = 0$
    seja $D_0 := \bigl(V, \{a_v : v \in V \setminus \{r\}\}\bigr)$
    if $D_0$ é $\,$ uma $r$-arborescência: return $D_0$
    seja $C$ um ciclo em $D_0$
    $T' :=$ chu-liu-edmonds($D/C \mapsto x_C$, $w_\lambda/C \mapsto x_C$, $r$)
    $T := $ expand($T'$)
    return $T$
\end{lstlisting}
\end{algobox}


\subsection*{Complexidade}

Não é difícil ver que as operações envolvidas nas linhas 2 a 6 e na linha 8
podem ser implementadas de forma a serem executadas em tempo $O(|A|)$.
Como cada chamada recursiva contrai ao menos um vértice, o número total
de chamadas é limitado por $O(|V|)$. Portanto, o consumo de tempo do
algoritmo está em $O(|V||A|)$.

Quanto ao consumo de memória, é possível realizar as operações de contração
de modo que o uso total de memória adicional permaneça em $O(|V||A|)$.
A implementação descrita a seguir, disponível em
\url{https://github.com/lorenypsum/GraphVisualizer},
apresenta consumo de tempo e memória em $O(|V||A|)$.


\section{Implementação em Python}

Esta seção descreve a implementação do algoritmo de Chu--Liu/Edmonds em \texttt{Python}, 
estruturada para refletir com precisão as etapas discutidas anteriormente. Cada operação fundamental — redução dos custos, construção do subdigrafo gerador de arcos de custo reduzido iguais a zero, contração de ciclos e reexpansão — é implementada 
utilizando como suporte a biblioteca \texttt{networkx}.


\subsection*{Representação de digrafos e detecção de ciclos}

A implementação utiliza a biblioteca NetworkX\footnote{Disponível em
\url{https://networkx.org/}.}. Digrafos ponderados são representados por instâncias da
classe \texttt{DiGraph}. Internamente, essa classe usa dicionários do
\texttt{Python} para armazenar vértices, arcos e atributos, o que garante
operações eficientes na prática. Por exemplo, adicionar ou remover um arco
tem consumo amortizado de tempo em $O(1)$; iterar sobre os sucessores de um
vértice $u$ tem consumo de tempo em $O(|\delta^+(u)|)$; e iterar sobre todos
os arcos tem consumo de tempo em $O(m)$, em que $m$ é o número de arcos
do digrafo.

\subsubsection*{Métodos da API NetworkX}

Elencamos a seguir alguns métodos da API \texttt{NetworkX} utilizados na implementação. 
Para uma instância \texttt{D} de \texttt{DiGraph}:

\paragraph*{Consulta de estrutura}
\begin{itemize}\setlength{\itemsep}{2pt}
  \item \texttt{D.nodes()}: devolve um iterável sobre o conjunto dos vértices
  de \texttt{D}.
  
  \item \texttt{D.in\_edges(v, data="w")}: devolve um iterável de triplas da
  forma \texttt{(u, v, w)}, em que \texttt{uv} é um arco de \texttt{D} com peso
  \texttt{w}.

  \item \texttt{D.out\_edges(u, data="w")}: devolve um iterável de triplas da
  forma \texttt{(u, v, w)}, em que \texttt{uv} é um arco de \texttt{D} com peso
  \texttt{w}.

  \item \texttt{D[u][v]["w"]}: devolve o peso do arco \texttt{uv} de \texttt{D},
  isto é, o valor armazenado no atributo \texttt{"w"} associado a esse arco.
\end{itemize}

\paragraph*{Modificação de estrutura}
\begin{itemize}\setlength{\itemsep}{2pt}
  \item \texttt{D.add\_edge(u, v, w=p)}: adiciona o arco \texttt{uv} a \texttt{D}
  com peso \texttt{p} armazenado no atributo \texttt{"w"}. Os vértices \texttt{u}
  e/ou \texttt{v} são criados automaticamente se ainda não existirem em \texttt{D}.
  
  \item \texttt{D.remove\_edges\_from(edges)}: recebe um iterável \texttt{edges}
  de arcos e remove de \texttt{D} cada arco \texttt{(u, v)} em \texttt{edges}.
  
  \item \texttt{D.remove\_nodes\_from(nodes)}: recebe um iterável \texttt{nodes}
  de vértices e remove de \texttt{D} cada vértice \texttt{v} em \texttt{nodes}
  (bem como todos os arcos incidentes em \texttt{v}).
\end{itemize}


\subsection*{Remoção de arcos que entram na raiz}

Recorde que o algoritmo de Chu--Liu--Edmonds recebe um $r$-digrafo ponderado
e que, por definição, em um $r$-digrafo ponderado nenhum arco entra em $r$.
Escrevemos esta função como uma etapa de pré-processamento justamente para
garantir que a raiz $r$ não possua arcos de entrada antes de iniciar
o algoritmo principal.


Em detalhes, essa função recebe como entrada uma instância \texttt{D} de
\texttt{DiGraph} e um vértice \texttt{r} de \texttt{D}.
A função modifica \texttt{D} removendo todos os arcos que entram em \texttt{r}
e tem consumo amortizado de tempo em $O(k)$, em que $k$ é o número de arcos
que entram em \texttt{r}. Destacamos que é necessário armazenar,
em uma lista, todos esses arcos usando o método \texttt{in\_edges} (linha 2),
pois esse método devolve um iterador que é invalidado assim que alguma operação
modifica a estrutura de dados que o produziu.


\begin{tcolorbox}[
		enhanced, breakable,
		colframe=blue!60!black, colback=blue!2,
		colbacktitle=blue!15, coltitle=black,
		title={Remoção de arcos que entram na raiz},
		boxed title style={sharp corners, boxrule=0.6pt},
		sharp corners, boxrule=0.6pt
	]
	\emph{Entrada:} \texttt{D: Digraph} e \texttt{r}.
	
	\emph{Pré-condição:} \texttt{r} vértice de \texttt{D}.
	
	\emph{Modifica:} \texttt{D}.
	
	\emph{Pós-condição:} \texttt{D} não possui nenhum arco que entra em \texttt{r}.
	\tcblower
	\begin{lstlisting}[mathescape=true, language=Python]
def remove_edges_to(D: nx.DiGraph, r: int):
    in_edges = list(D.in_edges(r))
    D.remove_edges_from(in_edges)
\end{lstlisting}
\end{tcolorbox}

\subsection{Redução de custos}

A função \texttt{reduce\_weights} tem como propósito realizar a redução de custos
por vértice. Em outras palavras, é nessa função que, para um vértice $v$, calculamos
$\lambda(v)$ e obtemos os custos $\lambda$-reduzidos dos arcos que entram em $v$.
A função recebe \texttt{D: DiGraph} e um vértice \texttt{v}.
A variável \texttt{in\_edges} é um iterável de triplas da forma
\texttt{(u, v, w)}, em que \texttt{(u, v)} é um arco de \texttt{D} e
\texttt{w} é o seu peso, obtidas por meio do método
\texttt{D.in\_edges(node, data="w")}.
A variável \texttt{yv} é o peso mínimo entre esses arcos.
Em seguida, os pesos dos arcos que entram em \texttt{v} são decrementados
de \texttt{yv}. O consumo de tempo está em $O(k)$, em que $k$ é o número de
arcos que entram em \texttt{node}.

\begin{tcolorbox}[
		enhanced, breakable,
		colframe=blue!60!black, colback=blue!2,
		colbacktitle=blue!15, coltitle=black,
		title={Redução de custos por vértice (normalização)},
		boxed title style={sharp corners, boxrule=0.6pt},
		sharp corners, boxrule=0.6pt
	]
	\emph{Entrada:} \texttt{D: DiGraph}, \texttt{v: int}.
	
	\emph{Pré-condição:} \texttt{v} é vértice de \texttt{D} e possui ao menos um arco de entrada.
	
	\emph{Modifica:} \texttt{D}.
	
	\emph{Pós-condição:} \texttt{D[u][v][''w'']} é o custo $\lambda$-reduzido do arco \texttt{(u, v)} para cada \texttt{u} que é predecessor de \texttt{v}.  
	\tcblower
	\begin{lstlisting}[language=Python]
def reduce_weights(D: nx.DiGraph, v: int):    
    in_edges = D.in_edges(v, data=True)
    yv = min((data["w"] for _, _, data in in_edges))
    for u, _, _ in in_edges:
        D[u][v]["w"] -= yv   
\end{lstlisting}
\end{tcolorbox}

A Figura~\ref{fig:normalize-example} ilustra o funcionamento da redução:

\begin{figure}[H]
	\centering
	\begin{tikzpicture}[>=Stealth, node distance=2.5cm, scale=0.9, every node/.style={scale=0.9}]
    % Grafo antes da normalização
    \node[circle,draw,fill=blue!20,minimum size=8mm] (u1) at (0,2) {\(u_1\)};
    \node[circle,draw,fill=blue!20,minimum size=8mm] (u2) at (0,0) {\(u_2\)};
    \node[circle,draw,fill=blue!20,minimum size=8mm] (u3) at (0,-2) {\(u_3\)};
    \node[circle,draw,fill=green!20,minimum size=8mm] (v1) at (3,0) {\(v\)};

    \draw[->,thick] (u1) to node[above] {5} (v1);
    \draw[->,thick,red] (u2) to node[above] {3} (v1);
    \draw[->,thick] (u3) to node[below] {7} (v1);

    \node[above] at (1.5,2.8) {\textbf{Antes:} \(y(v)=\min\{5,3,7\}=3\)};

    % Seta de transformação
    \node at (5,0) {\Large\(\longrightarrow\)};
    \node[align=center] at (5,-0.8) {\small\texttt{normalize\_incoming\_}\\[-2pt]\small\texttt{edge\_weights(D, v)}};

    % Grafo depois da normalização
    \node[circle,draw,fill=blue!20,minimum size=8mm] (u1b) at (7,2) {\(u_1\)};
    \node[circle,draw,fill=blue!20,minimum size=8mm] (u2b) at (7,0) {\(u_2\)};
    \node[circle,draw,fill=blue!20,minimum size=8mm] (u3b) at (7,-2) {\(u_3\)};
    \node[circle,draw,fill=green!20,minimum size=8mm] (v2) at (10,0) {\(v\)};

    \draw[->,thick] (u1b) to node[right, xshift=10pt] {\(5-3=2\)} (v2);
    \draw[->,thick,red] (u2b) to node[above, xshift=-4pt] {\(3-3=0\)} (v2);
    \draw[->,thick] (u3b) to node[right, xshift=10pt] {\(7-3=4\)} (v2);

    \node[above] at (8.5,2.8) {\textbf{Depois:} ao menos uma entrada tem custo 0};
\end{tikzpicture}

	\caption{À esquerda, vértice \(v\) com três arcos de entrada (pesos 5, 3 e 7). À direita, após aplicar \texttt{reduce\_weights(D, v)}: o menor peso \(3\) é subtraído de todas as entradas, resultando em custos reduzidos 2, 0 e 4. O arco \((u_2,v)\) (em vermelho) tem custo zero e será selecionado para \(D_0\).}
	\label{fig:normalize-example}
\end{figure}


\subsection{Construção de $D_0$}

Vamos mostrar agora como construir o subdigrafo $D_0$ de $D$.
Lembre-se de que $D_0$ é o subdigrafo gerador de $D$ em que, para cada
vértice $v \neq r$, selecionamos um arco que entra em $v$ com peso zero.

A função é bastante simples. Ela recebe um \texttt{D: DiGraph} e um vértice
\texttt{r} de \texttt{D} tais que, em cada vértice \texttt{v} diferente de
\texttt{r}, entra ao menos um arco de peso zero.
A função devolve um subdigrafo gerador \texttt{D0: DiGraph}, construído ao
se iterar sobre o conjunto dos vértices \texttt{v} de \texttt{D} distintos
de \texttt{r} e selecionar exatamente um arco de peso zero que entra em
\texttt{v}.


\begin{tcolorbox}[
		enhanced, breakable,
		colframe=blue!60!black, colback=blue!2,
		colbacktitle=blue!15, coltitle=black,
		title={Construção de $D_0$},
		boxed title style={sharp corners, boxrule=0.6pt},
		sharp corners, boxrule=0.6pt
	]
	\emph{Entrada:} \texttt{D: DiGraph}, \texttt{r: int}.
	
	\emph{Pré-condição:} \texttt{r} é vértice de \texttt{D} e para cada vértice
	distinto de \texttt{r} existe um arco de peso zero (atributo \texttt{"w"}) 
	que nele entra. 
	
	\emph{Saída:} \texttt{D0: Digraph} subdigrafo gerador de \texttt{D} tal 
	que em cada vértice diferente de \texttt{r} entra exatamente 
	um arco de custo reduzido igual a zero.
	\tcblower
	\begin{lstlisting}[mathescape=true, language=Python]
def get_Dzero(D: nx.DiGraph, r: int):
    D_zero = nx.DiGraph()
    for v in D.nodes():
        if v != r:
            in_edges = D.in_edges(v, data=True)
            u = next((u for u, _, data in in_edges 
                        if data["w"] == 0))
            D_zero.add_edge(u, v)
    return D_zero
\end{lstlisting}
\end{tcolorbox}


As funções de redução de custo e construção de $D_0$ juntas implementam os passos das linhas
2, 3, e 4 do Algoritmo de Chu--Liu--Edmonds.

\subsection{Detecção de ciclo}

O próximo passo consiste em mostrar como decidir se o digrafo $D_0$ é uma
$r$-arborescência e, caso isso não ocorra, como encontrar um ciclo para
futura contração. A decisão sobre se $D_0$ é uma arborescência é delegada
a uma função de biblioteca, chamada \texttt{is\_arborescence}. 
Note que, em virtude da forma como $D_0$ é construído, se $D_0$ é uma
arborescência, então $D_0$ é necessariamente uma $r$-arborescência.
No que segue, vamos assumir que $D_0$ não é uma arborescência.


A função recebe \texttt{D\_zero: DiGraph} e supõe que
(i) \texttt{D\_zero} não é uma arborescência; (ii) existe exatamente
um vértice de \texttt{D\_zero} no qual não entra nenhum arco; e (iii)
em cada um dos demais vértices, entra ao menos um arco.
A função devolve um subdigrafo \texttt{C: DiGraph} de \texttt{D\_zero}
que é um ciclo.

Uma função de biblioteca, \texttt{find\_cycle}, é usada para encontrar
um ciclo em \texttt{D\_zero}. Os arcos desse ciclo — resultado da chamada
\[ \texttt{find\_cycle(D\_zero, orientation="original")} \] — determinam o
subconjunto de vértices do ciclo. A função devolve o subdigrafo de \texttt{D\_zero}
induzido por esse subconjunto vértices.


\begin{tcolorbox}[
		enhanced, breakable,
		colframe=blue!60!black, colback=blue!2,
		colbacktitle=blue!15, coltitle=black,
		title={Detecção de ciclo dirigido em $D_0$},
		boxed title style={sharp corners, boxrule=0.6pt},
		sharp corners, boxrule=0.6pt
	]
	\emph{Entrada:} \texttt{D\_zero: DiGraph}.
	
	\emph{Pré-condição:} \texttt{D\_zero} não é uma arborescência, e existe exatamente
	um vértice de \texttt{D\_zero} no qual não entra nenhum arco e nos demais vértices
	entra ao menos um arco.
	
	\emph{Saída:} \texttt{C: Digraph} subdigrafo de \texttt{D\_zero} que é um ciclo.
	\tcblower
	\begin{lstlisting}[mathescape=true, language=Python]
def find_cycle(D_zero: nx.DiGraph):
    nodes_in_cycle = set()
    for u, v, _ in nx.find_cycle(D_zero, orientation="original"):
        nodes_in_cycle.update([u, v])
    return D_zero.subgraph(nodes_in_cycle).to_directed()  
\end{lstlisting}
\end{tcolorbox}


\subsection{Contração de um ciclo}

Vamos agora mostrar como implementar a contração de um ciclo $C$ de $D_0$.
A função \texttt{contract\_cycle} recebe um digrafo \texttt{D},
um\footnote{Isso é irrelevante: \texttt{C} poderia ser qualquer subdigrafo
não vazio de \texttt{D}.}
subdigrafo \texttt{C} de \texttt{D}
e um nome \texttt{label} para o supervértice do digrafo contraído.
A função modifica \texttt{D} de tal forma que, após a chamada,
\texttt{D} representa o digrafo $\mathtt{D}/\mathtt{C} \mapsto \mathtt{label}$
e devolve dois dicionários
\[
  \texttt{in\_to\_cycle, out\_from\_cycle}
  : \texttt{dict[int, tuple[int, float]]},
\]
cuja formação é explicada a seguir.

Considere um vértice \texttt{u} de \texttt{D} que está fora de \texttt{C}
e que possui ao menos um sucessor em \texttt{C}.
Dizemos que um arco \texttt{uv} de \texttt{D} é \textbf{essencial de} \texttt{u}
\textbf{para} \texttt{C} se \texttt{v} é vértice de \texttt{C} e o custo de
\texttt{uv} é mínimo entre os custos dos arcos que saem de \texttt{u} e
entram em \texttt{C}.
De forma similar, considere um vértice \texttt{v} de \texttt{D} que está
fora de \texttt{C} e que possui ao menos um antecessor em \texttt{C}.
Dizemos que um arco \texttt{uv} de \texttt{D} é \textbf{essencial de} \texttt{C}
\textbf{para} \texttt{v} se \texttt{u} é vértice de \texttt{C} e o custo de
\texttt{uv} é mínimo entre os custos dos arcos que saem de \texttt{C}
e entram em \texttt{v}.

A função constrói os dicionários com os arcos essenciais e seus pesos.
Assim, para cada \texttt{u} que é antecessor de \texttt{C},
\texttt{in\_to\_cycle[u] = (v, w)} se, e somente se, \texttt{uv} é um arco
essencial de \texttt{u} para \texttt{C}, de custo \texttt{w}.
De forma similar, para cada \texttt{v} que é sucessor de \texttt{C},
\texttt{out\_from\_cycle[v] = (u, w)} se, e somente se, \texttt{uv} é um arco
essencial de \texttt{C} para \texttt{v}, de custo \texttt{w}.

É fácil ver que o consumo de tempo da função está em $O(m)$, em que
$m$ é o número de arcos de \texttt{D}.


\begin{tcolorbox}[
		enhanced, breakable,
		colframe=blue!60!black, colback=blue!2,
		colbacktitle=blue!15, coltitle=black,
		title={Contração de ciclo},
		boxed title style={sharp corners, boxrule=0.6pt},
		sharp corners, boxrule=0.6pt
	]
	\emph{Entrada:} \texttt{D: DiGraph}, \texttt{C: DiGraph}, \texttt{label: int}.
	
	\emph{Pré-condição:} \texttt{C} subdigrafo de \texttt{D} e \texttt{label} não é um vértice de \texttt{D}.
	
	\emph{Modifica:} \texttt{D}.
	
	\emph{Pós-condição:} \texttt{D} é o digrafo $\mathtt{D}/\mathtt{C} \mapsto \mathtt{label}$. 
	
	\emph{Saída:} \texttt{in\_to\_cycle, out\_from\_cycle: dict[int, tuple[int, float]]}.
	
	Para cada \texttt{u} que é antecessor de \texttt{C},
	\texttt{in\_to\_cycle[u] = (v, w)} se, e somente se, \texttt{uv} é um arco
	essencial de \texttt{u} para \texttt{C} de custo \texttt{w}.

	Para cada \texttt{u} que é sucessor de \texttt{C},
	\texttt{in\_to\_cycle[u] = (v, w)} se, e somente se, \texttt{vu} é um arco
	é um arco essencial de \texttt{C} para \texttt{u} de custo \texttt{w}.

	 
\tcblower
\begin{lstlisting}[mathescape=true, language=Python]
def contract_cycle(D: nx.DiGraph, C: nx.DiGraph, label: int):
    cycle_nodes: set[int] = set(C.nodes())
    in_to_cycle: dict[int, tuple[int, float]] = {}
    for u in D.nodes:
        if u not in cycle_nodes:
            min_weight_edge_to_cycle = min(
                ((v, data["w"])
                    for _, v, data in D.out_edges(u, data=True)
                    if v in cycle_nodes),
                key=lambda x: x[1],
                default=None,)
            if min_weight_edge_to_cycle:
                in_to_cycle[u] = min_weight_edge_to_cycle
    for u, (v, w) in in_to_cycle.items():
        D.add_edge(u, label, w=w)
    out_from_cycle: dict[int, tuple[int, float]] = {}
    for v in D.nodes:
        if v not in cycle_nodes:
            min_weight_edge_from_cycle = min(
                ((u, data["w"])
                    for u, _, data in D.in_edges(v, data=True)
                    if u in cycle_nodes),
                key=lambda x: x[1],
                default=None,)
            if min_weight_edge_from_cycle:
                out_from_cycle[v] = min_weight_edge_from_cycle
    for v, (u, w) in out_from_cycle.items():
        D.add_edge(label, v, w=w)
    D.remove_nodes_from(cycle_nodes)
    return in_to_cycle, out_from_cycle  
\end{lstlisting}
\end{tcolorbox}

\subsection{Procedimento principal}

Vamos agora apresentar a função principal, que orquestra todas as funções
auxiliares descritas anteriormente e completa a implementação do
algoritmo de Chu--Liu--Edmonds.
A função recebe um digrafo ponderado \texttt{D},
um vértice raiz \texttt{r} e um inteiro \texttt{label} que satisfazem:
\begin{itemize}
  \item os vértices de \texttt{D} são inteiros no conjunto
        $\{0, 1, \dots, n-1\}$, para algum $n \ge 1$;
  \item \texttt{r} é um vértice de \texttt{D};
  \item \texttt{D} possui ao menos uma $r$-arborescência;
  \item nenhum arco de \texttt{D} entra em \texttt{r}; e
  \item \texttt{label} é um inteiro maior ou igual a $n$.
\end{itemize}
A função devolve um digrafo ponderado que é uma $r$-arborescência
geradora de custo mínimo de \texttt{D}.

A seguir, comentamos brevemente o papel de cada bloco de instruções
do código.

Na linha 2, \texttt{D\_copy} é uma cópia de \texttt{D}.
As linhas 3 a 5 são responsáveis por calcular o custo reduzido de
cada arco do digrafo \texttt{D\_copy} (as modificações dos custos
são feitas em \texttt{D\_copy}).
A linha 6 determina o digrafo \texttt{D\_zero}, que contém exatamente
um arco de custo reduzido igual a zero entrando em cada vértice distinto
de \texttt{r}.
A linha 7 determina se \texttt{D\_zero} é uma arborescência. Se esse for
o caso, a função restaura os pesos originais nos arcos de \texttt{D\_zero}
e devolve \texttt{D\_zero}.

Suponha que \texttt{D\_zero} não seja uma arborescência.
A linha 11 determina um ciclo em \texttt{D\_zero}, armazenando-o em \texttt{C}.
A linha 12 contrai o ciclo \texttt{C} em \texttt{D\_copy}; o digrafo
\texttt{D\_copy} é modificado de tal forma que corresponda ao digrafo
contraído.
A linha 13 determina em \texttt{F\_prime} uma $r$-arborescência geradora
de custo mínimo do digrafo \texttt{D\_copy}.
As demais linhas são responsáveis pelo processo de expansão de \texttt{F\_prime}:
elas modificam \texttt{F\_prime} de tal forma que \texttt{F\_prime} se torne
uma $r$-arborescência geradora de custo mínimo de \texttt{D}.


O código completo da função principal é apresentado a seguir:

\begin{tcolorbox}[
		enhanced, breakable,
		colframe=blue!60!black, colback=blue!2,
		colbacktitle=blue!15, coltitle=black,
		title={Procedimento principal (recursivo)},
		boxed title style={sharp corners, boxrule=0.6pt},
		sharp corners, boxrule=0.6pt
	]
	\emph{Entrada:} \texttt{D: DiGraph}, \texttt{r: int}, \texttt{label: int}.
	
	\emph{Pré-condição:} \texttt{D} é um digrafo ponderado; 
	os vértices de \texttt{D} são inteiros no conjunto $\{0, 1, \dots, n-1\}$
	para algum $n \ge 0$; \texttt{r} é um vértice de \texttt{D}; 
	\texttt{D} possui ao menos uma \texttt{r}-arborescência;
	nenhum arco de \texttt{D} entra em \texttt{r}; e 
	\texttt{label} é maior ou igual a $n$.
	
	\emph{Saída:} \texttt{T: DiGraph} é uma \texttt{r}-arborescência de custo mínimo de \texttt{D}.
	\tcblower
	\begin{lstlisting}[mathescape=true, language=Python]
def chuliu_edmonds(D: nx.DiGraph, r: int, label: int):
	D_copy = cast(nx.DiGraph, D.copy())
	for v in D_copy.nodes:
        if v != r:
            reduce_weights(D_copy, v)
    D_zero = get_Dzero(D_copy, r)
    if nx.is_arborescence(D_zero):
        for u, v in D_zero.edges:
            D_zero[u][v]["w"] = D[u][v]["w"]
        return D_zero
    C = find_cycle(D_zero)
    in_to_cycle, out_from_cycle = contract_cycle(D_copy, C, label)
    F_prime = chuliu_edmonds(D_copy, r, label + 1)
    in_edge = next(iter(F_prime.in_edges(label, data=True)))
    u, _, _ = cast(tuple, in_edge)
    v, _ = in_to_cycle[u]
    F_prime.add_edge(u, v)
    for u_c, v_c in C.edges:
        if v != v_c: F_prime.add_edge(u_c, v_c)
    for _, z, _ in list(F_prime.out_edges(label, data=True)):
        u_cycle, _ = out_from_cycle[z]
        F_prime.add_edge(u_cycle, z)
    F_prime.remove_node(label)
    for u, v in F_prime.edges:
        F_prime[u][v]["w"] = D[u][v]["w"]
    return F_prime 
\end{lstlisting}
\end{tcolorbox}


