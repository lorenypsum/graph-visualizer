\chapter{Definições Preliminares}

Neste capítulo, apresentamos os conceitos matemáticos fundamentais necessários para a compreensão dos algoritmos de arborescência de custo mínimo. Focamos exclusivamente nas definições que serão utilizadas nos capítulos subsequentes, estabelecendo a notação e a terminologia básica de grafos direcionados, arborescências e otimização.

\section{Grafos Direcionados}

Um \textbf{grafo dirigido} (ou \textbf{digrafo}) \(D = (V, A)\) é composto por um conjunto \(V\) de \emph{vértices} e um conjunto \(A\) de \emph{arcos} (arestas direcionadas). Cada arco \((u, v) \in A\) conecta o vértice \(u\) (cauda) ao vértice \(v\) (cabeça).

\begin{figure}[H]
    \centering
    \includegraphics[width=0.7\linewidth]{figures/fig_def_digrafo.pdf}
    \caption{Digrafo: arcos têm direção. No arco $a=(u,v)$, $u$ é a cauda e $v$ é a cabeça.}
    \label{fig:def-digrafo}
\end{figure}

\subsection{Graus e caminhos}

O \textbf{grau de entrada} de um vértice \(v\), denotado \(d^-(v)\), é o número de arcos que chegam a \(v\). O \textbf{grau de saída} \(d^+(v)\) é o número de arcos que saem de \(v\).

Uma \textbf{trilha} (caminho direcionado) é uma sequência de vértices \(P = (v_1, v_2, \ldots, v_{k+1})\) tal que cada par consecutivo \((v_i, v_{i+1})\) é um arco em \(A\).

Um \textbf{circuito} (ciclo direcionado) é uma trilha que começa e termina no mesmo vértice: \(v_1 = v_{k+1}\).

\begin{figure}[H]
    \centering
    \includegraphics[width=0.7\linewidth]{figures/fig_ciclo_direcionado.pdf}
    \caption{Ciclo direcionado: o ciclo $C=(v_1,v_2,v_3,v_4,v_1)$ está destacado.}
    \label{fig:ciclo-direcionado}
\end{figure}

\subsection{Componentes fortemente conexas}

Uma \textbf{componente fortemente conexa} (CFC) é um subdigrafo maximal onde existe um caminho direcionado entre qualquer par de vértices. Dois vértices \(u\) e \(v\) pertencem à mesma CFC se, e somente se, existe caminho de \(u\) para \(v\) e de \(v\) para \(u\).

Ao contrair cada CFC a um único vértice, obtemos o \textbf{grafo condensado}, que é sempre um DAG (grafo acíclico dirigido).

\begin{figure}[H]
    \centering
    \includegraphics[width=0.7\linewidth]{figures/figure_028.pdf}
    \caption{Componentes fortemente conexas: cada $C_i$ é fortemente conexo e maximal.}
\end{figure}

\section{Arborescências}

Uma \textbf{arborescência} enraizada em \(r\) é um digrafo acíclico onde existe exatamente um caminho direcionado de \(r\) para cada vértice \(v \in V \setminus \{r\}\). Equivalentemente, a raiz \(r\) tem grau de entrada zero (\(d^-(r) = 0\)), todo outro vértice tem grau de entrada um (\(d^-(v) = 1\) para \(v \neq r\)) e o número de arcos é \(|A| = |V| - 1\).

\begin{figure}[H]
    \centering
    \includegraphics[width=0.7\linewidth]{figures/fig_arborescencia.pdf}
    \caption{Arborescência enraizada em $r$: caminho único da raiz para cada vértice.}
    \label{fig:arborescencia}
\end{figure}

\subsection{Arborescência de custo mínimo}

Dado um digrafo \(D = (V, A)\) com função de custo \(c: A \to \mathbb{R}_{\geq 0}\) e raiz \(r \in V\), uma \textbf{arborescência de custo mínimo} é uma arborescência enraizada em \(r\) que minimiza o custo total:
\[
    C(T) = \sum_{a \in A_T} c(a)
\]

\begin{figure}[H]
    \centering
    \includegraphics[width=0.7\linewidth]{figures/fig_r_arborescencia_custo_minimo.pdf}
    \caption{Arborescência de custo mínimo enraizada em $r$ com custo total $17$.}
    \label{fig:r-arborescencia-custo-minimo}
\end{figure}

\section{Condições de Otimalidade}

\subsection{Custos reduzidos}

Para um conjunto \(X \subseteq V \setminus \{r\}\), definimos \(\delta^-(X)\) como o conjunto de arcos que entram em \(X\) vindos de fora:
\[
    \delta^-(X) = \{(u, v) \in A : u \notin X, v \in X\}
\]

Dada uma família de pesos \(y(X) \geq 0\) para cada subconjunto não vazio, o \textbf{custo reduzido} de um arco \(a = (u,v)\) é:
\[
    c'(a) = c(a) - \sum_{\substack{X\subseteq V\setminus\{r\},\\ u\notin X,\ v\in X}} y(X)
\]

\subsection{Teorema de Fulkerson}

Uma arborescência \(T\) enraizada em \(r\) tem custo mínimo se, e somente se, existem pesos \(y(X) \geq 0\) tais que:
\begin{enumerate}
    \item \(c'(a) \geq 0\) para todo arco \(a \in A\)
    \item \(c'(a) = 0\) para todo arco \(a \in T\)
    \item Para todo \(X\) com \(y(X) > 0\), exatamente um arco de \(T\) entra em \(X\)
\end{enumerate}

Este teorema fornece um certificado de otimalidade verificável: os custos reduzidos conectam a solução primal (arcos escolhidos) ao dual (pesos dos subconjuntos).

\section{Complexidade Computacional}

\subsection{Notação assintótica}

Expressamos a complexidade de algoritmos usando notação Big-O: \(O(f(n))\) denota limite superior (o tempo cresce no máximo proporcional a \(f(n)\)), \(\Omega(f(n))\) denota limite inferior (o tempo cresce no mínimo proporcional a \(f(n)\)) e \(\Theta(f(n))\) denota limite preciso (o tempo cresce exatamente proporcional a \(f(n)\)).

Para o problema da arborescência de custo mínimo, o algoritmo de Chu--Liu--Edmonds tem complexidade \(O(mn)\) na versão básica e \(O(m \log n)\) com heaps, enquanto o algoritmo de András Frank tem complexidade \(O(mn)\) na versão básica e \(O(m \log n)\) com heaps na Fase II, onde \(n = |V|\) e \(m = |A|\).

\subsection{Algoritmos gulosos}

Um \textbf{algoritmo guloso} toma decisões locais ótimas em cada etapa, esperando que levem a uma solução globalmente ótima. Suas características incluem escolha local baseada em critério de otimização imediato, decisões definitivas (não reconsidera escolhas anteriores) e eficiência, embora nem sempre garanta otimalidade global.

Os algoritmos de Chu--Liu--Edmonds e Frank são gulosos: selecionam arcos de menor custo (reduzido) e ajustam a estrutura iterativamente até obter a arborescência ótima.
