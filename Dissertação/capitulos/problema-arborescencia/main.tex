\chapter{O problema da Arborescência de Custo Mínimo}

O problema da arborescência de custo mínimo é um problema clássico em otimização combinatória e teoria dos grafos. O problema consiste em dado um digrafo enraizado \(D = (V, A)\) com custos \(c: A \to \mathbb{R}\) nos arcos e um vértice raiz \(r \in V\), o objetivo é encontrar uma arborescência de \(r\) que minimize a soma dos custos dos arcos selecionados.

\section{Definições fundamentais}

Uma \emph{arborescência} enraizada em \(r\) é um subgrafo conexo e acíclico de \(D\) que contém exatamente um caminho direcionado de \(r\) para cada vértice \(v \in V \setminus \{r\}\). Equivalentemente, é uma árvore geradora direcionada onde cada vértice, exceto a raiz, possui exatamente um arco de entrada \cite{schrijver2003comb}.

Formalmente, o problema consiste em encontrar um conjunto de arcos \(B \subseteq A\) tal que:
\begin{enumerate}
    \item \((V, B)\) forma uma arborescência enraizada em \(r\)
    \item O custo total \(\sum_{a \in B} c(a)\) é mínimo
\end{enumerate}

Este problema pode ser visto como uma generalização do problema da árvore geradora mínima para digrafos, mas apresenta características distintas devido à natureza direcionada dos arcos.

\section{Algoritmos clássicos}

Dois algoritmos fundamentais resolvem eficientemente o problema da arborescência de custo mínimo: o algoritmo de Chu--Liu--Edmonds e o algoritmo de András Frank.

\subsection{Algoritmo de Chu--Liu--Edmonds}

Proposto independentemente por Chu e Liu \cite{chu1965} e por Edmonds \cite{edmonds1967optimum}, este algoritmo opera iterativamente selecionando para cada vértice \(v \neq r\) um arco de entrada de custo mínimo. Quando ciclos são detectados, o algoritmo contrai o ciclo em um único vértice, ajusta os custos apropriadamente e resolve recursivamente o problema no grafo contraído. Ao final, expande os ciclos contraídos para reconstruir a arborescência ótima no grafo original.

\subsection{Algoritmo de András Frank}

O algoritmo de Frank \cite{frank1981,frank2014} adota uma abordagem diferente, operando em duas fases distintas. A Fase~I identifica \emph{subconjuntos minimais} via componentes fortemente conexas, reduzindo custos até criar arcos de custo reduzido zero. A Fase~II constrói a arborescência a partir desses arcos, garantindo otimalidade.

Esta abordagem processa múltiplos vértices simultaneamente, explorando a estrutura do grafo de forma mais global. A complexidade temporal também é \(O(mn)\), com possibilidade de melhoria para \(O(m \log n)\) usando filas de prioridade na Fase~II.

\section{Equivalência e diferenças estruturais}

Embora ambos os algoritmos produzam soluções ótimas, diferem significativamente em sua organização interna. Chu--Liu--Edmonds opera vértice a vértice com contrações imediatas de ciclos, enquanto Frank identifica subconjuntos minimais processando múltiplos vértices simultaneamente.

Ambos contraem ciclos detectados e expandem ao final, com número de contrações limitado por \(O(n)\) \cite{schrijver2003comb}. Esta limitação é fundamental para garantir a eficiência computacional dos métodos.

Os capítulos seguintes apresentam implementações detalhadas de ambos os algoritmos, comparando seus desempenhos experimentais e desenvolvendo ferramentas didáticas interativas para facilitar a compreensão de suas operações.



