\chapter{Conclusão}
\label{cap:conclusao}
Este projeto foi concebido com o propósito de apresentar de forma didática os algoritmos envolvidos no problema da $r$-arborescência geradora mínima e consideramos que alcançamos esse objetivo com sucesso.

Implementamos os algoritmos de Chu--Liu--Edmonds e András Frank, ambos fundamentais para a resolução do problema em questão. Através de uma análise detalhada e comparativa, conseguimos não apenas validar suas eficiências teóricas, mas também demonstrar suas performances práticas em diferentes cenários.

A partir desse desenvolvimento, criamos uma aplicação \textit{web} interativa que permite aos usuários visualizar e compreender passo a passo o funcionamento desses algoritmos. A ferramenta foi projetada para ser intuitiva e acessível, facilitando o aprendizado tanto para estudantes quanto para profissionais interessados na área.


\section*{Contribuições}

Este trabalho realiza a interseção entre teoria dos grafos e design pedagógico, culminando no desenvolvimento de um visualizador interativo de arborescências dirigidas que promove a compreensão de algoritmos cruciais para o problema de busca $r$-arborescência geradora mínima.

A primeira e principal contribuição dessa dissertação é a implementação detalhada dos algoritmos de Chu--Liu--Edmonds e András Frank, acompanhada de uma análise comparativa de suas eficiências teóricas e práticas. Essa análise não apenas valida os fundamentos matemáticos subjacentes, mas também oferece insights valiosos sobre o desempenho dos algoritmos em diferentes contextos.

A segunda contribuição significativa é a criação de uma aplicação \textit{web} interativa que serve como uma ferramenta educacional inovadora. Essa aplicação permite aos usuários explorar visualmente os algoritmos, facilitando a compreensão de conceitos complexos por meio de uma interface intuitiva e acessível. A integração de visualizações dinâmicas com explicações passo a passo representa um avanço na forma como algoritmos são ensinados e aprendidos.
Acreditamos que seja uma ferramenta que contribui para o ecossistemas de recursos educacionais em ciência da computação e teoria dos grafos.

\section*{Limitações}
Apesar dos avanços alcançados, a implementação apresenta algumas limitações que devem ser consideradas. A complexidade dos algoritmos pode levar a tempos de execução elevados para grafos muito grandes, o que pode impactar a experiência do usuário. Além disso, a interface, embora intuitiva, pode beneficiar-se de melhorias adicionais em termos de acessibilidade e usabilidade, especialmente para usuários com menos experiência em manipulação de grafos. Outro ponto de atenção está relacionado a dependência de bibliotecas externas, como Cytoscape.js e PyScript, que podem introduzir desafios de compatibilidade e manutenção a longo prazo. 

\section*{Trabalhos Futuros}
Para aprimorar a ferramenta, futuras iterações podem focar em otimizações de desempenho para lidar com digrafos maiores de forma mais eficiente. A interface pode ser refinada com base em testes de usabilidade, incorporando feedback de uma base de usuários diversificada. A adição de funcionalidades avançadas, como suporte a diferentes tipos de grafos e algoritmos adicionais, pode expandir o escopo da ferramenta. A integração de análises de aprendizado, como rastreamento do progresso do usuário e sugestões personalizadas, também pode enriquecer a experiência educacional. Finalmente, a realização de estudos formais para avaliar o impacto pedagógico da ferramenta em ambientes educacionais contribuirá para validar sua eficácia e orientar futuras melhorias.

\clearpage
\appendix
