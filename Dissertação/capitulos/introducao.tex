\chapter{Introdução}

Encontrar uma \textit{r-arborescência geradora mínimas} em digrafos é um problema clássico em Ciência da Computação estudado desde os anos 1960, com formulações fundamentais apresentadas por Jack Edmonds em 1967 \cite{edmonds}.

Nesse trabalho implementamos dois algoritmos para esse problema: o algoritmo de Chu--Liu--Edmonds \cite{chu,edmonds} e o procedimento em duas fases de András Frank \cite{frankco,frankks}. 

Além disso, desenvolvemos uma aplicação \textit{web} interativa que permite a visualização passo a passo dos algoritmos, facilitando o entendimento de suas operações e estruturas subjacentes.

\section{Estrutura do Trabalho}

Resumidamente, o trabalho abrange as seguintes frentes:

No Capítulo~\ref{cap:preliminares}, apresentamos definições e conceitos básicos sobre digrafos, arborescências etc, e estabelecemos a notação utilizada. 

No Capítulo~\ref{cap:chuliu-edmonds}, detalhamos o algoritmo de Chu--Liu--Edmonds e sua implemantação em \texttt{Python}.

No Capítulo~\ref{cap:frank}, detalhamos o procedimento em duas fases de András Frank e sua implementação em \texttt{Python}.

Discutimos no capítulo~\ref{cap:resultados} os resultados experimentais comparativos entre os dois algoritmos, avaliando tempo de execução, consumo de memória e características estruturais dos digrafos processados.

No Capítulo~\ref{cap:didatica}, exploramos fundamentos teóricos da didática de assuntos abstratos, destacando a importância de visualizações interativas para o ensino de algoritmos complexos.

No Capítulo~\ref{cap:implementacao-web}, descrevemos a implementação da aplicação \textit{web}, incluindo arquitetura, tecnologias utilizadas e aspectos de design centrado no usuário com base em princípios de interação humano-computacional.

Finalmente, no Capítulo~\ref{cap:conclusao}, apresentamos as conclusões do trabalho, suas contribuições, limitações e sugestões para trabalhos futuros.