\chapter{Introdução}

Encontrar uma \textit{r-arborescência geradora mínima} em digrafos é um problema clássico em Ciência da Computação estudado desde os anos 1960, com formulações fundamentais apresentadas por Jack Edmonds em 1967 \cite{edmonds}.

Neste trabalho implementamos dois métodos para esse problema: o algoritmo de Chu--Liu--Edmonds \cite{chu,edmonds} e o procedimento em duas fases de András Frank \cite{frankco,frankks}. 

Além disso, desenvolvemos uma aplicação \textit{web} interativa que permite a visualização passo a passo dessas metodologias, facilitando a compreensão de suas operações e estruturas subjacentes.

\section{Estrutura do Trabalho}

Resumidamente, esta dissertação abrange as seguintes frentes:

O capítulo~\ref{cap:preliminares} introduz definições e conceitos básicos sobre digrafos, arborescências etc, estabelecendo a notação empregada ao longo do texto. 

O capítulo~\ref{cap:chuliu-edmonds} detalha o algoritmo de Chu--Liu--Edmonds e sua implementação em \texttt{Python}.

O capítulo~\ref{cap:andras-frank} apresenta o procedimento em duas fases de András Frank e sua respectiva implementação computacional.

O capítulo~\ref{cap:resultados} discute os resultados experimentais comparativos entre as duas abordagens, avaliando desempenho temporal, consumo de memória e características estruturais dos digrafos processados.

O capítulo~\ref{cap:didatica} explora fundamentos teóricos da didática de assuntos abstratos, destacando a importância de visualizações interativas para o aprendizado de algoritmos complexos.

O capítulo~\ref{cap:implementacao-web} descreve a construção da aplicação \textit{web}, incluindo arquitetura, tecnologias empregadas e aspectos de design centrado no usuário com base em princípios de interação humano-computacional.

Finalmente, o capítulo~\ref{cap:conclusao} sintetiza as conclusões da pesquisa, suas contribuições, limitações e direções para trabalhos futuros.