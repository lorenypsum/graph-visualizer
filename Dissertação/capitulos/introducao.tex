\chapter{Introdução}

Encontrar uma \textit{r-arborescência geradora mínimas} em digrafos é um problema clássico em Ciência da Computação estudado desde os anos 1960, com formulações fundamentais apresentadas por Jack Edmonds em 1967 \cite{edmonds}.

Nesse trabalho implementamos dois algoritmos para esse problema: o algoritmo de Chu--Liu--Edmonds \cite{chu,edmonds} e o procedimento em duas fases de András Frank \cite{frankco,frankks}. 

Além disso, desenvolvemos uma aplicação \textit{web} interativa que permite a visualização passo a passo dos algoritmos, facilitando o entendimento de suas operações e estruturas subjacentes.


\section{Estrutura do Trabalho}

Resumidamente, o trabalho abrange as seguintes frentes:

No (Capítulo~\ref{cap:preliminares}), apresentamos definições e conceitos básicos sobre digrafos, arborescências, custos e dualidade.

No (Capítulo~\ref{cap:chuliu-edmonds}), detalhamos o algoritmo de Chu--Liu--Edmonds, enfatizando seus invariantes e a prova de correção.

\begin{enumerate}
    \item \textbf{Definições Preliminares}: fixação de notação e conceitos fundamentais (digrafos, arborescências, custos e dualidade).
    \item \textbf{Algoritmo de Chu--Liu/Edmonds}: detalhamento da abordagem primal por normalização/contração de ciclos, com ênfase em invariantes e correção.
    \item \textbf{Procedimento em Duas Fases de András Frank}: detalhamento da abordagem dual por cortes c-viáveis, com ênfase em invariantes e correção.
    \item \textbf{Discussão e Resultados: Chu-liu/Edmonds vs Frank}: comparação de desempenho e estrutura das soluções, com testes em larga escala.
    \item \textbf{Didática do Abstrato}: discussão de desafios do ensino de matemática no ensino superior especialmente quando há o envolvimento de conceitos abstratos e como representações visuais atuam em favor da compreensão.
    \item \textbf{A interação humano--computacional em ação - uma aplicação \textit{web} interativa}: apresentação de conceitos em interação humano-computacional, a influência desses conceitos nas decisões tomadas para o desenvolvimento da aplicação \textit{web} e detalhamento da implementação.
    \item \textbf{Conclusão}: síntese das contribuições, limitações e trabalhos futuros.
\end{enumerate}

Deste modo, o trabalho entrega implementações verificadas de Chu--Liu/Edmonds e András Frank, um visualizador \textit{web} interativo e testes de volume que confirmam a equivalência de custos, úteis ao estudo e ao ensino de arborescências.
