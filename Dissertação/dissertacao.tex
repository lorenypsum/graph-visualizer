\documentclass[
	% -- opções da classe memoir --
	11pt,					% tamanho da fonte
	openright,				% capítulos começam em pág ímpar (insere página vazia caso preciso)
	oneside,				% para impressão só frente
	a4paper,				% tamanho do papel. 
	% -- opções da classe abntex2 --
	%chapter=TITLE,			% títulos de capítulos convertidos em letras maiúsculas
	%section=TITLE,			% títulos de seções convertidos em letras maiúsculas
	%subsection=TITLE,		% títulos de subseções convertidos em letras maiúsculas
	%subsubsection=TITLE,	% títulos de subsubseções convertidos em letras maiúsculas
	% -- opções do pacote babel --
	english,				% idioma adicional para hifenização
	%french,				% idioma adicional para hifenização
	%spanish,				% idioma adicional para hifenização
	brazil					% o último idioma é o principal do documento
	]{abntex2}

% ---------------------
% Pacotes OBRIGATÓRIOS
% ---------------------
%\usepackage{lmodern}
\usepackage[adobe-utopia]{mathdesign}		
\usepackage[T1]{fontenc}			% Selecao de codigos de fonte.
\usepackage[utf8]{inputenc}		% Codificacao do documento (conversão automática dos acentos)
\usepackage{lastpage}			% Usado pela Ficha catalográfica
\usepackage{indentfirst}			% Indenta o primeiro parágrafo de cada seção.
\usepackage{color}				% Controle das cores
\usepackage{graphicx}	        % Inclusão de gráficos

% Configuração global para limitar o tamanho das figuras
\makeatletter
\def\maxwidth{\ifdim\Gin@nat@width>\linewidth \linewidth \else \Gin@nat@width \fi}
\def\maxheight{\ifdim\Gin@nat@height>0.75\textheight 0.75\textheight \else \Gin@nat@height \fi}
\makeatother
\setkeys{Gin}{width=\maxwidth,height=\maxheight,keepaspectratio}

\usepackage{epsfig,subfig}		% Inclusão de figuras
\usepackage{microtype} 			% Melhorias de justificação
\usepackage{verbatim}           %comentarios de mais de uma linha
% ---------------------

% ---------------------
% Pacotes ADICIONAIS
% ---------------------
\usepackage{lipsum}						% Geração de dummy text
\usepackage{siunitx,amsmath,mathrsfs,cleveref}	% Comandos matemáticos avançados 
\usepackage{setspace}  					% Para permitir espaçamento simples, 1 1/2 e duplo
\usepackage{tabularx} 					% Para poder ter tabelas com colunas de largura auto-ajustável
\usepackage{afterpage} 					% Para executar um comando depois do fim da página corrente
\usepackage{url} 						% Para formatar URLs (endereços da Web)
\usepackage[dvipsnames]{xcolor}
\usepackage{soul}
\sethlcolor{White}
\usepackage{float}                      % Para colocar imagens exatamente na posição desejada
\usepackage{mathtools}                  % Símbolos matemáticos
\usepackage{colortbl}                   % Colorir células de tabelas
\usepackage{color}
\usepackage{array}
\usepackage{calrsfs}
% ---------------------
%\usepackage{times}
%\usepackage{newpxtext}
%\usepackage{mathptmx}
\usepackage{svg}

% tracejado e riscado
\usepackage{cancel}
\usepackage[normalem]{ulem}
% ---------------------
% Pacotes de CITAÇÕES
% ---------------------
\usepackage[brazilian,hyperpageref]{backref}	% Paginas com as citações na bibl
\usepackage[alf,abnt-etal-cite=2]{abntex2cite}				% Citações padrão ABNT (alfa)
%\usepackage[num]{abntex2cite}				% Citações padrão ABNT (numericas)
\usepackage{algorithm}
\usepackage{algpseudocode}
\usepackage{multirow}
\usepackage{amsthm}
\usepackage{enumitem}
% ---------------------

\usepackage{listings}
\lstset{
  literate=%
   {á}{{\'a}}1 {â}{{\^a}}1 {ã}{{\~a}}1 {à}{{\`a}}1
   {é}{{\'e}}1 {ê}{{\^e}}1
   {í}{{\'i}}1
   {ó}{{\'o}}1 {ô}{{\^o}}1 {õ}{{\~o}}1
   {ú}{{\'u}}1
   {ç}{{\c{c}}}1
}

\usepackage{booktabs}
%\usepackage{makeidx}
%\usepackage{imakeidx}
\usepackage[scaled=0.8]{newtxtt}
\usepackage[scaled=0.85]{beramono}
\usepackage{tikz}
\usetikzlibrary{fit}
\usetikzlibrary{graphs,arrows.meta,positioning,fit,calc,decorations.pathmorphing}

%\usegdlibrary{force}
\usepackage{bbm}

\usepackage[most]{tcolorbox}
\usepackage{etoolbox}
\usepackage{needspace}
\usepackage{multicol}
\usepackage{booktabs}
\usepackage{adjustbox}

\usepackage{rotating} % para sidewaystable
\usepackage{caption}
\usepackage{booktabs} % para melhor aparência das tabelas		
\robustify{\citeonline}

% Macros e comandos personalizados

% Comandos matemáticos
\newcommand{\N}{\mathbb{N}}
\newcommand{\Z}{\mathbb{Z}}
\newcommand{\Q}{\mathbb{Q}}
\newcommand{\R}{\mathbb{R}}
\newcommand{\C}{\mathbb{C}}

% Comandos para grafos e relações
\newcommand{\grafo}[1]{\mathcal{#1}}
\newcommand{\vertices}[1]{V(#1)}
\newcommand{\arestas}[1]{E(#1)}
\newcommand{\pto}{\mathrel{\rightharpoonup}}
\newcommand{\dom}{\operatorname{dom}}
\newcommand{\dist}{\operatorname{dist}}
\newcommand{\seq}[1]{\langle #1 \rangle}

% Definição de ambientes personalizados

\newtheorem{teorema}{Teorema}[chapter]
\newtheorem{lema}[teorema]{Lema}
\newtheorem{proposicao}[teorema]{Proposição}
\newtheorem{corolario}[teorema]{Corolário}

\theoremstyle{definition}
\newtheorem{definicao}{Definição}[chapter]
\newtheorem{exemplo}{Exemplo}[chapter]

\theoremstyle{remark}
\newtheorem{observacao}{Observação}[chapter]


% Configurações de CITAÇÕES para abntex2
\input{extras/conf_citacoes}

% Inclusão de dados para CAPA e FOLHA DE ROSTO (título, autor, orientador, etc.)
% Dados para capa e folha de rosto

\titulo{Análise e Implementação de Algoritmos de Busca de uma r-Arborescência Inversa de Custo Mínimo em Grafos Dirigidos com Aplicação Didática Interativa}

\autor{Lorena Silva Sampaio \and Samira Haddad}

\local{Brasil}
\data{2025}

\orientador{Prof. Dr. Mário Leston}

\instituicao{%
	Universidade
	\par
	Faculdade
	\par
	Programa de Pós-Graduação}

\tipotrabalho{Dissertação (Mestrado)}

\preambulo{Dissertação apresentada ao Programa de Pós-Graduação como requisito parcial para obtenção do título de Mestre.}


% Inclui Configurações de aparência do PDF Final
% Configurações de aparência do PDF final

\makeatletter
\hypersetup{
	pdftitle={\@title},
	pdfauthor={\@author},
	pdfsubject={\imprimirpreambulo},
	pdfcreator={LaTeX with abnTeX2},
	pdfkeywords={abnt}{latex}{abntex}{abntex2}{dissertação},
	colorlinks=true,
	linkcolor=blue,
	citecolor=blue,
	filecolor=magenta,
	urlcolor=blue,
	bookmarksdepth=4
}
\makeatother

% Configurações para código fonte

\lstdefinestyle{pystyle}{
language=Python,
basicstyle=\ttfamily\small,
numbers=left,
numbersep=6pt,
tabsize=4,
showstringspaces=false,
breaklines=true,
keywordstyle=\color{blue!70!black}\bfseries,
commentstyle=\color{green!50!black}\itshape,
stringstyle=\color{red!60!black},
inputencoding=utf8,
extendedchars=true,
% Mapear caracteres UTF-8 comuns em PT/BR e símbolos especiais
literate={á}{{\'{a}}}1 {é}{{\'{e}}}1 {í}{{\'{\i}}}1 {ó}{{\'{o}}}1 {ú}{{\'{u}}}1
{Á}{{\'{A}}}1 {É}{{\'{E}}}1 {Í}{{\'{\I}}}1 {Ó}{{\'{O}}}1 {Ú}{{\'{U}}}1
{à}{{\`{a}}}1 {À}{{\`{A}}}1 {â}{{\^{a}}}1 {Â}{{\^{A}}}1 {ã}{{\~{a}}}1 {Ã}{{\~{A}}}1
{ê}{{\^{e}}}1 {Ê}{{\^{E}}}1 {ô}{{\^{o}}}1 {Ô}{{\^{O}}}1 {õ}{{\~{o}}}1 {Õ}{{\~{O}}}1
{ç}{{\c{c}}}1 {Ç}{{\c{C}}}1 {ü}{{\"{u}}}1 {Ü}{{\"{U}}}1
{ñ}{{\~{n}}}1 {Ñ}{{\~{N}}}1
{"}{{``}}1 {"}{{''}}1 {'}{{'}}1 {–}{{-}}1 {—}{{-}}1 {…}{{...}}1
{→}{{$\to$}}1 {←}{{$\leftarrow$}}1 {↔}{{$\leftrightarrow$}}1
{∈}{{$\in$}}1 {∉}{{$\notin$}}1 {⊆}{{$\subseteq$}}1
{∪}{{$\cup$}}1 {∩}{{$\cap$}}1 {∅}{{$\emptyset$}}1
{≤}{{$\leq$}}1 {≥}{{$\geq$}}1 {≠}{{$\neq$}}1
{✓}{{[OK]}}1 {✗}{{[X]}}1 {✅}{{$\checkmark$}}1 {❌}{{$\times$}}1
}

\lstset{style=pystyle}


% ----------------------------------------------------------
% Ajuste de margens para melhorar a legibilidade
% (linhas menos compridas)
% ----------------------------------------------------------
\setlrmarginsandblock{3.5cm}{3.5cm}{*}  % margens esquerda/direita
\setulmarginsandblock{3cm}{3cm}{*}      % margens superior/inferior
\checkandfixthelayout

% O tamanho da identação do parágrafo é dado por:
\setlength{\parindent}{1.3cm}

% Controle do espaçamento entre um parágrafo e outro:
\setlength{\parskip}{0.2cm}  % tente também \onelineskip

% ---------------------
% Compila o indice
% ---------------------
\makeindex
%\makeindex[program=makeidx,options=-s indexstyle,columns=2,intoc=true]
% ---------------------

%%%%%%%%%%%%%%%%%%%%%%%%%%%
%%  INICIO DO DOCUMENTO  %%
%%%%%%%%%%%%%%%%%%%%%%%%%%%
\begin{document}

% Retira espaço extra obsoleto entre as frases.
\frenchspacing


% Espaçamento 1,5 entre linhas (confortável para leitura de TCC)
\OnehalfSpacing

% ----------------------------------------------------------
% ELEMENTOS PRÉ-TEXTUAIS (Capa, Resumo, Abstract, etc.)
% ----------------------------------------------------------
\pretextual

% Capa
\include{pretextual/capa}

% Folha de rosto (o * indica que haverá a ficha bibliográfica)
\imprimirfolhaderosto*

\begin{comment}
% Imprimir Ficha Catalografica
% Ficha catalográfica
\begin{fichacatalografica}
	\vspace*{\fill}
	\begin{center}
		Ficha catalográfica a ser elaborada pela biblioteca.
	\end{center}
	\vspace*{\fill}
\end{fichacatalografica}


% Inserir Folha de Aprovação
% Folha de aprovação
\begin{folhadeaprovacao}
	\begin{center}
		{\ABNTEXchapterfont\large\imprimirautor}

		\vspace*{\fill}\vspace*{\fill}
		\begin{center}
			\ABNTEXchapterfont\bfseries\Large\imprimirtitulo
		\end{center}
		\vspace*{\fill}

		\hspace{.45\textwidth}
		\begin{minipage}{.5\textwidth}
			\imprimirpreambulo
		\end{minipage}%
		\vspace*{\fill}
	\end{center}

	Trabalho aprovado. \imprimirlocal, \today:

	\assinatura{\textbf{Prof. Dr. Mário Leston Rey} \\ Orientador}
	\assinatura{\textbf{Profa. Dra. Cristiane Maria Sato} \\ Convidada}
	\assinatura{\textbf{Prof. Dr. Aritanan Borges Garcia Gruber} \\ Convidado}

	\begin{center}
		\vspace*{0.5cm}
		{\large\imprimirlocal}
		\par
		{\large\imprimirdata}
		\vspace*{1cm}
	\end{center}

\end{folhadeaprovacao}

\end{comment}

% Dedicatória
% Dedicatória
\begin{dedicatoria}
	\vspace*{\fill}
	\centering
	\noindent
	\textit{Dedicatória (opcional).}
	\vspace*{\fill}
\end{dedicatoria}

% Agradecimentos
% Agradecimentos
\begin{agradecimentos}
	Agradecimentos (opcional).
\end{agradecimentos}

% Epígrafe
% Epígrafe
\begin{epigrafe}
	\vspace*{\fill}
	\begin{flushright}
		\textit{``Epígrafe (opcional).''}
	\end{flushright}
\end{epigrafe}

% Inserir Folha de Aprovação
% Folha de aprovação
\begin{folhadeaprovacao}
	\begin{center}
		{\ABNTEXchapterfont\large\imprimirautor}

		\vspace*{\fill}\vspace*{\fill}
		\begin{center}
			\ABNTEXchapterfont\bfseries\Large\imprimirtitulo
		\end{center}
		\vspace*{\fill}

		\hspace{.45\textwidth}
		\begin{minipage}{.5\textwidth}
			\imprimirpreambulo
		\end{minipage}%
		\vspace*{\fill}
	\end{center}

	Trabalho aprovado. \imprimirlocal, \today:

	\assinatura{\textbf{Prof. Dr. Mário Leston Rey} \\ Orientador}
	\assinatura{\textbf{Profa. Dra. Cristiane Maria Sato} \\ Convidada}
	\assinatura{\textbf{Prof. Dr. Aritanan Borges Garcia Gruber} \\ Convidado}

	\begin{center}
		\vspace*{0.5cm}
		{\large\imprimirlocal}
		\par
		{\large\imprimirdata}
		\vspace*{1cm}
	\end{center}

\end{folhadeaprovacao}

% Resumo e Abstract
% Resumo em português
\setlength{\absparsep}{18pt}
\begin{resumo}
	Este trabalho apresenta uma análise e implementação de algoritmos de busca de r-arborescência geradora mínima em digrafos com aplicação didática interativa.

	\textbf{Palavras-chave}: Digrafos. r-Arborescência Geradora. Algoritmos. Aplicação Web.
\end{resumo}

% Resumo em inglês
\begin{resumo}[Abstract]
	\begin{otherlanguage*}{english}
		This work presents an analysis and implementation of algorithms for finding a minimum spanning r-arborescence in digraphs with interactive didactic application.

		\textbf{Keywords}: Digraphs. Spanning r-Arborescence. Algorithms. Web Application.
	\end{otherlanguage*}
\end{resumo}


%Lista de ilustrações
\pdfbookmark[0]{\listfigurename}{lof}
\listoffigures*
\cleardoublepage

% Lista de abreviaturas e siglas
%\include{pretextual/siglas}

\begin{comment}
% Lista de tabelas
\pdfbookmark[0]{\listtablename}{lot}
\listoftables
\cleardoublepage


% Lista de símbolos
\begin{simbolos}
    \item[$ \Gamma $] Letra grega Gama
    \item[$ \Lambda $] Lambda
    \item[$ \zeta $] Letra grega minúscula zeta
    \item[$ \in $] Pertence
\end{simbolos}
\end{comment}

% Inserir o SUMÁRIO
\pdfbookmark[0]{\contentsname}{toc}
\tableofcontents*
\cleardoublepage

% ----------------------------------------------------------
% ELEMENTOS TEXTUAIS (Capítulos)
% ----------------------------------------------------------
\textual
% Elementos textuais com numeração arábica
% A classe abntex2 já gerencia a numeração de páginas automaticamente

% Inclui cada capitulo do Trabalho
\chapter{Introdução}
\label{cap:introducao}

Este trabalho tem como objetivo apresentar uma análise e implementação de algoritmos de busca de uma r-arborescência inversa de custo mínimo em grafos dirigidos.

\section{Motivação}

A teoria dos grafos é uma área fundamental da matemática discreta com diversas aplicações práticas.

\section{Objetivos}

O objetivo principal deste trabalho é...

\section{Organização do Trabalho}

Este trabalho está organizado da seguinte forma...

\chapter{Algoritmo de András Frank}

Neste capítulo, apresentaremos o algoritmo de András Frank, que também determina
uma arborescência de custo mínimo em um digrafo ponderado.
O algoritmo baseia-se em uma abordagem primal--dual:
a elevação gulosa de potenciais duais\footnote{\textbf{Potencial dual} é uma variável associada a cada vértice, que representa um "desconto" aplicado ao custo dos arcos que entram nesse vértice e servem para ajustar os custos de modo que certas restrições fiquem "justas", - satisfeitas com igualdade, ou seja, o valor da variável atinge exatamente o limite imposto pela restrição.} e (ii) a construção de uma solução gulosa
a partir dos arcos de custo reduzido zero.
Diferentemente do algoritmo de Chu--Liu--Edmonds, que opera diretamente sobre os custos,
o método de Frank manipula potenciais duais associados aos vértices,
gerando arcos \emph{justos} (de custo reduzido zero) que formam a arborescência ótima.
O propósito deste capítulo é fornecer uma descrição precisa tanto do algoritmo
quanto da implementação desenvolvida neste trabalho.

\section{O algoritmo}
O algoritmo de András Frank também recebe uma tripla \((D,c,r)\), em que \(D=(V,A)\) é um digrafo,
\(c\colon A\to\mathbb{R}\) é uma função custo e \(r\in V\) é a raiz,
sob a hipótese de que \(D\) admite ao menos uma \(r\)-arborescência e devolve uma \(r\)-arborescência \(c\)-mínima de \(D\).

Assim como no capítulo anterior, adotamos a terminologia de \textbf{\(r\)-digrafo ponderado}
para uma tripla \((D,c,r)\) em que $(D, c)$ é um digrafo ponderado,
$r$ é um vértice de $D$, $\delta^-(r) = \varnothing$
e $D$ possui uma $r$-arborescência.

Vamos desenvolver as ideias do algoritmo utilizando o mesmo dígrafo que apresentamos
no capítulo anterior.

\subsection*{Abordagem Gulosa--Dual}

Considere novamente o dígrafo \(D\) figura 26 a seguir, com custos nos arcos note que é o mesmo digrafo da figura 1 que utilizamos para ilustrar o algoritmo de Chu--Liu--Edmonds. E portanto, suponha que T é uma arborescência de custo mínimo, com seus arcos destacados em azul.

\begin{figure}[h]
	\centering
	\begin{tikzpicture}[>=Stealth]
		% estilos
		\tikzset{
			vtx/.style={circle, draw,fill=blue!8, thick, inner sep=1.2pt, minimum size=6mm},
			chosen/.style={->, very thick, draw=blue},
			other/.style={->, semithick, draw=black!35},
			costB/.style={font=\scriptsize, fill=white, inner sep=1pt, text=blue},
			costG/.style={font=\scriptsize, fill=white, inner sep=1pt, text=black!55}
		}

		% layout (mais espaçado)
		\node[vtx] (r) at (0,0) {$r$};
		\node[vtx] (a) at (3,1.6) {$a$};
		\node[vtx] (b) at (3,-1.6) {$b$};
		\node[vtx] (c) at (6,2.2) {$c$};
		\node[vtx] (d) at (6,0.0) {$d$};
		\node[vtx] (e) at (6,-2.2) {$e$};

		% --- arcos NÃO escolhidos (cinza), agora com curvaturas mais limpas ---
		\draw[other] (b) to[bend left=12] node[costG, above] {4} (a);   % b->a
		\draw[other] (a) to[bend left=14] node[costG, below] {5} (b);   % a->b

		% r->c: externo por cima, suave (pontos de controle explícitos)
		\draw[other]
		(r) .. controls (0.8,2.4) and (5.2,3.0) ..
		node[costG, yshift=4pt, pos=0.53] {3} (c);

		% a->d (referência)
		\draw[other] (a) -- node[costG, right] {3} (d);

		% c->e: leve curvatura pela direita (sem "barriga")
		\draw[other]
		(c) .. controls (7.6,2.6) and (7.6,-2.6) ..
		node[costG, xshift=8pt, pos=0.52] {5} (e);

		% --- arcos ESCOLHIDOS (AZUL) — formam a r-arborescência T ---
		\draw[chosen] (r) -- node[costB, above] {1} (a);   % r->a
		\draw[chosen] (r) -- node[costB, below] {2} (b);   % r->b
		\draw[chosen] (a) -- node[costB, above] {1} (c);   % a->c
		\draw[chosen] (b) -- node[costB, above] {1} (d);   % b->d
		\draw[chosen] (b) -- node[costB, below] {2} (e);   % b->e
	\end{tikzpicture}

	\caption{O dígrafo \(D\) com custos nos arcos. Os arcos em \textcolor{blue}{azul}
		formam a \(r\)-arborescência de custo mínimo \(T\) (custo total = 7).}
\end{figure}

O algoritmo de Frank utiliza \textbf{potenciais} \(y: V \to \mathbb{R}\) associados aos vértices
para definir custos reduzidos nos arcos. Para cada arco \(a = (u,v) \in A\), o
\textbf{custo \(y\)-reduzido} é dado por
\[
	c_y(u,v) := c(u,v) - y(v),
\]
onde \(v\) é o vértice de destino do arco. Note que, diferentemente do procedimento de redução do capítulo anterior (onde subtraíamos o mínimo custo de entrada em cada vértice), aqui o potencial \(y(v)\) é uma variável que será ajustada pelo algoritmo através da seguinte noção.

Um arco \(a = (u,v)\) é dito \textbf{justo} (ou \textbf{tight}) se \(c_y(u,v) = 0\), isto é, se o custo original coincide com o potencial do vértice de destino: \(c(u,v) = y(v)\). A ideia central do algoritmo de Frank consiste em duas etapas: (i)~elevar os potenciais \(y(v)\) para cada \(v \neq r\) até que cada vértice possua ao menos um arco justo entrando nele; (ii)~construir uma arborescência ótima utilizando exclusivamente arcos justos.

\begin{tcolorbox}[
		enhanced, breakable,
		colback=blue!5!white, colframe=blue!70!black,
		colbacktitle=blue!85!black, coltitle=white,
		fonttitle=\bfseries,
		title={Arcos justos: interpretação e cálculo},
		sharp corners, boxrule=0.8pt
	]
	Um arco \((u,v)\) é \textbf{justo} quando seu custo reduzido é zero, ou seja, quando o custo original \(c(u,v)\) coincide exatamente com o potencial \(y(v)\) do vértice de destino. Formalmente:
	\[
		c_y(u,v) = c(u,v) - y(v) = 0 \quad\Longleftrightarrow\quad c(u,v) = y(v).
	\]

	Em outras palavras, o potencial \(y(v)\) funciona como um ``desconto'' aplicado aos arcos que chegam em \(v\). Quando esse desconto é suficiente para anular o custo reduzido, o arco torna-se justo e candidato a participar da solução ótima.

	\medskip
	\textbf{Exemplos numéricos (referência: figura~26):}
	\begin{itemize}\setlength{\itemsep}{3pt}
		\item Arco \((r,a)\): \(c(r,a) = 1\), \(y(a) = 1 \implies c_y(r,a) = 1-1 = 0\). \quad\textcolor{blue!70!black}{\textbf{Justo.}}
		\item Arco \((b,a)\): \(c(b,a) = 4\), \(y(a) = 1 \implies c_y(b,a) = 4-1 = 3\). \quad\textcolor{red!70!black}{\textbf{Não justo.}}
		\item Arco \((a,c)\): \(c(a,c) = 1\), \(y(c) = 1 \implies c_y(a,c) = 1-1 = 0\). \quad\textcolor{blue!70!black}{\textbf{Justo.}}
	\end{itemize}

	\textbf{Propriedade fundamental:} o potencial \(y(v)\) afeta \emph{exclusivamente} os arcos que entram em \(v\). Por exemplo, \(y(c)\) influencia apenas arcos da forma \((\cdot, c)\), como \((r,c)\) e \((a,c)\), não afetando arcos que saem de \(c\).

	\medskip
	O algoritmo eleva progressivamente os potenciais até que cada vértice \(v \neq r\) possua ao menos um arco justo de entrada. No exemplo da figura~26, os arcos justos após a elevação correspondem precisamente aos arcos da arborescência ótima.
\end{tcolorbox}

% Figura 27: Exemplo visual com múltiplos arcos e vértice raiz
\begin{figure}[H]
	\centering
	\begin{tikzpicture}[>=Stealth]
		% estilos
		\tikzset{
			vtx/.style={circle, draw, fill=blue!8, thick, inner sep=1.2pt, minimum size=8mm},
			arc/.style={->, very thick, draw=blue},
			arcTight/.style={->, very thick, draw=red!70!black},
			cost/.style={font=\scriptsize, fill=white, inner sep=1pt, text=blue},
			pot/.style={font=\small, text=red!70!black},
			label/.style={font=\small, text=black!80}
		}

		% vértices
		\node[vtx] (r) at (0,0) {$r$};
		\node[vtx] (u) at (3,1.5) {$u$};
		\node[vtx] (v) at (3,-1.5) {$v$};

		% arcos
		\draw[arcTight] (r) -- node[cost, above left] {$c(r,u)=2$} (u);
		\draw[arc] (r) -- node[cost, below left] {$c(r,v)=4$} (v);
		\draw[arcTight] (u) -- node[cost, right] {$c(u,v)=3$} (v);

		% potenciais
		\node[pot] at (3,2.5) {$y(u)=2$};
		\node[pot] at (3,-2.5) {$y(v)=3$};

		% destaques
		\draw[red!70, thick, dashed] (u) circle [radius=0.7];
		\draw[red!70, thick, dashed] (v) circle [radius=0.7];

		% custos reduzidos
		\node[label] at (-0.0,1.5) {$c_y(r,u)=2-2=0$};
		\node[label] at (-0.0,-1.5) {$c_y(r,v)=4-3=1$};
		\node[label] at (5.0,-0.5) {$c_y(u,v)=3-3=0$};
	\end{tikzpicture}
	\caption{O vértice $r$ é a raiz. Os potenciais são $y(u)=2$ e $y(v)=3$. O arco $(r,u)$ tem custo reduzido $0$ (justo), $(r,v)$ tem custo reduzido $1$ (não justo), $(u,v)$ tem custo reduzido $0$ (justo). Os potenciais só afetam os arcos que chegam em cada vértice.}
	\label{fig:potencial-destino}
\end{figure}

\subsection*{Elevação de Potenciais}

Inicialmente, definimos \(y(v) = 0\) para todo \(v \in V\).
Para cada vértice \(v \neq r\), calculamos
\[
	\Delta(v) := \min\{\, c(a) : a \in \delta^{-}(v) \,\} - y(v).
\]
O valor \(\Delta(v)\) representa quanto precisamos elevar \(y(v)\) para que exista
ao menos um arco de custo reduzido zero entrando em \(v\).

Vamos ilustrar esse processo no dígrafo \(D\).

\begin{figure}[H]
	\centering
	\begin{tikzpicture}[>=Stealth]
		% estilos
		\tikzset{
			vtx/.style={circle, draw,fill=blue!8, thick, inner sep=1.2pt, minimum size=6mm},
			arc/.style={->, semithick, draw=black!35},
			costG/.style={font=\scriptsize, fill=white, inner sep=1pt, text=black!55}
		}

		% layout
		\node[vtx] (r) at (0,0) {$r$};
		\node[vtx] (a) at (3,1.6) {$a$};
		\node[vtx] (b) at (3,-1.6) {$b$};
		\node[vtx] (c) at (6,2.2) {$c$};
		\node[vtx] (d) at (6,0.0) {$d$};
		\node[vtx] (e) at (6,-2.2) {$e$};

		% arcos com custos originais
		\draw[arc] (r) -- node[costG, above] {1} (a);
		\draw[arc] (r) -- node[costG, below] {2} (b);
		\draw[arc] (r) .. controls (0.8,2.4) and (5.2,3.0) .. node[costG, yshift=4pt, pos=0.53] {3} (c);
		\draw[arc] (b) to[bend left=12] node[costG, above] {4} (a);
		\draw[arc] (a) to[bend left=14] node[costG, below] {5} (b);
		\draw[arc] (a) -- node[costG, above] {1} (c);
		\draw[arc] (a) -- node[costG, right] {3} (d);
		\draw[arc] (b) -- node[costG, above] {1} (d);
		\draw[arc] (b) -- node[costG, below] {2} (e);
		\draw[arc] (c) .. controls (7.6,2.6) and (7.6,-2.6) .. node[costG, xshift=8pt, pos=0.52] {5} (e);

		% potenciais iniciais
		\node[above=6mm of a, font=\small] {$y(a) = 0$};
		\node[below=2mm of b, font=\small] {$y(b) = 0$};
		\node[above=2mm of c, font=\small] {$y(c) = 0$};
		\node[below=2mm of d, font=\small] {$y(d) = 0$};
		\node[below=2mm of e, font=\small] {$y(e) = 0$};
	\end{tikzpicture}
	\caption{Dígrafo \(D\) com potenciais iniciais \(y(v) = 0\) para todo \(v \in V\).}
\end{figure}

Calculamos \(\Delta(v)\) para cada vértice \(v \neq r\):
\begin{align*}
	\Delta(a) & = \min\{1, 4\} - 0 = 1, \\
	\Delta(b) & = \min\{2, 5\} - 0 = 2, \\
	\Delta(c) & = \min\{3, 1\} - 0 = 1, \\
	\Delta(d) & = \min\{3, 1\} - 0 = 1, \\
	\Delta(e) & = \min\{2, 5\} - 0 = 2.
\end{align*}

Elevamos cada potencial:
\begin{align*}
	y(a) & \leftarrow y(a) + \Delta(a) = 0 + 1 = 1, \\
	y(b) & \leftarrow y(b) + \Delta(b) = 0 + 2 = 2, \\
	y(c) & \leftarrow y(c) + \Delta(c) = 0 + 1 = 1, \\
	y(d) & \leftarrow y(d) + \Delta(d) = 0 + 1 = 1, \\
	y(e) & \leftarrow y(e) + \Delta(e) = 0 + 2 = 2.
\end{align*}

Após a elevação, os custos reduzidos são dados por \(c_y(u,v) = c(u,v) - y(v)\) para cada arco \((u,v)\):

\begin{figure}[H]
	\centering
	\begin{tikzpicture}[>=Stealth]
		% estilos
		\tikzset{
			vtx/.style={circle, draw,fill=blue!8, thick, inner sep=1.2pt, minimum size=6mm},
			tight/.style={->, very thick, draw=blue},
			loose/.style={->, semithick, draw=black!35},
			costB/.style={font=\scriptsize, fill=white, inner sep=1pt, text=blue},
			costG/.style={font=\scriptsize, fill=white, inner sep=1pt, text=black!55}
		}

		% layout
		\node[vtx] (r) at (0,0) {$r$};
		\node[vtx] (a) at (3,1.6) {$a$};
		\node[vtx] (b) at (3,-1.6) {$b$};
		\node[vtx] (c) at (6,2.2) {$c$};
		\node[vtx] (d) at (6,0.0) {$d$};
		\node[vtx] (e) at (6,-2.2) {$e$};

		% arcos justos (custo reduzido = 0)
		\draw[tight] (r) -- node[costB, above] {0} (a);   % 1 - 1 = 0
		\draw[tight] (r) -- node[costB, below] {0} (b);   % 2 - 2 = 0
		\draw[tight] (a) -- node[costB, above] {0} (c);   % 1 - 1 = 0
		\draw[tight] (b) -- node[costB, above] {0} (d);   % 1 - 1 = 0
		\draw[tight] (b) -- node[costB, below] {0} (e);   % 2 - 2 = 0

		% arcos não justos
		\draw[loose] (r) .. controls (0.8,2.4) and (5.2,3.0) .. node[costG, yshift=4pt, pos=0.53] {2} (c);  % 3 - 1 = 2
		\draw[loose] (b) to[bend left=12] node[costG, above] {3} (a);   % 4 - 1 = 3
		\draw[loose] (a) to[bend left=14] node[costG, below] {3} (b);   % 5 - 2 = 3
		\draw[loose] (a) -- node[costG, right] {2} (d);   % 3 - 1 = 2
		\draw[loose] (c) .. controls (7.6,2.6) and (7.6,-2.6) .. node[costG, xshift=8pt, pos=0.52] {3} (e);  % 5 - 2 = 3

		% potenciais finais
		\node[above=6mm of a, font=\small] {$y(a) = 1$};
		\node[below=2mm of b, font=\small] {$y(b) = 2$};
		\node[above=2mm of c, font=\small] {$y(c) = 1$};
		\node[below=2mm of d, font=\small] {$y(d) = 1$};
		\node[below=2mm of e, font=\small] {$y(e) = 2$};
	\end{tikzpicture}
	\caption{Dígrafo após elevação de potenciais. Arcos em \textcolor{blue}{azul} são
		justos (\(c_y = 0\)); arcos em cinza têm custo reduzido positivo.}
\end{figure}

Observe que cada vértice \(v \neq r\) possui agora ao menos um arco justo entrando nele.
Mais ainda: neste exemplo, os arcos justos formam uma \(r\)-arborescência! Precisamos justificar por que a arborescência formada pelos arcos justos é ótima.

\subsection*{Construção da Arborescência}

Se os arcos justos formam uma \(r\)-arborescência, então encontramos a solução ótima.
Caso contrário, os arcos justos contêm ciclos, e precisamos tratá-los adequadamente.

No nosso exemplo, os arcos justos são:
\[
	H = \{\, (r,a), (r,b), (a,c), (b,d), (b,e) \,\}.
\]
Esses arcos formam a \(r\)-arborescência \(T\) de custo mínimo.

Para verificar a otimalidade, note que uma propriedade fundamental dos potenciais duais é:

\begin{proposicao}
	Para toda função \(y: V \to \mathbb{R}\) com \(y(r) = 0\),
	uma \(r\)-arborescência \(T\) é \(c\)-mínima em \(D\) se, e somente se,
	todos os arcos de \(T\) são justos (i.e., \(c_y(a) = 0\) para todo \(a \in T\))
	e \(c_y(a) \geq 0\) para todo \(a \in A\).
\end{proposicao}

\begin{proof}[Prova.]
	Seja \(F\) uma \(r\)-arborescência qualquer. Para cada \(v \in V \setminus \{r\}\),
	seja \(a_v\) o único arco de \(F\) que entra em \(v\). Então
	\begin{align*}
		c(F)
		 & = \sum_{v \in V \setminus \{r\}} c(a_v)                                          \\
		 & = \sum_{v \in V \setminus \{r\}} \bigl(c_y(a_v) + y(v)\bigr)                     \\
		 & = \sum_{v \in V \setminus \{r\}} c_y(a_v) + \sum_{v \in V \setminus \{r\}} y(v).
	\end{align*}
	Como a soma \(\sum_{v \in V \setminus \{r\}} y(v)\) é constante para todas as
	\(r\)-arborescências, minimizar \(c(F)\) equivale a minimizar \(\sum_{v} c_y(a_v)\).
	Se todos os arcos de \(T\) são justos e \(c_y(a) \geq 0\) para todo \(a\),
	então \(\sum_{v} c_y(a_v^T) = 0 \leq \sum_{v} c_y(a_v^F)\) para qualquer \(F\),
	provando que \(T\) é ótima.
\end{proof}

No nosso exemplo, todos os arcos de \(T\) são justos e \(c_y(a) \geq 0\) para todo
arco \(a \in A\), portanto \(T\) é a \(r\)-arborescência de custo mínimo. Evidentemente, nem sempre será o caso que os arcos justos formam uma arborescência e precisaremos tratar ciclos.

\subsection*{Tratamento de Ciclos}

Em geral, após a elevação de potenciais, os arcos justos podem conter ciclos.
Nesse caso, o algoritmo de Frank procede de forma similar ao de Chu--Liu--Edmonds:
contrai cada ciclo justo em um supervértice, resolve o problema recursivamente
no dígrafo contraído, e depois reexpande a solução.

Vamos ilustrar esse processo com um exemplo modificado. Suponha que adicionamos o arco
\((c, a)\) com custo 1 ao dígrafo original:

\begin{figure}[H]
	\centering
	\begin{tikzpicture}[>=Stealth]
		% estilos
		\tikzset{
			vtx/.style={circle, draw,fill=blue!8, thick, inner sep=1.2pt, minimum size=6mm},
			arc/.style={->, semithick, draw=black!35},
			costG/.style={font=\scriptsize, fill=white, inner sep=1pt, text=black!55}
		}

		% layout
		\node[vtx] (r) at (0,0) {$r$};
		\node[vtx] (a) at (3,1.6) {$a$};
		\node[vtx] (b) at (3,-1.6) {$b$};
		\node[vtx] (c) at (6,2.2) {$c$};
		\node[vtx] (d) at (6,0.0) {$d$};
		\node[vtx] (e) at (6,-2.2) {$e$};

		% arcos originais
		\draw[arc] (r) -- node[costG, above] {1} (a);
		\draw[arc] (r) -- node[costG, below] {2} (b);
		\draw[arc] (r) .. controls (0.8,2.4) and (5.2,3.0) .. node[costG, yshift=4pt, pos=0.53] {3} (c);
		\draw[arc] (b) to[bend left=12] node[costG, above] {4} (a);
		\draw[arc] (a) to[bend left=14] node[costG, below] {5} (b);
		\draw[arc] (a) -- node[costG, above] {1} (c);
		\draw[arc] (a) -- node[costG, right] {3} (d);
		\draw[arc] (b) -- node[costG, above] {1} (d);
		\draw[arc] (b) -- node[costG, below] {2} (e);
		\draw[arc] (c) .. controls (7.6,2.6) and (7.6,-2.6) .. node[costG, xshift=8pt, pos=0.52] {5} (e);

		% NOVO arco (c, a)
		\draw[arc, draw=red!60] (c) to[bend left=20] node[costG, left, text=red!60] {1} (a);
	\end{tikzpicture}
	\caption{Dígrafo modificado com o arco \((c,a)\) de custo 1 (em vermelho).}
\end{figure}

Após a elevação de potenciais com \(y(a) = 1, y(c) = 1\), o arco \((c,a)\) torna-se
justo: \(c_y(c,a) = 1 - 1 = 0\).

\begin{figure}[H]
	\centering
	\begin{tikzpicture}[>=Stealth]
		% estilos
		\tikzset{
			vtx/.style={circle, draw,fill=blue!8, thick, inner sep=1.2pt, minimum size=6mm},
			tight/.style={->, very thick, draw=blue},
			cycle/.style={->, ultra thick, draw=red!70},
			loose/.style={->, semithick, draw=black!35},
			costB/.style={font=\scriptsize, fill=white, inner sep=1pt, text=blue},
			costR/.style={font=\scriptsize, fill=white, inner sep=1pt, text=red!70},
			costG/.style={font=\scriptsize, fill=white, inner sep=1pt, text=black!55}
		}

		% layout
		\node[vtx] (r) at (0,0) {$r$};
		\node[vtx] (a) at (3,1.6) {$a$};
		\node[vtx] (b) at (3,-1.6) {$b$};
		\node[vtx] (c) at (6,2.2) {$c$};
		\node[vtx] (d) at (6,0.0) {$d$};
		\node[vtx] (e) at (6,-2.2) {$e$};

		% arcos justos que formam ciclo
		\draw[cycle] (a) -- node[costR, above] {0} (c);
		\draw[cycle] (c) to[bend left=20] node[costR, left] {0} (a);

		% outros arcos justos
		\draw[tight] (r) -- node[costB, above] {0} (a);
		\draw[tight] (r) -- node[costB, below] {0} (b);
		\draw[tight] (b) -- node[costB, above] {0} (d);
		\draw[tight] (b) -- node[costB, below] {0} (e);

		% arcos não justos
		\draw[loose] (r) .. controls (0.8,2.4) and (5.2,3.0) .. node[costG, yshift=4pt, pos=0.53] {2} (c);
		\draw[loose] (b) to[bend left=12] node[costG, above] {3} (a);
		\draw[loose] (a) to[bend left=14] node[costG, below] {3} (b);
		\draw[loose] (a) -- node[costG, right] {2} (d);   % 3 - 1 = 2
		\draw[loose] (c) .. controls (7.6,2.6) and (7.6,-2.6) .. node[costG, xshift=8pt, pos=0.52] {3} (e);  % 5 - 2 = 3
	\end{tikzpicture}
	\caption{Arcos justos após elevação. O ciclo \(C = (a, c, a)\) é destacado em
		\textcolor{red!70}{vermelho}.}
\end{figure}

Nesse caso, contraímos o ciclo \(C = (a, c)\) em um supervértice \(x_C\), ajustamos
os custos dos arcos incidentes, e resolvemos o problema no dígrafo contraído.
A reexpansão é feita removendo um arco do ciclo para manter a estrutura de arborescência,
de forma análoga ao algoritmo de Chu--Liu--Edmonds.

\section{Descrição do algoritmo}

O algoritmo de András Frank difere substancialmente do algoritmo de Chu--Liu--Edmonds em sua abordagem: enquanto Chu--Liu--Edmonds opera diretamente sobre os custos dos arcos, o método de Frank utiliza uma técnica gulosa--dual baseada em \emph{potenciais} associados aos vértices. A ideia central é elevar progressivamente os potenciais \(y(v)\) para cada vértice \(v\neq r\), de modo que os custos \emph{reduzidos} \(c_y(u,v) := c(u,v) - y(v)\) revelem quais arcos devem participar da solução ótima. Um arco \((u,v)\) é dito \emph{justo} quando seu custo reduzido é zero, isto é, quando \(c(u,v) = y(v)\). O algoritmo garante que toda r-arborescência ótima pode ser formada apenas por arcos justos.

A seguir apresentamos uma descrição formal do algoritmo de András Frank e detalhes de implementação serão discutidos na próxima seção.

\begin{algobox}{András Frank}{andras-frank}
	Entrada: dígrafo \(D=(V,A)\), custos \(c:A\to\mathbb{R}_{\ge 0}\), raiz \(r\).\footnote{Se algum \(v\neq r\) não possui arco de entrada, não existe r-arborescência.}

	\textbf{Fase 1: Elevação de potenciais e construção de \(A_0\)}
	\begin{enumerate}\setlength{\itemsep}{2pt}
		\item \textbf{Inicialização:} defina \(y(v) := 0\) para todo \(v \in V\).
		\item \textbf{Iteração:} enquanto existir conjunto \(X \subseteq V \setminus \{r\}\) sem arco justo entrando:
		      \begin{itemize}\setlength{\itemsep}{2pt}
			      \item Calcule \(\Delta(X) := \min\{c(u,v) - y(v) : u \notin X, v \in X\}\).
			      \item Para cada \(v \in X\), atualize \(y(v) := y(v) + \Delta(X)\).
		      \end{itemize}
		\item Defina \(A_0 := \{a \in A : c_y(a) = 0\}\), o conjunto de arcos justos.
	\end{enumerate}

	\textbf{Fase 2: Construção da arborescência}
	\begin{enumerate}\setlength{\itemsep}{2pt}\setcounter{enumi}{3}
		\item Se \((V,A_0)\) forma uma r-arborescência, devolva \(A_0\). Por otimalidade dos potenciais duais, trata-se de uma r-arborescência de custo mínimo.
		\item Caso contrário, identifique um ciclo dirigido \(C\) em \(A_0\) (com \(r\notin C\)). \textbf{Contração:} contraia \(C\) em um supervértice \(x_C\) e defina custos reduzidos \(c'\) por
		      \begin{align*}
			      c'(u,x_C) & := c_y(u,w) = c(u,w) - y(w) &  & \text{para } u\notin C,\ w\in C, \\
			      c'(x_C,v) & := c_y(w,v) = c(w,v) - y(v) &  & \text{para } w\in C,\ v\notin C,
		      \end{align*}
		      descartando laços em \(x_C\) e permitindo paralelos. Denote o dígrafo contraído por \(D'=(V',A')\).
		\item \textbf{Recursão:} compute uma r-arborescência ótima \(T'\) de \(D'\) com custos \(c'\).
		\item \textbf{Expansão:} seja \((u,x_C)\in T'\) o único arco que entra em \(x_C\). No dígrafo original, ele corresponde a \((u,w)\) com \(w\in C\). Forme
		      \[
			      T := \bigl(T'\setminus\{\text{arcos incidentes a } x_C\}\bigr)\ \cup\ \{(u,w)\}\ \cup\ \bigl((A_0\cap A(C))\setminus\{a_w\}\bigr),
		      \]
		      onde \(a_w\) é o arco de \(C\) que entra em \(w\). Então \(T\) é uma r-arborescência ótima de \(D\).
	\end{enumerate}
\end{algobox}

\subsection{Corretude}

A corretude do algoritmo de András Frank baseia-se na teoria de dualidade em programação linear aplicada ao problema de arborescência de custo mínimo:

\begin{enumerate}\setlength{\itemsep}{2pt}
	\item \emph{Potenciais duais:} os valores \(y(v)\) funcionam como variáveis duais. Para qualquer r-arborescência \(T\) e qualquer escolha de potenciais \(y\), vale
	      \[
		      c(T) = \sum_{a\in T} c(a) = \sum_{a\in T} c_y(a) + \sum_{v\neq r} y(v),
	      \]
	      pois há exatamente um arco de \(T\) entrando em cada \(v\neq r\). O termo \(\sum_{v\neq r} y(v)\) é constante após a elevação completa dos potenciais; assim, minimizar \(c(T)\) equivale a minimizar \(\sum_{a\in T} c_y(a)\).

	\item \emph{Condições de otimalidade:} uma r-arborescência \(T\) é ótima se e somente se:
	      \begin{itemize}\setlength{\itemsep}{2pt}
		      \item Todos os arcos de \(T\) são justos: \(c_y(a) = 0\) para todo \(a \in T\).
		      \item Todos os arcos do dígrafo têm custo reduzido não-negativo: \(c_y(a) \geq 0\) para todo \(a \in A\).
	      \end{itemize}
	      Sob essas condições, \(\sum_{a\in T} c_y(a) = 0 \leq \sum_{a\in T'} c_y(a)\) para qualquer outra r-arborescência \(T'\), provando otimalidade.

	\item \emph{Elevação de potenciais:} o algoritmo eleva os potenciais de forma gulosa, garantindo que ao final cada \(v\neq r\) tenha ao menos um arco justo entrando. Isso é feito identificando componentes fortemente conexas em \(A\_0\): conjuntos \(X\) que ainda não possuem arcos justos entrando têm seus potenciais elevados pelo mínimo necessário \(\Delta(X)\).

	\item \emph{Caso acíclico:} se \(A_0\) forma uma r-arborescência, todos os arcos são justos e, pelas condições de otimalidade, a solução é ótima.

	\item \emph{Caso com ciclo (contração/expansão):} se \(A\_0\) contém um ciclo dirigido \(C\), todos os seus arcos são justos (custo reduzido zero). A contração preserva a estrutura de custos reduzidos: no dígrafo contraído, os novos custos \(c'\) já incorporam os potenciais, e uma solução ótima em \(D'\) mapeia para uma solução ótima em \(D\) pela reversibilidade da operação.
\end{enumerate}

Em termos intuitivos, os potenciais \(y(v)\) funcionam como "descontos" aplicados aos arcos que entram em cada vértice, revelando quais arcos são "justos" (custo líquido zero) e, portanto, candidatos à solução ótima. Ciclos de arcos justos podem ser contraídos sem perder otimalidade, simplificando o problema recursivamente.

\subsection{Complexidade}

A implementação, baseada em componentes fortemente conexas, detecta em cada iteração, quais conjuntos \(X\) necessitam elevação de potenciais. Calcular componentes fortemente conexas custa \(O(n+m)\) usando algoritmos como Tarjan ou Kosaraju. Para cada componente (exceto a raiz), eleva-se o potencial calculando \(\Delta(X)\) em \(O(m)\), atualizando os custos reduzidos.

No pior caso, cada iteração reduz o número de componentes em pelo menos uma unidade, resultando em \(O(n)\) iterações. Cada iteração processa todos os arcos para atualizar custos reduzidos e recalcular componentes, resultando em \(O(nm)\) no total para a Fase~1. A Fase~2 constrói a arborescência percorrendo \(A_0\) uma vez, custando \(O(n)\).

O uso de memória é \(O(n+m)\), incluindo as estruturas para armazenar o dígrafo, potenciais e componentes. A implementação a seguir adota a versão \(O(nm)\) por simplicidade e está disponível no repositório do projeto (\url{https://github.com/lorenypsum/GraphVisualizer}).

\section{Implementação em Python}

Esta seção descreve a implementação do algoritmo de András Frank em Python, estruturada para refletir com precisão as duas fases formais discutidas anteriormente. A Fase~1 realiza a elevação de potenciais e identifica os arcos justos, enquanto a Fase~2 constrói a arborescência de custo mínimo a partir desses arcos. Utilizamos a biblioteca NetworkX para manipulação de digrafos, aproveitando suas funcionalidades para representar grafos, calcular componentes fortemente conexas e gerenciar atributos de arcos.

A entrada consiste em um dígrafo orientado \(D = (V, A)\), com custos dos arcos registrados no atributo \texttt{"w"}, e uma raiz \(r \in V\). As hipóteses adotadas são: (i) o dígrafo é conexo a partir de \(r\), isto é, todo vértice \(v \neq r\) é alcançável a partir da raiz; (ii) para todo subconjunto \(X \subseteq V \setminus \{r\}\), existe ao menos um arco entrando em \(X\); e (iii) todos os custos são não negativos.

A saída é um subdigrafo \(T\) de \(D\) com \(|A_T| = |V| - 1\) arcos, tal que cada vértice \(v \neq r\) possui grau de entrada igual a 1, todos os vértices são alcançáveis a partir de \(r\), e o custo total \(\sum_{a \in A_T} c(a)\) é mínimo.

A estrutura do código é modular: funções auxiliares tratam cada etapa do algoritmo — cálculo de componentes fortemente conexas, elevação de potenciais, construção do subdigrafo \(A_0\) e construção da arborescência final. Todas operam sobre objetos \texttt{nx.DiGraph} e são coordenadas por uma função principal que gerencia o fluxo das duas fases. As subseções seguintes detalham cada função auxiliar, abordando lógica, parâmetros, saídas e complexidade.

\subsection{Construção do dígrafo \texorpdfstring{\(D_0\)}{D_0} inicial}

Começamos escrevendo uma função que constrói o dígrafo inicial \(D_0\) que será utilizado na Fase~1 do algoritmo. O dígrafo \(D_0\) é inicializado como um grafo vazio contendo apenas os vértices do dígrafo original, sem arcos. Essa estrutura será gradualmente populada com arcos de custo reduzido zero à medida que os potenciais são elevados.

Recebe como entrada um dígrafo \texttt{D} (objeto \texttt{nx.DiGraph}). A implementação cria um novo dígrafo vazio \texttt{D\_zero} (linha 2) e adiciona todos os vértices de \texttt{D} a \texttt{D\_zero} (linhas 3-4), preservando a estrutura de vértices sem incluir arcos inicialmente.

A função devolve o dígrafo \texttt{D\_zero} contendo todos os vértices de \texttt{D} mas nenhum arco. O dígrafo original \texttt{D} não é modificado. A complexidade é \(O(n)\), onde \(n = |V|\), pois itera sobre todos os vértices uma única vez.

\begin{tcolorbox}[
		enhanced, breakable,
		colframe=blue!60!black, colback=blue!2,
		colbacktitle=blue!15, coltitle=black,
		title={Construção do dígrafo $D_0$ inicial},
		boxed title style={sharp corners, boxrule=0.6pt},
		sharp corners, boxrule=0.6pt
	]
	\emph{Constrói um dígrafo vazio contendo apenas os vértices de D, sem arcos. Este dígrafo será populado com arcos justos durante a elevação de potenciais.}
	\tcblower
	\begin{lstlisting}[language=Python]
def build_D_zero(D):
    D_zero = nx.DiGraph()
    for v in D.nodes():
        D_zero.add_node(v)
    return D_zero
\end{lstlisting}
\end{tcolorbox}

\subsection{Identificação de arcos entrando em conjunto \texorpdfstring{\(X\)}{X}}

Esta função auxiliar identifica todos os arcos que entram em um conjunto \(X \subseteq V\), isto é, arcos \((u,v)\) tais que \(u \notin X\) e \(v \in X\). Essa operação é fundamental para calcular o mínimo custo de entrada em \(X\) durante a elevação de potenciais.

Recebe como entrada um dígrafo \texttt{D} e um conjunto de vértices \texttt{X}. A implementação cria uma lista vazia \texttt{arcs} (linha 2) e itera sobre todos os arcos do dígrafo com seus dados (linha 3), incluindo o peso. Para cada arco \((u,v,\text{data})\), verifica se \(u \notin X\) e \(v \in X\) (linha 4), adicionando à lista apenas os arcos que cruzam a fronteira de \(X\) (linha 5).

A função devolve uma lista de tuplas \((u, v, \text{data})\) representando os arcos que entram em \(X\), onde \texttt{data} contém o atributo \texttt{"w"} com o peso do arco. A complexidade é \(O(m)\), onde \(m = |A|\), pois examina cada arco uma vez.

\begin{tcolorbox}[
		enhanced, breakable,
		colframe=blue!60!black, colback=blue!2,
		colbacktitle=blue!15, coltitle=black,
		title={Identificação de arcos entrando em conjunto X},
		boxed title style={sharp corners, boxrule=0.6pt},
		sharp corners, boxrule=0.6pt
	]
	\emph{Identifica todos os arcos $(u,v)$ do dígrafo D tais que $u \notin X$ e $v \in X$, devolvendo uma lista com as tuplas $(u, v, data)$ onde data contém o peso do arco.}
	\tcblower
	\begin{lstlisting}[language=Python]
def get_arcs_entering_X(D, X):
    arcs = []
    for u, v, data in D.edges(data=True):
        if u not in X and v in X:
            arcs.append((u, v, data))
    return arcs
\end{lstlisting}
\end{tcolorbox}

A figura a seguir ilustra o funcionamento da função \texttt{get\_arcs\_entering\_X} em um digrafo que vamos denotar por \(D_{32}\). O dígrafo possui uma raiz \(r_0\) conectada aos vértices \(u_1, u_2, u_3\). Os vértices em laranja pertencem ao conjunto \(X = \{v_1, v_2, v_3\}\), e a função identifica apenas os arcos em vermelho, que saem de vértices fora de \(X\) e entram em vértices dentro de \(X\). Arcos da raiz, arcos internos a \(X\), externos a \(X\), ou saindo de \(X\) não são retornados.

\begin{figure}[H]
	\centering
	\begin{tikzpicture}[>=Stealth]
		% estilos
		\tikzset{
			vtx/.style={circle, draw, fill=blue!8, thick, inner sep=1.2pt, minimum size=7mm},
			vtxRoot/.style={circle, draw, fill=red!15, thick, inner sep=1.2pt, minimum size=7mm},
			vtxX/.style={circle, draw=orange!80!black, fill=orange!15, thick, inner sep=1.2pt, minimum size=7mm},
			arc/.style={->, >=Stealth, thick, draw=black!40},
			arcEntering/.style={->, >=Stealth, very thick, draw=red!70!black},
			cost/.style={font=\scriptsize, fill=white, inner sep=1pt}
		}

		% Raiz
		\node[vtxRoot] (r0) at (2, 4) {$r_0$};

		% Vértices fora de X
		\node[vtx] (u1) at (0, 2) {$u_1$};
		\node[vtx] (u2) at (2, 2) {$u_2$};
		\node[vtx] (u3) at (4, 2) {$u_3$};

		% Vértices dentro de X (destacados)
		\node[vtxX] (v1) at (0.5, -0.5) {$v_1$};
		\node[vtxX] (v2) at (2, -0.5) {$v_2$};
		\node[vtxX] (v3) at (3.5, -0.5) {$v_3$};

		% Arcos da raiz (não retornados, pois r0 não está em consideração para X)
		\draw[arc] (r0) -- node[cost, left, pos=0.4] {1} (u1);
		\draw[arc] (r0) -- node[cost, right] {2} (u2);
		\draw[arc] (r0) -- node[cost, right, pos=0.4] {1} (u3);

		% Arcos externos a X (não retornados)
		\draw[arc] (u1) to[bend left=15] node[cost, above] {1} (u2);
		\draw[arc] (u2) to[bend left=15] node[cost, above] {3} (u3);

		% Arcos entrando em X (destacados em vermelho)
		\draw[arcEntering] (u1) -- node[cost, left] {3} (v1);
		\draw[arcEntering] (u2) -- node[cost, right] {2} (v2);
		\draw[arcEntering] (u3) -- node[cost, right] {4} (v3);
		\draw[arcEntering] (u1) to[bend right=20] node[cost, below, pos=0.3] {5} (v2);

		% Arcos internos a X (não retornados)
		\draw[arc] (v1) -- node[cost, above] {1} (v2);
		\draw[arc] (v2) -- node[cost, above] {2} (v3);

		% Arcos saindo de X (não retornados)
		\draw[arc] (v1) to[bend right=30] node[cost, left, pos=0.3] {2} (u1);

		% Região X destacada
		\draw[orange!60!black, thick, dashed, rounded corners=8pt]
		(-0.3, 0.3) rectangle (4.3, -1.3);
		\node[orange!80!black, font=\bfseries] at (5, 0.3) {$X$};
	\end{tikzpicture}
	\caption{Ilustração da função \texttt{get\_arcs\_entering\_X} em \(D_{32}\). A raiz \(r_0\) (em vermelho claro) conecta-se aos vértices \(u_1, u_2, u_3\). Os vértices em \textcolor{orange!80!black}{laranja} pertencem ao conjunto \(X = \{v_1, v_2, v_3\}\). A função identifica apenas os arcos \textcolor{red!70!black}{em vermelho}: aqueles que saem de vértices fora de \(X\) e entram em vértices dentro de \(X\). Arcos da raiz, arcos internos a \(X\), externos a \(X\), ou saindo de \(X\) não são retornados.}
\end{figure}

\subsection{Cálculo do peso mínimo de corte}

Esta função calcula o peso mínimo entre todos os arcos fornecidos, correspondendo ao valor \(\Delta(X)\) necessário para elevar os potenciais dos vértices em \(X\).

Recebe como entrada uma lista \texttt{arcos} de tuplas \((u, v, \text{data})\). A implementação usa a função \texttt{min} com uma compreensão de gerador (linha 2) que extrai o atributo \texttt{"w"} de cada tupla em \texttt{data}.

A função devolve o peso mínimo encontrado entre todos os arcos da lista. A complexidade é \(O(k)\), onde \(k\) é o número de arcos na lista, pois examina cada arco uma vez para encontrar o mínimo.

\begin{tcolorbox}[
		enhanced, breakable,
		colframe=blue!60!black, colback=blue!2,
		colbacktitle=blue!15, coltitle=black,
		title={Cálculo do peso mínimo de corte},
		boxed title style={sharp corners, boxrule=0.6pt},
		sharp corners, boxrule=0.6pt
	]
	\emph{Calcula o peso mínimo entre todos os arcos fornecidos, correspondendo ao valor $\Delta(X)$ usado na elevação de potenciais.}
	\tcblower
	\begin{lstlisting}[language=Python]
def get_minimum_weight_cut(arcs):
    return min(data["w"] for _, _, data in arcs)
\end{lstlisting}
\end{tcolorbox}

A seguir temos uma ilustração do funcionamento da função \texttt{get\_minimum\_weight\_cut} em \(D_{32}\). Considerando os arcos em vermelho que entram em \(X\) (identificados pela função anterior), esta função calcula o peso mínimo entre eles. O arco em verde possui o menor peso (2), correspondendo ao valor \(\Delta(X) = 2\) que será devolvido pela função.

\begin{figure}[H]
	\centering
	\begin{tikzpicture}[>=Stealth]
		% estilos
		\tikzset{
			vtx/.style={circle, draw, fill=blue!8, thick, inner sep=1.2pt, minimum size=7mm},
			vtxRoot/.style={circle, draw, fill=red!15, thick, inner sep=1.2pt, minimum size=7mm},
			vtxX/.style={circle, draw=orange!80!black, fill=orange!15, thick, inner sep=1.2pt, minimum size=7mm},
			arc/.style={->, >=Stealth, thick, draw=black!40},
			arcEntering/.style={->, >=Stealth, very thick, draw=red!70!black},
			arcMin/.style={->, >=Stealth, ultra thick, draw=green!60!black},
			cost/.style={font=\scriptsize, fill=white, inner sep=1pt},
			costMin/.style={font=\scriptsize\bfseries, fill=green!10, inner sep=2pt, draw=green!60!black, rounded corners=1pt}
		}

		% Raiz
		\node[vtxRoot] (r0) at (2, 4) {$r_0$};

		% Vértices fora de X
		\node[vtx] (u1) at (0, 2) {$u_1$};
		\node[vtx] (u2) at (2, 2) {$u_2$};
		\node[vtx] (u3) at (4, 2) {$u_3$};

		% Vértices dentro de X (destacados)
		\node[vtxX] (v1) at (0.5, -0.5) {$v_1$};
		\node[vtxX] (v2) at (2, -0.5) {$v_2$};
		\node[vtxX] (v3) at (3.5, -0.5) {$v_3$};

		% Arcos da raiz
		\draw[arc] (r0) -- node[cost, left, pos=0.4] {1} (u1);
		\draw[arc] (r0) -- node[cost, right] {2} (u2);
		\draw[arc] (r0) -- node[cost, right, pos=0.4] {1} (u3);

		% Arcos externos a X
		\draw[arc] (u1) to[bend left=15] node[cost, above] {1} (u2);
		\draw[arc] (u2) to[bend left=15] node[cost, above] {3} (u3);

		% Arcos entrando em X (em vermelho, mas o mínimo em verde)
		\draw[arcEntering] (u1) -- node[cost, left] {3} (v1);
		\draw[arcMin] (u2) -- node[costMin, right] {2} (v2);
		\draw[arcEntering] (u3) -- node[cost, right] {4} (v3);
		\draw[arcEntering] (u1) to[bend right=20] node[cost, below, pos=0.3] {5} (v2);

		% Arcos internos a X
		\draw[arc] (v1) -- node[cost, above] {1} (v2);
		\draw[arc] (v2) -- node[cost, above] {2} (v3);

		% Arcos saindo de X
		\draw[arc] (v1) to[bend right=30] node[cost, left, pos=0.3] {2} (u1);

		% Região X destacada
		\draw[orange!60!black, thick, dashed, rounded corners=8pt]
		(-0.3, 0.3) rectangle (4.3, -1.3);
		\node[orange!80!black, font=\bfseries] at (5, 0.3) {$X$};

		% Anotação do resultado
		\node[font=\bfseries, green!60!black] at (2, -2.2) {$\Delta(X) = \min\{3, 2, 4, 5\} = 2$};
	\end{tikzpicture}
	\caption{Ilustração da função \texttt{get\_minimum\_weight\_cut} em \(D_{32}\). Considerando os arcos \textcolor{red!70!black}{em vermelho} que entram em \(X\) (identificados pela função anterior), esta função calcula o peso mínimo entre eles. O arco \textcolor{green!60!black}{em verde} possui o menor peso (2), correspondendo ao valor \(\Delta(X) = 2\).}
\end{figure}

\subsection{Atualização de pesos em \texorpdfstring{\(X\)}{X}}

Esta função auxiliar atualiza os pesos dos arcos que entram em um conjunto \(X\), subtraindo o valor \(\Delta(X)\) de cada peso. Arcos que atingem peso zero são adicionados a \(A_0\) e a \(D_0\).

Recebe como entrada um dígrafo \texttt{D}, lista de \texttt{arcs} entrando em \(X\), o valor \texttt{min\_weight} a ser subtraído, uma lista \texttt{A\_zero} para armazenar arcos de peso zero, e o dígrafo \texttt{D\_zero} para adicionar os arcos justos.

A implementação itera sobre cada arco \((u,v,\_)\) da lista (linha 2), subtrai \texttt{min\_weight} do peso armazenado em \texttt{D[u][v]["w"]} (linha 3), e verifica se o peso resultante é zero (linha 4). Caso sim, adiciona-se \((u,v)\) à lista \texttt{A\_zero} (linha 5) e ao dígrafo \texttt{D\_zero} (linha 6).

A função não devolve valor, pois modifica diretamente as estruturas passadas como parâmetros: o dígrafo \texttt{D} tem seus pesos atualizados, \texttt{A\_zero} acumula arcos justos, e \texttt{D\_zero} é populado com esses arcos. A complexidade é \(O(k)\), onde \(k\) é o número de arcos em \texttt{arcs}.

\begin{tcolorbox}[
		enhanced, breakable,
		colframe=blue!60!black, colback=blue!2,
		colbacktitle=blue!15, coltitle=black,
		title={Atualização de pesos em X},
		boxed title style={sharp corners, boxrule=0.6pt},
		sharp corners, boxrule=0.6pt
	]
	\emph{Atualiza os pesos dos arcos que entram em X, subtraindo o valor mínimo. Arcos que atingem peso zero são registrados em $A_0$ e adicionados a $D_0$.}
	\tcblower
	\begin{lstlisting}[language=Python]
def update_weights_in_X(D, arcs, min_weight, A_zero, D_zero):
    for u, v, _ in arcs:
        D[u][v]["w"] -= min_weight
        if D[u][v]["w"] == 0:
            A_zero.append((u, v))
            D_zero.add_edge(u, v)
\end{lstlisting}
\end{tcolorbox}

A seguir ilustramos o funcionamento da função \texttt{update\_weights\_in\_X} em \(D_{32}\). A figura mostra o dígrafo antes e depois da atualização dos pesos. No estado inicial (esquerda), temos \(\Delta(X) = 2\). A função subtrai esse valor de todos os arcos que entram em \(X\). No estado final (direita), os arcos que atingiram peso zero (destacados em azul) são adicionados a \(A_0\) e \(D_0\).

\begin{figure}[H]
	\centering
	\begin{tikzpicture}[>=Stealth]
		% estilos
		\tikzset{
			vtx/.style={circle, draw, fill=blue!8, thick, inner sep=1.2pt, minimum size=6.5mm},
			vtxRoot/.style={circle, draw, fill=red!15, thick, inner sep=1.2pt, minimum size=6.5mm},
			vtxX/.style={circle, draw=orange!80!black, fill=orange!15, thick, inner sep=1.2pt, minimum size=6.5mm},
			arc/.style={->, >=Stealth, thick, draw=black!40},
			arcEntering/.style={->, >=Stealth, very thick, draw=red!70!black},
			arcTight/.style={->, >=Stealth, ultra thick, draw=blue!70!black},
			cost/.style={font=\scriptsize, fill=white, inner sep=1pt},
			costTight/.style={font=\scriptsize\bfseries, fill=blue!10, inner sep=2pt, draw=blue!70!black, rounded corners=1pt}
		}

		% ===== ANTES (esquerda) =====
		\begin{scope}
			\node[font=\bfseries, blue!80!black] at (2, 4.2) {Antes: pesos originais};

			% Raiz
			\node[vtxRoot] (r0) at (2, 3) {$r_0$};

			% Vértices fora de X
			\node[vtx] (u1) at (0, 1.5) {$u_1$};
			\node[vtx] (u2) at (2, 1.5) {$u_2$};
			\node[vtx] (u3) at (4, 1.5) {$u_3$};

			% Vértices dentro de X
			\node[vtxX] (v1) at (0.5, -0.5) {$v_1$};
			\node[vtxX] (v2) at (2, -0.5) {$v_2$};
			\node[vtxX] (v3) at (3.5, -0.5) {$v_3$};

			% Arcos da raiz
			\draw[arc] (r0) -- node[cost, left, pos=0.4] {1} (u1);
			\draw[arc] (r0) -- node[cost, right] {2} (u2);
			\draw[arc] (r0) -- node[cost, right, pos=0.4] {1} (u3);

			% Arcos externos a X
			\draw[arc] (u1) to[bend left=15] node[cost, above] {1} (u2);
			\draw[arc] (u2) to[bend left=15] node[cost, above] {3} (u3);

			% Arcos entrando em X
			\draw[arcEntering] (u1) -- node[cost, left] {3} (v1);
			\draw[arcEntering] (u2) -- node[cost, right] {2} (v2);
			\draw[arcEntering] (u3) -- node[cost, right] {4} (v3);
			\draw[arcEntering] (u1) to[bend right=20] node[cost, below, pos=0.3] {5} (v2);

			% Arcos internos a X
			\draw[arc] (v1) -- node[cost, above] {1} (v2);
			\draw[arc] (v2) -- node[cost, above] {2} (v3);

			% Arcos saindo de X
			\draw[arc] (v1) to[bend right=30] node[cost, left, pos=0.3] {2} (u1);

			% Região X
			\draw[orange!60!black, thick, dashed, rounded corners=8pt]
			(-0.3, 0.3) rectangle (4.3, -1.3);
			\node[orange!80!black, font=\bfseries\small] at (4.8, 0.1) {$X$};

			\node[font=\small, green!60!black] at (2, -2.2) {$\Delta(X) = 2$};
		\end{scope}

		% Seta indicando transformação
		\draw[->, ultra thick, blue!60!black] (5.3, 0.5) -- node[above, font=\small] {subtrai $\Delta(X)$} (6.7, 0.5);

		% ===== DEPOIS (direita) =====
		\begin{scope}[xshift=7.5cm]
			\node[font=\bfseries, blue!80!black] at (2, 4.2) {Depois: pesos atualizados};

			% Raiz
			\node[vtxRoot] (r0b) at (2, 3) {$r_0$};

			% Vértices fora de X
			\node[vtx] (u1b) at (0, 1.5) {$u_1$};
			\node[vtx] (u2b) at (2, 1.5) {$u_2$};
			\node[vtx] (u3b) at (4, 1.5) {$u_3$};

			% Vértices dentro de X
			\node[vtxX] (v1b) at (0.5, -0.5) {$v_1$};
			\node[vtxX] (v2b) at (2, -0.5) {$v_2$};
			\node[vtxX] (v3b) at (3.5, -0.5) {$v_3$};

			% Arcos da raiz (inalterados)
			\draw[arc] (r0b) -- node[cost, left, pos=0.4] {1} (u1b);
			\draw[arc] (r0b) -- node[cost, right] {2} (u2b);
			\draw[arc] (r0b) -- node[cost, right, pos=0.4] {1} (u3b);

			% Arcos externos a X (inalterados)
			\draw[arc] (u1b) to[bend left=15] node[cost, above] {1} (u2b);
			\draw[arc] (u2b) to[bend left=15] node[cost, above] {3} (u3b);

			% Arcos entrando em X (atualizados)
			\draw[arcEntering] (u1b) -- node[cost, left] {1} (v1b);
			\draw[arcTight] (u2b) -- node[costTight, right] {0} (v2b);
			\draw[arcEntering] (u3b) -- node[cost, right] {2} (v3b);
			\draw[arcEntering] (u1b) to[bend right=20] node[cost, below, pos=0.3] {3} (v2b);

			% Arcos internos a X (inalterados)
			\draw[arc] (v1b) -- node[cost, above] {1} (v2b);
			\draw[arc] (v2b) -- node[cost, above] {2} (v3b);

			% Arcos saindo de X (inalterados)
			\draw[arc] (v1b) to[bend right=30] node[cost, left, pos=0.3] {2} (u1b);

			% Região X
			\draw[orange!60!black, thick, dashed, rounded corners=8pt]
			(-0.3, 0.3) rectangle (4.3, -1.3);
			\node[orange!80!black, font=\bfseries\small] at (4.8, 0.1) {$X$};

			\node[font=\small, blue!70!black] at (2, -2.2) {Arco \textcolor{blue!70!black}{justo} $\to A_0$};
		\end{scope}
	\end{tikzpicture}
	\caption{Ilustração da função \texttt{update\_weights\_in\_X} em \(D_{32}\). À esquerda, o dígrafo antes da atualização, com os arcos \textcolor{red!70!black}{em vermelho} entrando em \(X\) e \(\Delta(X) = 2\). À direita, após subtrair \(\Delta(X)\) de cada arco entrando em \(X\): o peso $(u_1, v_1)$ reduz de 3 para 1, $(u_2, v_2)$ de 2 para \textcolor{blue!70!black}{\textbf{0}} (torna-se justo), $(u_3, v_3)$ de 4 para 2, e $(u_1, v_2)$ de 5 para 3. O arco justo é adicionado a \(A_0\) e \(D_0\). Note que os arcos da raiz e arcos internos/externos a \(X\) permanecem inalterados.}
\end{figure}

\subsection{Verificação de arborescência}

Esta função verifica se um dígrafo \(D\) contém uma r-arborescência com raiz \(r_0\). Utiliza busca em profundidade (DFS) a partir da raiz para verificar se todos os vértices são alcançáveis.

Recebe como entrada um dígrafo \texttt{D} e a raiz \texttt{r0}. A implementação constrói uma árvore DFS a partir de \texttt{r0} usando \texttt{nx.dfs\_tree} (linha 2), que devolve um subdigrafo contendo apenas os vértices alcançáveis a partir da raiz seguindo arcos. Em seguida, compara o número de vértices da árvore DFS com o número total de vértices de \texttt{D} (linha 3).

A função devolve \texttt{True} se todos os vértices são alcançáveis (indicando presença de r-arborescência), \texttt{False} caso contrário. A complexidade é \(O(n+m)\), onde \(n=|V|\) e \(m=|A|\), devido à busca em profundidade. Precisamos dessa verificação para garantir que a Fase~1 produza um dígrafo \(D_0\) que contenha uma r-arborescência antes de prosseguir para a Fase~2.

\begin{tcolorbox}[
		enhanced, breakable,
		colframe=blue!60!black, colback=blue!2,
		colbacktitle=blue!15, coltitle=black,
		title={Verificação de arborescência},
		boxed title style={sharp corners, boxrule=0.6pt},
		sharp corners, boxrule=0.6pt
	]
	\emph{Verifica se o dígrafo D contém uma r-arborescência com raiz r0, usando busca em profundidade para testar alcançabilidade de todos os vértices.}
	\tcblower
	\begin{lstlisting}[language=Python]
def has_arborescence(D, r0):
    tree = nx.dfs_tree(D, r0)
    return tree.number_of_nodes() == D.number_of_nodes()
\end{lstlisting}
\end{tcolorbox}

\subsection{Fase 1: Elevação de potenciais e construção de \texorpdfstring{\(A_0\)}{A_0}}

A seguir apresentamos a função principal da Fase~1, responsável por elevar os potenciais dos vértices iterativamente até que cada conjunto de vértices possua ao menos um arco justo entrando. O processo utiliza componentes fortemente conexas para identificar quais conjuntos necessitam elevação.

Recebe como entrada um dígrafo \texttt{D\_original} e a raiz \texttt{r0}. A implementação cria uma cópia do dígrafo original (linha 2) para preservar a entrada, inicializa estruturas auxiliares \texttt{A\_zero} (lista de arcos justos), \texttt{Dual\_list} (lista de pares \((X, \Delta(X))\) para fins de validação dual), e \texttt{D\_zero} (dígrafo de arcos justos) (linhas 3-5). Um contador de iterações é inicializado na linha 6.

O loop principal (linhas 7-22) itera enquanto houver conjuntos sem arcos justos entrando. Em cada iteração, incrementa-se o contador (linha 8), calcula-se as componentes fortemente conexas de \(D_0\) usando \texttt{nx.condensation} (linha 9), que devolve um grafo acíclico dirigido (DAG, do inglês \emph{directed acyclic graph}) onde cada vértice representa uma componente e contém o atributo \texttt{"members"} com os vértices originais. Utilizamos componentes fortemente conexas porque elas identificam naturalmente os conjuntos maximais de vértices que ainda não possuem arcos justos entrando, evitando a necessidade de rastrear manualmente quais conjuntos já foram processados.

Em seguida, identificam-se as fontes (componentes sem arcos entrando) no grafo de condensação (linha 10). Se há apenas uma fonte, significa que todos os vértices estão em uma única componente alcançável pela raiz através de arcos justos, garantindo que \(D_0\) contém uma r-arborescência e encerrando o loop (linhas 11-12).

Para cada fonte \(u\) no grafo de condensação (linha 13), obtém-se o conjunto \(X\) de vértices da componente (linha 14). Se \(r_0 \in X\), a fonte é ignorada (linhas 15-16), pois a componente contendo a raiz não necessita elevação. Caso contrário, identificam-se os arcos entrando em \(X\) usando \texttt{get\_arcs\_entering\_X} (linha 17), calcula-se o peso mínimo \(\Delta(X)\) usando \texttt{get\_minimum\_weight\_cut} (linha 18), e atualiza-se os pesos com \texttt{update\_weights\_in\_X}, registrando novos arcos justos (linha 19). A elevação simultânea de potenciais para todos os vértices de \(X\) mantém a propriedade de que arcos internos a \(X\) permanecem com o mesmo custo reduzido relativo, preservando a correção do algoritmo. Finalmente, adiciona-se \((X, \Delta(X))\) à lista dual se \(\Delta(X) > 0\) (linhas 20-21), permitindo verificação posterior das condições de otimalidade dual.

A função devolve \texttt{A\_zero} (lista de arcos justos) e \texttt{Dual\_list} (pares \((X, \Delta(X))\) para validação). A complexidade é \(O(nm)\) no pior caso, com \(O(n)\) iterações, cada uma custando \(O(m)\) para calcular componentes e atualizar pesos.

\begin{tcolorbox}[
		enhanced, breakable,
		colframe=blue!60!black, colback=blue!2,
		colbacktitle=blue!15, coltitle=black,
		title={Fase 1: Elevação de potenciais e construção de $A_0$},
		boxed title style={sharp corners, boxrule=0.6pt},
		sharp corners, boxrule=0.6pt
	]
	\emph{Eleva iterativamente os potenciais dos vértices até que cada conjunto possua ao menos um arco justo entrando. Devolve a lista $A_0$ de arcos justos e a lista de pares $(X, \Delta(X))$ para validação dual.}
	\tcblower
	\begin{lstlisting}[language=Python, basicstyle=\ttfamily\fontsize{8}{9}\selectfont]
def phase1(D_original, r0):
    D_copy = D_original.copy()
    A_zero = []
    Dual_list = []
    D_zero = build_D_zero(D_copy)
    iteration = 0
    while True:
        iteration += 1
        C = nx.condensation(D_zero)
        sources = [x for x in C.nodes() if C.in_degree(x) == 0]
        if len(sources) == 1:
            break
        for u in sources:
            X = C.nodes[u]["members"]
            if r0 in X:
                continue
            arcs = get_arcs_entering_X(D_copy, X)
            min_weight = get_minimum_weight_cut(arcs)
            update_weights_in_X(D_copy, arcs, min_weight, A_zero, D_zero)
            if min_weight != 0:
                Dual_list.append((X, min_weight)) 
    return A_zero, Dual_list
\end{lstlisting}
\end{tcolorbox}

A seguir ilustramos a execução da função \texttt{phase1} no dígrafo \(D_{32}\) estendido com a raiz \(r_0\). A figura mostra três estados principais: o dígrafo original \(D\) com seus pesos, o dígrafo \(D_0\) inicial (vazio, sem arcos), e o dígrafo \(D_0\) final após as iterações de elevação de potenciais, contendo apenas os arcos justos (em azul) que formam uma r-arborescência.

\begin{figure}[H]
	\centering
	\begin{tikzpicture}[>=Stealth, scale=0.85, every node/.style={scale=0.85}]
		% Estilos
		\tikzset{
			vtx/.style={circle, draw, fill=blue!8, thick, inner sep=1.2pt, minimum size=6.5mm},
			vtxRoot/.style={circle, draw, fill=red!15, thick, inner sep=1.2pt, minimum size=6.5mm},
			vtxX/.style={circle, draw, fill=orange!20, thick, inner sep=1.2pt, minimum size=6.5mm},
			arc/.style={->, thick, gray!60},
			arcTight/.style={->, ultra thick, blue!70!black},
			cost/.style={font=\scriptsize, fill=white, inner sep=1pt, rounded corners=1pt},
			costTight/.style={font=\scriptsize\bfseries, fill=blue!10, inner sep=2pt, draw=blue!70!black, rounded corners=1pt}
		}

		% ===== DÍGRAFO ORIGINAL D (esquerda) =====
		\begin{scope}
			\node[font=\bfseries, blue!80!black] at (2, 4.2) {Dígrafo original $D$};

			% Raiz
			\node[vtxRoot] (r0) at (2, 3) {$r_0$};

			% Vértices fora de X
			\node[vtx] (u1) at (0, 1.5) {$u_1$};
			\node[vtx] (u2) at (2, 1.5) {$u_2$};
			\node[vtx] (u3) at (4, 1.5) {$u_3$};

			% Vértices dentro de X
			\node[vtxX] (v1) at (0.5, -0.5) {$v_1$};
			\node[vtxX] (v2) at (2, -0.5) {$v_2$};
			\node[vtxX] (v3) at (3.5, -0.5) {$v_3$};

			% Arcos da raiz
			\draw[arc] (r0) -- node[cost, left, pos=0.4] {1} (u1);
			\draw[arc] (r0) -- node[cost, right] {2} (u2);
			\draw[arc] (r0) -- node[cost, right, pos=0.4] {1} (u3);

			% Arcos externos a X
			\draw[arc] (u1) to[bend left=15] node[cost, above] {1} (u2);
			\draw[arc] (u2) to[bend left=15] node[cost, above] {3} (u3);

			% Arcos entrando em X
			\draw[arc] (u1) -- node[cost, left] {3} (v1);
			\draw[arc] (u2) -- node[cost, right] {2} (v2);
			\draw[arc] (u3) -- node[cost, right] {4} (v3);
			\draw[arc] (u1) to[bend right=20] node[cost, below, pos=0.3] {5} (v2);

			% Arcos internos a X
			\draw[arc] (v1) -- node[cost, above] {1} (v2);
			\draw[arc] (v2) -- node[cost, above] {2} (v3);

			% Arcos saindo de X
			\draw[arc] (v1) to[bend right=30] node[cost, left, pos=0.3] {2} (u1);

			\node[font=\small, gray!70!black] at (2, -2) {Pesos originais};
		\end{scope}

		% ===== D_0 INICIAL (centro) =====
		\begin{scope}[xshift=6cm]
			\node[font=\bfseries, blue!80!black] at (2, 4.2) {$D_0$ inicial (linha 5)};

			% Vértices apenas, sem arcos
			\node[vtxRoot] (r0i) at (2, 3) {$r_0$};
			\node[vtx] (u1i) at (0, 1.5) {$u_1$};
			\node[vtx] (u2i) at (2, 1.5) {$u_2$};
			\node[vtx] (u3i) at (4, 1.5) {$u_3$};
			\node[vtxX] (v1i) at (0.5, -0.5) {$v_1$};
			\node[vtxX] (v2i) at (2, -0.5) {$v_2$};
			\node[vtxX] (v3i) at (3.5, -0.5) {$v_3$};

			\node[font=\small, gray!70!black] at (2, -2) {Sem arcos (vazio)};
		\end{scope}

		% ===== D_0 FINAL (direita) =====
		\begin{scope}[xshift=12cm]
			\node[font=\bfseries, blue!80!black] at (2, 4.2) {$D_0$ final (após loop)};

			% Vértices
			\node[vtxRoot] (r0f) at (2, 3) {$r_0$};
			\node[vtx] (u1f) at (0, 1.5) {$u_1$};
			\node[vtx] (u2f) at (2, 1.5) {$u_2$};
			\node[vtx] (u3f) at (4, 1.5) {$u_3$};
			\node[vtxX] (v1f) at (0.5, -0.5) {$v_1$};
			\node[vtxX] (v2f) at (2, -0.5) {$v_2$};
			\node[vtxX] (v3f) at (3.5, -0.5) {$v_3$};

			% Arcos justos (peso 0)
			\draw[arcTight] (r0f) -- node[costTight, left, pos=0.4] {0} (u1f);
			\draw[arcTight] (r0f) -- node[costTight, right, pos=0.4] {0} (u3f);
			\draw[arcTight] (u1f) to[bend left=15] node[costTight, above] {0} (u2f);
			\draw[arcTight] (u1f) -- node[costTight, left] {0} (v1f);
			\draw[arcTight] (u2f) -- node[costTight, right] {0} (v2f);
			\draw[arcTight] (v2f) -- node[costTight, above] {0} (v3f);

			\node[font=\small, blue!70!black] at (2, -2) {Arcos justos em $A_0$};
		\end{scope}

		% Setas indicando progressão
		\draw[->, ultra thick, gray!50] (4.5, 1) -- node[above, font=\small] {elevação} (5.5, 1);
		\draw[->, ultra thick, gray!50] (10.5, 1) -- node[above, font=\small] {iterações} (11.5, 1);
	\end{tikzpicture}
	\caption{Execução da função \texttt{phase1} em \(D_{32}\). À esquerda, o dígrafo original \(D\) com pesos dos arcos. Ao centro, \(D_0\) inicial após linha 5 (\texttt{build\_D\_zero}), contendo apenas vértices sem arcos. À direita, \(D_0\) final após o loop de elevação de potenciais, contendo apenas os arcos \textcolor{blue!70!black}{justos} (custo reduzido zero) que formam uma r-arborescência. Durante as iterações (linhas 7-22), os potenciais são elevados para cada componente sem arcos justos entrando, até que todos os vértices sejam alcançáveis a partir de \(r_0\) através de arcos em \(D_0\).}
\end{figure}

A seguir ilustramos a execução da função \texttt{phase2} em \(D_{32}\), mostrando como a r-arborescência é construída incrementalmente a partir do conjunto \(A_0\) de arcos justos obtidos da Fase~1. A figura apresenta quatro estados: o conjunto inicial \(A_0\) com 6 arcos justos, e três estados intermediários da arborescência \texttt{Arb} sendo construída, destacando em verde os arcos adicionados em cada iteração.

\begin{figure}[H]
	\centering
	\begin{tikzpicture}[>=Stealth, scale=0.72, every node/.style={scale=0.72}]
		% Estilos
		\tikzset{
			vtx/.style={circle, draw, fill=blue!8, thick, inner sep=1.2pt, minimum size=6.5mm},
			vtxRoot/.style={circle, draw, fill=red!15, thick, inner sep=1.2pt, minimum size=6.5mm},
			vtxIn/.style={circle, draw, fill=green!15, thick, inner sep=1.2pt, minimum size=6.5mm},
			vtxOut/.style={circle, draw, fill=gray!10, thick, inner sep=1.2pt, minimum size=6.5mm},
			arc/.style={->, thick, blue!70!black},
			arcNew/.style={->, ultra thick, green!70!black},
			cost/.style={font=\scriptsize, fill=white, inner sep=1pt}
		}

		% ===== A_0 (esquerda superior) =====
		\begin{scope}
			\node[font=\bfseries, blue!80!black] at (2, 4.5) {$A_0$ (arcos justos)};

			\node[vtxRoot] (r0) at (2, 3.2) {$r_0$};
			\node[vtx] (u1) at (0, 1.7) {$u_1$};
			\node[vtx] (u2) at (2, 1.7) {$u_2$};
			\node[vtx] (u3) at (4, 1.7) {$u_3$};
			\node[vtx] (v1) at (0.5, 0) {$v_1$};
			\node[vtx] (v2) at (2, 0) {$v_2$};
			\node[vtx] (v3) at (3.5, 0) {$v_3$};

			\draw[arc] (r0) -- node[cost, left, pos=0.3] {0} (u1);
			\draw[arc] (r0) -- node[cost, right, pos=0.3] {0} (u3);
			\draw[arc] (u1) to[bend left=12] node[cost, above, pos=0.4] {0} (u2);
			\draw[arc] (u1) -- node[cost, left] {0} (v1);
			\draw[arc] (u2) -- node[cost, right] {0} (v2);
			\draw[arc] (v2) -- node[cost, above] {0} (v3);

			\node[font=\small, gray!70!black] at (2, -1) {6 arcos com custo 0};
		\end{scope}

		% ===== Iteração 1 (direita superior) =====
		\begin{scope}[xshift=6.5cm]
			\node[font=\bfseries, blue!80!black] at (2, 4.5) {Iteração 1: $|$Arb$| = 1$};

			\node[vtxRoot] (r01) at (2, 3.2) {$r_0$};
			\node[vtxIn] (u11) at (0, 1.7) {$u_1$};
			\node[vtxOut] (u21) at (2, 1.7) {$u_2$};
			\node[vtxOut] (u31) at (4, 1.7) {$u_3$};
			\node[vtxOut] (v11) at (0.5, 0) {$v_1$};
			\node[vtxOut] (v21) at (2, 0) {$v_2$};
			\node[vtxOut] (v31) at (3.5, 0) {$v_3$};

			\draw[arcNew] (r01) -- node[cost, left, pos=0.3, fill=green!10] {1} (u11);

			\node[font=\small, green!70!black] at (2, -1) {Adiciona $(r_0, u_1)$};
		\end{scope}

		% ===== Iteração 2 (esquerda inferior) =====
		\begin{scope}[yshift=-6.2cm]
			\node[font=\bfseries, blue!80!black] at (2, 4.5) {Iteração 2: $|$Arb$| = 2$};

			\node[vtxRoot] (r02) at (2, 3.2) {$r_0$};
			\node[vtxIn] (u12) at (0, 1.7) {$u_1$};
			\node[vtxOut] (u22) at (2, 1.7) {$u_2$};
			\node[vtxIn] (u32) at (4, 1.7) {$u_3$};
			\node[vtxOut] (v12) at (0.5, 0) {$v_1$};
			\node[vtxOut] (v22) at (2, 0) {$v_2$};
			\node[vtxOut] (v32) at (3.5, 0) {$v_3$};

			\draw[arc] (r02) -- node[cost, left, pos=0.3] {1} (u12);
			\draw[arcNew] (r02) -- node[cost, right, pos=0.3, fill=green!10] {1} (u32);

			\node[font=\small, green!70!black] at (2, -1) {Adiciona $(r_0, u_3)$};
		\end{scope}

		% ===== Iteração 4 (direita inferior) =====
		\begin{scope}[xshift=6.5cm, yshift=-6.2cm]
			\node[font=\bfseries, blue!80!black] at (2, 4.5) {Iteração 4: $|$Arb$| = 4$};

			\node[vtxRoot] (r04) at (2, 3.2) {$r_0$};
			\node[vtxIn] (u14) at (0, 1.7) {$u_1$};
			\node[vtxIn] (u24) at (2, 1.7) {$u_2$};
			\node[vtxIn] (u34) at (4, 1.7) {$u_3$};
			\node[vtxIn] (v14) at (0.5, 0) {$v_1$};
			\node[vtxOut] (v24) at (2, 0) {$v_2$};
			\node[vtxOut] (v34) at (3.5, 0) {$v_3$};

			\draw[arc] (r04) -- node[cost, left, pos=0.3] {1} (u14);
			\draw[arc] (r04) -- node[cost, right, pos=0.3] {1} (u34);
			\draw[arc] (u14) to[bend left=12] node[cost, above, pos=0.4] {1} (u24);
			\draw[arcNew] (u14) -- node[cost, left, fill=green!10] {3} (v14);

			\node[font=\small, green!70!black] at (2, -1) {Adiciona $(u_1, v_1)$};
		\end{scope}

		% Setas de progressão
		\draw[->, ultra thick, gray!50] (4.8, 2) -- node[above, font=\small] {linha 6-10} (5.7, 2);
		\draw[->, ultra thick, gray!50] (2, -1.5) -- node[left, font=\small] {linha 6-10} (2, -2.4);
		\draw[->, ultra thick, gray!50] (4.8, -4.2) -- node[above, font=\small] {linha 6-10} (5.7, -4.2);
	\end{tikzpicture}
	\caption{Execução da função \texttt{phase2} em \(D_{32}\). Superior esquerdo: conjunto \(A_0\) com 6 arcos justos (custo reduzido zero) obtidos da Fase~1. Superior direito: Iteração 1 — adiciona arco $(r_0, u_1)$ pois $r_0 \in$ Arb e $u_1 \notin$ Arb. Inferior esquerdo: Iteração 2 — adiciona $(r_0, u_3)$. Inferior direito: Iteração 4 — após adicionar $(u_1, u_2)$ na iteração 3, adiciona $(u_1, v_1)$. Vértices \textcolor{green!70!black}{verdes} foram recém-adicionados, vértices \textcolor{gray!70!black}{cinzas} ainda não pertencem a Arb. O processo continua até que todos os 6 arcos de \(A_0\) sejam incluídos, formando uma r-arborescência com pesos originais de \(D\).}
\end{figure}

As funções auxiliares implementadas nesta seção correspondem diretamente aos passos da Fase~1 do algoritmo de András Frank da seguinte forma:

\begin{tcolorbox}[
		enhanced, breakable,
		colframe=purple!60!black, colback=purple!5,
		colbacktitle=purple!15, coltitle=black,
		title={Correspondência entre Teoria e Implementação — Fase 1},
		boxed title style={sharp corners, boxrule=0.6pt},
		sharp corners, boxrule=0.6pt
	]
	\textbf{Passo 1 — Inicialização:}
	\begin{itemize}
		\item \textbf{Descrição teórica:} Defina \(y(v) := 0\) para todo \(v \in V\).
		\item \textbf{Implementação:} A função \texttt{build\_D\_zero(D)} inicializa o dígrafo \(D_0\) vazio, que será populado apenas com arcos justos. Implicitamente, os potenciais iniciam em zero, pois os pesos no dígrafo \(D\) representam os custos reduzidos \(c_y(u,v) = c(u,v) - y(v)\). Com \(y(v) = 0\), temos \(c_y = c\) no início.
	\end{itemize}

	\vspace{2mm}
	\textbf{Passo 2 — Iteração de elevação de potenciais:}
	\begin{itemize}
		\item \textbf{Descrição teórica:} Enquanto existir conjunto \(X \subseteq V \setminus \{r\}\) sem arco justo entrando:
		      \begin{itemize}
			      \item Calcule \(\Delta(X) := \min\{c(u,v) - y(v) : u \notin X, v \in X\}\).
			      \item Para cada \(v \in X\), atualize \(y(v) := y(v) + \Delta(X)\).
		      \end{itemize}
		\item \textbf{Implementação:} Esse processo de iteração é realizado pela composição sequencial de três funções auxiliares:
		      \begin{enumerate}
			      \item \texttt{get\_arcs\_entering\_X(D, X)}: Identifica o conjunto \(\{(u,v) : u \notin X, v \in X\}\), isto é, todos os arcos que cruzam a fronteira de \(X\). Essa função corresponde diretamente à definição do conjunto sobre o qual o mínimo é calculado na fórmula \(\Delta(X)\).

			      \item \texttt{get\_minimum\_weight\_cut(arcs)}: Calcula \(\min\{\text{data}["w"] : (u,v,\text{data}) \in \text{arcs}\}\), que é exatamente \(\Delta(X) = \min\{c_y(u,v) : u \notin X, v \in X\}\). Como os pesos no dígrafo já representam custos reduzidos (são atualizados a cada iteração), essa função devolve precisamente o valor teórico de \(\Delta(X)\).

			      \item \texttt{update\_weights\_in\_X(D, arcs, min\_weight, A\_zero, D\_zero)}: Implementa a atualização dos potenciais. Para cada arco \((u,v)\) entrando em \(X\), subtrai \texttt{min\_weight} de \texttt{D[u][v]["w"]}, efetivamente calculando o novo custo reduzido:
			            \[c_y'(u,v) = c_y(u,v) - \Delta(X) = \]
			            \[c(u,v) - y(v) - \Delta(X) = \]
			            \[c(u,v) - \bigl(y(v) + \Delta(X)\bigr) = c(u,v) - y'(v), \]
			            onde \(y'(v) = y(v) + \Delta(X)\) é o novo potencial. Arcos cujo custo reduzido atinge zero (\texttt{D[u][v]["w"] == 0}) são adicionados a \(A_0\) e \(D_0\), tornando-se \textbf{justos}.
		      \end{enumerate}
	\end{itemize}

	\vspace{2mm}
	\textbf{Passo 3 — Construção de \(A_0\):}
	\begin{itemize}
		\item \textbf{Descrição teórica:} Defina \(A_0 := \{a \in A : c_y(a) = 0\}\), o conjunto de arcos justos.
		\item \textbf{Implementação:} A construção de \(A_0\) ocorre de forma incremental durante as iterações do Passo~2. Cada chamada a \texttt{update\_weights\_in\_X} verifica quais arcos atingiram custo reduzido zero e os adiciona tanto à lista \texttt{A\_zero} quanto ao dígrafo \texttt{D\_zero}. Esse processo continua até que a função principal da Fase~1 (que será apresentada adiante) determine que cada componente fortemente conexa em \(D_0\) (exceto a raiz) possui ao menos um arco justo entrando, garantindo que \(A_0\) é suficiente para formar a base de uma \(r\)-arborescência.
		\item O resultado final é o conjunto completo \(A_0 = \{a \in A : c_y(a) = 0\}\), usado na Fase~2 para construir a arborescência ótima através de uma seleção gulosa de arcos justos.
	\end{itemize}
\end{tcolorbox}

\subsection{Fase 2: Construção da arborescência}
Esta é a função principal da Fase~2, responsável por construir a r-arborescência de custo mínimo a partir do conjunto \(A_0\) de arcos justos. A construção é incremental: inicia-se com a raiz e adiciona-se iterativamente arcos de \(A_0\) que conectam vértices já incluídos a novos vértices, garantindo que cada vértice não-raiz receba exatamente um arco de entrada.

Recebe como entrada um dígrafo \texttt{D\_original}, a raiz \texttt{r0}, e a lista \texttt{A\_zero} de arcos justos. A implementação cria um novo dígrafo vazio \texttt{Arb} (linha 2) e adiciona a raiz (linha 3).

O loop principal (linhas 5-12) itera \(n-1\) vezes, onde \(n = |V|\), pois uma r-arborescência tem exatamente \(|V|-1\) arcos. Em cada iteração:
\begin{enumerate}\setlength{\itemsep}{2pt}
	\item Percorre os arcos \((u,v)\) de \texttt{A\_zero} (linha 6).
	\item Verifica se \(u\) já está em \texttt{Arb} e \(v\) ainda não (linha 7).
	\item Se sim, obtém os dados do arco do dígrafo original (linha 8) e adiciona \((u,v)\) a \texttt{Arb} (linha 9).
	\item Interrompe o loop interno para reiniciar a busca, garantindo descoberta em largura (linha 10).
\end{enumerate}

A função devolve o dígrafo \texttt{Arb} representando a r-arborescência de custo mínimo. A complexidade é \(O(nm)\) no pior caso, pois cada uma das \(O(n)\) iterações pode percorrer todos os \(O(m)\) arcos de \texttt{A\_zero}.

\begin{tcolorbox}[
		enhanced, breakable,
		colframe=blue!60!black, colback=blue!2,
		colbacktitle=blue!15, coltitle=black,
		title={Fase 2: Construção da arborescência},
		boxed title style={sharp corners, boxrule=0.6pt},
		sharp corners, boxrule=0.6pt
	]
	\emph{Constrói incrementalmente a r-arborescência a partir de $A_0$, adicionando iterativamente arcos que conectam vértices já incluídos a novos vértices.}
	\tcblower
	\begin{lstlisting}[language=Python]
def phase2(D_original, r0, A_zero):
    Arb = nx.DiGraph()
    Arb.add_node(r0)
    n = len(D_original.nodes())
    for _ in range(n - 1):
        for u, v in A_zero:
            if u in Arb.nodes() and v not in Arb.nodes():
                edge_data = D_original.get_edge_data(u, v)
                Arb.add_edge(u, v, **edge_data)
                break
    return Arb
\end{lstlisting}
\end{tcolorbox}

Apresentamos também uma versão alternativa da Fase~2 que utiliza busca em largura (BFS) para construir a r-arborescência de forma mais eficiente. Diferentemente da versão anterior que itera \(n-1\) vezes sobre todos os arcos, esta implementação usa uma fila de prioridade para explorar os arcos em ordem, evitando buscas lineares repetidas.

Recebe como entrada um dígrafo \texttt{D\_original}, a raiz \texttt{r0}, e a lista \texttt{A\_zero} de arcos justos. A implementação começa criando um dígrafo auxiliar \texttt{Arb} (linha 2) onde cada arco de \texttt{A\_zero} recebe um peso igual ao seu índice na lista (linhas 3-4), estabelecendo uma ordem de exploração. Inicializa-se o conjunto \texttt{V} de vértices visitados contendo apenas a raiz (linha 5) e uma fila de prioridade vazia \texttt{q} (linha 6).

Todos os arcos que saem da raiz em \texttt{Arb} são adicionados à fila de prioridade (linhas 7-8), usando o peso (índice) como critério de ordenação. Cria-se então o dígrafo \texttt{A} que conterá a arborescência resultante (linha 9).

O loop principal (linhas 10-17) extrai arcos da fila de prioridade em ordem crescente de índice. Para cada arco \((u,v)\) extraído (linha 11), verifica-se se o vértice destino \(v\) já foi visitado (linha 12); em caso positivo, o arco é ignorado via \texttt{continue} (linha 13). Caso contrário, adiciona-se o arco \((u,v)\) à arborescência \texttt{A} com o peso original de \texttt{D\_original} (linha 14), marca-se \(v\) como visitado (linha 15), e todos os arcos que saem de \(v\) em \texttt{Arb} são adicionados à fila de prioridade para exploração futura (linhas 16-17).

A função devolve o dígrafo \texttt{A} representando a r-arborescência de custo mínimo. A complexidade é \(O(m \log m)\), onde \(m = |A_0|\), devido às operações de inserção e remoção na fila de prioridade. Esta versão é mais eficiente que a anterior quando \(|A_0|\) é grande, pois evita percorrer todos os arcos em cada iteração.

\begin{tcolorbox}[
		enhanced, breakable,
		colframe=blue!60!black, colback=blue!2,
		colbacktitle=blue!15, coltitle=black,
		title={Fase 2 (versão BFS): Construção da arborescência com fila de prioridade},
		boxed title style={sharp corners, boxrule=0.6pt},
		sharp corners, boxrule=0.6pt
	]
	\emph{Constrói a r-arborescência usando busca em largura guiada por fila de prioridade, explorando arcos de $A_0$ em ordem e evitando buscas lineares repetidas. Complexidade $O(m \log m)$.}
	\tcblower
	\begin{lstlisting}[language=Python]
def phase2_v2(D_original, r0, A_zero):
    Arb = nx.DiGraph()
    for i, (u, v) in enumerate(A_zero):
        Arb.add_edge(u, v, w=i)
    V = {r0}
    q = []
    for u, v, data in Arb.out_edges(r0, data=True):
        heapq.heappush(q, (data["w"], u, v))
    A = nx.DiGraph()
    while q:
        _, u, v = heapq.heappop(q)
        if v in V:
            continue
        A.add_edge(u, v, w=D_original[u][v]["w"])
        V.add(v)
        for x, y, data in Arb.out_edges(v, data=True):
            heapq.heappush(q, (data["w"], x, y))
    return A
\end{lstlisting}
\end{tcolorbox}

As duas versões da Fase~2 implementadas acima correspondem diretamente ao Passo~4 da descrição teórica do algoritmo de András Frank:

\begin{tcolorbox}[
		enhanced, breakable,
		colframe=purple!60!black, colback=purple!5,
		colbacktitle=purple!15, coltitle=black,
		title={Correspondência entre Teoria e Implementação — Fase 2},
		boxed title style={sharp corners, boxrule=0.6pt},
		sharp corners, boxrule=0.6pt
	]
	\textbf{Passo 4 — Construção da arborescência (caso acíclico):}
	\begin{itemize}
		\item \textbf{Descrição teórica:} Se \((V,A_0)\) forma uma r-arborescência, devolva \(A_0\). Por otimalidade dos potenciais duais, trata-se de uma r-arborescência de custo mínimo.
		\item \textbf{Implementação:} Ambas as versões de \texttt{phase2} constroem uma r-arborescência a partir do conjunto \(A_0\) de arcos justos obtido na Fase~1. A corretude baseia-se no fato de que todos os arcos em \(A_0\) têm custo reduzido zero, e a Fase~1 garante que existe uma r-arborescência formada exclusivamente por arcos justos.
		      \begin{itemize}
			      \item \textbf{Versão 1 (\texttt{phase2}):} Construção incremental por exploração exaustiva. Em cada uma das \(n-1\) iterações, percorre todos os arcos de \(A_0\) procurando um arco \((u,v)\) tal que \(u\) já pertence à arborescência parcial e \(v\) ainda não. Essa abordagem simples corresponde diretamente à ideia teórica de construir a arborescência adicionando um vértice por vez, conectando-o à estrutura existente através de um arco justo. Complexidade: \(O(nm)\).

			      \item \textbf{Versão 2 (\texttt{phase2\_v2}):} Construção por busca em largura guiada por fila de prioridade. Cria um dígrafo auxiliar onde arcos são indexados, usa a fila de prioridade para explorar arcos sistematicamente a partir da raiz, evitando buscas lineares repetidas. Essa versão otimizada mantém a mesma correção teórica — construir uma r-arborescência usando apenas arcos de \(A_0\) — mas melhora a eficiência prática. Complexidade: \(O(m \log m)\).
		      \end{itemize}
	\end{itemize}

	\vspace{2mm}
	\textbf{Nota sobre Passos 5-7 (contração/recursão/expansão):}
	\begin{itemize}
		\item A descrição teórica do algoritmo de András Frank inclui os Passos~5-7 para tratar o caso onde \(A_0\) contém ciclos dirigidos, exigindo contração, resolução recursiva e reexpansão, de forma análoga ao algoritmo de Chu--Liu--Edmonds.
		\item Na implementação apresentada, optamos por uma abordagem não-recursiva baseada em componentes fortemente conexas. A Fase~1 já garante que \(A_0\) formará uma r-arborescência ao término das iterações de elevação de potenciais, eliminando a necessidade de tratar ciclos explicitamente na Fase~2. Essa simplificação é possível porque a elevação de potenciais progressivamente "quebra" todos os ciclos ao criar novos arcos justos que conectam diferentes componentes, até que reste apenas uma única componente fortemente conexa contendo todos os vértices.
		\item Portanto, quando a Fase~2 é executada, o conjunto \(A_0\) já está livre de ciclos e forma uma r-arborescência, correspondendo diretamente ao caso tratado pelo Passo~4 da descrição teórica. A verificação prévia \texttt{has\_arborescence(D, r0)} (realizada pela função principal) confirma essa propriedade antes de invocar a Fase~2.
	\end{itemize}
\end{tcolorbox}

\subsection{Verificação de otimalidade dual}

Esta função verifica se a condição de otimalidade dual é satisfeita para a r-arborescência construída. Segundo a teoria de programação linear dual aplicada ao problema de arborescência de custo mínimo, uma solução é ótima se e somente se cada conjunto \(X \subseteq V \setminus \{r\}\) que teve seu potencial elevado durante a Fase~1 possui exatamente um arco entrando na arborescência final.

Recebe como entrada a arborescência \texttt{Arb} e a lista \texttt{Dual\_list} contendo pares \((X, \Delta(X))\) onde \(X\) é um conjunto de vértices cujos potenciais foram elevados e \(\Delta(X) > 0\) é o valor da elevação. A implementação itera sobre cada par \((X, z)\) na lista dual (linha 2), e para cada conjunto \(X\), percorre todos os arcos \((u,v)\) da arborescência (linha 3). Inicializa um contador \texttt{count} em zero (linha 4) e verifica se o arco cruza a fronteira de \(X\), isto é, se \(u \notin X\) e \(v \in X\) (linha 5). Quando essa condição é satisfeita, incrementa o contador (linha 6) e imediatamente verifica se já há mais de um arco entrando em \(X\) (linha 7). Caso positivo, a condição de otimalidade dual é violada e a função devolve \texttt{False} (linha 8).

A função devolve \texttt{True} se todos os conjuntos em \texttt{Dual\_list} possuem exatamente um arco entrando na arborescência, confirmando que a solução satisfaz as condições de folga complementar da programação linear dual. A complexidade é \(O(km)\), onde \(k = |\texttt{Dual\_list}|\) e \(m = |A|\), pois para cada conjunto dual verifica-se todos os arcos da arborescência.

Esta verificação é fundamental para garantir a correção do algoritmo: a Fase~1 constrói uma solução dual viável (potenciais \(y(v)\)), a Fase~2 constrói uma solução primal viável (arborescência), e esta função confirma que ambas satisfazem as condições de folga complementar, implicando otimalidade pelo teorema da dualidade forte.

\begin{tcolorbox}[
		enhanced, breakable,
		colframe=blue!60!black, colback=blue!2,
		colbacktitle=blue!15, coltitle=black,
		title={Verificação de otimalidade dual},
		boxed title style={sharp corners, boxrule=0.6pt},
		sharp corners, boxrule=0.6pt
	]
	\emph{Verifica se a condição de otimalidade dual é satisfeita, confirmando que cada conjunto dual possui exatamente um arco entrando na arborescência.}
	\tcblower
	\begin{lstlisting}[language=Python]
def check_dual_optimality_condition(Arb, Dual_list):
    for X, z in Dual_list:
        count = 0
        for u, v in Arb.edges():
            if u not in X and v in X:
                count += 1
                if count > 1:
                    return False
    return True
\end{lstlisting}
\end{tcolorbox}

\subsection{O algoritmo completo de András Frank}

Finalmente, apresentamos a função principal que implementa o algoritmo de András Frank para encontrar uma r-arborescência de custo mínimo em um dígrafo com pesos. A função integra as fases de construção dos potenciais duais, obtenção dos arcos justos, construção da arborescência e verificação de otimalidade dual.

\begin{tcolorbox}[
		enhanced, breakable,
		colframe=blue!60!black, colback=blue!2,
		colbacktitle=blue!15, coltitle=black,
		title={Verificação de otimalidade dual},
		boxed title style={sharp corners, boxrule=0.6pt},
		sharp corners, boxrule=0.6pt
	]
	\emph{Implementa o algoritmo completo de András Frank, integrando as fases de construção dos potenciais duais, obtenção dos arcos justos, construção da arborescência e verificação de otimalidade dual.}
	\tcblower
	\begin{lstlisting}[language=Python]
def andras_frank_algorithm(D):
    A_zero, Dual_list = phase1(D,"r0")
    arborescence_frank = phase2(D, "r0", A_zero)
    arborescence_frank_v2 = phase2_v2(D, "r0", A_zero)
    dual_frank = check_dual_optimality_condition(
        arborescence_frank, Dual_list)
    dual_frank_v2 = check_dual_optimality_condition(
        arborescence_frank_v2, Dual_list)
    return arborescence_frank, arborescence_frank_v2, dual_frank, dual_frank_v2
\end{lstlisting}
\end{tcolorbox}

\subsection*{Exemplo de execução do algoritmo}

Aqui ilustraremos cada fase do algoritmo de András Frank: inicialização dos potenciais, iterações de elevação, construção de \(A_0\), e formação da arborescência final. Utilizaremos o dígrafo \(D_{32}\) estendido com a raiz \(r_0\), mostrado na Figura~\ref{fig:frank-exemplo-inicial}.

\begin{figure}[H]
	\centering
	\begin{tikzpicture}[>=Stealth, scale=0.9, every node/.style={scale=0.9}]
		\tikzset{
			vtx/.style={circle, draw, fill=blue!8, thick, inner sep=1.2pt, minimum size=6.5mm},
			vtxRoot/.style={circle, draw, fill=red!15, thick, inner sep=1.2pt, minimum size=6.5mm},
			arc/.style={->, thick, gray!60},
			cost/.style={font=\scriptsize, fill=white, inner sep=1pt, rounded corners=1pt}
		}

		\node[font=\bfseries, blue!80!black] at (3, 4.5) {Dígrafo inicial $D$ com raiz $r_0$};

		% Raiz
		\node[vtxRoot] (r0) at (3, 3) {$r_0$};

		% Vértices camada 1
		\node[vtx] (u1) at (0, 1.5) {$u_1$};
		\node[vtx] (u2) at (3, 1.5) {$u_2$};
		\node[vtx] (u3) at (6, 1.5) {$u_3$};

		% Vértices camada 2
		\node[vtx] (v1) at (1, 0) {$v_1$};
		\node[vtx] (v2) at (3, 0) {$v_2$};
		\node[vtx] (v3) at (5, 0) {$v_3$};

		% Arcos da raiz
		\draw[arc] (r0) -- node[cost, left, pos=0.4] {1} (u1);
		\draw[arc] (r0) -- node[cost, right] {2} (u2);
		\draw[arc] (r0) -- node[cost, right, pos=0.4] {1} (u3);

		% Arcos camada 1
		\draw[arc] (u1) to[bend left=15] node[cost, above] {1} (u2);
		\draw[arc] (u2) to[bend left=15] node[cost, above] {3} (u3);

		% Arcos entrando camada 2
		\draw[arc] (u1) -- node[cost, left] {3} (v1);
		\draw[arc] (u2) -- node[cost, right] {2} (v2);
		\draw[arc] (u3) -- node[cost, right] {4} (v3);
		\draw[arc] (u1) to[bend right=20] node[cost, below, pos=0.3] {5} (v2);

		% Arcos internos camada 2
		\draw[arc] (v1) -- node[cost, above] {1} (v2);
		\draw[arc] (v2) -- node[cost, above] {2} (v3);

		% Arco retorno
		\draw[arc] (v1) to[bend right=30] node[cost, left, pos=0.3] {2} (u1);

		\node[font=\small, gray!70!black] at (3, -1.2) {Pesos originais $c(u,v)$};
	\end{tikzpicture}
	\caption{Dígrafo inicial \(D\) com raiz \(r_0\) e 6 vértices adicionais. O dígrafo contém múltiplos arcos com pesos variados. A Fase~1 do algoritmo iniciará com potenciais \(y(v) = 0\) para todos os vértices, correspondendo aos custos reduzidos iniciais \(c_y(u,v) = c(u,v)\).}
	\label{fig:frank-exemplo-inicial}
\end{figure}

\paragraph*{Fase 1 — Iteração 1:} O algoritmo inicia com todos os potenciais \(y(v) = 0\) e \(D_0\) vazio (sem arcos justos). Calcula-se as componentes fortemente conexas de \(D_0\): cada vértice forma sua própria componente. As fontes são todos os vértices exceto \(r_0\). Para o conjunto \(X = \{u_1, u_2, u_3\}\), identifica-se os arcos entrando: \((r_0, u_1)\) com peso 1, \((r_0, u_2)\) com peso 2, \((r_0, u_3)\) com peso 1. O mínimo é \(\Delta(X) = 1\). Subtraindo esse valor, os arcos \((r_0, u_1)\) e \((r_0, u_3)\) tornam-se justos (peso 0) e são adicionados a \(A_0\).

\begin{figure}[H]
	\centering
	\begin{tikzpicture}[>=Stealth, scale=0.85, every node/.style={scale=0.85}]
		\tikzset{
			vtx/.style={circle, draw, fill=blue!8, thick, inner sep=1.2pt, minimum size=6.5mm},
			vtxRoot/.style={circle, draw, fill=red!15, thick, inner sep=1.2pt, minimum size=6.5mm},
			vtxX/.style={circle, draw, fill=orange!20, thick, inner sep=1.2pt, minimum size=6.5mm},
			arc/.style={->, thick, gray!60},
			arcTight/.style={->, ultra thick, blue!70!black},
			cost/.style={font=\scriptsize, fill=white, inner sep=1pt},
			costTight/.style={font=\scriptsize\bfseries, fill=blue!10, inner sep=2pt, draw=blue!70!black, rounded corners=1pt}
		}

		% Antes
		\begin{scope}
			\node[font=\bfseries, blue!80!black] at (2, 4.5) {Antes: $\Delta(X_1) = 1$};

			\node[vtxRoot] (r0) at (2, 3) {$r_0$};
			\node[vtxX] (u1) at (0, 1.5) {$u_1$};
			\node[vtxX] (u2) at (2, 1.5) {$u_2$};
			\node[vtxX] (u3) at (4, 1.5) {$u_3$};

			\draw[arc] (r0) -- node[cost, left, pos=0.4] {1} (u1);
			\draw[arc] (r0) -- node[cost, right] {2} (u2);
			\draw[arc] (r0) -- node[cost, right, pos=0.4] {1} (u3);

			\node[font=\small, orange!80!black] at (2, 0.5) {$X_1 = \{u_1, u_2, u_3\}$};
		\end{scope}

		% Depois
		\begin{scope}[xshift=7cm]
			\node[font=\bfseries, blue!80!black] at (2, 4.5) {Depois: arcos justos};

			\node[vtxRoot] (r0b) at (2, 3) {$r_0$};
			\node[vtx] (u1b) at (0, 1.5) {$u_1$};
			\node[vtx] (u2b) at (2, 1.5) {$u_2$};
			\node[vtx] (u3b) at (4, 1.5) {$u_3$};

			\draw[arcTight] (r0b) -- node[costTight, left, pos=0.4] {0} (u1b);
			\draw[arc] (r0b) -- node[cost, right] {1} (u2b);
			\draw[arcTight] (r0b) -- node[costTight, right, pos=0.4] {0} (u3b);

			\node[font=\small, blue!70!black] at (2, 0.5) {2 arcos adicionados a $A_0$};
		\end{scope}
	\end{tikzpicture}
	\caption{Fase~1, Iteração 1: Elevação de potenciais para \(X_1 = \{u_1, u_2, u_3\}\). À esquerda, o conjunto \(X_1\) (em laranja) sem arcos justos entrando. Calcula-se \(\Delta(X_1) = \min\{1, 2, 1\} = 1\). À direita, após subtrair \(\Delta(X_1)\), os arcos \((r_0, u_1)\) e \((r_0, u_3)\) tornam-se \textcolor{blue!70!black}{justos} (custo reduzido 0) e são adicionados a \(A_0\).}
	\label{fig:frank-exemplo-iter1}
\end{figure}

\paragraph*{Fase 1 — Iteração 2:} Com \(A_0 = \{(r_0, u_1), (r_0, u_3)\}\), as componentes fortemente conexas são: \(\{r_0, u_1, u_3\}\) e \(\{u_2\}\), \(\{v_1\}\), \(\{v_2\}\), \(\{v_3\}\). A fonte (excluindo a raiz) é \(\{u_2\}\). Para \(X = \{u_2\}\), o único arco entrando é \((r_0, u_2)\) com peso atual 1 (após iteração anterior). Portanto \(\Delta(X) = 1\), e após subtração, \((r_0, u_2)\) torna-se justo. Também na mesma iteração, para \(X = \{v_1, v_2, v_3\}\), calcula-se \(\Delta(X) = \min\{3, 2, 4, 5\} = 2\), e os arcos são atualizados conforme ilustrado.

\begin{figure}[H]
	\centering
	\begin{tikzpicture}[>=Stealth, scale=0.75, every node/.style={scale=0.75}]
		\tikzset{
			vtx/.style={circle, draw, fill=blue!8, thick, inner sep=1.2pt, minimum size=6mm},
			vtxRoot/.style={circle, draw, fill=red!15, thick, inner sep=1.2pt, minimum size=6mm},
			vtxX/.style={circle, draw, fill=orange!20, thick, inner sep=1.2pt, minimum size=6mm},
			arc/.style={->, thick, gray!60},
			arcTight/.style={->, ultra thick, blue!70!black},
			cost/.style={font=\scriptsize, fill=white, inner sep=1pt},
			costTight/.style={font=\scriptsize\bfseries, fill=blue!10, inner sep=2pt, draw=blue!70!black, rounded corners=1pt}
		}

		\node[font=\bfseries, blue!80!black] at (3, 5.5) {Iteração 2: $\Delta(\{v_1, v_2, v_3\}) = 2$};

		\node[vtxRoot] (r0) at (3, 4) {$r_0$};
		\node[vtx] (u1) at (0.5, 2.5) {$u_1$};
		\node[vtx] (u2) at (3, 2.5) {$u_2$};
		\node[vtx] (u3) at (5.5, 2.5) {$u_3$};
		\node[vtxX] (v1) at (1.5, 0.5) {$v_1$};
		\node[vtxX] (v2) at (3, 0.5) {$v_2$};
		\node[vtxX] (v3) at (4.5, 0.5) {$v_3$};

		% Arcos já justos
		\draw[arcTight] (r0) -- node[costTight, left, pos=0.3] {0} (u1);
		\draw[arcTight] (r0) -- node[costTight, right, pos=0.3] {0} (u3);
		\draw[arcTight] (u1) to[bend left=12] node[costTight, above] {0} (u2);

		% Arcos entrando em X (antes de ficarem justos)
		\draw[arc] (u1) -- node[cost, left, fill=orange!10] {3→1} (v1);
		\draw[arc] (u2) -- node[cost, right, fill=orange!10] {2→0} (v2);
		\draw[arc] (u3) -- node[cost, right] {4→2} (v3);
		\draw[arc] (u1) to[bend right=15] node[cost, below, pos=0.35] {5→3} (v2);

		\node[font=\small, orange!80!black] at (3, -0.5) {Após subtração: $(u_2, v_2)$ torna-se justo};
	\end{tikzpicture}
	\caption{Fase~1, Iteração 2: Para o conjunto \(X = \{v_1, v_2, v_3\}\) (em laranja), calcula-se \(\Delta(X) = \min\{3, 2, 4, 5\} = 2\). Após subtrair esse valor dos arcos entrando em \(X\), o arco \((u_2, v_2)\) atinge custo reduzido 0 e é adicionado a \(A_0\). Note que \((u_1, u_2)\) também se tornou justo na mesma iteração ao processar \(X = \{u_2\}\).}
	\label{fig:frank-exemplo-iter2}
\end{figure}

\paragraph*{Fase 1 — Estado final:} Após mais iterações elevando potenciais para os conjuntos restantes, o dígrafo \(D_0\) de arcos justos fica completo, contendo uma r-arborescência. A Figura~\ref{fig:frank-exemplo-A0-final} mostra o conjunto final \(A_0\) com 6 arcos de custo reduzido zero.

\begin{figure}[H]
	\centering
	\begin{tikzpicture}[>=Stealth, scale=0.85, every node/.style={scale=0.85}]
		\tikzset{
			vtx/.style={circle, draw, fill=blue!8, thick, inner sep=1.2pt, minimum size=6.5mm},
			vtxRoot/.style={circle, draw, fill=red!15, thick, inner sep=1.2pt, minimum size=6.5mm},
			arcTight/.style={->, ultra thick, blue!70!black},
			costTight/.style={font=\scriptsize\bfseries, fill=blue!10, inner sep=2pt, draw=blue!70!black, rounded corners=1pt}
		}

		\node[font=\bfseries, blue!80!black] at (3, 4.5) {$A_0$ final: arcos justos após Fase~1};

		\node[vtxRoot] (r0) at (3, 3) {$r_0$};
		\node[vtx] (u1) at (0.5, 1.5) {$u_1$};
		\node[vtx] (u2) at (3, 1.5) {$u_2$};
		\node[vtx] (u3) at (5.5, 1.5) {$u_3$};
		\node[vtx] (v1) at (1.5, 0) {$v_1$};
		\node[vtx] (v2) at (3, 0) {$v_2$};
		\node[vtx] (v3) at (4.5, 0) {$v_3$};

		\draw[arcTight] (r0) -- node[costTight, left, pos=0.3] {0} (u1);
		\draw[arcTight] (r0) -- node[costTight, right, pos=0.3] {0} (u3);
		\draw[arcTight] (u1) to[bend left=12] node[costTight, above] {0} (u2);
		\draw[arcTight] (u1) -- node[costTight, left] {0} (v1);
		\draw[arcTight] (u2) -- node[costTight, right] {0} (v2);
		\draw[arcTight] (v2) -- node[costTight, above] {0} (v3);

		\node[font=\small, blue!70!black] at (3, -1.2) {$|A_0| = 6$ arcos com custo reduzido $c_y = 0$};
	\end{tikzpicture}
	\caption{Conjunto final \(A_0\) de arcos justos após conclusão da Fase~1. Todos os 6 arcos \textcolor{blue!70!black}{destacados} possuem custo reduzido zero: \((r_0, u_1)\), \((r_0, u_3)\), \((u_1, u_2)\), \((u_1, v_1)\), \((u_2, v_2)\), \((v_2, v_3)\). O dígrafo formado por esses arcos é acíclico e alcança todos os vértices a partir de \(r_0\), satisfazendo as condições para prosseguir à Fase~2.}
	\label{fig:frank-exemplo-A0-final}
\end{figure}

\paragraph*{Fase 2 — Construção da arborescência:} Com \(A_0\) completo e acíclico, a Fase~2 constrói incrementalmente a arborescência final. Inicia-se com \(\texttt{Arb} = \{r_0\}\) e em cada iteração adiciona-se um arco \((u,v) \in A_0\) tal que \(u \in \texttt{Arb}\) e \(v \notin \texttt{Arb}\). A Figura~\ref{fig:frank-exemplo-arborescencia-final} mostra a arborescência resultante.

\begin{figure}[H]
	\centering
	\begin{tikzpicture}[>=Stealth, scale=0.85, every node/.style={scale=0.85}]
		\tikzset{
			vtx/.style={circle, draw, fill=green!15, thick, inner sep=1.2pt, minimum size=6.5mm},
			vtxRoot/.style={circle, draw, fill=red!15, thick, inner sep=1.2pt, minimum size=6.5mm},
			arcArb/.style={->, ultra thick, green!70!black},
			costArb/.style={font=\scriptsize\bfseries, fill=green!10, inner sep=2pt, draw=green!70!black, rounded corners=1pt}
		}

		\node[font=\bfseries, blue!80!black] at (3, 4.5) {Arborescência final (Fase~2)};

		\node[vtxRoot] (r0) at (3, 3) {$r_0$};
		\node[vtx] (u1) at (0.5, 1.5) {$u_1$};
		\node[vtx] (u2) at (3, 1.5) {$u_2$};
		\node[vtx] (u3) at (5.5, 1.5) {$u_3$};
		\node[vtx] (v1) at (1.5, 0) {$v_1$};
		\node[vtx] (v2) at (3, 0) {$v_2$};
		\node[vtx] (v3) at (4.5, 0) {$v_3$};

		\draw[arcArb] (r0) -- node[costArb, left, pos=0.3] {1} (u1);
		\draw[arcArb] (r0) -- node[costArb, right, pos=0.3] {1} (u3);
		\draw[arcArb] (u1) to[bend left=12] node[costArb, above] {1} (u2);
		\draw[arcArb] (u1) -- node[costArb, left] {3} (v1);
		\draw[arcArb] (u2) -- node[costArb, right] {2} (v2);
		\draw[arcArb] (v2) -- node[costArb, above] {2} (v3);

		\node[font=\small, green!70!black] at (3, -1.2) {Custo total: $1+1+1+3+2+2 = 10$};
	\end{tikzpicture}
	\caption{Arborescência de custo mínimo final obtida pela Fase~2. Os pesos mostrados são os \emph{originais} de \(D\), restaurados após a construção. Cada vértice não-raiz possui exatamente um arco de entrada, não há ciclos, e todos os vértices são alcançáveis a partir de \(r_0\). A verificação de otimalidade dual confirma que cada conjunto \(X\) que teve potenciais elevados possui exatamente um arco da arborescência cruzando sua fronteira, garantindo que a solução é ótima.}
	\label{fig:frank-exemplo-arborescencia-final}
\end{figure}

\paragraph*{Verificação de otimalidade dual:} A função \texttt{check\_dual\_optimality\_condition} confirma que para cada par \((X, \Delta(X))\) em \texttt{Dual\_list} (conjuntos cujos potenciais foram elevados com \(\Delta(X) > 0\)), existe exatamente um arco da arborescência final cruzando a fronteira de \(X\). Essa condição, juntamente com os arcos justos, garante que as condições de folga complementar da programação linear dual são satisfeitas, implicando que a arborescência encontrada é de custo mínimo global.

\subsection{Correspondência entre teoria e implementação}

A implementação em Python do algoritmo de András Frank segue fielmente a descrição teórica primal-dual apresentada anteriormente. A tabela abaixo estabelece o paralelo direto entre os passos teóricos e sua realização no código:

\begin{table}[H]
	\centering
	\scriptsize
	\renewcommand{\arraystretch}{1.3}
	\begin{tabular}{>{\raggedright\arraybackslash}p{0.44\textwidth}|>{\raggedright\arraybackslash}p{0.48\textwidth}}
		\toprule
		\textbf{\normalsize Descrição Teórica} & \textbf{\normalsize Implementação Python} \\
		\midrule
		\rowcolor{blue!5}
		\textbf{Passo 1:} Inicialização

		Defina \(y(v) := 0\) para todo \(v \in V\).

		Inicialize \(A_0 := \emptyset\).

		Construa dígrafo vazio \(D_0\) (arcos justos).
		                                       &
		\textbf{Função \texttt{phase1} — Linhas 2--5:}

		\texttt{D\_copy = D\_original.copy()}

		\texttt{A\_zero = []}

		\texttt{D\_zero = build\_D\_zero(D\_copy)}

		\vspace{1mm}
		Potenciais \(y(v) = 0\) implícitos, custos \(c_y = c\).
		\\
		\midrule
		\rowcolor{green!5}
		\textbf{Passo 2:} Elevação de potenciais

		Enquanto \(\exists X \subseteq V \setminus \{r\}\) sem arco justo:

		\quad Calcule \(\Delta(X) := \min\{c_y(u,v) : u \notin X, v \in X\}\)

		\quad Atualize \(y(v) := y(v) + \Delta(X)\), \(\forall v \in X\)

		\quad Adicione arcos com \(c_y = 0\) a \(A_0\)
		                                       &
		\textbf{Loop principal — Linhas 7--22:}

		\texttt{C = nx.condensation(D\_zero)}

		\texttt{sources = [x for x in C.nodes()]}

		\texttt{\quad if C.in\_degree(x) == 0]}

		Para fonte \texttt{u} (exceto raiz):

		\texttt{\quad X = C.nodes[u]["members"]}

		\texttt{\quad arcs = get\_arcs\_entering\_X(D, X)}

		\texttt{\quad min\_w = get\_minimum\_weight\_cut(arcs)}

		\texttt{\quad update\_weights\_in\_X(D, arcs, min\_w,}

		\texttt{\quad\quad A\_zero, D\_zero)}
		\\
		\midrule
		\rowcolor{orange!5}
		\textbf{Passo 2(a):} Identificar arcos entrando

		Determine \(\{(u,v) \in A : u \notin X, v \in X\}\)
		                                       &
		\textbf{Função \texttt{get\_arcs\_entering\_X}:}

		\texttt{return [(u, v, data)}

		\texttt{\quad for u, v, data in D.edges(data=True)}

		\texttt{\quad if u not in X and v in X]}
		\\
		\midrule
		\rowcolor{purple!5}
		\textbf{Passo 2(b):} Calcular \(\Delta(X)\)

		\(\Delta(X) := \min\{c_y(u,v) : u \notin X, v \in X\}\)
		                                       &
		\textbf{Função \texttt{get\_minimum\_weight\_cut}:}

		\texttt{return min(data["w"]}

		\texttt{\quad for \_, \_, data in arcs)}
		\\
		\midrule
		\rowcolor{yellow!5}
		\textbf{Passo 2(c):} Atualizar pesos

		Para \((u,v)\) entrando em \(X\):

		\quad \(c_y(u,v) := c_y(u,v) - \Delta(X)\)

		\quad Se \(c_y(u,v) = 0\), adicione a \(A_0\)
		                                       &
		\textbf{Função \texttt{update\_weights\_in\_X}:}

		\texttt{for u, v, \_ in arcs:}

		\texttt{\quad D[u][v]["w"] -= min\_weight}

		\texttt{\quad if D[u][v]["w"] == 0:}

		\texttt{\quad\quad A\_zero.append((u, v))}

		\texttt{\quad\quad D\_zero.add\_edge(u, v)}
		\\
		\midrule
		\rowcolor{cyan!5}
		\textbf{Passo 3:} Verificar término

		Se \(D_0\) contém r-arborescência, encerre.
		                                       &
		\textbf{Condição — Linhas 11--12:}

		\texttt{if len(sources) == 1: break}

		\vspace{1mm}
		Uma fonte \(\Rightarrow\) r-arborescência acíclica.
		\\
		\midrule
		\rowcolor{red!5}
		\textbf{Passo 4:} Construir arborescência

		Construa \(F\) a partir de \(A_0\), conectando vértices incrementalmente.
		                                       &
		\textbf{Função \texttt{phase2} (incremental):}

		\texttt{Arb = nx.DiGraph(); Arb.add\_node(r0)}

		\texttt{for \_ in range(n - 1):}

		\texttt{\quad for u, v in A\_zero:}

		\texttt{\quad\quad if u in Arb and v not in Arb:}

		\texttt{\quad\quad\quad Arb.add\_edge(u, v, **data)}

		\texttt{\quad\quad\quad break}

		\vspace{1mm}
		Complexidade: \(O(nm)\).
		\\
		\midrule
		\rowcolor{red!5}
		\textbf{Passo 4:} Versão otimizada

		Mesma ideia, com fila de prioridade.
		                                       &
		\textbf{Função \texttt{phase2\_v2} (BFS):}

		\texttt{Arb = nx.DiGraph()}

		\texttt{for i, (u, v) in enumerate(A\_zero):}

		\texttt{\quad Arb.add\_edge(u, v, w=i)}

		\texttt{q = [] \# fila de prioridade}

		\texttt{while q:}

		\texttt{\quad \_, u, v = heapq.heappop(q)}

		\texttt{\quad if v in V: continue}

		\texttt{\quad A.add\_edge(u, v, w=D[u][v]["w"])}

		\vspace{1mm}
		Complexidade: \(O(m \log m)\).
		\\
		\midrule
		\rowcolor{pink!5}
		\textbf{Otimalidade dual}

		Para cada \(X\) elevado (\(\Delta(X) > 0\)), exatamente um arco de \(F\) cruza \(\delta^-(X)\).
		                                       &
		\textbf{Função \texttt{check\_dual\_optimality}:}

		\texttt{for X, z in Dual\_list:}

		\texttt{\quad count = 0}

		\texttt{\quad for u, v in Arb.edges():}

		\texttt{\quad\quad if u not in X and v in X:}

		\texttt{\quad\quad\quad count += 1}

		\texttt{\quad\quad\quad if count > 1: return False}

		\texttt{return True}
		\\
		\bottomrule
	\end{tabular}
	\caption{Correspondência entre a descrição teórica do algoritmo de András Frank e sua implementação em Python. Cores: inicialização (azul), elevação de potenciais (verde/laranja/roxo/amarelo), verificação (ciano), construção (vermelho) e validação dual (rosa).}
	\label{tab:frank-teoria-implementacao}
\end{table}

Esta correspondência demonstra que a implementação traduz fielmente a abordagem primal-dual em código executável. As funções auxiliares (\texttt{get\_arcs\_entering\_X}, \texttt{get\_minimum\_weight\_cut}, \texttt{update\_weights\_in\_X}, \texttt{phase1}, \texttt{phase2}, \texttt{phase2\_v2},
\texttt{check\_dual\_}-
\texttt{optimality\_condition} encapsulam exatamente as operações descritas na teoria, preservando as propriedades de correção e as garantias de otimalidade do algoritmo original. A utilização de componentes fortemente conexas (\texttt{nx.condensation}) para identificar conjuntos sem arcos justos entrando é uma implementação eficiente da verificação teórica, evitando enumeração explícita de todos os subconjuntos de vértices.

\chapter{Algoritmo de András Frank}

Neste capítulo, apresentaremos o algoritmo de András Frank, que também determina
uma arborescência de custo mínimo em um digrafo ponderado.
O algoritmo baseia-se em uma abordagem primal--dual:
a elevação gulosa de potenciais duais\footnote{\textbf{Potencial dual} é uma variável associada a cada vértice, que representa um "desconto" aplicado ao custo dos arcos que entram nesse vértice e servem para ajustar os custos de modo que certas restrições fiquem "justas", - satisfeitas com igualdade, ou seja, o valor da variável atinge exatamente o limite imposto pela restrição.} e (ii) a construção de uma solução gulosa
a partir dos arcos de custo reduzido zero.
Diferentemente do algoritmo de Chu--Liu--Edmonds, que opera diretamente sobre os custos,
o método de Frank manipula potenciais duais associados aos vértices,
gerando arcos \emph{justos} (de custo reduzido zero) que formam a arborescência ótima.
O propósito deste capítulo é fornecer uma descrição precisa tanto do algoritmo
quanto da implementação desenvolvida neste trabalho.

\section{O algoritmo}
O algoritmo de András Frank também recebe uma tripla \((D,c,r)\), em que \(D=(V,A)\) é um digrafo,
\(c\colon A\to\mathbb{R}\) é uma função custo e \(r\in V\) é a raiz,
sob a hipótese de que \(D\) admite ao menos uma \(r\)-arborescência e devolve uma \(r\)-arborescência \(c\)-mínima de \(D\).

Assim como no capítulo anterior, adotamos a terminologia de \textbf{\(r\)-digrafo ponderado}
para uma tripla \((D,c,r)\) em que $(D, c)$ é um digrafo ponderado,
$r$ é um vértice de $D$, $\delta^-(r) = \varnothing$
e $D$ possui uma $r$-arborescência.

Vamos desenvolver as ideias do algoritmo utilizando o mesmo dígrafo que apresentamos
no capítulo anterior.

\subsection*{Abordagem Gulosa--Dual}

Considere novamente o dígrafo \(D\) figura 26 a seguir, com custos nos arcos note que é o mesmo digrafo da figura 1 que utilizamos para ilustrar o algoritmo de Chu--Liu--Edmonds. E portanto, suponha que T é uma arborescência de custo mínimo, com seus arcos destacados em azul.

\begin{figure}[h]
	\centering
	\begin{tikzpicture}[>=Stealth]
		% estilos
		\tikzset{
			vtx/.style={circle, draw,fill=blue!8, thick, inner sep=1.2pt, minimum size=6mm},
			chosen/.style={->, very thick, draw=blue},
			other/.style={->, semithick, draw=black!35},
			costB/.style={font=\scriptsize, fill=white, inner sep=1pt, text=blue},
			costG/.style={font=\scriptsize, fill=white, inner sep=1pt, text=black!55}
		}

		% layout (mais espaçado)
		\node[vtx] (r) at (0,0) {$r$};
		\node[vtx] (a) at (3,1.6) {$a$};
		\node[vtx] (b) at (3,-1.6) {$b$};
		\node[vtx] (c) at (6,2.2) {$c$};
		\node[vtx] (d) at (6,0.0) {$d$};
		\node[vtx] (e) at (6,-2.2) {$e$};

		% --- arcos NÃO escolhidos (cinza), agora com curvaturas mais limpas ---
		\draw[other] (b) to[bend left=12] node[costG, above] {4} (a);   % b->a
		\draw[other] (a) to[bend left=14] node[costG, below] {5} (b);   % a->b

		% r->c: externo por cima, suave (pontos de controle explícitos)
		\draw[other]
		(r) .. controls (0.8,2.4) and (5.2,3.0) ..
		node[costG, yshift=4pt, pos=0.53] {3} (c);

		% a->d (referência)
		\draw[other] (a) -- node[costG, right] {3} (d);

		% c->e: leve curvatura pela direita (sem "barriga")
		\draw[other]
		(c) .. controls (7.6,2.6) and (7.6,-2.6) ..
		node[costG, xshift=8pt, pos=0.52] {5} (e);

		% --- arcos ESCOLHIDOS (AZUL) — formam a r-arborescência T ---
		\draw[chosen] (r) -- node[costB, above] {1} (a);   % r->a
		\draw[chosen] (r) -- node[costB, below] {2} (b);   % r->b
		\draw[chosen] (a) -- node[costB, above] {1} (c);   % a->c
		\draw[chosen] (b) -- node[costB, above] {1} (d);   % b->d
		\draw[chosen] (b) -- node[costB, below] {2} (e);   % b->e
	\end{tikzpicture}

	\caption{O dígrafo \(D\) com custos nos arcos. Os arcos em \textcolor{blue}{azul}
		formam a \(r\)-arborescência de custo mínimo \(T\) (custo total = 7).}
\end{figure}

O algoritmo de Frank utiliza \textbf{potenciais} \(y: V \to \mathbb{R}\) associados aos vértices
para definir custos reduzidos nos arcos. Para cada arco \(a = (u,v) \in A\), o
\textbf{custo \(y\)-reduzido} é dado por
\[
	c_y(u,v) := c(u,v) - y(v),
\]
onde \(v\) é o vértice de destino do arco. Note que, diferentemente do procedimento de redução do capítulo anterior (onde subtraíamos o mínimo custo de entrada em cada vértice), aqui o potencial \(y(v)\) é uma variável que será ajustada pelo algoritmo através da seguinte noção.

Um arco \(a = (u,v)\) é dito \textbf{justo} (ou \textbf{tight}) se \(c_y(u,v) = 0\), isto é, se o custo original coincide com o potencial do vértice de destino: \(c(u,v) = y(v)\). A ideia central do algoritmo de Frank consiste em duas etapas: (i)~elevar os potenciais \(y(v)\) para cada \(v \neq r\) até que cada vértice possua ao menos um arco justo entrando nele; (ii)~construir uma arborescência ótima utilizando exclusivamente arcos justos.

\begin{tcolorbox}[
		enhanced, breakable,
		colback=blue!5!white, colframe=blue!70!black,
		colbacktitle=blue!85!black, coltitle=white,
		fonttitle=\bfseries,
		title={Arcos justos: interpretação e cálculo},
		sharp corners, boxrule=0.8pt
	]
	Um arco \((u,v)\) é \textbf{justo} quando seu custo reduzido é zero, ou seja, quando o custo original \(c(u,v)\) coincide exatamente com o potencial \(y(v)\) do vértice de destino. Formalmente:
	\[
		c_y(u,v) = c(u,v) - y(v) = 0 \quad\Longleftrightarrow\quad c(u,v) = y(v).
	\]

	Em outras palavras, o potencial \(y(v)\) funciona como um ``desconto'' aplicado aos arcos que chegam em \(v\). Quando esse desconto é suficiente para anular o custo reduzido, o arco torna-se justo e candidato a participar da solução ótima.

	\medskip
	\textbf{Exemplos numéricos (referência: figura~26):}
	\begin{itemize}\setlength{\itemsep}{3pt}
		\item Arco \((r,a)\): \(c(r,a) = 1\), \(y(a) = 1 \implies c_y(r,a) = 1-1 = 0\). \quad\textcolor{blue!70!black}{\textbf{Justo.}}
		\item Arco \((b,a)\): \(c(b,a) = 4\), \(y(a) = 1 \implies c_y(b,a) = 4-1 = 3\). \quad\textcolor{red!70!black}{\textbf{Não justo.}}
		\item Arco \((a,c)\): \(c(a,c) = 1\), \(y(c) = 1 \implies c_y(a,c) = 1-1 = 0\). \quad\textcolor{blue!70!black}{\textbf{Justo.}}
	\end{itemize}

	\textbf{Propriedade fundamental:} o potencial \(y(v)\) afeta \emph{exclusivamente} os arcos que entram em \(v\). Por exemplo, \(y(c)\) influencia apenas arcos da forma \((\cdot, c)\), como \((r,c)\) e \((a,c)\), não afetando arcos que saem de \(c\).

	\medskip
	O algoritmo eleva progressivamente os potenciais até que cada vértice \(v \neq r\) possua ao menos um arco justo de entrada. No exemplo da figura~26, os arcos justos após a elevação correspondem precisamente aos arcos da arborescência ótima.
\end{tcolorbox}

% Figura 27: Exemplo visual com múltiplos arcos e vértice raiz
\begin{figure}[H]
	\centering
	\begin{tikzpicture}[>=Stealth]
		% estilos
		\tikzset{
			vtx/.style={circle, draw, fill=blue!8, thick, inner sep=1.2pt, minimum size=8mm},
			arc/.style={->, very thick, draw=blue},
			arcTight/.style={->, very thick, draw=red!70!black},
			cost/.style={font=\scriptsize, fill=white, inner sep=1pt, text=blue},
			pot/.style={font=\small, text=red!70!black},
			label/.style={font=\small, text=black!80}
		}

		% vértices
		\node[vtx] (r) at (0,0) {$r$};
		\node[vtx] (u) at (3,1.5) {$u$};
		\node[vtx] (v) at (3,-1.5) {$v$};

		% arcos
		\draw[arcTight] (r) -- node[cost, above left] {$c(r,u)=2$} (u);
		\draw[arc] (r) -- node[cost, below left] {$c(r,v)=4$} (v);
		\draw[arcTight] (u) -- node[cost, right] {$c(u,v)=3$} (v);

		% potenciais
		\node[pot] at (3,2.5) {$y(u)=2$};
		\node[pot] at (3,-2.5) {$y(v)=3$};

		% destaques
		\draw[red!70, thick, dashed] (u) circle [radius=0.7];
		\draw[red!70, thick, dashed] (v) circle [radius=0.7];

		% custos reduzidos
		\node[label] at (-0.0,1.5) {$c_y(r,u)=2-2=0$};
		\node[label] at (-0.0,-1.5) {$c_y(r,v)=4-3=1$};
		\node[label] at (5.0,-0.5) {$c_y(u,v)=3-3=0$};
	\end{tikzpicture}
	\caption{O vértice $r$ é a raiz. Os potenciais são $y(u)=2$ e $y(v)=3$. O arco $(r,u)$ tem custo reduzido $0$ (justo), $(r,v)$ tem custo reduzido $1$ (não justo), $(u,v)$ tem custo reduzido $0$ (justo). Os potenciais só afetam os arcos que chegam em cada vértice.}
	\label{fig:potencial-destino}
\end{figure}

\subsection*{Elevação de Potenciais}

Inicialmente, definimos \(y(v) = 0\) para todo \(v \in V\).
Para cada vértice \(v \neq r\), calculamos
\[
	\Delta(v) := \min\{\, c(a) : a \in \delta^{-}(v) \,\} - y(v).
\]
O valor \(\Delta(v)\) representa quanto precisamos elevar \(y(v)\) para que exista
ao menos um arco de custo reduzido zero entrando em \(v\).

Vamos ilustrar esse processo no dígrafo \(D\).

\begin{figure}[H]
	\centering
	\begin{tikzpicture}[>=Stealth]
		% estilos
		\tikzset{
			vtx/.style={circle, draw,fill=blue!8, thick, inner sep=1.2pt, minimum size=6mm},
			arc/.style={->, semithick, draw=black!35},
			costG/.style={font=\scriptsize, fill=white, inner sep=1pt, text=black!55}
		}

		% layout
		\node[vtx] (r) at (0,0) {$r$};
		\node[vtx] (a) at (3,1.6) {$a$};
		\node[vtx] (b) at (3,-1.6) {$b$};
		\node[vtx] (c) at (6,2.2) {$c$};
		\node[vtx] (d) at (6,0.0) {$d$};
		\node[vtx] (e) at (6,-2.2) {$e$};

		% arcos com custos originais
		\draw[arc] (r) -- node[costG, above] {1} (a);
		\draw[arc] (r) -- node[costG, below] {2} (b);
		\draw[arc] (r) .. controls (0.8,2.4) and (5.2,3.0) .. node[costG, yshift=4pt, pos=0.53] {3} (c);
		\draw[arc] (b) to[bend left=12] node[costG, above] {4} (a);
		\draw[arc] (a) to[bend left=14] node[costG, below] {5} (b);
		\draw[arc] (a) -- node[costG, above] {1} (c);
		\draw[arc] (a) -- node[costG, right] {3} (d);
		\draw[arc] (b) -- node[costG, above] {1} (d);
		\draw[arc] (b) -- node[costG, below] {2} (e);
		\draw[arc] (c) .. controls (7.6,2.6) and (7.6,-2.6) .. node[costG, xshift=8pt, pos=0.52] {5} (e);

		% potenciais iniciais
		\node[above=6mm of a, font=\small] {$y(a) = 0$};
		\node[below=2mm of b, font=\small] {$y(b) = 0$};
		\node[above=2mm of c, font=\small] {$y(c) = 0$};
		\node[below=2mm of d, font=\small] {$y(d) = 0$};
		\node[below=2mm of e, font=\small] {$y(e) = 0$};
	\end{tikzpicture}
	\caption{Dígrafo \(D\) com potenciais iniciais \(y(v) = 0\) para todo \(v \in V\).}
\end{figure}

Calculamos \(\Delta(v)\) para cada vértice \(v \neq r\):
\begin{align*}
	\Delta(a) & = \min\{1, 4\} - 0 = 1, \\
	\Delta(b) & = \min\{2, 5\} - 0 = 2, \\
	\Delta(c) & = \min\{3, 1\} - 0 = 1, \\
	\Delta(d) & = \min\{3, 1\} - 0 = 1, \\
	\Delta(e) & = \min\{2, 5\} - 0 = 2.
\end{align*}

Elevamos cada potencial:
\begin{align*}
	y(a) & \leftarrow y(a) + \Delta(a) = 0 + 1 = 1, \\
	y(b) & \leftarrow y(b) + \Delta(b) = 0 + 2 = 2, \\
	y(c) & \leftarrow y(c) + \Delta(c) = 0 + 1 = 1, \\
	y(d) & \leftarrow y(d) + \Delta(d) = 0 + 1 = 1, \\
	y(e) & \leftarrow y(e) + \Delta(e) = 0 + 2 = 2.
\end{align*}

Após a elevação, os custos reduzidos são dados por \(c_y(u,v) = c(u,v) - y(v)\) para cada arco \((u,v)\):

\begin{figure}[H]
	\centering
	\begin{tikzpicture}[>=Stealth]
		% estilos
		\tikzset{
			vtx/.style={circle, draw,fill=blue!8, thick, inner sep=1.2pt, minimum size=6mm},
			tight/.style={->, very thick, draw=blue},
			loose/.style={->, semithick, draw=black!35},
			costB/.style={font=\scriptsize, fill=white, inner sep=1pt, text=blue},
			costG/.style={font=\scriptsize, fill=white, inner sep=1pt, text=black!55}
		}

		% layout
		\node[vtx] (r) at (0,0) {$r$};
		\node[vtx] (a) at (3,1.6) {$a$};
		\node[vtx] (b) at (3,-1.6) {$b$};
		\node[vtx] (c) at (6,2.2) {$c$};
		\node[vtx] (d) at (6,0.0) {$d$};
		\node[vtx] (e) at (6,-2.2) {$e$};

		% arcos justos (custo reduzido = 0)
		\draw[tight] (r) -- node[costB, above] {0} (a);   % 1 - 1 = 0
		\draw[tight] (r) -- node[costB, below] {0} (b);   % 2 - 2 = 0
		\draw[tight] (a) -- node[costB, above] {0} (c);   % 1 - 1 = 0
		\draw[tight] (b) -- node[costB, above] {0} (d);   % 1 - 1 = 0
		\draw[tight] (b) -- node[costB, below] {0} (e);   % 2 - 2 = 0

		% arcos não justos
		\draw[loose] (r) .. controls (0.8,2.4) and (5.2,3.0) .. node[costG, yshift=4pt, pos=0.53] {2} (c);  % 3 - 1 = 2
		\draw[loose] (b) to[bend left=12] node[costG, above] {3} (a);   % 4 - 1 = 3
		\draw[loose] (a) to[bend left=14] node[costG, below] {3} (b);   % 5 - 2 = 3
		\draw[loose] (a) -- node[costG, right] {2} (d);   % 3 - 1 = 2
		\draw[loose] (c) .. controls (7.6,2.6) and (7.6,-2.6) .. node[costG, xshift=8pt, pos=0.52] {3} (e);  % 5 - 2 = 3

		% potenciais finais
		\node[above=6mm of a, font=\small] {$y(a) = 1$};
		\node[below=2mm of b, font=\small] {$y(b) = 2$};
		\node[above=2mm of c, font=\small] {$y(c) = 1$};
		\node[below=2mm of d, font=\small] {$y(d) = 1$};
		\node[below=2mm of e, font=\small] {$y(e) = 2$};
	\end{tikzpicture}
	\caption{Dígrafo após elevação de potenciais. Arcos em \textcolor{blue}{azul} são
		justos (\(c_y = 0\)); arcos em cinza têm custo reduzido positivo.}
\end{figure}

Observe que cada vértice \(v \neq r\) possui agora ao menos um arco justo entrando nele.
Mais ainda: neste exemplo, os arcos justos formam uma \(r\)-arborescência! Precisamos justificar por que a arborescência formada pelos arcos justos é ótima.

\subsection*{Construção da Arborescência}

Se os arcos justos formam uma \(r\)-arborescência, então encontramos a solução ótima.
Caso contrário, os arcos justos contêm ciclos, e precisamos tratá-los adequadamente.

No nosso exemplo, os arcos justos são:
\[
	H = \{\, (r,a), (r,b), (a,c), (b,d), (b,e) \,\}.
\]
Esses arcos formam a \(r\)-arborescência \(T\) de custo mínimo.

Para verificar a otimalidade, note que uma propriedade fundamental dos potenciais duais é:

\begin{proposicao}
	Para toda função \(y: V \to \mathbb{R}\) com \(y(r) = 0\),
	uma \(r\)-arborescência \(T\) é \(c\)-mínima em \(D\) se, e somente se,
	todos os arcos de \(T\) são justos (i.e., \(c_y(a) = 0\) para todo \(a \in T\))
	e \(c_y(a) \geq 0\) para todo \(a \in A\).
\end{proposicao}

\begin{proof}[Prova.]
	Seja \(F\) uma \(r\)-arborescência qualquer. Para cada \(v \in V \setminus \{r\}\),
	seja \(a_v\) o único arco de \(F\) que entra em \(v\). Então
	\begin{align*}
		c(F)
		 & = \sum_{v \in V \setminus \{r\}} c(a_v)                                          \\
		 & = \sum_{v \in V \setminus \{r\}} \bigl(c_y(a_v) + y(v)\bigr)                     \\
		 & = \sum_{v \in V \setminus \{r\}} c_y(a_v) + \sum_{v \in V \setminus \{r\}} y(v).
	\end{align*}
	Como a soma \(\sum_{v \in V \setminus \{r\}} y(v)\) é constante para todas as
	\(r\)-arborescências, minimizar \(c(F)\) equivale a minimizar \(\sum_{v} c_y(a_v)\).
	Se todos os arcos de \(T\) são justos e \(c_y(a) \geq 0\) para todo \(a\),
	então \(\sum_{v} c_y(a_v^T) = 0 \leq \sum_{v} c_y(a_v^F)\) para qualquer \(F\),
	provando que \(T\) é ótima.
\end{proof}

No nosso exemplo, todos os arcos de \(T\) são justos e \(c_y(a) \geq 0\) para todo
arco \(a \in A\), portanto \(T\) é a \(r\)-arborescência de custo mínimo. Evidentemente, nem sempre será o caso que os arcos justos formam uma arborescência e precisaremos tratar ciclos.

\subsection*{Tratamento de Ciclos}

Em geral, após a elevação de potenciais, os arcos justos podem conter ciclos.
Nesse caso, o algoritmo de Frank procede de forma similar ao de Chu--Liu--Edmonds:
contrai cada ciclo justo em um supervértice, resolve o problema recursivamente
no dígrafo contraído, e depois reexpande a solução.

Vamos ilustrar esse processo com um exemplo modificado. Suponha que adicionamos o arco
\((c, a)\) com custo 1 ao dígrafo original:

\begin{figure}[H]
	\centering
	\begin{tikzpicture}[>=Stealth]
		% estilos
		\tikzset{
			vtx/.style={circle, draw,fill=blue!8, thick, inner sep=1.2pt, minimum size=6mm},
			arc/.style={->, semithick, draw=black!35},
			costG/.style={font=\scriptsize, fill=white, inner sep=1pt, text=black!55}
		}

		% layout
		\node[vtx] (r) at (0,0) {$r$};
		\node[vtx] (a) at (3,1.6) {$a$};
		\node[vtx] (b) at (3,-1.6) {$b$};
		\node[vtx] (c) at (6,2.2) {$c$};
		\node[vtx] (d) at (6,0.0) {$d$};
		\node[vtx] (e) at (6,-2.2) {$e$};

		% arcos originais
		\draw[arc] (r) -- node[costG, above] {1} (a);
		\draw[arc] (r) -- node[costG, below] {2} (b);
		\draw[arc] (r) .. controls (0.8,2.4) and (5.2,3.0) .. node[costG, yshift=4pt, pos=0.53] {3} (c);
		\draw[arc] (b) to[bend left=12] node[costG, above] {4} (a);
		\draw[arc] (a) to[bend left=14] node[costG, below] {5} (b);
		\draw[arc] (a) -- node[costG, above] {1} (c);
		\draw[arc] (a) -- node[costG, right] {3} (d);
		\draw[arc] (b) -- node[costG, above] {1} (d);
		\draw[arc] (b) -- node[costG, below] {2} (e);
		\draw[arc] (c) .. controls (7.6,2.6) and (7.6,-2.6) .. node[costG, xshift=8pt, pos=0.52] {5} (e);

		% NOVO arco (c, a)
		\draw[arc, draw=red!60] (c) to[bend left=20] node[costG, left, text=red!60] {1} (a);
	\end{tikzpicture}
	\caption{Dígrafo modificado com o arco \((c,a)\) de custo 1 (em vermelho).}
\end{figure}

Após a elevação de potenciais com \(y(a) = 1, y(c) = 1\), o arco \((c,a)\) torna-se
justo: \(c_y(c,a) = 1 - 1 = 0\).

\begin{figure}[H]
	\centering
	\begin{tikzpicture}[>=Stealth]
		% estilos
		\tikzset{
			vtx/.style={circle, draw,fill=blue!8, thick, inner sep=1.2pt, minimum size=6mm},
			tight/.style={->, very thick, draw=blue},
			cycle/.style={->, ultra thick, draw=red!70},
			loose/.style={->, semithick, draw=black!35},
			costB/.style={font=\scriptsize, fill=white, inner sep=1pt, text=blue},
			costR/.style={font=\scriptsize, fill=white, inner sep=1pt, text=red!70},
			costG/.style={font=\scriptsize, fill=white, inner sep=1pt, text=black!55}
		}

		% layout
		\node[vtx] (r) at (0,0) {$r$};
		\node[vtx] (a) at (3,1.6) {$a$};
		\node[vtx] (b) at (3,-1.6) {$b$};
		\node[vtx] (c) at (6,2.2) {$c$};
		\node[vtx] (d) at (6,0.0) {$d$};
		\node[vtx] (e) at (6,-2.2) {$e$};

		% arcos justos que formam ciclo
		\draw[cycle] (a) -- node[costR, above] {0} (c);
		\draw[cycle] (c) to[bend left=20] node[costR, left] {0} (a);

		% outros arcos justos
		\draw[tight] (r) -- node[costB, above] {0} (a);
		\draw[tight] (r) -- node[costB, below] {0} (b);
		\draw[tight] (b) -- node[costB, above] {0} (d);
		\draw[tight] (b) -- node[costB, below] {0} (e);

		% arcos não justos
		\draw[loose] (r) .. controls (0.8,2.4) and (5.2,3.0) .. node[costG, yshift=4pt, pos=0.53] {2} (c);
		\draw[loose] (b) to[bend left=12] node[costG, above] {3} (a);
		\draw[loose] (a) to[bend left=14] node[costG, below] {3} (b);
		\draw[loose] (a) -- node[costG, right] {2} (d);   % 3 - 1 = 2
		\draw[loose] (c) .. controls (7.6,2.6) and (7.6,-2.6) .. node[costG, xshift=8pt, pos=0.52] {3} (e);  % 5 - 2 = 3
	\end{tikzpicture}
	\caption{Arcos justos após elevação. O ciclo \(C = (a, c, a)\) é destacado em
		\textcolor{red!70}{vermelho}.}
\end{figure}

Nesse caso, contraímos o ciclo \(C = (a, c)\) em um supervértice \(x_C\), ajustamos
os custos dos arcos incidentes, e resolvemos o problema no dígrafo contraído.
A reexpansão é feita removendo um arco do ciclo para manter a estrutura de arborescência,
de forma análoga ao algoritmo de Chu--Liu--Edmonds.

\section{Descrição do algoritmo}

O algoritmo de András Frank difere substancialmente do algoritmo de Chu--Liu--Edmonds em sua abordagem: enquanto Chu--Liu--Edmonds opera diretamente sobre os custos dos arcos, o método de Frank utiliza uma técnica gulosa--dual baseada em \emph{potenciais} associados aos vértices. A ideia central é elevar progressivamente os potenciais \(y(v)\) para cada vértice \(v\neq r\), de modo que os custos \emph{reduzidos} \(c_y(u,v) := c(u,v) - y(v)\) revelem quais arcos devem participar da solução ótima. Um arco \((u,v)\) é dito \emph{justo} quando seu custo reduzido é zero, isto é, quando \(c(u,v) = y(v)\). O algoritmo garante que toda r-arborescência ótima pode ser formada apenas por arcos justos.

A seguir apresentamos uma descrição formal do algoritmo de András Frank e detalhes de implementação serão discutidos na próxima seção.

\begin{algobox}{András Frank}{andras-frank}
	Entrada: dígrafo \(D=(V,A)\), custos \(c:A\to\mathbb{R}_{\ge 0}\), raiz \(r\).\footnote{Se algum \(v\neq r\) não possui arco de entrada, não existe r-arborescência.}

	\textbf{Fase 1: Elevação de potenciais e construção de \(A_0\)}
	\begin{enumerate}\setlength{\itemsep}{2pt}
		\item \textbf{Inicialização:} defina \(y(v) := 0\) para todo \(v \in V\).
		\item \textbf{Iteração:} enquanto existir conjunto \(X \subseteq V \setminus \{r\}\) sem arco justo entrando:
		      \begin{itemize}\setlength{\itemsep}{2pt}
			      \item Calcule \(\Delta(X) := \min\{c(u,v) - y(v) : u \notin X, v \in X\}\).
			      \item Para cada \(v \in X\), atualize \(y(v) := y(v) + \Delta(X)\).
		      \end{itemize}
		\item Defina \(A_0 := \{a \in A : c_y(a) = 0\}\), o conjunto de arcos justos.
	\end{enumerate}

	\textbf{Fase 2: Construção da arborescência}
	\begin{enumerate}\setlength{\itemsep}{2pt}\setcounter{enumi}{3}
		\item Se \((V,A_0)\) forma uma r-arborescência, devolva \(A_0\). Por otimalidade dos potenciais duais, trata-se de uma r-arborescência de custo mínimo.
		\item Caso contrário, identifique um ciclo dirigido \(C\) em \(A_0\) (com \(r\notin C\)). \textbf{Contração:} contraia \(C\) em um supervértice \(x_C\) e defina custos reduzidos \(c'\) por
		      \begin{align*}
			      c'(u,x_C) & := c_y(u,w) = c(u,w) - y(w) &  & \text{para } u\notin C,\ w\in C, \\
			      c'(x_C,v) & := c_y(w,v) = c(w,v) - y(v) &  & \text{para } w\in C,\ v\notin C,
		      \end{align*}
		      descartando laços em \(x_C\) e permitindo paralelos. Denote o dígrafo contraído por \(D'=(V',A')\).
		\item \textbf{Recursão:} compute uma r-arborescência ótima \(T'\) de \(D'\) com custos \(c'\).
		\item \textbf{Expansão:} seja \((u,x_C)\in T'\) o único arco que entra em \(x_C\). No dígrafo original, ele corresponde a \((u,w)\) com \(w\in C\). Forme
		      \[
			      T := \bigl(T'\setminus\{\text{arcos incidentes a } x_C\}\bigr)\ \cup\ \{(u,w)\}\ \cup\ \bigl((A_0\cap A(C))\setminus\{a_w\}\bigr),
		      \]
		      onde \(a_w\) é o arco de \(C\) que entra em \(w\). Então \(T\) é uma r-arborescência ótima de \(D\).
	\end{enumerate}
\end{algobox}

\subsection{Corretude}

A corretude do algoritmo de András Frank baseia-se na teoria de dualidade em programação linear aplicada ao problema de arborescência de custo mínimo:

\begin{enumerate}\setlength{\itemsep}{2pt}
	\item \emph{Potenciais duais:} os valores \(y(v)\) funcionam como variáveis duais. Para qualquer r-arborescência \(T\) e qualquer escolha de potenciais \(y\), vale
	      \[
		      c(T) = \sum_{a\in T} c(a) = \sum_{a\in T} c_y(a) + \sum_{v\neq r} y(v),
	      \]
	      pois há exatamente um arco de \(T\) entrando em cada \(v\neq r\). O termo \(\sum_{v\neq r} y(v)\) é constante após a elevação completa dos potenciais; assim, minimizar \(c(T)\) equivale a minimizar \(\sum_{a\in T} c_y(a)\).

	\item \emph{Condições de otimalidade:} uma r-arborescência \(T\) é ótima se e somente se:
	      \begin{itemize}\setlength{\itemsep}{2pt}
		      \item Todos os arcos de \(T\) são justos: \(c_y(a) = 0\) para todo \(a \in T\).
		      \item Todos os arcos do dígrafo têm custo reduzido não-negativo: \(c_y(a) \geq 0\) para todo \(a \in A\).
	      \end{itemize}
	      Sob essas condições, \(\sum_{a\in T} c_y(a) = 0 \leq \sum_{a\in T'} c_y(a)\) para qualquer outra r-arborescência \(T'\), provando otimalidade.

	\item \emph{Elevação de potenciais:} o algoritmo eleva os potenciais de forma gulosa, garantindo que ao final cada \(v\neq r\) tenha ao menos um arco justo entrando. Isso é feito identificando componentes fortemente conexas em \(A\_0\): conjuntos \(X\) que ainda não possuem arcos justos entrando têm seus potenciais elevados pelo mínimo necessário \(\Delta(X)\).

	\item \emph{Caso acíclico:} se \(A_0\) forma uma r-arborescência, todos os arcos são justos e, pelas condições de otimalidade, a solução é ótima.

	\item \emph{Caso com ciclo (contração/expansão):} se \(A\_0\) contém um ciclo dirigido \(C\), todos os seus arcos são justos (custo reduzido zero). A contração preserva a estrutura de custos reduzidos: no dígrafo contraído, os novos custos \(c'\) já incorporam os potenciais, e uma solução ótima em \(D'\) mapeia para uma solução ótima em \(D\) pela reversibilidade da operação.
\end{enumerate}

Em termos intuitivos, os potenciais \(y(v)\) funcionam como "descontos" aplicados aos arcos que entram em cada vértice, revelando quais arcos são "justos" (custo líquido zero) e, portanto, candidatos à solução ótima. Ciclos de arcos justos podem ser contraídos sem perder otimalidade, simplificando o problema recursivamente.

\subsection{Complexidade}

A implementação, baseada em componentes fortemente conexas, detecta em cada iteração, quais conjuntos \(X\) necessitam elevação de potenciais. Calcular componentes fortemente conexas custa \(O(n+m)\) usando algoritmos como Tarjan ou Kosaraju. Para cada componente (exceto a raiz), eleva-se o potencial calculando \(\Delta(X)\) em \(O(m)\), atualizando os custos reduzidos.

No pior caso, cada iteração reduz o número de componentes em pelo menos uma unidade, resultando em \(O(n)\) iterações. Cada iteração processa todos os arcos para atualizar custos reduzidos e recalcular componentes, resultando em \(O(nm)\) no total para a Fase~1. A Fase~2 constrói a arborescência percorrendo \(A_0\) uma vez, custando \(O(n)\).

O uso de memória é \(O(n+m)\), incluindo as estruturas para armazenar o dígrafo, potenciais e componentes. A implementação a seguir adota a versão \(O(nm)\) por simplicidade e está disponível no repositório do projeto (\url{https://github.com/lorenypsum/GraphVisualizer}).

\section{Implementação em Python}

Esta seção descreve a implementação do algoritmo de András Frank em Python, estruturada para refletir com precisão as duas fases formais discutidas anteriormente. A Fase~1 realiza a elevação de potenciais e identifica os arcos justos, enquanto a Fase~2 constrói a arborescência de custo mínimo a partir desses arcos. Utilizamos a biblioteca NetworkX para manipulação de digrafos, aproveitando suas funcionalidades para representar grafos, calcular componentes fortemente conexas e gerenciar atributos de arcos.

A entrada consiste em um dígrafo orientado \(D = (V, A)\), com custos dos arcos registrados no atributo \texttt{"w"}, e uma raiz \(r \in V\). As hipóteses adotadas são: (i) o dígrafo é conexo a partir de \(r\), isto é, todo vértice \(v \neq r\) é alcançável a partir da raiz; (ii) para todo subconjunto \(X \subseteq V \setminus \{r\}\), existe ao menos um arco entrando em \(X\); e (iii) todos os custos são não negativos.

A saída é um subdigrafo \(T\) de \(D\) com \(|A_T| = |V| - 1\) arcos, tal que cada vértice \(v \neq r\) possui grau de entrada igual a 1, todos os vértices são alcançáveis a partir de \(r\), e o custo total \(\sum_{a \in A_T} c(a)\) é mínimo.

A estrutura do código é modular: funções auxiliares tratam cada etapa do algoritmo — cálculo de componentes fortemente conexas, elevação de potenciais, construção do subdigrafo \(A_0\) e construção da arborescência final. Todas operam sobre objetos \texttt{nx.DiGraph} e são coordenadas por uma função principal que gerencia o fluxo das duas fases. As subseções seguintes detalham cada função auxiliar, abordando lógica, parâmetros, saídas e complexidade.

\subsection{Construção do dígrafo \texorpdfstring{\(D_0\)}{D_0} inicial}

Começamos escrevendo uma função que constrói o dígrafo inicial \(D_0\) que será utilizado na Fase~1 do algoritmo. O dígrafo \(D_0\) é inicializado como um grafo vazio contendo apenas os vértices do dígrafo original, sem arcos. Essa estrutura será gradualmente populada com arcos de custo reduzido zero à medida que os potenciais são elevados.

Recebe como entrada um dígrafo \texttt{D} (objeto \texttt{nx.DiGraph}). A implementação cria um novo dígrafo vazio \texttt{D\_zero} (linha 2) e adiciona todos os vértices de \texttt{D} a \texttt{D\_zero} (linhas 3-4), preservando a estrutura de vértices sem incluir arcos inicialmente.

A função devolve o dígrafo \texttt{D\_zero} contendo todos os vértices de \texttt{D} mas nenhum arco. O dígrafo original \texttt{D} não é modificado. A complexidade é \(O(n)\), onde \(n = |V|\), pois itera sobre todos os vértices uma única vez.

\begin{tcolorbox}[
		enhanced, breakable,
		colframe=blue!60!black, colback=blue!2,
		colbacktitle=blue!15, coltitle=black,
		title={Construção do dígrafo $D_0$ inicial},
		boxed title style={sharp corners, boxrule=0.6pt},
		sharp corners, boxrule=0.6pt
	]
	\emph{Constrói um dígrafo vazio contendo apenas os vértices de D, sem arcos. Este dígrafo será populado com arcos justos durante a elevação de potenciais.}
	\tcblower
	\begin{lstlisting}[language=Python]
def build_D_zero(D):
    D_zero = nx.DiGraph()
    for v in D.nodes():
        D_zero.add_node(v)
    return D_zero
\end{lstlisting}
\end{tcolorbox}

\subsection{Identificação de arcos entrando em conjunto \texorpdfstring{\(X\)}{X}}

Esta função auxiliar identifica todos os arcos que entram em um conjunto \(X \subseteq V\), isto é, arcos \((u,v)\) tais que \(u \notin X\) e \(v \in X\). Essa operação é fundamental para calcular o mínimo custo de entrada em \(X\) durante a elevação de potenciais.

Recebe como entrada um dígrafo \texttt{D} e um conjunto de vértices \texttt{X}. A implementação cria uma lista vazia \texttt{arcs} (linha 2) e itera sobre todos os arcos do dígrafo com seus dados (linha 3), incluindo o peso. Para cada arco \((u,v,\text{data})\), verifica se \(u \notin X\) e \(v \in X\) (linha 4), adicionando à lista apenas os arcos que cruzam a fronteira de \(X\) (linha 5).

A função devolve uma lista de tuplas \((u, v, \text{data})\) representando os arcos que entram em \(X\), onde \texttt{data} contém o atributo \texttt{"w"} com o peso do arco. A complexidade é \(O(m)\), onde \(m = |A|\), pois examina cada arco uma vez.

\begin{tcolorbox}[
		enhanced, breakable,
		colframe=blue!60!black, colback=blue!2,
		colbacktitle=blue!15, coltitle=black,
		title={Identificação de arcos entrando em conjunto X},
		boxed title style={sharp corners, boxrule=0.6pt},
		sharp corners, boxrule=0.6pt
	]
	\emph{Identifica todos os arcos $(u,v)$ do dígrafo D tais que $u \notin X$ e $v \in X$, devolvendo uma lista com as tuplas $(u, v, data)$ onde data contém o peso do arco.}
	\tcblower
	\begin{lstlisting}[language=Python]
def get_arcs_entering_X(D, X):
    arcs = []
    for u, v, data in D.edges(data=True):
        if u not in X and v in X:
            arcs.append((u, v, data))
    return arcs
\end{lstlisting}
\end{tcolorbox}

A figura a seguir ilustra o funcionamento da função \texttt{get\_arcs\_entering\_X} em um digrafo que vamos denotar por \(D_{32}\). O dígrafo possui uma raiz \(r_0\) conectada aos vértices \(u_1, u_2, u_3\). Os vértices em laranja pertencem ao conjunto \(X = \{v_1, v_2, v_3\}\), e a função identifica apenas os arcos em vermelho, que saem de vértices fora de \(X\) e entram em vértices dentro de \(X\). Arcos da raiz, arcos internos a \(X\), externos a \(X\), ou saindo de \(X\) não são retornados.

\begin{figure}[H]
	\centering
	\begin{tikzpicture}[>=Stealth]
		% estilos
		\tikzset{
			vtx/.style={circle, draw, fill=blue!8, thick, inner sep=1.2pt, minimum size=7mm},
			vtxRoot/.style={circle, draw, fill=red!15, thick, inner sep=1.2pt, minimum size=7mm},
			vtxX/.style={circle, draw=orange!80!black, fill=orange!15, thick, inner sep=1.2pt, minimum size=7mm},
			arc/.style={->, >=Stealth, thick, draw=black!40},
			arcEntering/.style={->, >=Stealth, very thick, draw=red!70!black},
			cost/.style={font=\scriptsize, fill=white, inner sep=1pt}
		}

		% Raiz
		\node[vtxRoot] (r0) at (2, 4) {$r_0$};

		% Vértices fora de X
		\node[vtx] (u1) at (0, 2) {$u_1$};
		\node[vtx] (u2) at (2, 2) {$u_2$};
		\node[vtx] (u3) at (4, 2) {$u_3$};

		% Vértices dentro de X (destacados)
		\node[vtxX] (v1) at (0.5, -0.5) {$v_1$};
		\node[vtxX] (v2) at (2, -0.5) {$v_2$};
		\node[vtxX] (v3) at (3.5, -0.5) {$v_3$};

		% Arcos da raiz (não retornados, pois r0 não está em consideração para X)
		\draw[arc] (r0) -- node[cost, left, pos=0.4] {1} (u1);
		\draw[arc] (r0) -- node[cost, right] {2} (u2);
		\draw[arc] (r0) -- node[cost, right, pos=0.4] {1} (u3);

		% Arcos externos a X (não retornados)
		\draw[arc] (u1) to[bend left=15] node[cost, above] {1} (u2);
		\draw[arc] (u2) to[bend left=15] node[cost, above] {3} (u3);

		% Arcos entrando em X (destacados em vermelho)
		\draw[arcEntering] (u1) -- node[cost, left] {3} (v1);
		\draw[arcEntering] (u2) -- node[cost, right] {2} (v2);
		\draw[arcEntering] (u3) -- node[cost, right] {4} (v3);
		\draw[arcEntering] (u1) to[bend right=20] node[cost, below, pos=0.3] {5} (v2);

		% Arcos internos a X (não retornados)
		\draw[arc] (v1) -- node[cost, above] {1} (v2);
		\draw[arc] (v2) -- node[cost, above] {2} (v3);

		% Arcos saindo de X (não retornados)
		\draw[arc] (v1) to[bend right=30] node[cost, left, pos=0.3] {2} (u1);

		% Região X destacada
		\draw[orange!60!black, thick, dashed, rounded corners=8pt]
		(-0.3, 0.3) rectangle (4.3, -1.3);
		\node[orange!80!black, font=\bfseries] at (5, 0.3) {$X$};
	\end{tikzpicture}
	\caption{Ilustração da função \texttt{get\_arcs\_entering\_X} em \(D_{32}\). A raiz \(r_0\) (em vermelho claro) conecta-se aos vértices \(u_1, u_2, u_3\). Os vértices em \textcolor{orange!80!black}{laranja} pertencem ao conjunto \(X = \{v_1, v_2, v_3\}\). A função identifica apenas os arcos \textcolor{red!70!black}{em vermelho}: aqueles que saem de vértices fora de \(X\) e entram em vértices dentro de \(X\). Arcos da raiz, arcos internos a \(X\), externos a \(X\), ou saindo de \(X\) não são retornados.}
\end{figure}

\subsection{Cálculo do peso mínimo de corte}

Esta função calcula o peso mínimo entre todos os arcos fornecidos, correspondendo ao valor \(\Delta(X)\) necessário para elevar os potenciais dos vértices em \(X\).

Recebe como entrada uma lista \texttt{arcos} de tuplas \((u, v, \text{data})\). A implementação usa a função \texttt{min} com uma compreensão de gerador (linha 2) que extrai o atributo \texttt{"w"} de cada tupla em \texttt{data}.

A função devolve o peso mínimo encontrado entre todos os arcos da lista. A complexidade é \(O(k)\), onde \(k\) é o número de arcos na lista, pois examina cada arco uma vez para encontrar o mínimo.

\begin{tcolorbox}[
		enhanced, breakable,
		colframe=blue!60!black, colback=blue!2,
		colbacktitle=blue!15, coltitle=black,
		title={Cálculo do peso mínimo de corte},
		boxed title style={sharp corners, boxrule=0.6pt},
		sharp corners, boxrule=0.6pt
	]
	\emph{Calcula o peso mínimo entre todos os arcos fornecidos, correspondendo ao valor $\Delta(X)$ usado na elevação de potenciais.}
	\tcblower
	\begin{lstlisting}[language=Python]
def get_minimum_weight_cut(arcs):
    return min(data["w"] for _, _, data in arcs)
\end{lstlisting}
\end{tcolorbox}

A seguir temos uma ilustração do funcionamento da função \texttt{get\_minimum\_weight\_cut} em \(D_{32}\). Considerando os arcos em vermelho que entram em \(X\) (identificados pela função anterior), esta função calcula o peso mínimo entre eles. O arco em verde possui o menor peso (2), correspondendo ao valor \(\Delta(X) = 2\) que será devolvido pela função.

\begin{figure}[H]
	\centering
	\begin{tikzpicture}[>=Stealth]
		% estilos
		\tikzset{
			vtx/.style={circle, draw, fill=blue!8, thick, inner sep=1.2pt, minimum size=7mm},
			vtxRoot/.style={circle, draw, fill=red!15, thick, inner sep=1.2pt, minimum size=7mm},
			vtxX/.style={circle, draw=orange!80!black, fill=orange!15, thick, inner sep=1.2pt, minimum size=7mm},
			arc/.style={->, >=Stealth, thick, draw=black!40},
			arcEntering/.style={->, >=Stealth, very thick, draw=red!70!black},
			arcMin/.style={->, >=Stealth, ultra thick, draw=green!60!black},
			cost/.style={font=\scriptsize, fill=white, inner sep=1pt},
			costMin/.style={font=\scriptsize\bfseries, fill=green!10, inner sep=2pt, draw=green!60!black, rounded corners=1pt}
		}

		% Raiz
		\node[vtxRoot] (r0) at (2, 4) {$r_0$};

		% Vértices fora de X
		\node[vtx] (u1) at (0, 2) {$u_1$};
		\node[vtx] (u2) at (2, 2) {$u_2$};
		\node[vtx] (u3) at (4, 2) {$u_3$};

		% Vértices dentro de X (destacados)
		\node[vtxX] (v1) at (0.5, -0.5) {$v_1$};
		\node[vtxX] (v2) at (2, -0.5) {$v_2$};
		\node[vtxX] (v3) at (3.5, -0.5) {$v_3$};

		% Arcos da raiz
		\draw[arc] (r0) -- node[cost, left, pos=0.4] {1} (u1);
		\draw[arc] (r0) -- node[cost, right] {2} (u2);
		\draw[arc] (r0) -- node[cost, right, pos=0.4] {1} (u3);

		% Arcos externos a X
		\draw[arc] (u1) to[bend left=15] node[cost, above] {1} (u2);
		\draw[arc] (u2) to[bend left=15] node[cost, above] {3} (u3);

		% Arcos entrando em X (em vermelho, mas o mínimo em verde)
		\draw[arcEntering] (u1) -- node[cost, left] {3} (v1);
		\draw[arcMin] (u2) -- node[costMin, right] {2} (v2);
		\draw[arcEntering] (u3) -- node[cost, right] {4} (v3);
		\draw[arcEntering] (u1) to[bend right=20] node[cost, below, pos=0.3] {5} (v2);

		% Arcos internos a X
		\draw[arc] (v1) -- node[cost, above] {1} (v2);
		\draw[arc] (v2) -- node[cost, above] {2} (v3);

		% Arcos saindo de X
		\draw[arc] (v1) to[bend right=30] node[cost, left, pos=0.3] {2} (u1);

		% Região X destacada
		\draw[orange!60!black, thick, dashed, rounded corners=8pt]
		(-0.3, 0.3) rectangle (4.3, -1.3);
		\node[orange!80!black, font=\bfseries] at (5, 0.3) {$X$};

		% Anotação do resultado
		\node[font=\bfseries, green!60!black] at (2, -2.2) {$\Delta(X) = \min\{3, 2, 4, 5\} = 2$};
	\end{tikzpicture}
	\caption{Ilustração da função \texttt{get\_minimum\_weight\_cut} em \(D_{32}\). Considerando os arcos \textcolor{red!70!black}{em vermelho} que entram em \(X\) (identificados pela função anterior), esta função calcula o peso mínimo entre eles. O arco \textcolor{green!60!black}{em verde} possui o menor peso (2), correspondendo ao valor \(\Delta(X) = 2\).}
\end{figure}

\subsection{Atualização de pesos em \texorpdfstring{\(X\)}{X}}

Esta função auxiliar atualiza os pesos dos arcos que entram em um conjunto \(X\), subtraindo o valor \(\Delta(X)\) de cada peso. Arcos que atingem peso zero são adicionados a \(A_0\) e a \(D_0\).

Recebe como entrada um dígrafo \texttt{D}, lista de \texttt{arcs} entrando em \(X\), o valor \texttt{min\_weight} a ser subtraído, uma lista \texttt{A\_zero} para armazenar arcos de peso zero, e o dígrafo \texttt{D\_zero} para adicionar os arcos justos.

A implementação itera sobre cada arco \((u,v,\_)\) da lista (linha 2), subtrai \texttt{min\_weight} do peso armazenado em \texttt{D[u][v]["w"]} (linha 3), e verifica se o peso resultante é zero (linha 4). Caso sim, adiciona-se \((u,v)\) à lista \texttt{A\_zero} (linha 5) e ao dígrafo \texttt{D\_zero} (linha 6).

A função não devolve valor, pois modifica diretamente as estruturas passadas como parâmetros: o dígrafo \texttt{D} tem seus pesos atualizados, \texttt{A\_zero} acumula arcos justos, e \texttt{D\_zero} é populado com esses arcos. A complexidade é \(O(k)\), onde \(k\) é o número de arcos em \texttt{arcs}.

\begin{tcolorbox}[
		enhanced, breakable,
		colframe=blue!60!black, colback=blue!2,
		colbacktitle=blue!15, coltitle=black,
		title={Atualização de pesos em X},
		boxed title style={sharp corners, boxrule=0.6pt},
		sharp corners, boxrule=0.6pt
	]
	\emph{Atualiza os pesos dos arcos que entram em X, subtraindo o valor mínimo. Arcos que atingem peso zero são registrados em $A_0$ e adicionados a $D_0$.}
	\tcblower
	\begin{lstlisting}[language=Python]
def update_weights_in_X(D, arcs, min_weight, A_zero, D_zero):
    for u, v, _ in arcs:
        D[u][v]["w"] -= min_weight
        if D[u][v]["w"] == 0:
            A_zero.append((u, v))
            D_zero.add_edge(u, v)
\end{lstlisting}
\end{tcolorbox}

A seguir ilustramos o funcionamento da função \texttt{update\_weights\_in\_X} em \(D_{32}\). A figura mostra o dígrafo antes e depois da atualização dos pesos. No estado inicial (esquerda), temos \(\Delta(X) = 2\). A função subtrai esse valor de todos os arcos que entram em \(X\). No estado final (direita), os arcos que atingiram peso zero (destacados em azul) são adicionados a \(A_0\) e \(D_0\).

\begin{figure}[H]
	\centering
	\begin{tikzpicture}[>=Stealth]
		% estilos
		\tikzset{
			vtx/.style={circle, draw, fill=blue!8, thick, inner sep=1.2pt, minimum size=6.5mm},
			vtxRoot/.style={circle, draw, fill=red!15, thick, inner sep=1.2pt, minimum size=6.5mm},
			vtxX/.style={circle, draw=orange!80!black, fill=orange!15, thick, inner sep=1.2pt, minimum size=6.5mm},
			arc/.style={->, >=Stealth, thick, draw=black!40},
			arcEntering/.style={->, >=Stealth, very thick, draw=red!70!black},
			arcTight/.style={->, >=Stealth, ultra thick, draw=blue!70!black},
			cost/.style={font=\scriptsize, fill=white, inner sep=1pt},
			costTight/.style={font=\scriptsize\bfseries, fill=blue!10, inner sep=2pt, draw=blue!70!black, rounded corners=1pt}
		}

		% ===== ANTES (esquerda) =====
		\begin{scope}
			\node[font=\bfseries, blue!80!black] at (2, 4.2) {Antes: pesos originais};

			% Raiz
			\node[vtxRoot] (r0) at (2, 3) {$r_0$};

			% Vértices fora de X
			\node[vtx] (u1) at (0, 1.5) {$u_1$};
			\node[vtx] (u2) at (2, 1.5) {$u_2$};
			\node[vtx] (u3) at (4, 1.5) {$u_3$};

			% Vértices dentro de X
			\node[vtxX] (v1) at (0.5, -0.5) {$v_1$};
			\node[vtxX] (v2) at (2, -0.5) {$v_2$};
			\node[vtxX] (v3) at (3.5, -0.5) {$v_3$};

			% Arcos da raiz
			\draw[arc] (r0) -- node[cost, left, pos=0.4] {1} (u1);
			\draw[arc] (r0) -- node[cost, right] {2} (u2);
			\draw[arc] (r0) -- node[cost, right, pos=0.4] {1} (u3);

			% Arcos externos a X
			\draw[arc] (u1) to[bend left=15] node[cost, above] {1} (u2);
			\draw[arc] (u2) to[bend left=15] node[cost, above] {3} (u3);

			% Arcos entrando em X
			\draw[arcEntering] (u1) -- node[cost, left] {3} (v1);
			\draw[arcEntering] (u2) -- node[cost, right] {2} (v2);
			\draw[arcEntering] (u3) -- node[cost, right] {4} (v3);
			\draw[arcEntering] (u1) to[bend right=20] node[cost, below, pos=0.3] {5} (v2);

			% Arcos internos a X
			\draw[arc] (v1) -- node[cost, above] {1} (v2);
			\draw[arc] (v2) -- node[cost, above] {2} (v3);

			% Arcos saindo de X
			\draw[arc] (v1) to[bend right=30] node[cost, left, pos=0.3] {2} (u1);

			% Região X
			\draw[orange!60!black, thick, dashed, rounded corners=8pt]
			(-0.3, 0.3) rectangle (4.3, -1.3);
			\node[orange!80!black, font=\bfseries\small] at (4.8, 0.1) {$X$};

			\node[font=\small, green!60!black] at (2, -2.2) {$\Delta(X) = 2$};
		\end{scope}

		% Seta indicando transformação
		\draw[->, ultra thick, blue!60!black] (5.3, 0.5) -- node[above, font=\small] {subtrai $\Delta(X)$} (6.7, 0.5);

		% ===== DEPOIS (direita) =====
		\begin{scope}[xshift=7.5cm]
			\node[font=\bfseries, blue!80!black] at (2, 4.2) {Depois: pesos atualizados};

			% Raiz
			\node[vtxRoot] (r0b) at (2, 3) {$r_0$};

			% Vértices fora de X
			\node[vtx] (u1b) at (0, 1.5) {$u_1$};
			\node[vtx] (u2b) at (2, 1.5) {$u_2$};
			\node[vtx] (u3b) at (4, 1.5) {$u_3$};

			% Vértices dentro de X
			\node[vtxX] (v1b) at (0.5, -0.5) {$v_1$};
			\node[vtxX] (v2b) at (2, -0.5) {$v_2$};
			\node[vtxX] (v3b) at (3.5, -0.5) {$v_3$};

			% Arcos da raiz (inalterados)
			\draw[arc] (r0b) -- node[cost, left, pos=0.4] {1} (u1b);
			\draw[arc] (r0b) -- node[cost, right] {2} (u2b);
			\draw[arc] (r0b) -- node[cost, right, pos=0.4] {1} (u3b);

			% Arcos externos a X (inalterados)
			\draw[arc] (u1b) to[bend left=15] node[cost, above] {1} (u2b);
			\draw[arc] (u2b) to[bend left=15] node[cost, above] {3} (u3b);

			% Arcos entrando em X (atualizados)
			\draw[arcEntering] (u1b) -- node[cost, left] {1} (v1b);
			\draw[arcTight] (u2b) -- node[costTight, right] {0} (v2b);
			\draw[arcEntering] (u3b) -- node[cost, right] {2} (v3b);
			\draw[arcEntering] (u1b) to[bend right=20] node[cost, below, pos=0.3] {3} (v2b);

			% Arcos internos a X (inalterados)
			\draw[arc] (v1b) -- node[cost, above] {1} (v2b);
			\draw[arc] (v2b) -- node[cost, above] {2} (v3b);

			% Arcos saindo de X (inalterados)
			\draw[arc] (v1b) to[bend right=30] node[cost, left, pos=0.3] {2} (u1b);

			% Região X
			\draw[orange!60!black, thick, dashed, rounded corners=8pt]
			(-0.3, 0.3) rectangle (4.3, -1.3);
			\node[orange!80!black, font=\bfseries\small] at (4.8, 0.1) {$X$};

			\node[font=\small, blue!70!black] at (2, -2.2) {Arco \textcolor{blue!70!black}{justo} $\to A_0$};
		\end{scope}
	\end{tikzpicture}
	\caption{Ilustração da função \texttt{update\_weights\_in\_X} em \(D_{32}\). À esquerda, o dígrafo antes da atualização, com os arcos \textcolor{red!70!black}{em vermelho} entrando em \(X\) e \(\Delta(X) = 2\). À direita, após subtrair \(\Delta(X)\) de cada arco entrando em \(X\): o peso $(u_1, v_1)$ reduz de 3 para 1, $(u_2, v_2)$ de 2 para \textcolor{blue!70!black}{\textbf{0}} (torna-se justo), $(u_3, v_3)$ de 4 para 2, e $(u_1, v_2)$ de 5 para 3. O arco justo é adicionado a \(A_0\) e \(D_0\). Note que os arcos da raiz e arcos internos/externos a \(X\) permanecem inalterados.}
\end{figure}

\subsection{Verificação de arborescência}

Esta função verifica se um dígrafo \(D\) contém uma r-arborescência com raiz \(r_0\). Utiliza busca em profundidade (DFS) a partir da raiz para verificar se todos os vértices são alcançáveis.

Recebe como entrada um dígrafo \texttt{D} e a raiz \texttt{r0}. A implementação constrói uma árvore DFS a partir de \texttt{r0} usando \texttt{nx.dfs\_tree} (linha 2), que devolve um subdigrafo contendo apenas os vértices alcançáveis a partir da raiz seguindo arcos. Em seguida, compara o número de vértices da árvore DFS com o número total de vértices de \texttt{D} (linha 3).

A função devolve \texttt{True} se todos os vértices são alcançáveis (indicando presença de r-arborescência), \texttt{False} caso contrário. A complexidade é \(O(n+m)\), onde \(n=|V|\) e \(m=|A|\), devido à busca em profundidade. Precisamos dessa verificação para garantir que a Fase~1 produza um dígrafo \(D_0\) que contenha uma r-arborescência antes de prosseguir para a Fase~2.

\begin{tcolorbox}[
		enhanced, breakable,
		colframe=blue!60!black, colback=blue!2,
		colbacktitle=blue!15, coltitle=black,
		title={Verificação de arborescência},
		boxed title style={sharp corners, boxrule=0.6pt},
		sharp corners, boxrule=0.6pt
	]
	\emph{Verifica se o dígrafo D contém uma r-arborescência com raiz r0, usando busca em profundidade para testar alcançabilidade de todos os vértices.}
	\tcblower
	\begin{lstlisting}[language=Python]
def has_arborescence(D, r0):
    tree = nx.dfs_tree(D, r0)
    return tree.number_of_nodes() == D.number_of_nodes()
\end{lstlisting}
\end{tcolorbox}

\subsection{Fase 1: Elevação de potenciais e construção de \texorpdfstring{\(A_0\)}{A_0}}

A seguir apresentamos a função principal da Fase~1, responsável por elevar os potenciais dos vértices iterativamente até que cada conjunto de vértices possua ao menos um arco justo entrando. O processo utiliza componentes fortemente conexas para identificar quais conjuntos necessitam elevação.

Recebe como entrada um dígrafo \texttt{D\_original} e a raiz \texttt{r0}. A implementação cria uma cópia do dígrafo original (linha 2) para preservar a entrada, inicializa estruturas auxiliares \texttt{A\_zero} (lista de arcos justos), \texttt{Dual\_list} (lista de pares \((X, \Delta(X))\) para fins de validação dual), e \texttt{D\_zero} (dígrafo de arcos justos) (linhas 3-5). Um contador de iterações é inicializado na linha 6.

O loop principal (linhas 7-22) itera enquanto houver conjuntos sem arcos justos entrando. Em cada iteração, incrementa-se o contador (linha 8), calcula-se as componentes fortemente conexas de \(D_0\) usando \texttt{nx.condensation} (linha 9), que devolve um grafo acíclico dirigido (DAG, do inglês \emph{directed acyclic graph}) onde cada vértice representa uma componente e contém o atributo \texttt{"members"} com os vértices originais. Utilizamos componentes fortemente conexas porque elas identificam naturalmente os conjuntos maximais de vértices que ainda não possuem arcos justos entrando, evitando a necessidade de rastrear manualmente quais conjuntos já foram processados.

Em seguida, identificam-se as fontes (componentes sem arcos entrando) no grafo de condensação (linha 10). Se há apenas uma fonte, significa que todos os vértices estão em uma única componente alcançável pela raiz através de arcos justos, garantindo que \(D_0\) contém uma r-arborescência e encerrando o loop (linhas 11-12).

Para cada fonte \(u\) no grafo de condensação (linha 13), obtém-se o conjunto \(X\) de vértices da componente (linha 14). Se \(r_0 \in X\), a fonte é ignorada (linhas 15-16), pois a componente contendo a raiz não necessita elevação. Caso contrário, identificam-se os arcos entrando em \(X\) usando \texttt{get\_arcs\_entering\_X} (linha 17), calcula-se o peso mínimo \(\Delta(X)\) usando \texttt{get\_minimum\_weight\_cut} (linha 18), e atualiza-se os pesos com \texttt{update\_weights\_in\_X}, registrando novos arcos justos (linha 19). A elevação simultânea de potenciais para todos os vértices de \(X\) mantém a propriedade de que arcos internos a \(X\) permanecem com o mesmo custo reduzido relativo, preservando a correção do algoritmo. Finalmente, adiciona-se \((X, \Delta(X))\) à lista dual se \(\Delta(X) > 0\) (linhas 20-21), permitindo verificação posterior das condições de otimalidade dual.

A função devolve \texttt{A\_zero} (lista de arcos justos) e \texttt{Dual\_list} (pares \((X, \Delta(X))\) para validação). A complexidade é \(O(nm)\) no pior caso, com \(O(n)\) iterações, cada uma custando \(O(m)\) para calcular componentes e atualizar pesos.

\begin{tcolorbox}[
		enhanced, breakable,
		colframe=blue!60!black, colback=blue!2,
		colbacktitle=blue!15, coltitle=black,
		title={Fase 1: Elevação de potenciais e construção de $A_0$},
		boxed title style={sharp corners, boxrule=0.6pt},
		sharp corners, boxrule=0.6pt
	]
	\emph{Eleva iterativamente os potenciais dos vértices até que cada conjunto possua ao menos um arco justo entrando. Devolve a lista $A_0$ de arcos justos e a lista de pares $(X, \Delta(X))$ para validação dual.}
	\tcblower
	\begin{lstlisting}[language=Python, basicstyle=\ttfamily\fontsize{8}{9}\selectfont]
def phase1(D_original, r0):
    D_copy = D_original.copy()
    A_zero = []
    Dual_list = []
    D_zero = build_D_zero(D_copy)
    iteration = 0
    while True:
        iteration += 1
        C = nx.condensation(D_zero)
        sources = [x for x in C.nodes() if C.in_degree(x) == 0]
        if len(sources) == 1:
            break
        for u in sources:
            X = C.nodes[u]["members"]
            if r0 in X:
                continue
            arcs = get_arcs_entering_X(D_copy, X)
            min_weight = get_minimum_weight_cut(arcs)
            update_weights_in_X(D_copy, arcs, min_weight, A_zero, D_zero)
            if min_weight != 0:
                Dual_list.append((X, min_weight)) 
    return A_zero, Dual_list
\end{lstlisting}
\end{tcolorbox}

A seguir ilustramos a execução da função \texttt{phase1} no dígrafo \(D_{32}\) estendido com a raiz \(r_0\). A figura mostra três estados principais: o dígrafo original \(D\) com seus pesos, o dígrafo \(D_0\) inicial (vazio, sem arcos), e o dígrafo \(D_0\) final após as iterações de elevação de potenciais, contendo apenas os arcos justos (em azul) que formam uma r-arborescência.

\begin{figure}[H]
	\centering
	\begin{tikzpicture}[>=Stealth, scale=0.85, every node/.style={scale=0.85}]
		% Estilos
		\tikzset{
			vtx/.style={circle, draw, fill=blue!8, thick, inner sep=1.2pt, minimum size=6.5mm},
			vtxRoot/.style={circle, draw, fill=red!15, thick, inner sep=1.2pt, minimum size=6.5mm},
			vtxX/.style={circle, draw, fill=orange!20, thick, inner sep=1.2pt, minimum size=6.5mm},
			arc/.style={->, thick, gray!60},
			arcTight/.style={->, ultra thick, blue!70!black},
			cost/.style={font=\scriptsize, fill=white, inner sep=1pt, rounded corners=1pt},
			costTight/.style={font=\scriptsize\bfseries, fill=blue!10, inner sep=2pt, draw=blue!70!black, rounded corners=1pt}
		}

		% ===== DÍGRAFO ORIGINAL D (esquerda) =====
		\begin{scope}
			\node[font=\bfseries, blue!80!black] at (2, 4.2) {Dígrafo original $D$};

			% Raiz
			\node[vtxRoot] (r0) at (2, 3) {$r_0$};

			% Vértices fora de X
			\node[vtx] (u1) at (0, 1.5) {$u_1$};
			\node[vtx] (u2) at (2, 1.5) {$u_2$};
			\node[vtx] (u3) at (4, 1.5) {$u_3$};

			% Vértices dentro de X
			\node[vtxX] (v1) at (0.5, -0.5) {$v_1$};
			\node[vtxX] (v2) at (2, -0.5) {$v_2$};
			\node[vtxX] (v3) at (3.5, -0.5) {$v_3$};

			% Arcos da raiz
			\draw[arc] (r0) -- node[cost, left, pos=0.4] {1} (u1);
			\draw[arc] (r0) -- node[cost, right] {2} (u2);
			\draw[arc] (r0) -- node[cost, right, pos=0.4] {1} (u3);

			% Arcos externos a X
			\draw[arc] (u1) to[bend left=15] node[cost, above] {1} (u2);
			\draw[arc] (u2) to[bend left=15] node[cost, above] {3} (u3);

			% Arcos entrando em X
			\draw[arc] (u1) -- node[cost, left] {3} (v1);
			\draw[arc] (u2) -- node[cost, right] {2} (v2);
			\draw[arc] (u3) -- node[cost, right] {4} (v3);
			\draw[arc] (u1) to[bend right=20] node[cost, below, pos=0.3] {5} (v2);

			% Arcos internos a X
			\draw[arc] (v1) -- node[cost, above] {1} (v2);
			\draw[arc] (v2) -- node[cost, above] {2} (v3);

			% Arcos saindo de X
			\draw[arc] (v1) to[bend right=30] node[cost, left, pos=0.3] {2} (u1);

			\node[font=\small, gray!70!black] at (2, -2) {Pesos originais};
		\end{scope}

		% ===== D_0 INICIAL (centro) =====
		\begin{scope}[xshift=6cm]
			\node[font=\bfseries, blue!80!black] at (2, 4.2) {$D_0$ inicial (linha 5)};

			% Vértices apenas, sem arcos
			\node[vtxRoot] (r0i) at (2, 3) {$r_0$};
			\node[vtx] (u1i) at (0, 1.5) {$u_1$};
			\node[vtx] (u2i) at (2, 1.5) {$u_2$};
			\node[vtx] (u3i) at (4, 1.5) {$u_3$};
			\node[vtxX] (v1i) at (0.5, -0.5) {$v_1$};
			\node[vtxX] (v2i) at (2, -0.5) {$v_2$};
			\node[vtxX] (v3i) at (3.5, -0.5) {$v_3$};

			\node[font=\small, gray!70!black] at (2, -2) {Sem arcos (vazio)};
		\end{scope}

		% ===== D_0 FINAL (direita) =====
		\begin{scope}[xshift=12cm]
			\node[font=\bfseries, blue!80!black] at (2, 4.2) {$D_0$ final (após loop)};

			% Vértices
			\node[vtxRoot] (r0f) at (2, 3) {$r_0$};
			\node[vtx] (u1f) at (0, 1.5) {$u_1$};
			\node[vtx] (u2f) at (2, 1.5) {$u_2$};
			\node[vtx] (u3f) at (4, 1.5) {$u_3$};
			\node[vtxX] (v1f) at (0.5, -0.5) {$v_1$};
			\node[vtxX] (v2f) at (2, -0.5) {$v_2$};
			\node[vtxX] (v3f) at (3.5, -0.5) {$v_3$};

			% Arcos justos (peso 0)
			\draw[arcTight] (r0f) -- node[costTight, left, pos=0.4] {0} (u1f);
			\draw[arcTight] (r0f) -- node[costTight, right, pos=0.4] {0} (u3f);
			\draw[arcTight] (u1f) to[bend left=15] node[costTight, above] {0} (u2f);
			\draw[arcTight] (u1f) -- node[costTight, left] {0} (v1f);
			\draw[arcTight] (u2f) -- node[costTight, right] {0} (v2f);
			\draw[arcTight] (v2f) -- node[costTight, above] {0} (v3f);

			\node[font=\small, blue!70!black] at (2, -2) {Arcos justos em $A_0$};
		\end{scope}

		% Setas indicando progressão
		\draw[->, ultra thick, gray!50] (4.5, 1) -- node[above, font=\small] {elevação} (5.5, 1);
		\draw[->, ultra thick, gray!50] (10.5, 1) -- node[above, font=\small] {iterações} (11.5, 1);
	\end{tikzpicture}
	\caption{Execução da função \texttt{phase1} em \(D_{32}\). À esquerda, o dígrafo original \(D\) com pesos dos arcos. Ao centro, \(D_0\) inicial após linha 5 (\texttt{build\_D\_zero}), contendo apenas vértices sem arcos. À direita, \(D_0\) final após o loop de elevação de potenciais, contendo apenas os arcos \textcolor{blue!70!black}{justos} (custo reduzido zero) que formam uma r-arborescência. Durante as iterações (linhas 7-22), os potenciais são elevados para cada componente sem arcos justos entrando, até que todos os vértices sejam alcançáveis a partir de \(r_0\) através de arcos em \(D_0\).}
\end{figure}

A seguir ilustramos a execução da função \texttt{phase2} em \(D_{32}\), mostrando como a r-arborescência é construída incrementalmente a partir do conjunto \(A_0\) de arcos justos obtidos da Fase~1. A figura apresenta quatro estados: o conjunto inicial \(A_0\) com 6 arcos justos, e três estados intermediários da arborescência \texttt{Arb} sendo construída, destacando em verde os arcos adicionados em cada iteração.

\begin{figure}[H]
	\centering
	\begin{tikzpicture}[>=Stealth, scale=0.72, every node/.style={scale=0.72}]
		% Estilos
		\tikzset{
			vtx/.style={circle, draw, fill=blue!8, thick, inner sep=1.2pt, minimum size=6.5mm},
			vtxRoot/.style={circle, draw, fill=red!15, thick, inner sep=1.2pt, minimum size=6.5mm},
			vtxIn/.style={circle, draw, fill=green!15, thick, inner sep=1.2pt, minimum size=6.5mm},
			vtxOut/.style={circle, draw, fill=gray!10, thick, inner sep=1.2pt, minimum size=6.5mm},
			arc/.style={->, thick, blue!70!black},
			arcNew/.style={->, ultra thick, green!70!black},
			cost/.style={font=\scriptsize, fill=white, inner sep=1pt}
		}

		% ===== A_0 (esquerda superior) =====
		\begin{scope}
			\node[font=\bfseries, blue!80!black] at (2, 4.5) {$A_0$ (arcos justos)};

			\node[vtxRoot] (r0) at (2, 3.2) {$r_0$};
			\node[vtx] (u1) at (0, 1.7) {$u_1$};
			\node[vtx] (u2) at (2, 1.7) {$u_2$};
			\node[vtx] (u3) at (4, 1.7) {$u_3$};
			\node[vtx] (v1) at (0.5, 0) {$v_1$};
			\node[vtx] (v2) at (2, 0) {$v_2$};
			\node[vtx] (v3) at (3.5, 0) {$v_3$};

			\draw[arc] (r0) -- node[cost, left, pos=0.3] {0} (u1);
			\draw[arc] (r0) -- node[cost, right, pos=0.3] {0} (u3);
			\draw[arc] (u1) to[bend left=12] node[cost, above, pos=0.4] {0} (u2);
			\draw[arc] (u1) -- node[cost, left] {0} (v1);
			\draw[arc] (u2) -- node[cost, right] {0} (v2);
			\draw[arc] (v2) -- node[cost, above] {0} (v3);

			\node[font=\small, gray!70!black] at (2, -1) {6 arcos com custo 0};
		\end{scope}

		% ===== Iteração 1 (direita superior) =====
		\begin{scope}[xshift=6.5cm]
			\node[font=\bfseries, blue!80!black] at (2, 4.5) {Iteração 1: $|$Arb$| = 1$};

			\node[vtxRoot] (r01) at (2, 3.2) {$r_0$};
			\node[vtxIn] (u11) at (0, 1.7) {$u_1$};
			\node[vtxOut] (u21) at (2, 1.7) {$u_2$};
			\node[vtxOut] (u31) at (4, 1.7) {$u_3$};
			\node[vtxOut] (v11) at (0.5, 0) {$v_1$};
			\node[vtxOut] (v21) at (2, 0) {$v_2$};
			\node[vtxOut] (v31) at (3.5, 0) {$v_3$};

			\draw[arcNew] (r01) -- node[cost, left, pos=0.3, fill=green!10] {1} (u11);

			\node[font=\small, green!70!black] at (2, -1) {Adiciona $(r_0, u_1)$};
		\end{scope}

		% ===== Iteração 2 (esquerda inferior) =====
		\begin{scope}[yshift=-6.2cm]
			\node[font=\bfseries, blue!80!black] at (2, 4.5) {Iteração 2: $|$Arb$| = 2$};

			\node[vtxRoot] (r02) at (2, 3.2) {$r_0$};
			\node[vtxIn] (u12) at (0, 1.7) {$u_1$};
			\node[vtxOut] (u22) at (2, 1.7) {$u_2$};
			\node[vtxIn] (u32) at (4, 1.7) {$u_3$};
			\node[vtxOut] (v12) at (0.5, 0) {$v_1$};
			\node[vtxOut] (v22) at (2, 0) {$v_2$};
			\node[vtxOut] (v32) at (3.5, 0) {$v_3$};

			\draw[arc] (r02) -- node[cost, left, pos=0.3] {1} (u12);
			\draw[arcNew] (r02) -- node[cost, right, pos=0.3, fill=green!10] {1} (u32);

			\node[font=\small, green!70!black] at (2, -1) {Adiciona $(r_0, u_3)$};
		\end{scope}

		% ===== Iteração 4 (direita inferior) =====
		\begin{scope}[xshift=6.5cm, yshift=-6.2cm]
			\node[font=\bfseries, blue!80!black] at (2, 4.5) {Iteração 4: $|$Arb$| = 4$};

			\node[vtxRoot] (r04) at (2, 3.2) {$r_0$};
			\node[vtxIn] (u14) at (0, 1.7) {$u_1$};
			\node[vtxIn] (u24) at (2, 1.7) {$u_2$};
			\node[vtxIn] (u34) at (4, 1.7) {$u_3$};
			\node[vtxIn] (v14) at (0.5, 0) {$v_1$};
			\node[vtxOut] (v24) at (2, 0) {$v_2$};
			\node[vtxOut] (v34) at (3.5, 0) {$v_3$};

			\draw[arc] (r04) -- node[cost, left, pos=0.3] {1} (u14);
			\draw[arc] (r04) -- node[cost, right, pos=0.3] {1} (u34);
			\draw[arc] (u14) to[bend left=12] node[cost, above, pos=0.4] {1} (u24);
			\draw[arcNew] (u14) -- node[cost, left, fill=green!10] {3} (v14);

			\node[font=\small, green!70!black] at (2, -1) {Adiciona $(u_1, v_1)$};
		\end{scope}

		% Setas de progressão
		\draw[->, ultra thick, gray!50] (4.8, 2) -- node[above, font=\small] {linha 6-10} (5.7, 2);
		\draw[->, ultra thick, gray!50] (2, -1.5) -- node[left, font=\small] {linha 6-10} (2, -2.4);
		\draw[->, ultra thick, gray!50] (4.8, -4.2) -- node[above, font=\small] {linha 6-10} (5.7, -4.2);
	\end{tikzpicture}
	\caption{Execução da função \texttt{phase2} em \(D_{32}\). Superior esquerdo: conjunto \(A_0\) com 6 arcos justos (custo reduzido zero) obtidos da Fase~1. Superior direito: Iteração 1 — adiciona arco $(r_0, u_1)$ pois $r_0 \in$ Arb e $u_1 \notin$ Arb. Inferior esquerdo: Iteração 2 — adiciona $(r_0, u_3)$. Inferior direito: Iteração 4 — após adicionar $(u_1, u_2)$ na iteração 3, adiciona $(u_1, v_1)$. Vértices \textcolor{green!70!black}{verdes} foram recém-adicionados, vértices \textcolor{gray!70!black}{cinzas} ainda não pertencem a Arb. O processo continua até que todos os 6 arcos de \(A_0\) sejam incluídos, formando uma r-arborescência com pesos originais de \(D\).}
\end{figure}

As funções auxiliares implementadas nesta seção correspondem diretamente aos passos da Fase~1 do algoritmo de András Frank da seguinte forma:

\begin{tcolorbox}[
		enhanced, breakable,
		colframe=purple!60!black, colback=purple!5,
		colbacktitle=purple!15, coltitle=black,
		title={Correspondência entre Teoria e Implementação — Fase 1},
		boxed title style={sharp corners, boxrule=0.6pt},
		sharp corners, boxrule=0.6pt
	]
	\textbf{Passo 1 — Inicialização:}
	\begin{itemize}
		\item \textbf{Descrição teórica:} Defina \(y(v) := 0\) para todo \(v \in V\).
		\item \textbf{Implementação:} A função \texttt{build\_D\_zero(D)} inicializa o dígrafo \(D_0\) vazio, que será populado apenas com arcos justos. Implicitamente, os potenciais iniciam em zero, pois os pesos no dígrafo \(D\) representam os custos reduzidos \(c_y(u,v) = c(u,v) - y(v)\). Com \(y(v) = 0\), temos \(c_y = c\) no início.
	\end{itemize}

	\vspace{2mm}
	\textbf{Passo 2 — Iteração de elevação de potenciais:}
	\begin{itemize}
		\item \textbf{Descrição teórica:} Enquanto existir conjunto \(X \subseteq V \setminus \{r\}\) sem arco justo entrando:
		      \begin{itemize}
			      \item Calcule \(\Delta(X) := \min\{c(u,v) - y(v) : u \notin X, v \in X\}\).
			      \item Para cada \(v \in X\), atualize \(y(v) := y(v) + \Delta(X)\).
		      \end{itemize}
		\item \textbf{Implementação:} Esse processo de iteração é realizado pela composição sequencial de três funções auxiliares:
		      \begin{enumerate}
			      \item \texttt{get\_arcs\_entering\_X(D, X)}: Identifica o conjunto \(\{(u,v) : u \notin X, v \in X\}\), isto é, todos os arcos que cruzam a fronteira de \(X\). Essa função corresponde diretamente à definição do conjunto sobre o qual o mínimo é calculado na fórmula \(\Delta(X)\).

			      \item \texttt{get\_minimum\_weight\_cut(arcs)}: Calcula \(\min\{\text{data}["w"] : (u,v,\text{data}) \in \text{arcs}\}\), que é exatamente \(\Delta(X) = \min\{c_y(u,v) : u \notin X, v \in X\}\). Como os pesos no dígrafo já representam custos reduzidos (são atualizados a cada iteração), essa função devolve precisamente o valor teórico de \(\Delta(X)\).

			      \item \texttt{update\_weights\_in\_X(D, arcs, min\_weight, A\_zero, D\_zero)}: Implementa a atualização dos potenciais. Para cada arco \((u,v)\) entrando em \(X\), subtrai \texttt{min\_weight} de \texttt{D[u][v]["w"]}, efetivamente calculando o novo custo reduzido:
			            \[c_y'(u,v) = c_y(u,v) - \Delta(X) = \]
			            \[c(u,v) - y(v) - \Delta(X) = \]
			            \[c(u,v) - \bigl(y(v) + \Delta(X)\bigr) = c(u,v) - y'(v), \]
			            onde \(y'(v) = y(v) + \Delta(X)\) é o novo potencial. Arcos cujo custo reduzido atinge zero (\texttt{D[u][v]["w"] == 0}) são adicionados a \(A_0\) e \(D_0\), tornando-se \textbf{justos}.
		      \end{enumerate}
	\end{itemize}

	\vspace{2mm}
	\textbf{Passo 3 — Construção de \(A_0\):}
	\begin{itemize}
		\item \textbf{Descrição teórica:} Defina \(A_0 := \{a \in A : c_y(a) = 0\}\), o conjunto de arcos justos.
		\item \textbf{Implementação:} A construção de \(A_0\) ocorre de forma incremental durante as iterações do Passo~2. Cada chamada a \texttt{update\_weights\_in\_X} verifica quais arcos atingiram custo reduzido zero e os adiciona tanto à lista \texttt{A\_zero} quanto ao dígrafo \texttt{D\_zero}. Esse processo continua até que a função principal da Fase~1 (que será apresentada adiante) determine que cada componente fortemente conexa em \(D_0\) (exceto a raiz) possui ao menos um arco justo entrando, garantindo que \(A_0\) é suficiente para formar a base de uma \(r\)-arborescência.
		\item O resultado final é o conjunto completo \(A_0 = \{a \in A : c_y(a) = 0\}\), usado na Fase~2 para construir a arborescência ótima através de uma seleção gulosa de arcos justos.
	\end{itemize}
\end{tcolorbox}

\subsection{Fase 2: Construção da arborescência}
Esta é a função principal da Fase~2, responsável por construir a r-arborescência de custo mínimo a partir do conjunto \(A_0\) de arcos justos. A construção é incremental: inicia-se com a raiz e adiciona-se iterativamente arcos de \(A_0\) que conectam vértices já incluídos a novos vértices, garantindo que cada vértice não-raiz receba exatamente um arco de entrada.

Recebe como entrada um dígrafo \texttt{D\_original}, a raiz \texttt{r0}, e a lista \texttt{A\_zero} de arcos justos. A implementação cria um novo dígrafo vazio \texttt{Arb} (linha 2) e adiciona a raiz (linha 3).

O loop principal (linhas 5-12) itera \(n-1\) vezes, onde \(n = |V|\), pois uma r-arborescência tem exatamente \(|V|-1\) arcos. Em cada iteração:
\begin{enumerate}\setlength{\itemsep}{2pt}
	\item Percorre os arcos \((u,v)\) de \texttt{A\_zero} (linha 6).
	\item Verifica se \(u\) já está em \texttt{Arb} e \(v\) ainda não (linha 7).
	\item Se sim, obtém os dados do arco do dígrafo original (linha 8) e adiciona \((u,v)\) a \texttt{Arb} (linha 9).
	\item Interrompe o loop interno para reiniciar a busca, garantindo descoberta em largura (linha 10).
\end{enumerate}

A função devolve o dígrafo \texttt{Arb} representando a r-arborescência de custo mínimo. A complexidade é \(O(nm)\) no pior caso, pois cada uma das \(O(n)\) iterações pode percorrer todos os \(O(m)\) arcos de \texttt{A\_zero}.

\begin{tcolorbox}[
		enhanced, breakable,
		colframe=blue!60!black, colback=blue!2,
		colbacktitle=blue!15, coltitle=black,
		title={Fase 2: Construção da arborescência},
		boxed title style={sharp corners, boxrule=0.6pt},
		sharp corners, boxrule=0.6pt
	]
	\emph{Constrói incrementalmente a r-arborescência a partir de $A_0$, adicionando iterativamente arcos que conectam vértices já incluídos a novos vértices.}
	\tcblower
	\begin{lstlisting}[language=Python]
def phase2(D_original, r0, A_zero):
    Arb = nx.DiGraph()
    Arb.add_node(r0)
    n = len(D_original.nodes())
    for _ in range(n - 1):
        for u, v in A_zero:
            if u in Arb.nodes() and v not in Arb.nodes():
                edge_data = D_original.get_edge_data(u, v)
                Arb.add_edge(u, v, **edge_data)
                break
    return Arb
\end{lstlisting}
\end{tcolorbox}

Apresentamos também uma versão alternativa da Fase~2 que utiliza busca em largura (BFS) para construir a r-arborescência de forma mais eficiente. Diferentemente da versão anterior que itera \(n-1\) vezes sobre todos os arcos, esta implementação usa uma fila de prioridade para explorar os arcos em ordem, evitando buscas lineares repetidas.

Recebe como entrada um dígrafo \texttt{D\_original}, a raiz \texttt{r0}, e a lista \texttt{A\_zero} de arcos justos. A implementação começa criando um dígrafo auxiliar \texttt{Arb} (linha 2) onde cada arco de \texttt{A\_zero} recebe um peso igual ao seu índice na lista (linhas 3-4), estabelecendo uma ordem de exploração. Inicializa-se o conjunto \texttt{V} de vértices visitados contendo apenas a raiz (linha 5) e uma fila de prioridade vazia \texttt{q} (linha 6).

Todos os arcos que saem da raiz em \texttt{Arb} são adicionados à fila de prioridade (linhas 7-8), usando o peso (índice) como critério de ordenação. Cria-se então o dígrafo \texttt{A} que conterá a arborescência resultante (linha 9).

O loop principal (linhas 10-17) extrai arcos da fila de prioridade em ordem crescente de índice. Para cada arco \((u,v)\) extraído (linha 11), verifica-se se o vértice destino \(v\) já foi visitado (linha 12); em caso positivo, o arco é ignorado via \texttt{continue} (linha 13). Caso contrário, adiciona-se o arco \((u,v)\) à arborescência \texttt{A} com o peso original de \texttt{D\_original} (linha 14), marca-se \(v\) como visitado (linha 15), e todos os arcos que saem de \(v\) em \texttt{Arb} são adicionados à fila de prioridade para exploração futura (linhas 16-17).

A função devolve o dígrafo \texttt{A} representando a r-arborescência de custo mínimo. A complexidade é \(O(m \log m)\), onde \(m = |A_0|\), devido às operações de inserção e remoção na fila de prioridade. Esta versão é mais eficiente que a anterior quando \(|A_0|\) é grande, pois evita percorrer todos os arcos em cada iteração.

\begin{tcolorbox}[
		enhanced, breakable,
		colframe=blue!60!black, colback=blue!2,
		colbacktitle=blue!15, coltitle=black,
		title={Fase 2 (versão BFS): Construção da arborescência com fila de prioridade},
		boxed title style={sharp corners, boxrule=0.6pt},
		sharp corners, boxrule=0.6pt
	]
	\emph{Constrói a r-arborescência usando busca em largura guiada por fila de prioridade, explorando arcos de $A_0$ em ordem e evitando buscas lineares repetidas. Complexidade $O(m \log m)$.}
	\tcblower
	\begin{lstlisting}[language=Python]
def phase2_v2(D_original, r0, A_zero):
    Arb = nx.DiGraph()
    for i, (u, v) in enumerate(A_zero):
        Arb.add_edge(u, v, w=i)
    V = {r0}
    q = []
    for u, v, data in Arb.out_edges(r0, data=True):
        heapq.heappush(q, (data["w"], u, v))
    A = nx.DiGraph()
    while q:
        _, u, v = heapq.heappop(q)
        if v in V:
            continue
        A.add_edge(u, v, w=D_original[u][v]["w"])
        V.add(v)
        for x, y, data in Arb.out_edges(v, data=True):
            heapq.heappush(q, (data["w"], x, y))
    return A
\end{lstlisting}
\end{tcolorbox}

As duas versões da Fase~2 implementadas acima correspondem diretamente ao Passo~4 da descrição teórica do algoritmo de András Frank:

\begin{tcolorbox}[
		enhanced, breakable,
		colframe=purple!60!black, colback=purple!5,
		colbacktitle=purple!15, coltitle=black,
		title={Correspondência entre Teoria e Implementação — Fase 2},
		boxed title style={sharp corners, boxrule=0.6pt},
		sharp corners, boxrule=0.6pt
	]
	\textbf{Passo 4 — Construção da arborescência (caso acíclico):}
	\begin{itemize}
		\item \textbf{Descrição teórica:} Se \((V,A_0)\) forma uma r-arborescência, devolva \(A_0\). Por otimalidade dos potenciais duais, trata-se de uma r-arborescência de custo mínimo.
		\item \textbf{Implementação:} Ambas as versões de \texttt{phase2} constroem uma r-arborescência a partir do conjunto \(A_0\) de arcos justos obtido na Fase~1. A corretude baseia-se no fato de que todos os arcos em \(A_0\) têm custo reduzido zero, e a Fase~1 garante que existe uma r-arborescência formada exclusivamente por arcos justos.
		      \begin{itemize}
			      \item \textbf{Versão 1 (\texttt{phase2}):} Construção incremental por exploração exaustiva. Em cada uma das \(n-1\) iterações, percorre todos os arcos de \(A_0\) procurando um arco \((u,v)\) tal que \(u\) já pertence à arborescência parcial e \(v\) ainda não. Essa abordagem simples corresponde diretamente à ideia teórica de construir a arborescência adicionando um vértice por vez, conectando-o à estrutura existente através de um arco justo. Complexidade: \(O(nm)\).

			      \item \textbf{Versão 2 (\texttt{phase2\_v2}):} Construção por busca em largura guiada por fila de prioridade. Cria um dígrafo auxiliar onde arcos são indexados, usa a fila de prioridade para explorar arcos sistematicamente a partir da raiz, evitando buscas lineares repetidas. Essa versão otimizada mantém a mesma correção teórica — construir uma r-arborescência usando apenas arcos de \(A_0\) — mas melhora a eficiência prática. Complexidade: \(O(m \log m)\).
		      \end{itemize}
	\end{itemize}

	\vspace{2mm}
	\textbf{Nota sobre Passos 5-7 (contração/recursão/expansão):}
	\begin{itemize}
		\item A descrição teórica do algoritmo de András Frank inclui os Passos~5-7 para tratar o caso onde \(A_0\) contém ciclos dirigidos, exigindo contração, resolução recursiva e reexpansão, de forma análoga ao algoritmo de Chu--Liu--Edmonds.
		\item Na implementação apresentada, optamos por uma abordagem não-recursiva baseada em componentes fortemente conexas. A Fase~1 já garante que \(A_0\) formará uma r-arborescência ao término das iterações de elevação de potenciais, eliminando a necessidade de tratar ciclos explicitamente na Fase~2. Essa simplificação é possível porque a elevação de potenciais progressivamente "quebra" todos os ciclos ao criar novos arcos justos que conectam diferentes componentes, até que reste apenas uma única componente fortemente conexa contendo todos os vértices.
		\item Portanto, quando a Fase~2 é executada, o conjunto \(A_0\) já está livre de ciclos e forma uma r-arborescência, correspondendo diretamente ao caso tratado pelo Passo~4 da descrição teórica. A verificação prévia \texttt{has\_arborescence(D, r0)} (realizada pela função principal) confirma essa propriedade antes de invocar a Fase~2.
	\end{itemize}
\end{tcolorbox}

\subsection{Verificação de otimalidade dual}

Esta função verifica se a condição de otimalidade dual é satisfeita para a r-arborescência construída. Segundo a teoria de programação linear dual aplicada ao problema de arborescência de custo mínimo, uma solução é ótima se e somente se cada conjunto \(X \subseteq V \setminus \{r\}\) que teve seu potencial elevado durante a Fase~1 possui exatamente um arco entrando na arborescência final.

Recebe como entrada a arborescência \texttt{Arb} e a lista \texttt{Dual\_list} contendo pares \((X, \Delta(X))\) onde \(X\) é um conjunto de vértices cujos potenciais foram elevados e \(\Delta(X) > 0\) é o valor da elevação. A implementação itera sobre cada par \((X, z)\) na lista dual (linha 2), e para cada conjunto \(X\), percorre todos os arcos \((u,v)\) da arborescência (linha 3). Inicializa um contador \texttt{count} em zero (linha 4) e verifica se o arco cruza a fronteira de \(X\), isto é, se \(u \notin X\) e \(v \in X\) (linha 5). Quando essa condição é satisfeita, incrementa o contador (linha 6) e imediatamente verifica se já há mais de um arco entrando em \(X\) (linha 7). Caso positivo, a condição de otimalidade dual é violada e a função devolve \texttt{False} (linha 8).

A função devolve \texttt{True} se todos os conjuntos em \texttt{Dual\_list} possuem exatamente um arco entrando na arborescência, confirmando que a solução satisfaz as condições de folga complementar da programação linear dual. A complexidade é \(O(km)\), onde \(k = |\texttt{Dual\_list}|\) e \(m = |A|\), pois para cada conjunto dual verifica-se todos os arcos da arborescência.

Esta verificação é fundamental para garantir a correção do algoritmo: a Fase~1 constrói uma solução dual viável (potenciais \(y(v)\)), a Fase~2 constrói uma solução primal viável (arborescência), e esta função confirma que ambas satisfazem as condições de folga complementar, implicando otimalidade pelo teorema da dualidade forte.

\begin{tcolorbox}[
		enhanced, breakable,
		colframe=blue!60!black, colback=blue!2,
		colbacktitle=blue!15, coltitle=black,
		title={Verificação de otimalidade dual},
		boxed title style={sharp corners, boxrule=0.6pt},
		sharp corners, boxrule=0.6pt
	]
	\emph{Verifica se a condição de otimalidade dual é satisfeita, confirmando que cada conjunto dual possui exatamente um arco entrando na arborescência.}
	\tcblower
	\begin{lstlisting}[language=Python]
def check_dual_optimality_condition(Arb, Dual_list):
    for X, z in Dual_list:
        count = 0
        for u, v in Arb.edges():
            if u not in X and v in X:
                count += 1
                if count > 1:
                    return False
    return True
\end{lstlisting}
\end{tcolorbox}

\subsection{O algoritmo completo de András Frank}

Finalmente, apresentamos a função principal que implementa o algoritmo de András Frank para encontrar uma r-arborescência de custo mínimo em um dígrafo com pesos. A função integra as fases de construção dos potenciais duais, obtenção dos arcos justos, construção da arborescência e verificação de otimalidade dual.

\begin{tcolorbox}[
		enhanced, breakable,
		colframe=blue!60!black, colback=blue!2,
		colbacktitle=blue!15, coltitle=black,
		title={Verificação de otimalidade dual},
		boxed title style={sharp corners, boxrule=0.6pt},
		sharp corners, boxrule=0.6pt
	]
	\emph{Implementa o algoritmo completo de András Frank, integrando as fases de construção dos potenciais duais, obtenção dos arcos justos, construção da arborescência e verificação de otimalidade dual.}
	\tcblower
	\begin{lstlisting}[language=Python]
def andras_frank_algorithm(D):
    A_zero, Dual_list = phase1(D,"r0")
    arborescence_frank = phase2(D, "r0", A_zero)
    arborescence_frank_v2 = phase2_v2(D, "r0", A_zero)
    dual_frank = check_dual_optimality_condition(
        arborescence_frank, Dual_list)
    dual_frank_v2 = check_dual_optimality_condition(
        arborescence_frank_v2, Dual_list)
    return arborescence_frank, arborescence_frank_v2, dual_frank, dual_frank_v2
\end{lstlisting}
\end{tcolorbox}

\subsection*{Exemplo de execução do algoritmo}

Aqui ilustraremos cada fase do algoritmo de András Frank: inicialização dos potenciais, iterações de elevação, construção de \(A_0\), e formação da arborescência final. Utilizaremos o dígrafo \(D_{32}\) estendido com a raiz \(r_0\), mostrado na Figura~\ref{fig:frank-exemplo-inicial}.

\begin{figure}[H]
	\centering
	\begin{tikzpicture}[>=Stealth, scale=0.9, every node/.style={scale=0.9}]
		\tikzset{
			vtx/.style={circle, draw, fill=blue!8, thick, inner sep=1.2pt, minimum size=6.5mm},
			vtxRoot/.style={circle, draw, fill=red!15, thick, inner sep=1.2pt, minimum size=6.5mm},
			arc/.style={->, thick, gray!60},
			cost/.style={font=\scriptsize, fill=white, inner sep=1pt, rounded corners=1pt}
		}

		\node[font=\bfseries, blue!80!black] at (3, 4.5) {Dígrafo inicial $D$ com raiz $r_0$};

		% Raiz
		\node[vtxRoot] (r0) at (3, 3) {$r_0$};

		% Vértices camada 1
		\node[vtx] (u1) at (0, 1.5) {$u_1$};
		\node[vtx] (u2) at (3, 1.5) {$u_2$};
		\node[vtx] (u3) at (6, 1.5) {$u_3$};

		% Vértices camada 2
		\node[vtx] (v1) at (1, 0) {$v_1$};
		\node[vtx] (v2) at (3, 0) {$v_2$};
		\node[vtx] (v3) at (5, 0) {$v_3$};

		% Arcos da raiz
		\draw[arc] (r0) -- node[cost, left, pos=0.4] {1} (u1);
		\draw[arc] (r0) -- node[cost, right] {2} (u2);
		\draw[arc] (r0) -- node[cost, right, pos=0.4] {1} (u3);

		% Arcos camada 1
		\draw[arc] (u1) to[bend left=15] node[cost, above] {1} (u2);
		\draw[arc] (u2) to[bend left=15] node[cost, above] {3} (u3);

		% Arcos entrando camada 2
		\draw[arc] (u1) -- node[cost, left] {3} (v1);
		\draw[arc] (u2) -- node[cost, right] {2} (v2);
		\draw[arc] (u3) -- node[cost, right] {4} (v3);
		\draw[arc] (u1) to[bend right=20] node[cost, below, pos=0.3] {5} (v2);

		% Arcos internos camada 2
		\draw[arc] (v1) -- node[cost, above] {1} (v2);
		\draw[arc] (v2) -- node[cost, above] {2} (v3);

		% Arco retorno
		\draw[arc] (v1) to[bend right=30] node[cost, left, pos=0.3] {2} (u1);

		\node[font=\small, gray!70!black] at (3, -1.2) {Pesos originais $c(u,v)$};
	\end{tikzpicture}
	\caption{Dígrafo inicial \(D\) com raiz \(r_0\) e 6 vértices adicionais. O dígrafo contém múltiplos arcos com pesos variados. A Fase~1 do algoritmo iniciará com potenciais \(y(v) = 0\) para todos os vértices, correspondendo aos custos reduzidos iniciais \(c_y(u,v) = c(u,v)\).}
	\label{fig:frank-exemplo-inicial}
\end{figure}

\paragraph*{Fase 1 — Iteração 1:} O algoritmo inicia com todos os potenciais \(y(v) = 0\) e \(D_0\) vazio (sem arcos justos). Calcula-se as componentes fortemente conexas de \(D_0\): cada vértice forma sua própria componente. As fontes são todos os vértices exceto \(r_0\). Para o conjunto \(X = \{u_1, u_2, u_3\}\), identifica-se os arcos entrando: \((r_0, u_1)\) com peso 1, \((r_0, u_2)\) com peso 2, \((r_0, u_3)\) com peso 1. O mínimo é \(\Delta(X) = 1\). Subtraindo esse valor, os arcos \((r_0, u_1)\) e \((r_0, u_3)\) tornam-se justos (peso 0) e são adicionados a \(A_0\).

\begin{figure}[H]
	\centering
	\begin{tikzpicture}[>=Stealth, scale=0.85, every node/.style={scale=0.85}]
		\tikzset{
			vtx/.style={circle, draw, fill=blue!8, thick, inner sep=1.2pt, minimum size=6.5mm},
			vtxRoot/.style={circle, draw, fill=red!15, thick, inner sep=1.2pt, minimum size=6.5mm},
			vtxX/.style={circle, draw, fill=orange!20, thick, inner sep=1.2pt, minimum size=6.5mm},
			arc/.style={->, thick, gray!60},
			arcTight/.style={->, ultra thick, blue!70!black},
			cost/.style={font=\scriptsize, fill=white, inner sep=1pt},
			costTight/.style={font=\scriptsize\bfseries, fill=blue!10, inner sep=2pt, draw=blue!70!black, rounded corners=1pt}
		}

		% Antes
		\begin{scope}
			\node[font=\bfseries, blue!80!black] at (2, 4.5) {Antes: $\Delta(X_1) = 1$};

			\node[vtxRoot] (r0) at (2, 3) {$r_0$};
			\node[vtxX] (u1) at (0, 1.5) {$u_1$};
			\node[vtxX] (u2) at (2, 1.5) {$u_2$};
			\node[vtxX] (u3) at (4, 1.5) {$u_3$};

			\draw[arc] (r0) -- node[cost, left, pos=0.4] {1} (u1);
			\draw[arc] (r0) -- node[cost, right] {2} (u2);
			\draw[arc] (r0) -- node[cost, right, pos=0.4] {1} (u3);

			\node[font=\small, orange!80!black] at (2, 0.5) {$X_1 = \{u_1, u_2, u_3\}$};
		\end{scope}

		% Depois
		\begin{scope}[xshift=7cm]
			\node[font=\bfseries, blue!80!black] at (2, 4.5) {Depois: arcos justos};

			\node[vtxRoot] (r0b) at (2, 3) {$r_0$};
			\node[vtx] (u1b) at (0, 1.5) {$u_1$};
			\node[vtx] (u2b) at (2, 1.5) {$u_2$};
			\node[vtx] (u3b) at (4, 1.5) {$u_3$};

			\draw[arcTight] (r0b) -- node[costTight, left, pos=0.4] {0} (u1b);
			\draw[arc] (r0b) -- node[cost, right] {1} (u2b);
			\draw[arcTight] (r0b) -- node[costTight, right, pos=0.4] {0} (u3b);

			\node[font=\small, blue!70!black] at (2, 0.5) {2 arcos adicionados a $A_0$};
		\end{scope}
	\end{tikzpicture}
	\caption{Fase~1, Iteração 1: Elevação de potenciais para \(X_1 = \{u_1, u_2, u_3\}\). À esquerda, o conjunto \(X_1\) (em laranja) sem arcos justos entrando. Calcula-se \(\Delta(X_1) = \min\{1, 2, 1\} = 1\). À direita, após subtrair \(\Delta(X_1)\), os arcos \((r_0, u_1)\) e \((r_0, u_3)\) tornam-se \textcolor{blue!70!black}{justos} (custo reduzido 0) e são adicionados a \(A_0\).}
	\label{fig:frank-exemplo-iter1}
\end{figure}

\paragraph*{Fase 1 — Iteração 2:} Com \(A_0 = \{(r_0, u_1), (r_0, u_3)\}\), as componentes fortemente conexas são: \(\{r_0, u_1, u_3\}\) e \(\{u_2\}\), \(\{v_1\}\), \(\{v_2\}\), \(\{v_3\}\). A fonte (excluindo a raiz) é \(\{u_2\}\). Para \(X = \{u_2\}\), o único arco entrando é \((r_0, u_2)\) com peso atual 1 (após iteração anterior). Portanto \(\Delta(X) = 1\), e após subtração, \((r_0, u_2)\) torna-se justo. Também na mesma iteração, para \(X = \{v_1, v_2, v_3\}\), calcula-se \(\Delta(X) = \min\{3, 2, 4, 5\} = 2\), e os arcos são atualizados conforme ilustrado.

\begin{figure}[H]
	\centering
	\begin{tikzpicture}[>=Stealth, scale=0.75, every node/.style={scale=0.75}]
		\tikzset{
			vtx/.style={circle, draw, fill=blue!8, thick, inner sep=1.2pt, minimum size=6mm},
			vtxRoot/.style={circle, draw, fill=red!15, thick, inner sep=1.2pt, minimum size=6mm},
			vtxX/.style={circle, draw, fill=orange!20, thick, inner sep=1.2pt, minimum size=6mm},
			arc/.style={->, thick, gray!60},
			arcTight/.style={->, ultra thick, blue!70!black},
			cost/.style={font=\scriptsize, fill=white, inner sep=1pt},
			costTight/.style={font=\scriptsize\bfseries, fill=blue!10, inner sep=2pt, draw=blue!70!black, rounded corners=1pt}
		}

		\node[font=\bfseries, blue!80!black] at (3, 5.5) {Iteração 2: $\Delta(\{v_1, v_2, v_3\}) = 2$};

		\node[vtxRoot] (r0) at (3, 4) {$r_0$};
		\node[vtx] (u1) at (0.5, 2.5) {$u_1$};
		\node[vtx] (u2) at (3, 2.5) {$u_2$};
		\node[vtx] (u3) at (5.5, 2.5) {$u_3$};
		\node[vtxX] (v1) at (1.5, 0.5) {$v_1$};
		\node[vtxX] (v2) at (3, 0.5) {$v_2$};
		\node[vtxX] (v3) at (4.5, 0.5) {$v_3$};

		% Arcos já justos
		\draw[arcTight] (r0) -- node[costTight, left, pos=0.3] {0} (u1);
		\draw[arcTight] (r0) -- node[costTight, right, pos=0.3] {0} (u3);
		\draw[arcTight] (u1) to[bend left=12] node[costTight, above] {0} (u2);

		% Arcos entrando em X (antes de ficarem justos)
		\draw[arc] (u1) -- node[cost, left, fill=orange!10] {3→1} (v1);
		\draw[arc] (u2) -- node[cost, right, fill=orange!10] {2→0} (v2);
		\draw[arc] (u3) -- node[cost, right] {4→2} (v3);
		\draw[arc] (u1) to[bend right=15] node[cost, below, pos=0.35] {5→3} (v2);

		\node[font=\small, orange!80!black] at (3, -0.5) {Após subtração: $(u_2, v_2)$ torna-se justo};
	\end{tikzpicture}
	\caption{Fase~1, Iteração 2: Para o conjunto \(X = \{v_1, v_2, v_3\}\) (em laranja), calcula-se \(\Delta(X) = \min\{3, 2, 4, 5\} = 2\). Após subtrair esse valor dos arcos entrando em \(X\), o arco \((u_2, v_2)\) atinge custo reduzido 0 e é adicionado a \(A_0\). Note que \((u_1, u_2)\) também se tornou justo na mesma iteração ao processar \(X = \{u_2\}\).}
	\label{fig:frank-exemplo-iter2}
\end{figure}

\paragraph*{Fase 1 — Estado final:} Após mais iterações elevando potenciais para os conjuntos restantes, o dígrafo \(D_0\) de arcos justos fica completo, contendo uma r-arborescência. A Figura~\ref{fig:frank-exemplo-A0-final} mostra o conjunto final \(A_0\) com 6 arcos de custo reduzido zero.

\begin{figure}[H]
	\centering
	\begin{tikzpicture}[>=Stealth, scale=0.85, every node/.style={scale=0.85}]
		\tikzset{
			vtx/.style={circle, draw, fill=blue!8, thick, inner sep=1.2pt, minimum size=6.5mm},
			vtxRoot/.style={circle, draw, fill=red!15, thick, inner sep=1.2pt, minimum size=6.5mm},
			arcTight/.style={->, ultra thick, blue!70!black},
			costTight/.style={font=\scriptsize\bfseries, fill=blue!10, inner sep=2pt, draw=blue!70!black, rounded corners=1pt}
		}

		\node[font=\bfseries, blue!80!black] at (3, 4.5) {$A_0$ final: arcos justos após Fase~1};

		\node[vtxRoot] (r0) at (3, 3) {$r_0$};
		\node[vtx] (u1) at (0.5, 1.5) {$u_1$};
		\node[vtx] (u2) at (3, 1.5) {$u_2$};
		\node[vtx] (u3) at (5.5, 1.5) {$u_3$};
		\node[vtx] (v1) at (1.5, 0) {$v_1$};
		\node[vtx] (v2) at (3, 0) {$v_2$};
		\node[vtx] (v3) at (4.5, 0) {$v_3$};

		\draw[arcTight] (r0) -- node[costTight, left, pos=0.3] {0} (u1);
		\draw[arcTight] (r0) -- node[costTight, right, pos=0.3] {0} (u3);
		\draw[arcTight] (u1) to[bend left=12] node[costTight, above] {0} (u2);
		\draw[arcTight] (u1) -- node[costTight, left] {0} (v1);
		\draw[arcTight] (u2) -- node[costTight, right] {0} (v2);
		\draw[arcTight] (v2) -- node[costTight, above] {0} (v3);

		\node[font=\small, blue!70!black] at (3, -1.2) {$|A_0| = 6$ arcos com custo reduzido $c_y = 0$};
	\end{tikzpicture}
	\caption{Conjunto final \(A_0\) de arcos justos após conclusão da Fase~1. Todos os 6 arcos \textcolor{blue!70!black}{destacados} possuem custo reduzido zero: \((r_0, u_1)\), \((r_0, u_3)\), \((u_1, u_2)\), \((u_1, v_1)\), \((u_2, v_2)\), \((v_2, v_3)\). O dígrafo formado por esses arcos é acíclico e alcança todos os vértices a partir de \(r_0\), satisfazendo as condições para prosseguir à Fase~2.}
	\label{fig:frank-exemplo-A0-final}
\end{figure}

\paragraph*{Fase 2 — Construção da arborescência:} Com \(A_0\) completo e acíclico, a Fase~2 constrói incrementalmente a arborescência final. Inicia-se com \(\texttt{Arb} = \{r_0\}\) e em cada iteração adiciona-se um arco \((u,v) \in A_0\) tal que \(u \in \texttt{Arb}\) e \(v \notin \texttt{Arb}\). A Figura~\ref{fig:frank-exemplo-arborescencia-final} mostra a arborescência resultante.

\begin{figure}[H]
	\centering
	\begin{tikzpicture}[>=Stealth, scale=0.85, every node/.style={scale=0.85}]
		\tikzset{
			vtx/.style={circle, draw, fill=green!15, thick, inner sep=1.2pt, minimum size=6.5mm},
			vtxRoot/.style={circle, draw, fill=red!15, thick, inner sep=1.2pt, minimum size=6.5mm},
			arcArb/.style={->, ultra thick, green!70!black},
			costArb/.style={font=\scriptsize\bfseries, fill=green!10, inner sep=2pt, draw=green!70!black, rounded corners=1pt}
		}

		\node[font=\bfseries, blue!80!black] at (3, 4.5) {Arborescência final (Fase~2)};

		\node[vtxRoot] (r0) at (3, 3) {$r_0$};
		\node[vtx] (u1) at (0.5, 1.5) {$u_1$};
		\node[vtx] (u2) at (3, 1.5) {$u_2$};
		\node[vtx] (u3) at (5.5, 1.5) {$u_3$};
		\node[vtx] (v1) at (1.5, 0) {$v_1$};
		\node[vtx] (v2) at (3, 0) {$v_2$};
		\node[vtx] (v3) at (4.5, 0) {$v_3$};

		\draw[arcArb] (r0) -- node[costArb, left, pos=0.3] {1} (u1);
		\draw[arcArb] (r0) -- node[costArb, right, pos=0.3] {1} (u3);
		\draw[arcArb] (u1) to[bend left=12] node[costArb, above] {1} (u2);
		\draw[arcArb] (u1) -- node[costArb, left] {3} (v1);
		\draw[arcArb] (u2) -- node[costArb, right] {2} (v2);
		\draw[arcArb] (v2) -- node[costArb, above] {2} (v3);

		\node[font=\small, green!70!black] at (3, -1.2) {Custo total: $1+1+1+3+2+2 = 10$};
	\end{tikzpicture}
	\caption{Arborescência de custo mínimo final obtida pela Fase~2. Os pesos mostrados são os \emph{originais} de \(D\), restaurados após a construção. Cada vértice não-raiz possui exatamente um arco de entrada, não há ciclos, e todos os vértices são alcançáveis a partir de \(r_0\). A verificação de otimalidade dual confirma que cada conjunto \(X\) que teve potenciais elevados possui exatamente um arco da arborescência cruzando sua fronteira, garantindo que a solução é ótima.}
	\label{fig:frank-exemplo-arborescencia-final}
\end{figure}

\paragraph*{Verificação de otimalidade dual:} A função \texttt{check\_dual\_optimality\_condition} confirma que para cada par \((X, \Delta(X))\) em \texttt{Dual\_list} (conjuntos cujos potenciais foram elevados com \(\Delta(X) > 0\)), existe exatamente um arco da arborescência final cruzando a fronteira de \(X\). Essa condição, juntamente com os arcos justos, garante que as condições de folga complementar da programação linear dual são satisfeitas, implicando que a arborescência encontrada é de custo mínimo global.

\subsection{Correspondência entre teoria e implementação}

A implementação em Python do algoritmo de András Frank segue fielmente a descrição teórica primal-dual apresentada anteriormente. A tabela abaixo estabelece o paralelo direto entre os passos teóricos e sua realização no código:

\begin{table}[H]
	\centering
	\scriptsize
	\renewcommand{\arraystretch}{1.3}
	\begin{tabular}{>{\raggedright\arraybackslash}p{0.44\textwidth}|>{\raggedright\arraybackslash}p{0.48\textwidth}}
		\toprule
		\textbf{\normalsize Descrição Teórica} & \textbf{\normalsize Implementação Python} \\
		\midrule
		\rowcolor{blue!5}
		\textbf{Passo 1:} Inicialização

		Defina \(y(v) := 0\) para todo \(v \in V\).

		Inicialize \(A_0 := \emptyset\).

		Construa dígrafo vazio \(D_0\) (arcos justos).
		                                       &
		\textbf{Função \texttt{phase1} — Linhas 2--5:}

		\texttt{D\_copy = D\_original.copy()}

		\texttt{A\_zero = []}

		\texttt{D\_zero = build\_D\_zero(D\_copy)}

		\vspace{1mm}
		Potenciais \(y(v) = 0\) implícitos, custos \(c_y = c\).
		\\
		\midrule
		\rowcolor{green!5}
		\textbf{Passo 2:} Elevação de potenciais

		Enquanto \(\exists X \subseteq V \setminus \{r\}\) sem arco justo:

		\quad Calcule \(\Delta(X) := \min\{c_y(u,v) : u \notin X, v \in X\}\)

		\quad Atualize \(y(v) := y(v) + \Delta(X)\), \(\forall v \in X\)

		\quad Adicione arcos com \(c_y = 0\) a \(A_0\)
		                                       &
		\textbf{Loop principal — Linhas 7--22:}

		\texttt{C = nx.condensation(D\_zero)}

		\texttt{sources = [x for x in C.nodes()]}

		\texttt{\quad if C.in\_degree(x) == 0]}

		Para fonte \texttt{u} (exceto raiz):

		\texttt{\quad X = C.nodes[u]["members"]}

		\texttt{\quad arcs = get\_arcs\_entering\_X(D, X)}

		\texttt{\quad min\_w = get\_minimum\_weight\_cut(arcs)}

		\texttt{\quad update\_weights\_in\_X(D, arcs, min\_w,}

		\texttt{\quad\quad A\_zero, D\_zero)}
		\\
		\midrule
		\rowcolor{orange!5}
		\textbf{Passo 2(a):} Identificar arcos entrando

		Determine \(\{(u,v) \in A : u \notin X, v \in X\}\)
		                                       &
		\textbf{Função \texttt{get\_arcs\_entering\_X}:}

		\texttt{return [(u, v, data)}

		\texttt{\quad for u, v, data in D.edges(data=True)}

		\texttt{\quad if u not in X and v in X]}
		\\
		\midrule
		\rowcolor{purple!5}
		\textbf{Passo 2(b):} Calcular \(\Delta(X)\)

		\(\Delta(X) := \min\{c_y(u,v) : u \notin X, v \in X\}\)
		                                       &
		\textbf{Função \texttt{get\_minimum\_weight\_cut}:}

		\texttt{return min(data["w"]}

		\texttt{\quad for \_, \_, data in arcs)}
		\\
		\midrule
		\rowcolor{yellow!5}
		\textbf{Passo 2(c):} Atualizar pesos

		Para \((u,v)\) entrando em \(X\):

		\quad \(c_y(u,v) := c_y(u,v) - \Delta(X)\)

		\quad Se \(c_y(u,v) = 0\), adicione a \(A_0\)
		                                       &
		\textbf{Função \texttt{update\_weights\_in\_X}:}

		\texttt{for u, v, \_ in arcs:}

		\texttt{\quad D[u][v]["w"] -= min\_weight}

		\texttt{\quad if D[u][v]["w"] == 0:}

		\texttt{\quad\quad A\_zero.append((u, v))}

		\texttt{\quad\quad D\_zero.add\_edge(u, v)}
		\\
		\midrule
		\rowcolor{cyan!5}
		\textbf{Passo 3:} Verificar término

		Se \(D_0\) contém r-arborescência, encerre.
		                                       &
		\textbf{Condição — Linhas 11--12:}

		\texttt{if len(sources) == 1: break}

		\vspace{1mm}
		Uma fonte \(\Rightarrow\) r-arborescência acíclica.
		\\
		\midrule
		\rowcolor{red!5}
		\textbf{Passo 4:} Construir arborescência

		Construa \(F\) a partir de \(A_0\), conectando vértices incrementalmente.
		                                       &
		\textbf{Função \texttt{phase2} (incremental):}

		\texttt{Arb = nx.DiGraph(); Arb.add\_node(r0)}

		\texttt{for \_ in range(n - 1):}

		\texttt{\quad for u, v in A\_zero:}

		\texttt{\quad\quad if u in Arb and v not in Arb:}

		\texttt{\quad\quad\quad Arb.add\_edge(u, v, **data)}

		\texttt{\quad\quad\quad break}

		\vspace{1mm}
		Complexidade: \(O(nm)\).
		\\
		\midrule
		\rowcolor{red!5}
		\textbf{Passo 4:} Versão otimizada

		Mesma ideia, com fila de prioridade.
		                                       &
		\textbf{Função \texttt{phase2\_v2} (BFS):}

		\texttt{Arb = nx.DiGraph()}

		\texttt{for i, (u, v) in enumerate(A\_zero):}

		\texttt{\quad Arb.add\_edge(u, v, w=i)}

		\texttt{q = [] \# fila de prioridade}

		\texttt{while q:}

		\texttt{\quad \_, u, v = heapq.heappop(q)}

		\texttt{\quad if v in V: continue}

		\texttt{\quad A.add\_edge(u, v, w=D[u][v]["w"])}

		\vspace{1mm}
		Complexidade: \(O(m \log m)\).
		\\
		\midrule
		\rowcolor{pink!5}
		\textbf{Otimalidade dual}

		Para cada \(X\) elevado (\(\Delta(X) > 0\)), exatamente um arco de \(F\) cruza \(\delta^-(X)\).
		                                       &
		\textbf{Função \texttt{check\_dual\_optimality}:}

		\texttt{for X, z in Dual\_list:}

		\texttt{\quad count = 0}

		\texttt{\quad for u, v in Arb.edges():}

		\texttt{\quad\quad if u not in X and v in X:}

		\texttt{\quad\quad\quad count += 1}

		\texttt{\quad\quad\quad if count > 1: return False}

		\texttt{return True}
		\\
		\bottomrule
	\end{tabular}
	\caption{Correspondência entre a descrição teórica do algoritmo de András Frank e sua implementação em Python. Cores: inicialização (azul), elevação de potenciais (verde/laranja/roxo/amarelo), verificação (ciano), construção (vermelho) e validação dual (rosa).}
	\label{tab:frank-teoria-implementacao}
\end{table}

Esta correspondência demonstra que a implementação traduz fielmente a abordagem primal-dual em código executável. As funções auxiliares (\texttt{get\_arcs\_entering\_X}, \texttt{get\_minimum\_weight\_cut}, \texttt{update\_weights\_in\_X}, \texttt{phase1}, \texttt{phase2}, \texttt{phase2\_v2},
\texttt{check\_dual\_}-
\texttt{optimality\_condition} encapsulam exatamente as operações descritas na teoria, preservando as propriedades de correção e as garantias de otimalidade do algoritmo original. A utilização de componentes fortemente conexas (\texttt{nx.condensation}) para identificar conjuntos sem arcos justos entrando é uma implementação eficiente da verificação teórica, evitando enumeração explícita de todos os subconjuntos de vértices.

\chapter{Algoritmo de András Frank}

Neste capítulo, apresentaremos o algoritmo de András Frank, que também determina
uma arborescência de custo mínimo em um digrafo ponderado.
O algoritmo baseia-se em uma abordagem primal--dual:
a elevação gulosa de potenciais duais\footnote{\textbf{Potencial dual} é uma variável associada a cada vértice, que representa um "desconto" aplicado ao custo dos arcos que entram nesse vértice e servem para ajustar os custos de modo que certas restrições fiquem "justas", - satisfeitas com igualdade, ou seja, o valor da variável atinge exatamente o limite imposto pela restrição.} e (ii) a construção de uma solução gulosa
a partir dos arcos de custo reduzido zero.
Diferentemente do algoritmo de Chu--Liu--Edmonds, que opera diretamente sobre os custos,
o método de Frank manipula potenciais duais associados aos vértices,
gerando arcos \emph{justos} (de custo reduzido zero) que formam a arborescência ótima.
O propósito deste capítulo é fornecer uma descrição precisa tanto do algoritmo
quanto da implementação desenvolvida neste trabalho.

\section{O algoritmo}
O algoritmo de András Frank também recebe uma tripla \((D,c,r)\), em que \(D=(V,A)\) é um digrafo,
\(c\colon A\to\mathbb{R}\) é uma função custo e \(r\in V\) é a raiz,
sob a hipótese de que \(D\) admite ao menos uma \(r\)-arborescência e devolve uma \(r\)-arborescência \(c\)-mínima de \(D\).

Assim como no capítulo anterior, adotamos a terminologia de \textbf{\(r\)-digrafo ponderado}
para uma tripla \((D,c,r)\) em que $(D, c)$ é um digrafo ponderado,
$r$ é um vértice de $D$, $\delta^-(r) = \varnothing$
e $D$ possui uma $r$-arborescência.

Vamos desenvolver as ideias do algoritmo utilizando o mesmo dígrafo que apresentamos
no capítulo anterior.

\subsection*{Abordagem Gulosa--Dual}

Considere novamente o dígrafo \(D\) figura 26 a seguir, com custos nos arcos note que é o mesmo digrafo da figura 1 que utilizamos para ilustrar o algoritmo de Chu--Liu--Edmonds. E portanto, suponha que T é uma arborescência de custo mínimo, com seus arcos destacados em azul.

\begin{figure}[h]
	\centering
	\begin{tikzpicture}[>=Stealth]
		% estilos
		\tikzset{
			vtx/.style={circle, draw,fill=blue!8, thick, inner sep=1.2pt, minimum size=6mm},
			chosen/.style={->, very thick, draw=blue},
			other/.style={->, semithick, draw=black!35},
			costB/.style={font=\scriptsize, fill=white, inner sep=1pt, text=blue},
			costG/.style={font=\scriptsize, fill=white, inner sep=1pt, text=black!55}
		}

		% layout (mais espaçado)
		\node[vtx] (r) at (0,0) {$r$};
		\node[vtx] (a) at (3,1.6) {$a$};
		\node[vtx] (b) at (3,-1.6) {$b$};
		\node[vtx] (c) at (6,2.2) {$c$};
		\node[vtx] (d) at (6,0.0) {$d$};
		\node[vtx] (e) at (6,-2.2) {$e$};

		% --- arcos NÃO escolhidos (cinza), agora com curvaturas mais limpas ---
		\draw[other] (b) to[bend left=12] node[costG, above] {4} (a);   % b->a
		\draw[other] (a) to[bend left=14] node[costG, below] {5} (b);   % a->b

		% r->c: externo por cima, suave (pontos de controle explícitos)
		\draw[other]
		(r) .. controls (0.8,2.4) and (5.2,3.0) ..
		node[costG, yshift=4pt, pos=0.53] {3} (c);

		% a->d (referência)
		\draw[other] (a) -- node[costG, right] {3} (d);

		% c->e: leve curvatura pela direita (sem "barriga")
		\draw[other]
		(c) .. controls (7.6,2.6) and (7.6,-2.6) ..
		node[costG, xshift=8pt, pos=0.52] {5} (e);

		% --- arcos ESCOLHIDOS (AZUL) — formam a r-arborescência T ---
		\draw[chosen] (r) -- node[costB, above] {1} (a);   % r->a
		\draw[chosen] (r) -- node[costB, below] {2} (b);   % r->b
		\draw[chosen] (a) -- node[costB, above] {1} (c);   % a->c
		\draw[chosen] (b) -- node[costB, above] {1} (d);   % b->d
		\draw[chosen] (b) -- node[costB, below] {2} (e);   % b->e
	\end{tikzpicture}

	\caption{O dígrafo \(D\) com custos nos arcos. Os arcos em \textcolor{blue}{azul}
		formam a \(r\)-arborescência de custo mínimo \(T\) (custo total = 7).}
\end{figure}

O algoritmo de Frank utiliza \textbf{potenciais} \(y: V \to \mathbb{R}\) associados aos vértices
para definir custos reduzidos nos arcos. Para cada arco \(a = (u,v) \in A\), o
\textbf{custo \(y\)-reduzido} é dado por
\[
	c_y(u,v) := c(u,v) - y(v),
\]
onde \(v\) é o vértice de destino do arco. Note que, diferentemente do procedimento de redução do capítulo anterior (onde subtraíamos o mínimo custo de entrada em cada vértice), aqui o potencial \(y(v)\) é uma variável que será ajustada pelo algoritmo através da seguinte noção.

Um arco \(a = (u,v)\) é dito \textbf{justo} (ou \textbf{tight}) se \(c_y(u,v) = 0\), isto é, se o custo original coincide com o potencial do vértice de destino: \(c(u,v) = y(v)\). A ideia central do algoritmo de Frank consiste em duas etapas: (i)~elevar os potenciais \(y(v)\) para cada \(v \neq r\) até que cada vértice possua ao menos um arco justo entrando nele; (ii)~construir uma arborescência ótima utilizando exclusivamente arcos justos.

\begin{tcolorbox}[
		enhanced, breakable,
		colback=blue!5!white, colframe=blue!70!black,
		colbacktitle=blue!85!black, coltitle=white,
		fonttitle=\bfseries,
		title={Arcos justos: interpretação e cálculo},
		sharp corners, boxrule=0.8pt
	]
	Um arco \((u,v)\) é \textbf{justo} quando seu custo reduzido é zero, ou seja, quando o custo original \(c(u,v)\) coincide exatamente com o potencial \(y(v)\) do vértice de destino. Formalmente:
	\[
		c_y(u,v) = c(u,v) - y(v) = 0 \quad\Longleftrightarrow\quad c(u,v) = y(v).
	\]

	Em outras palavras, o potencial \(y(v)\) funciona como um ``desconto'' aplicado aos arcos que chegam em \(v\). Quando esse desconto é suficiente para anular o custo reduzido, o arco torna-se justo e candidato a participar da solução ótima.

	\medskip
	\textbf{Exemplos numéricos (referência: figura~26):}
	\begin{itemize}\setlength{\itemsep}{3pt}
		\item Arco \((r,a)\): \(c(r,a) = 1\), \(y(a) = 1 \implies c_y(r,a) = 1-1 = 0\). \quad\textcolor{blue!70!black}{\textbf{Justo.}}
		\item Arco \((b,a)\): \(c(b,a) = 4\), \(y(a) = 1 \implies c_y(b,a) = 4-1 = 3\). \quad\textcolor{red!70!black}{\textbf{Não justo.}}
		\item Arco \((a,c)\): \(c(a,c) = 1\), \(y(c) = 1 \implies c_y(a,c) = 1-1 = 0\). \quad\textcolor{blue!70!black}{\textbf{Justo.}}
	\end{itemize}

	\textbf{Propriedade fundamental:} o potencial \(y(v)\) afeta \emph{exclusivamente} os arcos que entram em \(v\). Por exemplo, \(y(c)\) influencia apenas arcos da forma \((\cdot, c)\), como \((r,c)\) e \((a,c)\), não afetando arcos que saem de \(c\).

	\medskip
	O algoritmo eleva progressivamente os potenciais até que cada vértice \(v \neq r\) possua ao menos um arco justo de entrada. No exemplo da figura~26, os arcos justos após a elevação correspondem precisamente aos arcos da arborescência ótima.
\end{tcolorbox}

% Figura 27: Exemplo visual com múltiplos arcos e vértice raiz
\begin{figure}[H]
	\centering
	\begin{tikzpicture}[>=Stealth]
		% estilos
		\tikzset{
			vtx/.style={circle, draw, fill=blue!8, thick, inner sep=1.2pt, minimum size=8mm},
			arc/.style={->, very thick, draw=blue},
			arcTight/.style={->, very thick, draw=red!70!black},
			cost/.style={font=\scriptsize, fill=white, inner sep=1pt, text=blue},
			pot/.style={font=\small, text=red!70!black},
			label/.style={font=\small, text=black!80}
		}

		% vértices
		\node[vtx] (r) at (0,0) {$r$};
		\node[vtx] (u) at (3,1.5) {$u$};
		\node[vtx] (v) at (3,-1.5) {$v$};

		% arcos
		\draw[arcTight] (r) -- node[cost, above left] {$c(r,u)=2$} (u);
		\draw[arc] (r) -- node[cost, below left] {$c(r,v)=4$} (v);
		\draw[arcTight] (u) -- node[cost, right] {$c(u,v)=3$} (v);

		% potenciais
		\node[pot] at (3,2.5) {$y(u)=2$};
		\node[pot] at (3,-2.5) {$y(v)=3$};

		% destaques
		\draw[red!70, thick, dashed] (u) circle [radius=0.7];
		\draw[red!70, thick, dashed] (v) circle [radius=0.7];

		% custos reduzidos
		\node[label] at (-0.0,1.5) {$c_y(r,u)=2-2=0$};
		\node[label] at (-0.0,-1.5) {$c_y(r,v)=4-3=1$};
		\node[label] at (5.0,-0.5) {$c_y(u,v)=3-3=0$};
	\end{tikzpicture}
	\caption{O vértice $r$ é a raiz. Os potenciais são $y(u)=2$ e $y(v)=3$. O arco $(r,u)$ tem custo reduzido $0$ (justo), $(r,v)$ tem custo reduzido $1$ (não justo), $(u,v)$ tem custo reduzido $0$ (justo). Os potenciais só afetam os arcos que chegam em cada vértice.}
	\label{fig:potencial-destino}
\end{figure}

\subsection*{Elevação de Potenciais}

Inicialmente, definimos \(y(v) = 0\) para todo \(v \in V\).
Para cada vértice \(v \neq r\), calculamos
\[
	\Delta(v) := \min\{\, c(a) : a \in \delta^{-}(v) \,\} - y(v).
\]
O valor \(\Delta(v)\) representa quanto precisamos elevar \(y(v)\) para que exista
ao menos um arco de custo reduzido zero entrando em \(v\).

Vamos ilustrar esse processo no dígrafo \(D\).

\begin{figure}[H]
	\centering
	\begin{tikzpicture}[>=Stealth]
		% estilos
		\tikzset{
			vtx/.style={circle, draw,fill=blue!8, thick, inner sep=1.2pt, minimum size=6mm},
			arc/.style={->, semithick, draw=black!35},
			costG/.style={font=\scriptsize, fill=white, inner sep=1pt, text=black!55}
		}

		% layout
		\node[vtx] (r) at (0,0) {$r$};
		\node[vtx] (a) at (3,1.6) {$a$};
		\node[vtx] (b) at (3,-1.6) {$b$};
		\node[vtx] (c) at (6,2.2) {$c$};
		\node[vtx] (d) at (6,0.0) {$d$};
		\node[vtx] (e) at (6,-2.2) {$e$};

		% arcos com custos originais
		\draw[arc] (r) -- node[costG, above] {1} (a);
		\draw[arc] (r) -- node[costG, below] {2} (b);
		\draw[arc] (r) .. controls (0.8,2.4) and (5.2,3.0) .. node[costG, yshift=4pt, pos=0.53] {3} (c);
		\draw[arc] (b) to[bend left=12] node[costG, above] {4} (a);
		\draw[arc] (a) to[bend left=14] node[costG, below] {5} (b);
		\draw[arc] (a) -- node[costG, above] {1} (c);
		\draw[arc] (a) -- node[costG, right] {3} (d);
		\draw[arc] (b) -- node[costG, above] {1} (d);
		\draw[arc] (b) -- node[costG, below] {2} (e);
		\draw[arc] (c) .. controls (7.6,2.6) and (7.6,-2.6) .. node[costG, xshift=8pt, pos=0.52] {5} (e);

		% potenciais iniciais
		\node[above=6mm of a, font=\small] {$y(a) = 0$};
		\node[below=2mm of b, font=\small] {$y(b) = 0$};
		\node[above=2mm of c, font=\small] {$y(c) = 0$};
		\node[below=2mm of d, font=\small] {$y(d) = 0$};
		\node[below=2mm of e, font=\small] {$y(e) = 0$};
	\end{tikzpicture}
	\caption{Dígrafo \(D\) com potenciais iniciais \(y(v) = 0\) para todo \(v \in V\).}
\end{figure}

Calculamos \(\Delta(v)\) para cada vértice \(v \neq r\):
\begin{align*}
	\Delta(a) & = \min\{1, 4\} - 0 = 1, \\
	\Delta(b) & = \min\{2, 5\} - 0 = 2, \\
	\Delta(c) & = \min\{3, 1\} - 0 = 1, \\
	\Delta(d) & = \min\{3, 1\} - 0 = 1, \\
	\Delta(e) & = \min\{2, 5\} - 0 = 2.
\end{align*}

Elevamos cada potencial:
\begin{align*}
	y(a) & \leftarrow y(a) + \Delta(a) = 0 + 1 = 1, \\
	y(b) & \leftarrow y(b) + \Delta(b) = 0 + 2 = 2, \\
	y(c) & \leftarrow y(c) + \Delta(c) = 0 + 1 = 1, \\
	y(d) & \leftarrow y(d) + \Delta(d) = 0 + 1 = 1, \\
	y(e) & \leftarrow y(e) + \Delta(e) = 0 + 2 = 2.
\end{align*}

Após a elevação, os custos reduzidos são dados por \(c_y(u,v) = c(u,v) - y(v)\) para cada arco \((u,v)\):

\begin{figure}[H]
	\centering
	\begin{tikzpicture}[>=Stealth]
		% estilos
		\tikzset{
			vtx/.style={circle, draw,fill=blue!8, thick, inner sep=1.2pt, minimum size=6mm},
			tight/.style={->, very thick, draw=blue},
			loose/.style={->, semithick, draw=black!35},
			costB/.style={font=\scriptsize, fill=white, inner sep=1pt, text=blue},
			costG/.style={font=\scriptsize, fill=white, inner sep=1pt, text=black!55}
		}

		% layout
		\node[vtx] (r) at (0,0) {$r$};
		\node[vtx] (a) at (3,1.6) {$a$};
		\node[vtx] (b) at (3,-1.6) {$b$};
		\node[vtx] (c) at (6,2.2) {$c$};
		\node[vtx] (d) at (6,0.0) {$d$};
		\node[vtx] (e) at (6,-2.2) {$e$};

		% arcos justos (custo reduzido = 0)
		\draw[tight] (r) -- node[costB, above] {0} (a);   % 1 - 1 = 0
		\draw[tight] (r) -- node[costB, below] {0} (b);   % 2 - 2 = 0
		\draw[tight] (a) -- node[costB, above] {0} (c);   % 1 - 1 = 0
		\draw[tight] (b) -- node[costB, above] {0} (d);   % 1 - 1 = 0
		\draw[tight] (b) -- node[costB, below] {0} (e);   % 2 - 2 = 0

		% arcos não justos
		\draw[loose] (r) .. controls (0.8,2.4) and (5.2,3.0) .. node[costG, yshift=4pt, pos=0.53] {2} (c);  % 3 - 1 = 2
		\draw[loose] (b) to[bend left=12] node[costG, above] {3} (a);   % 4 - 1 = 3
		\draw[loose] (a) to[bend left=14] node[costG, below] {3} (b);   % 5 - 2 = 3
		\draw[loose] (a) -- node[costG, right] {2} (d);   % 3 - 1 = 2
		\draw[loose] (c) .. controls (7.6,2.6) and (7.6,-2.6) .. node[costG, xshift=8pt, pos=0.52] {3} (e);  % 5 - 2 = 3

		% potenciais finais
		\node[above=6mm of a, font=\small] {$y(a) = 1$};
		\node[below=2mm of b, font=\small] {$y(b) = 2$};
		\node[above=2mm of c, font=\small] {$y(c) = 1$};
		\node[below=2mm of d, font=\small] {$y(d) = 1$};
		\node[below=2mm of e, font=\small] {$y(e) = 2$};
	\end{tikzpicture}
	\caption{Dígrafo após elevação de potenciais. Arcos em \textcolor{blue}{azul} são
		justos (\(c_y = 0\)); arcos em cinza têm custo reduzido positivo.}
\end{figure}

Observe que cada vértice \(v \neq r\) possui agora ao menos um arco justo entrando nele.
Mais ainda: neste exemplo, os arcos justos formam uma \(r\)-arborescência! Precisamos justificar por que a arborescência formada pelos arcos justos é ótima.

\subsection*{Construção da Arborescência}

Se os arcos justos formam uma \(r\)-arborescência, então encontramos a solução ótima.
Caso contrário, os arcos justos contêm ciclos, e precisamos tratá-los adequadamente.

No nosso exemplo, os arcos justos são:
\[
	H = \{\, (r,a), (r,b), (a,c), (b,d), (b,e) \,\}.
\]
Esses arcos formam a \(r\)-arborescência \(T\) de custo mínimo.

Para verificar a otimalidade, note que uma propriedade fundamental dos potenciais duais é:

\begin{proposicao}
	Para toda função \(y: V \to \mathbb{R}\) com \(y(r) = 0\),
	uma \(r\)-arborescência \(T\) é \(c\)-mínima em \(D\) se, e somente se,
	todos os arcos de \(T\) são justos (i.e., \(c_y(a) = 0\) para todo \(a \in T\))
	e \(c_y(a) \geq 0\) para todo \(a \in A\).
\end{proposicao}

\begin{proof}[Prova.]
	Seja \(F\) uma \(r\)-arborescência qualquer. Para cada \(v \in V \setminus \{r\}\),
	seja \(a_v\) o único arco de \(F\) que entra em \(v\). Então
	\begin{align*}
		c(F)
		 & = \sum_{v \in V \setminus \{r\}} c(a_v)                                          \\
		 & = \sum_{v \in V \setminus \{r\}} \bigl(c_y(a_v) + y(v)\bigr)                     \\
		 & = \sum_{v \in V \setminus \{r\}} c_y(a_v) + \sum_{v \in V \setminus \{r\}} y(v).
	\end{align*}
	Como a soma \(\sum_{v \in V \setminus \{r\}} y(v)\) é constante para todas as
	\(r\)-arborescências, minimizar \(c(F)\) equivale a minimizar \(\sum_{v} c_y(a_v)\).
	Se todos os arcos de \(T\) são justos e \(c_y(a) \geq 0\) para todo \(a\),
	então \(\sum_{v} c_y(a_v^T) = 0 \leq \sum_{v} c_y(a_v^F)\) para qualquer \(F\),
	provando que \(T\) é ótima.
\end{proof}

No nosso exemplo, todos os arcos de \(T\) são justos e \(c_y(a) \geq 0\) para todo
arco \(a \in A\), portanto \(T\) é a \(r\)-arborescência de custo mínimo. Evidentemente, nem sempre será o caso que os arcos justos formam uma arborescência e precisaremos tratar ciclos.

\subsection*{Tratamento de Ciclos}

Em geral, após a elevação de potenciais, os arcos justos podem conter ciclos.
Nesse caso, o algoritmo de Frank procede de forma similar ao de Chu--Liu--Edmonds:
contrai cada ciclo justo em um supervértice, resolve o problema recursivamente
no dígrafo contraído, e depois reexpande a solução.

Vamos ilustrar esse processo com um exemplo modificado. Suponha que adicionamos o arco
\((c, a)\) com custo 1 ao dígrafo original:

\begin{figure}[H]
	\centering
	\begin{tikzpicture}[>=Stealth]
		% estilos
		\tikzset{
			vtx/.style={circle, draw,fill=blue!8, thick, inner sep=1.2pt, minimum size=6mm},
			arc/.style={->, semithick, draw=black!35},
			costG/.style={font=\scriptsize, fill=white, inner sep=1pt, text=black!55}
		}

		% layout
		\node[vtx] (r) at (0,0) {$r$};
		\node[vtx] (a) at (3,1.6) {$a$};
		\node[vtx] (b) at (3,-1.6) {$b$};
		\node[vtx] (c) at (6,2.2) {$c$};
		\node[vtx] (d) at (6,0.0) {$d$};
		\node[vtx] (e) at (6,-2.2) {$e$};

		% arcos originais
		\draw[arc] (r) -- node[costG, above] {1} (a);
		\draw[arc] (r) -- node[costG, below] {2} (b);
		\draw[arc] (r) .. controls (0.8,2.4) and (5.2,3.0) .. node[costG, yshift=4pt, pos=0.53] {3} (c);
		\draw[arc] (b) to[bend left=12] node[costG, above] {4} (a);
		\draw[arc] (a) to[bend left=14] node[costG, below] {5} (b);
		\draw[arc] (a) -- node[costG, above] {1} (c);
		\draw[arc] (a) -- node[costG, right] {3} (d);
		\draw[arc] (b) -- node[costG, above] {1} (d);
		\draw[arc] (b) -- node[costG, below] {2} (e);
		\draw[arc] (c) .. controls (7.6,2.6) and (7.6,-2.6) .. node[costG, xshift=8pt, pos=0.52] {5} (e);

		% NOVO arco (c, a)
		\draw[arc, draw=red!60] (c) to[bend left=20] node[costG, left, text=red!60] {1} (a);
	\end{tikzpicture}
	\caption{Dígrafo modificado com o arco \((c,a)\) de custo 1 (em vermelho).}
\end{figure}

Após a elevação de potenciais com \(y(a) = 1, y(c) = 1\), o arco \((c,a)\) torna-se
justo: \(c_y(c,a) = 1 - 1 = 0\).

\begin{figure}[H]
	\centering
	\begin{tikzpicture}[>=Stealth]
		% estilos
		\tikzset{
			vtx/.style={circle, draw,fill=blue!8, thick, inner sep=1.2pt, minimum size=6mm},
			tight/.style={->, very thick, draw=blue},
			cycle/.style={->, ultra thick, draw=red!70},
			loose/.style={->, semithick, draw=black!35},
			costB/.style={font=\scriptsize, fill=white, inner sep=1pt, text=blue},
			costR/.style={font=\scriptsize, fill=white, inner sep=1pt, text=red!70},
			costG/.style={font=\scriptsize, fill=white, inner sep=1pt, text=black!55}
		}

		% layout
		\node[vtx] (r) at (0,0) {$r$};
		\node[vtx] (a) at (3,1.6) {$a$};
		\node[vtx] (b) at (3,-1.6) {$b$};
		\node[vtx] (c) at (6,2.2) {$c$};
		\node[vtx] (d) at (6,0.0) {$d$};
		\node[vtx] (e) at (6,-2.2) {$e$};

		% arcos justos que formam ciclo
		\draw[cycle] (a) -- node[costR, above] {0} (c);
		\draw[cycle] (c) to[bend left=20] node[costR, left] {0} (a);

		% outros arcos justos
		\draw[tight] (r) -- node[costB, above] {0} (a);
		\draw[tight] (r) -- node[costB, below] {0} (b);
		\draw[tight] (b) -- node[costB, above] {0} (d);
		\draw[tight] (b) -- node[costB, below] {0} (e);

		% arcos não justos
		\draw[loose] (r) .. controls (0.8,2.4) and (5.2,3.0) .. node[costG, yshift=4pt, pos=0.53] {2} (c);
		\draw[loose] (b) to[bend left=12] node[costG, above] {3} (a);
		\draw[loose] (a) to[bend left=14] node[costG, below] {3} (b);
		\draw[loose] (a) -- node[costG, right] {2} (d);   % 3 - 1 = 2
		\draw[loose] (c) .. controls (7.6,2.6) and (7.6,-2.6) .. node[costG, xshift=8pt, pos=0.52] {3} (e);  % 5 - 2 = 3
	\end{tikzpicture}
	\caption{Arcos justos após elevação. O ciclo \(C = (a, c, a)\) é destacado em
		\textcolor{red!70}{vermelho}.}
\end{figure}

Nesse caso, contraímos o ciclo \(C = (a, c)\) em um supervértice \(x_C\), ajustamos
os custos dos arcos incidentes, e resolvemos o problema no dígrafo contraído.
A reexpansão é feita removendo um arco do ciclo para manter a estrutura de arborescência,
de forma análoga ao algoritmo de Chu--Liu--Edmonds.

\section{Descrição do algoritmo}

O algoritmo de András Frank difere substancialmente do algoritmo de Chu--Liu--Edmonds em sua abordagem: enquanto Chu--Liu--Edmonds opera diretamente sobre os custos dos arcos, o método de Frank utiliza uma técnica gulosa--dual baseada em \emph{potenciais} associados aos vértices. A ideia central é elevar progressivamente os potenciais \(y(v)\) para cada vértice \(v\neq r\), de modo que os custos \emph{reduzidos} \(c_y(u,v) := c(u,v) - y(v)\) revelem quais arcos devem participar da solução ótima. Um arco \((u,v)\) é dito \emph{justo} quando seu custo reduzido é zero, isto é, quando \(c(u,v) = y(v)\). O algoritmo garante que toda r-arborescência ótima pode ser formada apenas por arcos justos.

A seguir apresentamos uma descrição formal do algoritmo de András Frank e detalhes de implementação serão discutidos na próxima seção.

\begin{algobox}{András Frank}{andras-frank}
	Entrada: dígrafo \(D=(V,A)\), custos \(c:A\to\mathbb{R}_{\ge 0}\), raiz \(r\).\footnote{Se algum \(v\neq r\) não possui arco de entrada, não existe r-arborescência.}

	\textbf{Fase 1: Elevação de potenciais e construção de \(A_0\)}
	\begin{enumerate}\setlength{\itemsep}{2pt}
		\item \textbf{Inicialização:} defina \(y(v) := 0\) para todo \(v \in V\).
		\item \textbf{Iteração:} enquanto existir conjunto \(X \subseteq V \setminus \{r\}\) sem arco justo entrando:
		      \begin{itemize}\setlength{\itemsep}{2pt}
			      \item Calcule \(\Delta(X) := \min\{c(u,v) - y(v) : u \notin X, v \in X\}\).
			      \item Para cada \(v \in X\), atualize \(y(v) := y(v) + \Delta(X)\).
		      \end{itemize}
		\item Defina \(A_0 := \{a \in A : c_y(a) = 0\}\), o conjunto de arcos justos.
	\end{enumerate}

	\textbf{Fase 2: Construção da arborescência}
	\begin{enumerate}\setlength{\itemsep}{2pt}\setcounter{enumi}{3}
		\item Se \((V,A_0)\) forma uma r-arborescência, devolva \(A_0\). Por otimalidade dos potenciais duais, trata-se de uma r-arborescência de custo mínimo.
		\item Caso contrário, identifique um ciclo dirigido \(C\) em \(A_0\) (com \(r\notin C\)). \textbf{Contração:} contraia \(C\) em um supervértice \(x_C\) e defina custos reduzidos \(c'\) por
		      \begin{align*}
			      c'(u,x_C) & := c_y(u,w) = c(u,w) - y(w) &  & \text{para } u\notin C,\ w\in C, \\
			      c'(x_C,v) & := c_y(w,v) = c(w,v) - y(v) &  & \text{para } w\in C,\ v\notin C,
		      \end{align*}
		      descartando laços em \(x_C\) e permitindo paralelos. Denote o dígrafo contraído por \(D'=(V',A')\).
		\item \textbf{Recursão:} compute uma r-arborescência ótima \(T'\) de \(D'\) com custos \(c'\).
		\item \textbf{Expansão:} seja \((u,x_C)\in T'\) o único arco que entra em \(x_C\). No dígrafo original, ele corresponde a \((u,w)\) com \(w\in C\). Forme
		      \[
			      T := \bigl(T'\setminus\{\text{arcos incidentes a } x_C\}\bigr)\ \cup\ \{(u,w)\}\ \cup\ \bigl((A_0\cap A(C))\setminus\{a_w\}\bigr),
		      \]
		      onde \(a_w\) é o arco de \(C\) que entra em \(w\). Então \(T\) é uma r-arborescência ótima de \(D\).
	\end{enumerate}
\end{algobox}

\subsection{Corretude}

A corretude do algoritmo de András Frank baseia-se na teoria de dualidade em programação linear aplicada ao problema de arborescência de custo mínimo:

\begin{enumerate}\setlength{\itemsep}{2pt}
	\item \emph{Potenciais duais:} os valores \(y(v)\) funcionam como variáveis duais. Para qualquer r-arborescência \(T\) e qualquer escolha de potenciais \(y\), vale
	      \[
		      c(T) = \sum_{a\in T} c(a) = \sum_{a\in T} c_y(a) + \sum_{v\neq r} y(v),
	      \]
	      pois há exatamente um arco de \(T\) entrando em cada \(v\neq r\). O termo \(\sum_{v\neq r} y(v)\) é constante após a elevação completa dos potenciais; assim, minimizar \(c(T)\) equivale a minimizar \(\sum_{a\in T} c_y(a)\).

	\item \emph{Condições de otimalidade:} uma r-arborescência \(T\) é ótima se e somente se:
	      \begin{itemize}\setlength{\itemsep}{2pt}
		      \item Todos os arcos de \(T\) são justos: \(c_y(a) = 0\) para todo \(a \in T\).
		      \item Todos os arcos do dígrafo têm custo reduzido não-negativo: \(c_y(a) \geq 0\) para todo \(a \in A\).
	      \end{itemize}
	      Sob essas condições, \(\sum_{a\in T} c_y(a) = 0 \leq \sum_{a\in T'} c_y(a)\) para qualquer outra r-arborescência \(T'\), provando otimalidade.

	\item \emph{Elevação de potenciais:} o algoritmo eleva os potenciais de forma gulosa, garantindo que ao final cada \(v\neq r\) tenha ao menos um arco justo entrando. Isso é feito identificando componentes fortemente conexas em \(A\_0\): conjuntos \(X\) que ainda não possuem arcos justos entrando têm seus potenciais elevados pelo mínimo necessário \(\Delta(X)\).

	\item \emph{Caso acíclico:} se \(A_0\) forma uma r-arborescência, todos os arcos são justos e, pelas condições de otimalidade, a solução é ótima.

	\item \emph{Caso com ciclo (contração/expansão):} se \(A\_0\) contém um ciclo dirigido \(C\), todos os seus arcos são justos (custo reduzido zero). A contração preserva a estrutura de custos reduzidos: no dígrafo contraído, os novos custos \(c'\) já incorporam os potenciais, e uma solução ótima em \(D'\) mapeia para uma solução ótima em \(D\) pela reversibilidade da operação.
\end{enumerate}

Em termos intuitivos, os potenciais \(y(v)\) funcionam como "descontos" aplicados aos arcos que entram em cada vértice, revelando quais arcos são "justos" (custo líquido zero) e, portanto, candidatos à solução ótima. Ciclos de arcos justos podem ser contraídos sem perder otimalidade, simplificando o problema recursivamente.

\subsection{Complexidade}

A implementação, baseada em componentes fortemente conexas, detecta em cada iteração, quais conjuntos \(X\) necessitam elevação de potenciais. Calcular componentes fortemente conexas custa \(O(n+m)\) usando algoritmos como Tarjan ou Kosaraju. Para cada componente (exceto a raiz), eleva-se o potencial calculando \(\Delta(X)\) em \(O(m)\), atualizando os custos reduzidos.

No pior caso, cada iteração reduz o número de componentes em pelo menos uma unidade, resultando em \(O(n)\) iterações. Cada iteração processa todos os arcos para atualizar custos reduzidos e recalcular componentes, resultando em \(O(nm)\) no total para a Fase~1. A Fase~2 constrói a arborescência percorrendo \(A_0\) uma vez, custando \(O(n)\).

O uso de memória é \(O(n+m)\), incluindo as estruturas para armazenar o dígrafo, potenciais e componentes. A implementação a seguir adota a versão \(O(nm)\) por simplicidade e está disponível no repositório do projeto (\url{https://github.com/lorenypsum/GraphVisualizer}).

\section{Implementação em Python}

Esta seção descreve a implementação do algoritmo de András Frank em Python, estruturada para refletir com precisão as duas fases formais discutidas anteriormente. A Fase~1 realiza a elevação de potenciais e identifica os arcos justos, enquanto a Fase~2 constrói a arborescência de custo mínimo a partir desses arcos. Utilizamos a biblioteca NetworkX para manipulação de digrafos, aproveitando suas funcionalidades para representar grafos, calcular componentes fortemente conexas e gerenciar atributos de arcos.

A entrada consiste em um dígrafo orientado \(D = (V, A)\), com custos dos arcos registrados no atributo \texttt{"w"}, e uma raiz \(r \in V\). As hipóteses adotadas são: (i) o dígrafo é conexo a partir de \(r\), isto é, todo vértice \(v \neq r\) é alcançável a partir da raiz; (ii) para todo subconjunto \(X \subseteq V \setminus \{r\}\), existe ao menos um arco entrando em \(X\); e (iii) todos os custos são não negativos.

A saída é um subdigrafo \(T\) de \(D\) com \(|A_T| = |V| - 1\) arcos, tal que cada vértice \(v \neq r\) possui grau de entrada igual a 1, todos os vértices são alcançáveis a partir de \(r\), e o custo total \(\sum_{a \in A_T} c(a)\) é mínimo.

A estrutura do código é modular: funções auxiliares tratam cada etapa do algoritmo — cálculo de componentes fortemente conexas, elevação de potenciais, construção do subdigrafo \(A_0\) e construção da arborescência final. Todas operam sobre objetos \texttt{nx.DiGraph} e são coordenadas por uma função principal que gerencia o fluxo das duas fases. As subseções seguintes detalham cada função auxiliar, abordando lógica, parâmetros, saídas e complexidade.

\subsection{Construção do dígrafo \texorpdfstring{\(D_0\)}{D_0} inicial}

Começamos escrevendo uma função que constrói o dígrafo inicial \(D_0\) que será utilizado na Fase~1 do algoritmo. O dígrafo \(D_0\) é inicializado como um grafo vazio contendo apenas os vértices do dígrafo original, sem arcos. Essa estrutura será gradualmente populada com arcos de custo reduzido zero à medida que os potenciais são elevados.

Recebe como entrada um dígrafo \texttt{D} (objeto \texttt{nx.DiGraph}). A implementação cria um novo dígrafo vazio \texttt{D\_zero} (linha 2) e adiciona todos os vértices de \texttt{D} a \texttt{D\_zero} (linhas 3-4), preservando a estrutura de vértices sem incluir arcos inicialmente.

A função devolve o dígrafo \texttt{D\_zero} contendo todos os vértices de \texttt{D} mas nenhum arco. O dígrafo original \texttt{D} não é modificado. A complexidade é \(O(n)\), onde \(n = |V|\), pois itera sobre todos os vértices uma única vez.

\begin{tcolorbox}[
		enhanced, breakable,
		colframe=blue!60!black, colback=blue!2,
		colbacktitle=blue!15, coltitle=black,
		title={Construção do dígrafo $D_0$ inicial},
		boxed title style={sharp corners, boxrule=0.6pt},
		sharp corners, boxrule=0.6pt
	]
	\emph{Constrói um dígrafo vazio contendo apenas os vértices de D, sem arcos. Este dígrafo será populado com arcos justos durante a elevação de potenciais.}
	\tcblower
	\begin{lstlisting}[language=Python]
def build_D_zero(D):
    D_zero = nx.DiGraph()
    for v in D.nodes():
        D_zero.add_node(v)
    return D_zero
\end{lstlisting}
\end{tcolorbox}

\subsection{Identificação de arcos entrando em conjunto \texorpdfstring{\(X\)}{X}}

Esta função auxiliar identifica todos os arcos que entram em um conjunto \(X \subseteq V\), isto é, arcos \((u,v)\) tais que \(u \notin X\) e \(v \in X\). Essa operação é fundamental para calcular o mínimo custo de entrada em \(X\) durante a elevação de potenciais.

Recebe como entrada um dígrafo \texttt{D} e um conjunto de vértices \texttt{X}. A implementação cria uma lista vazia \texttt{arcs} (linha 2) e itera sobre todos os arcos do dígrafo com seus dados (linha 3), incluindo o peso. Para cada arco \((u,v,\text{data})\), verifica se \(u \notin X\) e \(v \in X\) (linha 4), adicionando à lista apenas os arcos que cruzam a fronteira de \(X\) (linha 5).

A função devolve uma lista de tuplas \((u, v, \text{data})\) representando os arcos que entram em \(X\), onde \texttt{data} contém o atributo \texttt{"w"} com o peso do arco. A complexidade é \(O(m)\), onde \(m = |A|\), pois examina cada arco uma vez.

\begin{tcolorbox}[
		enhanced, breakable,
		colframe=blue!60!black, colback=blue!2,
		colbacktitle=blue!15, coltitle=black,
		title={Identificação de arcos entrando em conjunto X},
		boxed title style={sharp corners, boxrule=0.6pt},
		sharp corners, boxrule=0.6pt
	]
	\emph{Identifica todos os arcos $(u,v)$ do dígrafo D tais que $u \notin X$ e $v \in X$, devolvendo uma lista com as tuplas $(u, v, data)$ onde data contém o peso do arco.}
	\tcblower
	\begin{lstlisting}[language=Python]
def get_arcs_entering_X(D, X):
    arcs = []
    for u, v, data in D.edges(data=True):
        if u not in X and v in X:
            arcs.append((u, v, data))
    return arcs
\end{lstlisting}
\end{tcolorbox}

A figura a seguir ilustra o funcionamento da função \texttt{get\_arcs\_entering\_X} em um digrafo que vamos denotar por \(D_{32}\). O dígrafo possui uma raiz \(r_0\) conectada aos vértices \(u_1, u_2, u_3\). Os vértices em laranja pertencem ao conjunto \(X = \{v_1, v_2, v_3\}\), e a função identifica apenas os arcos em vermelho, que saem de vértices fora de \(X\) e entram em vértices dentro de \(X\). Arcos da raiz, arcos internos a \(X\), externos a \(X\), ou saindo de \(X\) não são retornados.

\begin{figure}[H]
	\centering
	\begin{tikzpicture}[>=Stealth]
		% estilos
		\tikzset{
			vtx/.style={circle, draw, fill=blue!8, thick, inner sep=1.2pt, minimum size=7mm},
			vtxRoot/.style={circle, draw, fill=red!15, thick, inner sep=1.2pt, minimum size=7mm},
			vtxX/.style={circle, draw=orange!80!black, fill=orange!15, thick, inner sep=1.2pt, minimum size=7mm},
			arc/.style={->, >=Stealth, thick, draw=black!40},
			arcEntering/.style={->, >=Stealth, very thick, draw=red!70!black},
			cost/.style={font=\scriptsize, fill=white, inner sep=1pt}
		}

		% Raiz
		\node[vtxRoot] (r0) at (2, 4) {$r_0$};

		% Vértices fora de X
		\node[vtx] (u1) at (0, 2) {$u_1$};
		\node[vtx] (u2) at (2, 2) {$u_2$};
		\node[vtx] (u3) at (4, 2) {$u_3$};

		% Vértices dentro de X (destacados)
		\node[vtxX] (v1) at (0.5, -0.5) {$v_1$};
		\node[vtxX] (v2) at (2, -0.5) {$v_2$};
		\node[vtxX] (v3) at (3.5, -0.5) {$v_3$};

		% Arcos da raiz (não retornados, pois r0 não está em consideração para X)
		\draw[arc] (r0) -- node[cost, left, pos=0.4] {1} (u1);
		\draw[arc] (r0) -- node[cost, right] {2} (u2);
		\draw[arc] (r0) -- node[cost, right, pos=0.4] {1} (u3);

		% Arcos externos a X (não retornados)
		\draw[arc] (u1) to[bend left=15] node[cost, above] {1} (u2);
		\draw[arc] (u2) to[bend left=15] node[cost, above] {3} (u3);

		% Arcos entrando em X (destacados em vermelho)
		\draw[arcEntering] (u1) -- node[cost, left] {3} (v1);
		\draw[arcEntering] (u2) -- node[cost, right] {2} (v2);
		\draw[arcEntering] (u3) -- node[cost, right] {4} (v3);
		\draw[arcEntering] (u1) to[bend right=20] node[cost, below, pos=0.3] {5} (v2);

		% Arcos internos a X (não retornados)
		\draw[arc] (v1) -- node[cost, above] {1} (v2);
		\draw[arc] (v2) -- node[cost, above] {2} (v3);

		% Arcos saindo de X (não retornados)
		\draw[arc] (v1) to[bend right=30] node[cost, left, pos=0.3] {2} (u1);

		% Região X destacada
		\draw[orange!60!black, thick, dashed, rounded corners=8pt]
		(-0.3, 0.3) rectangle (4.3, -1.3);
		\node[orange!80!black, font=\bfseries] at (5, 0.3) {$X$};
	\end{tikzpicture}
	\caption{Ilustração da função \texttt{get\_arcs\_entering\_X} em \(D_{32}\). A raiz \(r_0\) (em vermelho claro) conecta-se aos vértices \(u_1, u_2, u_3\). Os vértices em \textcolor{orange!80!black}{laranja} pertencem ao conjunto \(X = \{v_1, v_2, v_3\}\). A função identifica apenas os arcos \textcolor{red!70!black}{em vermelho}: aqueles que saem de vértices fora de \(X\) e entram em vértices dentro de \(X\). Arcos da raiz, arcos internos a \(X\), externos a \(X\), ou saindo de \(X\) não são retornados.}
\end{figure}

\subsection{Cálculo do peso mínimo de corte}

Esta função calcula o peso mínimo entre todos os arcos fornecidos, correspondendo ao valor \(\Delta(X)\) necessário para elevar os potenciais dos vértices em \(X\).

Recebe como entrada uma lista \texttt{arcos} de tuplas \((u, v, \text{data})\). A implementação usa a função \texttt{min} com uma compreensão de gerador (linha 2) que extrai o atributo \texttt{"w"} de cada tupla em \texttt{data}.

A função devolve o peso mínimo encontrado entre todos os arcos da lista. A complexidade é \(O(k)\), onde \(k\) é o número de arcos na lista, pois examina cada arco uma vez para encontrar o mínimo.

\begin{tcolorbox}[
		enhanced, breakable,
		colframe=blue!60!black, colback=blue!2,
		colbacktitle=blue!15, coltitle=black,
		title={Cálculo do peso mínimo de corte},
		boxed title style={sharp corners, boxrule=0.6pt},
		sharp corners, boxrule=0.6pt
	]
	\emph{Calcula o peso mínimo entre todos os arcos fornecidos, correspondendo ao valor $\Delta(X)$ usado na elevação de potenciais.}
	\tcblower
	\begin{lstlisting}[language=Python]
def get_minimum_weight_cut(arcs):
    return min(data["w"] for _, _, data in arcs)
\end{lstlisting}
\end{tcolorbox}

A seguir temos uma ilustração do funcionamento da função \texttt{get\_minimum\_weight\_cut} em \(D_{32}\). Considerando os arcos em vermelho que entram em \(X\) (identificados pela função anterior), esta função calcula o peso mínimo entre eles. O arco em verde possui o menor peso (2), correspondendo ao valor \(\Delta(X) = 2\) que será devolvido pela função.

\begin{figure}[H]
	\centering
	\begin{tikzpicture}[>=Stealth]
		% estilos
		\tikzset{
			vtx/.style={circle, draw, fill=blue!8, thick, inner sep=1.2pt, minimum size=7mm},
			vtxRoot/.style={circle, draw, fill=red!15, thick, inner sep=1.2pt, minimum size=7mm},
			vtxX/.style={circle, draw=orange!80!black, fill=orange!15, thick, inner sep=1.2pt, minimum size=7mm},
			arc/.style={->, >=Stealth, thick, draw=black!40},
			arcEntering/.style={->, >=Stealth, very thick, draw=red!70!black},
			arcMin/.style={->, >=Stealth, ultra thick, draw=green!60!black},
			cost/.style={font=\scriptsize, fill=white, inner sep=1pt},
			costMin/.style={font=\scriptsize\bfseries, fill=green!10, inner sep=2pt, draw=green!60!black, rounded corners=1pt}
		}

		% Raiz
		\node[vtxRoot] (r0) at (2, 4) {$r_0$};

		% Vértices fora de X
		\node[vtx] (u1) at (0, 2) {$u_1$};
		\node[vtx] (u2) at (2, 2) {$u_2$};
		\node[vtx] (u3) at (4, 2) {$u_3$};

		% Vértices dentro de X (destacados)
		\node[vtxX] (v1) at (0.5, -0.5) {$v_1$};
		\node[vtxX] (v2) at (2, -0.5) {$v_2$};
		\node[vtxX] (v3) at (3.5, -0.5) {$v_3$};

		% Arcos da raiz
		\draw[arc] (r0) -- node[cost, left, pos=0.4] {1} (u1);
		\draw[arc] (r0) -- node[cost, right] {2} (u2);
		\draw[arc] (r0) -- node[cost, right, pos=0.4] {1} (u3);

		% Arcos externos a X
		\draw[arc] (u1) to[bend left=15] node[cost, above] {1} (u2);
		\draw[arc] (u2) to[bend left=15] node[cost, above] {3} (u3);

		% Arcos entrando em X (em vermelho, mas o mínimo em verde)
		\draw[arcEntering] (u1) -- node[cost, left] {3} (v1);
		\draw[arcMin] (u2) -- node[costMin, right] {2} (v2);
		\draw[arcEntering] (u3) -- node[cost, right] {4} (v3);
		\draw[arcEntering] (u1) to[bend right=20] node[cost, below, pos=0.3] {5} (v2);

		% Arcos internos a X
		\draw[arc] (v1) -- node[cost, above] {1} (v2);
		\draw[arc] (v2) -- node[cost, above] {2} (v3);

		% Arcos saindo de X
		\draw[arc] (v1) to[bend right=30] node[cost, left, pos=0.3] {2} (u1);

		% Região X destacada
		\draw[orange!60!black, thick, dashed, rounded corners=8pt]
		(-0.3, 0.3) rectangle (4.3, -1.3);
		\node[orange!80!black, font=\bfseries] at (5, 0.3) {$X$};

		% Anotação do resultado
		\node[font=\bfseries, green!60!black] at (2, -2.2) {$\Delta(X) = \min\{3, 2, 4, 5\} = 2$};
	\end{tikzpicture}
	\caption{Ilustração da função \texttt{get\_minimum\_weight\_cut} em \(D_{32}\). Considerando os arcos \textcolor{red!70!black}{em vermelho} que entram em \(X\) (identificados pela função anterior), esta função calcula o peso mínimo entre eles. O arco \textcolor{green!60!black}{em verde} possui o menor peso (2), correspondendo ao valor \(\Delta(X) = 2\).}
\end{figure}

\subsection{Atualização de pesos em \texorpdfstring{\(X\)}{X}}

Esta função auxiliar atualiza os pesos dos arcos que entram em um conjunto \(X\), subtraindo o valor \(\Delta(X)\) de cada peso. Arcos que atingem peso zero são adicionados a \(A_0\) e a \(D_0\).

Recebe como entrada um dígrafo \texttt{D}, lista de \texttt{arcs} entrando em \(X\), o valor \texttt{min\_weight} a ser subtraído, uma lista \texttt{A\_zero} para armazenar arcos de peso zero, e o dígrafo \texttt{D\_zero} para adicionar os arcos justos.

A implementação itera sobre cada arco \((u,v,\_)\) da lista (linha 2), subtrai \texttt{min\_weight} do peso armazenado em \texttt{D[u][v]["w"]} (linha 3), e verifica se o peso resultante é zero (linha 4). Caso sim, adiciona-se \((u,v)\) à lista \texttt{A\_zero} (linha 5) e ao dígrafo \texttt{D\_zero} (linha 6).

A função não devolve valor, pois modifica diretamente as estruturas passadas como parâmetros: o dígrafo \texttt{D} tem seus pesos atualizados, \texttt{A\_zero} acumula arcos justos, e \texttt{D\_zero} é populado com esses arcos. A complexidade é \(O(k)\), onde \(k\) é o número de arcos em \texttt{arcs}.

\begin{tcolorbox}[
		enhanced, breakable,
		colframe=blue!60!black, colback=blue!2,
		colbacktitle=blue!15, coltitle=black,
		title={Atualização de pesos em X},
		boxed title style={sharp corners, boxrule=0.6pt},
		sharp corners, boxrule=0.6pt
	]
	\emph{Atualiza os pesos dos arcos que entram em X, subtraindo o valor mínimo. Arcos que atingem peso zero são registrados em $A_0$ e adicionados a $D_0$.}
	\tcblower
	\begin{lstlisting}[language=Python]
def update_weights_in_X(D, arcs, min_weight, A_zero, D_zero):
    for u, v, _ in arcs:
        D[u][v]["w"] -= min_weight
        if D[u][v]["w"] == 0:
            A_zero.append((u, v))
            D_zero.add_edge(u, v)
\end{lstlisting}
\end{tcolorbox}

A seguir ilustramos o funcionamento da função \texttt{update\_weights\_in\_X} em \(D_{32}\). A figura mostra o dígrafo antes e depois da atualização dos pesos. No estado inicial (esquerda), temos \(\Delta(X) = 2\). A função subtrai esse valor de todos os arcos que entram em \(X\). No estado final (direita), os arcos que atingiram peso zero (destacados em azul) são adicionados a \(A_0\) e \(D_0\).

\begin{figure}[H]
	\centering
	\begin{tikzpicture}[>=Stealth]
		% estilos
		\tikzset{
			vtx/.style={circle, draw, fill=blue!8, thick, inner sep=1.2pt, minimum size=6.5mm},
			vtxRoot/.style={circle, draw, fill=red!15, thick, inner sep=1.2pt, minimum size=6.5mm},
			vtxX/.style={circle, draw=orange!80!black, fill=orange!15, thick, inner sep=1.2pt, minimum size=6.5mm},
			arc/.style={->, >=Stealth, thick, draw=black!40},
			arcEntering/.style={->, >=Stealth, very thick, draw=red!70!black},
			arcTight/.style={->, >=Stealth, ultra thick, draw=blue!70!black},
			cost/.style={font=\scriptsize, fill=white, inner sep=1pt},
			costTight/.style={font=\scriptsize\bfseries, fill=blue!10, inner sep=2pt, draw=blue!70!black, rounded corners=1pt}
		}

		% ===== ANTES (esquerda) =====
		\begin{scope}
			\node[font=\bfseries, blue!80!black] at (2, 4.2) {Antes: pesos originais};

			% Raiz
			\node[vtxRoot] (r0) at (2, 3) {$r_0$};

			% Vértices fora de X
			\node[vtx] (u1) at (0, 1.5) {$u_1$};
			\node[vtx] (u2) at (2, 1.5) {$u_2$};
			\node[vtx] (u3) at (4, 1.5) {$u_3$};

			% Vértices dentro de X
			\node[vtxX] (v1) at (0.5, -0.5) {$v_1$};
			\node[vtxX] (v2) at (2, -0.5) {$v_2$};
			\node[vtxX] (v3) at (3.5, -0.5) {$v_3$};

			% Arcos da raiz
			\draw[arc] (r0) -- node[cost, left, pos=0.4] {1} (u1);
			\draw[arc] (r0) -- node[cost, right] {2} (u2);
			\draw[arc] (r0) -- node[cost, right, pos=0.4] {1} (u3);

			% Arcos externos a X
			\draw[arc] (u1) to[bend left=15] node[cost, above] {1} (u2);
			\draw[arc] (u2) to[bend left=15] node[cost, above] {3} (u3);

			% Arcos entrando em X
			\draw[arcEntering] (u1) -- node[cost, left] {3} (v1);
			\draw[arcEntering] (u2) -- node[cost, right] {2} (v2);
			\draw[arcEntering] (u3) -- node[cost, right] {4} (v3);
			\draw[arcEntering] (u1) to[bend right=20] node[cost, below, pos=0.3] {5} (v2);

			% Arcos internos a X
			\draw[arc] (v1) -- node[cost, above] {1} (v2);
			\draw[arc] (v2) -- node[cost, above] {2} (v3);

			% Arcos saindo de X
			\draw[arc] (v1) to[bend right=30] node[cost, left, pos=0.3] {2} (u1);

			% Região X
			\draw[orange!60!black, thick, dashed, rounded corners=8pt]
			(-0.3, 0.3) rectangle (4.3, -1.3);
			\node[orange!80!black, font=\bfseries\small] at (4.8, 0.1) {$X$};

			\node[font=\small, green!60!black] at (2, -2.2) {$\Delta(X) = 2$};
		\end{scope}

		% Seta indicando transformação
		\draw[->, ultra thick, blue!60!black] (5.3, 0.5) -- node[above, font=\small] {subtrai $\Delta(X)$} (6.7, 0.5);

		% ===== DEPOIS (direita) =====
		\begin{scope}[xshift=7.5cm]
			\node[font=\bfseries, blue!80!black] at (2, 4.2) {Depois: pesos atualizados};

			% Raiz
			\node[vtxRoot] (r0b) at (2, 3) {$r_0$};

			% Vértices fora de X
			\node[vtx] (u1b) at (0, 1.5) {$u_1$};
			\node[vtx] (u2b) at (2, 1.5) {$u_2$};
			\node[vtx] (u3b) at (4, 1.5) {$u_3$};

			% Vértices dentro de X
			\node[vtxX] (v1b) at (0.5, -0.5) {$v_1$};
			\node[vtxX] (v2b) at (2, -0.5) {$v_2$};
			\node[vtxX] (v3b) at (3.5, -0.5) {$v_3$};

			% Arcos da raiz (inalterados)
			\draw[arc] (r0b) -- node[cost, left, pos=0.4] {1} (u1b);
			\draw[arc] (r0b) -- node[cost, right] {2} (u2b);
			\draw[arc] (r0b) -- node[cost, right, pos=0.4] {1} (u3b);

			% Arcos externos a X (inalterados)
			\draw[arc] (u1b) to[bend left=15] node[cost, above] {1} (u2b);
			\draw[arc] (u2b) to[bend left=15] node[cost, above] {3} (u3b);

			% Arcos entrando em X (atualizados)
			\draw[arcEntering] (u1b) -- node[cost, left] {1} (v1b);
			\draw[arcTight] (u2b) -- node[costTight, right] {0} (v2b);
			\draw[arcEntering] (u3b) -- node[cost, right] {2} (v3b);
			\draw[arcEntering] (u1b) to[bend right=20] node[cost, below, pos=0.3] {3} (v2b);

			% Arcos internos a X (inalterados)
			\draw[arc] (v1b) -- node[cost, above] {1} (v2b);
			\draw[arc] (v2b) -- node[cost, above] {2} (v3b);

			% Arcos saindo de X (inalterados)
			\draw[arc] (v1b) to[bend right=30] node[cost, left, pos=0.3] {2} (u1b);

			% Região X
			\draw[orange!60!black, thick, dashed, rounded corners=8pt]
			(-0.3, 0.3) rectangle (4.3, -1.3);
			\node[orange!80!black, font=\bfseries\small] at (4.8, 0.1) {$X$};

			\node[font=\small, blue!70!black] at (2, -2.2) {Arco \textcolor{blue!70!black}{justo} $\to A_0$};
		\end{scope}
	\end{tikzpicture}
	\caption{Ilustração da função \texttt{update\_weights\_in\_X} em \(D_{32}\). À esquerda, o dígrafo antes da atualização, com os arcos \textcolor{red!70!black}{em vermelho} entrando em \(X\) e \(\Delta(X) = 2\). À direita, após subtrair \(\Delta(X)\) de cada arco entrando em \(X\): o peso $(u_1, v_1)$ reduz de 3 para 1, $(u_2, v_2)$ de 2 para \textcolor{blue!70!black}{\textbf{0}} (torna-se justo), $(u_3, v_3)$ de 4 para 2, e $(u_1, v_2)$ de 5 para 3. O arco justo é adicionado a \(A_0\) e \(D_0\). Note que os arcos da raiz e arcos internos/externos a \(X\) permanecem inalterados.}
\end{figure}

\subsection{Verificação de arborescência}

Esta função verifica se um dígrafo \(D\) contém uma r-arborescência com raiz \(r_0\). Utiliza busca em profundidade (DFS) a partir da raiz para verificar se todos os vértices são alcançáveis.

Recebe como entrada um dígrafo \texttt{D} e a raiz \texttt{r0}. A implementação constrói uma árvore DFS a partir de \texttt{r0} usando \texttt{nx.dfs\_tree} (linha 2), que devolve um subdigrafo contendo apenas os vértices alcançáveis a partir da raiz seguindo arcos. Em seguida, compara o número de vértices da árvore DFS com o número total de vértices de \texttt{D} (linha 3).

A função devolve \texttt{True} se todos os vértices são alcançáveis (indicando presença de r-arborescência), \texttt{False} caso contrário. A complexidade é \(O(n+m)\), onde \(n=|V|\) e \(m=|A|\), devido à busca em profundidade. Precisamos dessa verificação para garantir que a Fase~1 produza um dígrafo \(D_0\) que contenha uma r-arborescência antes de prosseguir para a Fase~2.

\begin{tcolorbox}[
		enhanced, breakable,
		colframe=blue!60!black, colback=blue!2,
		colbacktitle=blue!15, coltitle=black,
		title={Verificação de arborescência},
		boxed title style={sharp corners, boxrule=0.6pt},
		sharp corners, boxrule=0.6pt
	]
	\emph{Verifica se o dígrafo D contém uma r-arborescência com raiz r0, usando busca em profundidade para testar alcançabilidade de todos os vértices.}
	\tcblower
	\begin{lstlisting}[language=Python]
def has_arborescence(D, r0):
    tree = nx.dfs_tree(D, r0)
    return tree.number_of_nodes() == D.number_of_nodes()
\end{lstlisting}
\end{tcolorbox}

\subsection{Fase 1: Elevação de potenciais e construção de \texorpdfstring{\(A_0\)}{A_0}}

A seguir apresentamos a função principal da Fase~1, responsável por elevar os potenciais dos vértices iterativamente até que cada conjunto de vértices possua ao menos um arco justo entrando. O processo utiliza componentes fortemente conexas para identificar quais conjuntos necessitam elevação.

Recebe como entrada um dígrafo \texttt{D\_original} e a raiz \texttt{r0}. A implementação cria uma cópia do dígrafo original (linha 2) para preservar a entrada, inicializa estruturas auxiliares \texttt{A\_zero} (lista de arcos justos), \texttt{Dual\_list} (lista de pares \((X, \Delta(X))\) para fins de validação dual), e \texttt{D\_zero} (dígrafo de arcos justos) (linhas 3-5). Um contador de iterações é inicializado na linha 6.

O loop principal (linhas 7-22) itera enquanto houver conjuntos sem arcos justos entrando. Em cada iteração, incrementa-se o contador (linha 8), calcula-se as componentes fortemente conexas de \(D_0\) usando \texttt{nx.condensation} (linha 9), que devolve um grafo acíclico dirigido (DAG, do inglês \emph{directed acyclic graph}) onde cada vértice representa uma componente e contém o atributo \texttt{"members"} com os vértices originais. Utilizamos componentes fortemente conexas porque elas identificam naturalmente os conjuntos maximais de vértices que ainda não possuem arcos justos entrando, evitando a necessidade de rastrear manualmente quais conjuntos já foram processados.

Em seguida, identificam-se as fontes (componentes sem arcos entrando) no grafo de condensação (linha 10). Se há apenas uma fonte, significa que todos os vértices estão em uma única componente alcançável pela raiz através de arcos justos, garantindo que \(D_0\) contém uma r-arborescência e encerrando o loop (linhas 11-12).

Para cada fonte \(u\) no grafo de condensação (linha 13), obtém-se o conjunto \(X\) de vértices da componente (linha 14). Se \(r_0 \in X\), a fonte é ignorada (linhas 15-16), pois a componente contendo a raiz não necessita elevação. Caso contrário, identificam-se os arcos entrando em \(X\) usando \texttt{get\_arcs\_entering\_X} (linha 17), calcula-se o peso mínimo \(\Delta(X)\) usando \texttt{get\_minimum\_weight\_cut} (linha 18), e atualiza-se os pesos com \texttt{update\_weights\_in\_X}, registrando novos arcos justos (linha 19). A elevação simultânea de potenciais para todos os vértices de \(X\) mantém a propriedade de que arcos internos a \(X\) permanecem com o mesmo custo reduzido relativo, preservando a correção do algoritmo. Finalmente, adiciona-se \((X, \Delta(X))\) à lista dual se \(\Delta(X) > 0\) (linhas 20-21), permitindo verificação posterior das condições de otimalidade dual.

A função devolve \texttt{A\_zero} (lista de arcos justos) e \texttt{Dual\_list} (pares \((X, \Delta(X))\) para validação). A complexidade é \(O(nm)\) no pior caso, com \(O(n)\) iterações, cada uma custando \(O(m)\) para calcular componentes e atualizar pesos.

\begin{tcolorbox}[
		enhanced, breakable,
		colframe=blue!60!black, colback=blue!2,
		colbacktitle=blue!15, coltitle=black,
		title={Fase 1: Elevação de potenciais e construção de $A_0$},
		boxed title style={sharp corners, boxrule=0.6pt},
		sharp corners, boxrule=0.6pt
	]
	\emph{Eleva iterativamente os potenciais dos vértices até que cada conjunto possua ao menos um arco justo entrando. Devolve a lista $A_0$ de arcos justos e a lista de pares $(X, \Delta(X))$ para validação dual.}
	\tcblower
	\begin{lstlisting}[language=Python, basicstyle=\ttfamily\fontsize{8}{9}\selectfont]
def phase1(D_original, r0):
    D_copy = D_original.copy()
    A_zero = []
    Dual_list = []
    D_zero = build_D_zero(D_copy)
    iteration = 0
    while True:
        iteration += 1
        C = nx.condensation(D_zero)
        sources = [x for x in C.nodes() if C.in_degree(x) == 0]
        if len(sources) == 1:
            break
        for u in sources:
            X = C.nodes[u]["members"]
            if r0 in X:
                continue
            arcs = get_arcs_entering_X(D_copy, X)
            min_weight = get_minimum_weight_cut(arcs)
            update_weights_in_X(D_copy, arcs, min_weight, A_zero, D_zero)
            if min_weight != 0:
                Dual_list.append((X, min_weight)) 
    return A_zero, Dual_list
\end{lstlisting}
\end{tcolorbox}

A seguir ilustramos a execução da função \texttt{phase1} no dígrafo \(D_{32}\) estendido com a raiz \(r_0\). A figura mostra três estados principais: o dígrafo original \(D\) com seus pesos, o dígrafo \(D_0\) inicial (vazio, sem arcos), e o dígrafo \(D_0\) final após as iterações de elevação de potenciais, contendo apenas os arcos justos (em azul) que formam uma r-arborescência.

\begin{figure}[H]
	\centering
	\begin{tikzpicture}[>=Stealth, scale=0.85, every node/.style={scale=0.85}]
		% Estilos
		\tikzset{
			vtx/.style={circle, draw, fill=blue!8, thick, inner sep=1.2pt, minimum size=6.5mm},
			vtxRoot/.style={circle, draw, fill=red!15, thick, inner sep=1.2pt, minimum size=6.5mm},
			vtxX/.style={circle, draw, fill=orange!20, thick, inner sep=1.2pt, minimum size=6.5mm},
			arc/.style={->, thick, gray!60},
			arcTight/.style={->, ultra thick, blue!70!black},
			cost/.style={font=\scriptsize, fill=white, inner sep=1pt, rounded corners=1pt},
			costTight/.style={font=\scriptsize\bfseries, fill=blue!10, inner sep=2pt, draw=blue!70!black, rounded corners=1pt}
		}

		% ===== DÍGRAFO ORIGINAL D (esquerda) =====
		\begin{scope}
			\node[font=\bfseries, blue!80!black] at (2, 4.2) {Dígrafo original $D$};

			% Raiz
			\node[vtxRoot] (r0) at (2, 3) {$r_0$};

			% Vértices fora de X
			\node[vtx] (u1) at (0, 1.5) {$u_1$};
			\node[vtx] (u2) at (2, 1.5) {$u_2$};
			\node[vtx] (u3) at (4, 1.5) {$u_3$};

			% Vértices dentro de X
			\node[vtxX] (v1) at (0.5, -0.5) {$v_1$};
			\node[vtxX] (v2) at (2, -0.5) {$v_2$};
			\node[vtxX] (v3) at (3.5, -0.5) {$v_3$};

			% Arcos da raiz
			\draw[arc] (r0) -- node[cost, left, pos=0.4] {1} (u1);
			\draw[arc] (r0) -- node[cost, right] {2} (u2);
			\draw[arc] (r0) -- node[cost, right, pos=0.4] {1} (u3);

			% Arcos externos a X
			\draw[arc] (u1) to[bend left=15] node[cost, above] {1} (u2);
			\draw[arc] (u2) to[bend left=15] node[cost, above] {3} (u3);

			% Arcos entrando em X
			\draw[arc] (u1) -- node[cost, left] {3} (v1);
			\draw[arc] (u2) -- node[cost, right] {2} (v2);
			\draw[arc] (u3) -- node[cost, right] {4} (v3);
			\draw[arc] (u1) to[bend right=20] node[cost, below, pos=0.3] {5} (v2);

			% Arcos internos a X
			\draw[arc] (v1) -- node[cost, above] {1} (v2);
			\draw[arc] (v2) -- node[cost, above] {2} (v3);

			% Arcos saindo de X
			\draw[arc] (v1) to[bend right=30] node[cost, left, pos=0.3] {2} (u1);

			\node[font=\small, gray!70!black] at (2, -2) {Pesos originais};
		\end{scope}

		% ===== D_0 INICIAL (centro) =====
		\begin{scope}[xshift=6cm]
			\node[font=\bfseries, blue!80!black] at (2, 4.2) {$D_0$ inicial (linha 5)};

			% Vértices apenas, sem arcos
			\node[vtxRoot] (r0i) at (2, 3) {$r_0$};
			\node[vtx] (u1i) at (0, 1.5) {$u_1$};
			\node[vtx] (u2i) at (2, 1.5) {$u_2$};
			\node[vtx] (u3i) at (4, 1.5) {$u_3$};
			\node[vtxX] (v1i) at (0.5, -0.5) {$v_1$};
			\node[vtxX] (v2i) at (2, -0.5) {$v_2$};
			\node[vtxX] (v3i) at (3.5, -0.5) {$v_3$};

			\node[font=\small, gray!70!black] at (2, -2) {Sem arcos (vazio)};
		\end{scope}

		% ===== D_0 FINAL (direita) =====
		\begin{scope}[xshift=12cm]
			\node[font=\bfseries, blue!80!black] at (2, 4.2) {$D_0$ final (após loop)};

			% Vértices
			\node[vtxRoot] (r0f) at (2, 3) {$r_0$};
			\node[vtx] (u1f) at (0, 1.5) {$u_1$};
			\node[vtx] (u2f) at (2, 1.5) {$u_2$};
			\node[vtx] (u3f) at (4, 1.5) {$u_3$};
			\node[vtxX] (v1f) at (0.5, -0.5) {$v_1$};
			\node[vtxX] (v2f) at (2, -0.5) {$v_2$};
			\node[vtxX] (v3f) at (3.5, -0.5) {$v_3$};

			% Arcos justos (peso 0)
			\draw[arcTight] (r0f) -- node[costTight, left, pos=0.4] {0} (u1f);
			\draw[arcTight] (r0f) -- node[costTight, right, pos=0.4] {0} (u3f);
			\draw[arcTight] (u1f) to[bend left=15] node[costTight, above] {0} (u2f);
			\draw[arcTight] (u1f) -- node[costTight, left] {0} (v1f);
			\draw[arcTight] (u2f) -- node[costTight, right] {0} (v2f);
			\draw[arcTight] (v2f) -- node[costTight, above] {0} (v3f);

			\node[font=\small, blue!70!black] at (2, -2) {Arcos justos em $A_0$};
		\end{scope}

		% Setas indicando progressão
		\draw[->, ultra thick, gray!50] (4.5, 1) -- node[above, font=\small] {elevação} (5.5, 1);
		\draw[->, ultra thick, gray!50] (10.5, 1) -- node[above, font=\small] {iterações} (11.5, 1);
	\end{tikzpicture}
	\caption{Execução da função \texttt{phase1} em \(D_{32}\). À esquerda, o dígrafo original \(D\) com pesos dos arcos. Ao centro, \(D_0\) inicial após linha 5 (\texttt{build\_D\_zero}), contendo apenas vértices sem arcos. À direita, \(D_0\) final após o loop de elevação de potenciais, contendo apenas os arcos \textcolor{blue!70!black}{justos} (custo reduzido zero) que formam uma r-arborescência. Durante as iterações (linhas 7-22), os potenciais são elevados para cada componente sem arcos justos entrando, até que todos os vértices sejam alcançáveis a partir de \(r_0\) através de arcos em \(D_0\).}
\end{figure}

A seguir ilustramos a execução da função \texttt{phase2} em \(D_{32}\), mostrando como a r-arborescência é construída incrementalmente a partir do conjunto \(A_0\) de arcos justos obtidos da Fase~1. A figura apresenta quatro estados: o conjunto inicial \(A_0\) com 6 arcos justos, e três estados intermediários da arborescência \texttt{Arb} sendo construída, destacando em verde os arcos adicionados em cada iteração.

\begin{figure}[H]
	\centering
	\begin{tikzpicture}[>=Stealth, scale=0.72, every node/.style={scale=0.72}]
		% Estilos
		\tikzset{
			vtx/.style={circle, draw, fill=blue!8, thick, inner sep=1.2pt, minimum size=6.5mm},
			vtxRoot/.style={circle, draw, fill=red!15, thick, inner sep=1.2pt, minimum size=6.5mm},
			vtxIn/.style={circle, draw, fill=green!15, thick, inner sep=1.2pt, minimum size=6.5mm},
			vtxOut/.style={circle, draw, fill=gray!10, thick, inner sep=1.2pt, minimum size=6.5mm},
			arc/.style={->, thick, blue!70!black},
			arcNew/.style={->, ultra thick, green!70!black},
			cost/.style={font=\scriptsize, fill=white, inner sep=1pt}
		}

		% ===== A_0 (esquerda superior) =====
		\begin{scope}
			\node[font=\bfseries, blue!80!black] at (2, 4.5) {$A_0$ (arcos justos)};

			\node[vtxRoot] (r0) at (2, 3.2) {$r_0$};
			\node[vtx] (u1) at (0, 1.7) {$u_1$};
			\node[vtx] (u2) at (2, 1.7) {$u_2$};
			\node[vtx] (u3) at (4, 1.7) {$u_3$};
			\node[vtx] (v1) at (0.5, 0) {$v_1$};
			\node[vtx] (v2) at (2, 0) {$v_2$};
			\node[vtx] (v3) at (3.5, 0) {$v_3$};

			\draw[arc] (r0) -- node[cost, left, pos=0.3] {0} (u1);
			\draw[arc] (r0) -- node[cost, right, pos=0.3] {0} (u3);
			\draw[arc] (u1) to[bend left=12] node[cost, above, pos=0.4] {0} (u2);
			\draw[arc] (u1) -- node[cost, left] {0} (v1);
			\draw[arc] (u2) -- node[cost, right] {0} (v2);
			\draw[arc] (v2) -- node[cost, above] {0} (v3);

			\node[font=\small, gray!70!black] at (2, -1) {6 arcos com custo 0};
		\end{scope}

		% ===== Iteração 1 (direita superior) =====
		\begin{scope}[xshift=6.5cm]
			\node[font=\bfseries, blue!80!black] at (2, 4.5) {Iteração 1: $|$Arb$| = 1$};

			\node[vtxRoot] (r01) at (2, 3.2) {$r_0$};
			\node[vtxIn] (u11) at (0, 1.7) {$u_1$};
			\node[vtxOut] (u21) at (2, 1.7) {$u_2$};
			\node[vtxOut] (u31) at (4, 1.7) {$u_3$};
			\node[vtxOut] (v11) at (0.5, 0) {$v_1$};
			\node[vtxOut] (v21) at (2, 0) {$v_2$};
			\node[vtxOut] (v31) at (3.5, 0) {$v_3$};

			\draw[arcNew] (r01) -- node[cost, left, pos=0.3, fill=green!10] {1} (u11);

			\node[font=\small, green!70!black] at (2, -1) {Adiciona $(r_0, u_1)$};
		\end{scope}

		% ===== Iteração 2 (esquerda inferior) =====
		\begin{scope}[yshift=-6.2cm]
			\node[font=\bfseries, blue!80!black] at (2, 4.5) {Iteração 2: $|$Arb$| = 2$};

			\node[vtxRoot] (r02) at (2, 3.2) {$r_0$};
			\node[vtxIn] (u12) at (0, 1.7) {$u_1$};
			\node[vtxOut] (u22) at (2, 1.7) {$u_2$};
			\node[vtxIn] (u32) at (4, 1.7) {$u_3$};
			\node[vtxOut] (v12) at (0.5, 0) {$v_1$};
			\node[vtxOut] (v22) at (2, 0) {$v_2$};
			\node[vtxOut] (v32) at (3.5, 0) {$v_3$};

			\draw[arc] (r02) -- node[cost, left, pos=0.3] {1} (u12);
			\draw[arcNew] (r02) -- node[cost, right, pos=0.3, fill=green!10] {1} (u32);

			\node[font=\small, green!70!black] at (2, -1) {Adiciona $(r_0, u_3)$};
		\end{scope}

		% ===== Iteração 4 (direita inferior) =====
		\begin{scope}[xshift=6.5cm, yshift=-6.2cm]
			\node[font=\bfseries, blue!80!black] at (2, 4.5) {Iteração 4: $|$Arb$| = 4$};

			\node[vtxRoot] (r04) at (2, 3.2) {$r_0$};
			\node[vtxIn] (u14) at (0, 1.7) {$u_1$};
			\node[vtxIn] (u24) at (2, 1.7) {$u_2$};
			\node[vtxIn] (u34) at (4, 1.7) {$u_3$};
			\node[vtxIn] (v14) at (0.5, 0) {$v_1$};
			\node[vtxOut] (v24) at (2, 0) {$v_2$};
			\node[vtxOut] (v34) at (3.5, 0) {$v_3$};

			\draw[arc] (r04) -- node[cost, left, pos=0.3] {1} (u14);
			\draw[arc] (r04) -- node[cost, right, pos=0.3] {1} (u34);
			\draw[arc] (u14) to[bend left=12] node[cost, above, pos=0.4] {1} (u24);
			\draw[arcNew] (u14) -- node[cost, left, fill=green!10] {3} (v14);

			\node[font=\small, green!70!black] at (2, -1) {Adiciona $(u_1, v_1)$};
		\end{scope}

		% Setas de progressão
		\draw[->, ultra thick, gray!50] (4.8, 2) -- node[above, font=\small] {linha 6-10} (5.7, 2);
		\draw[->, ultra thick, gray!50] (2, -1.5) -- node[left, font=\small] {linha 6-10} (2, -2.4);
		\draw[->, ultra thick, gray!50] (4.8, -4.2) -- node[above, font=\small] {linha 6-10} (5.7, -4.2);
	\end{tikzpicture}
	\caption{Execução da função \texttt{phase2} em \(D_{32}\). Superior esquerdo: conjunto \(A_0\) com 6 arcos justos (custo reduzido zero) obtidos da Fase~1. Superior direito: Iteração 1 — adiciona arco $(r_0, u_1)$ pois $r_0 \in$ Arb e $u_1 \notin$ Arb. Inferior esquerdo: Iteração 2 — adiciona $(r_0, u_3)$. Inferior direito: Iteração 4 — após adicionar $(u_1, u_2)$ na iteração 3, adiciona $(u_1, v_1)$. Vértices \textcolor{green!70!black}{verdes} foram recém-adicionados, vértices \textcolor{gray!70!black}{cinzas} ainda não pertencem a Arb. O processo continua até que todos os 6 arcos de \(A_0\) sejam incluídos, formando uma r-arborescência com pesos originais de \(D\).}
\end{figure}

As funções auxiliares implementadas nesta seção correspondem diretamente aos passos da Fase~1 do algoritmo de András Frank da seguinte forma:

\begin{tcolorbox}[
		enhanced, breakable,
		colframe=purple!60!black, colback=purple!5,
		colbacktitle=purple!15, coltitle=black,
		title={Correspondência entre Teoria e Implementação — Fase 1},
		boxed title style={sharp corners, boxrule=0.6pt},
		sharp corners, boxrule=0.6pt
	]
	\textbf{Passo 1 — Inicialização:}
	\begin{itemize}
		\item \textbf{Descrição teórica:} Defina \(y(v) := 0\) para todo \(v \in V\).
		\item \textbf{Implementação:} A função \texttt{build\_D\_zero(D)} inicializa o dígrafo \(D_0\) vazio, que será populado apenas com arcos justos. Implicitamente, os potenciais iniciam em zero, pois os pesos no dígrafo \(D\) representam os custos reduzidos \(c_y(u,v) = c(u,v) - y(v)\). Com \(y(v) = 0\), temos \(c_y = c\) no início.
	\end{itemize}

	\vspace{2mm}
	\textbf{Passo 2 — Iteração de elevação de potenciais:}
	\begin{itemize}
		\item \textbf{Descrição teórica:} Enquanto existir conjunto \(X \subseteq V \setminus \{r\}\) sem arco justo entrando:
		      \begin{itemize}
			      \item Calcule \(\Delta(X) := \min\{c(u,v) - y(v) : u \notin X, v \in X\}\).
			      \item Para cada \(v \in X\), atualize \(y(v) := y(v) + \Delta(X)\).
		      \end{itemize}
		\item \textbf{Implementação:} Esse processo de iteração é realizado pela composição sequencial de três funções auxiliares:
		      \begin{enumerate}
			      \item \texttt{get\_arcs\_entering\_X(D, X)}: Identifica o conjunto \(\{(u,v) : u \notin X, v \in X\}\), isto é, todos os arcos que cruzam a fronteira de \(X\). Essa função corresponde diretamente à definição do conjunto sobre o qual o mínimo é calculado na fórmula \(\Delta(X)\).

			      \item \texttt{get\_minimum\_weight\_cut(arcs)}: Calcula \(\min\{\text{data}["w"] : (u,v,\text{data}) \in \text{arcs}\}\), que é exatamente \(\Delta(X) = \min\{c_y(u,v) : u \notin X, v \in X\}\). Como os pesos no dígrafo já representam custos reduzidos (são atualizados a cada iteração), essa função devolve precisamente o valor teórico de \(\Delta(X)\).

			      \item \texttt{update\_weights\_in\_X(D, arcs, min\_weight, A\_zero, D\_zero)}: Implementa a atualização dos potenciais. Para cada arco \((u,v)\) entrando em \(X\), subtrai \texttt{min\_weight} de \texttt{D[u][v]["w"]}, efetivamente calculando o novo custo reduzido:
			            \[c_y'(u,v) = c_y(u,v) - \Delta(X) = \]
			            \[c(u,v) - y(v) - \Delta(X) = \]
			            \[c(u,v) - \bigl(y(v) + \Delta(X)\bigr) = c(u,v) - y'(v), \]
			            onde \(y'(v) = y(v) + \Delta(X)\) é o novo potencial. Arcos cujo custo reduzido atinge zero (\texttt{D[u][v]["w"] == 0}) são adicionados a \(A_0\) e \(D_0\), tornando-se \textbf{justos}.
		      \end{enumerate}
	\end{itemize}

	\vspace{2mm}
	\textbf{Passo 3 — Construção de \(A_0\):}
	\begin{itemize}
		\item \textbf{Descrição teórica:} Defina \(A_0 := \{a \in A : c_y(a) = 0\}\), o conjunto de arcos justos.
		\item \textbf{Implementação:} A construção de \(A_0\) ocorre de forma incremental durante as iterações do Passo~2. Cada chamada a \texttt{update\_weights\_in\_X} verifica quais arcos atingiram custo reduzido zero e os adiciona tanto à lista \texttt{A\_zero} quanto ao dígrafo \texttt{D\_zero}. Esse processo continua até que a função principal da Fase~1 (que será apresentada adiante) determine que cada componente fortemente conexa em \(D_0\) (exceto a raiz) possui ao menos um arco justo entrando, garantindo que \(A_0\) é suficiente para formar a base de uma \(r\)-arborescência.
		\item O resultado final é o conjunto completo \(A_0 = \{a \in A : c_y(a) = 0\}\), usado na Fase~2 para construir a arborescência ótima através de uma seleção gulosa de arcos justos.
	\end{itemize}
\end{tcolorbox}

\subsection{Fase 2: Construção da arborescência}
Esta é a função principal da Fase~2, responsável por construir a r-arborescência de custo mínimo a partir do conjunto \(A_0\) de arcos justos. A construção é incremental: inicia-se com a raiz e adiciona-se iterativamente arcos de \(A_0\) que conectam vértices já incluídos a novos vértices, garantindo que cada vértice não-raiz receba exatamente um arco de entrada.

Recebe como entrada um dígrafo \texttt{D\_original}, a raiz \texttt{r0}, e a lista \texttt{A\_zero} de arcos justos. A implementação cria um novo dígrafo vazio \texttt{Arb} (linha 2) e adiciona a raiz (linha 3).

O loop principal (linhas 5-12) itera \(n-1\) vezes, onde \(n = |V|\), pois uma r-arborescência tem exatamente \(|V|-1\) arcos. Em cada iteração:
\begin{enumerate}\setlength{\itemsep}{2pt}
	\item Percorre os arcos \((u,v)\) de \texttt{A\_zero} (linha 6).
	\item Verifica se \(u\) já está em \texttt{Arb} e \(v\) ainda não (linha 7).
	\item Se sim, obtém os dados do arco do dígrafo original (linha 8) e adiciona \((u,v)\) a \texttt{Arb} (linha 9).
	\item Interrompe o loop interno para reiniciar a busca, garantindo descoberta em largura (linha 10).
\end{enumerate}

A função devolve o dígrafo \texttt{Arb} representando a r-arborescência de custo mínimo. A complexidade é \(O(nm)\) no pior caso, pois cada uma das \(O(n)\) iterações pode percorrer todos os \(O(m)\) arcos de \texttt{A\_zero}.

\begin{tcolorbox}[
		enhanced, breakable,
		colframe=blue!60!black, colback=blue!2,
		colbacktitle=blue!15, coltitle=black,
		title={Fase 2: Construção da arborescência},
		boxed title style={sharp corners, boxrule=0.6pt},
		sharp corners, boxrule=0.6pt
	]
	\emph{Constrói incrementalmente a r-arborescência a partir de $A_0$, adicionando iterativamente arcos que conectam vértices já incluídos a novos vértices.}
	\tcblower
	\begin{lstlisting}[language=Python]
def phase2(D_original, r0, A_zero):
    Arb = nx.DiGraph()
    Arb.add_node(r0)
    n = len(D_original.nodes())
    for _ in range(n - 1):
        for u, v in A_zero:
            if u in Arb.nodes() and v not in Arb.nodes():
                edge_data = D_original.get_edge_data(u, v)
                Arb.add_edge(u, v, **edge_data)
                break
    return Arb
\end{lstlisting}
\end{tcolorbox}

Apresentamos também uma versão alternativa da Fase~2 que utiliza busca em largura (BFS) para construir a r-arborescência de forma mais eficiente. Diferentemente da versão anterior que itera \(n-1\) vezes sobre todos os arcos, esta implementação usa uma fila de prioridade para explorar os arcos em ordem, evitando buscas lineares repetidas.

Recebe como entrada um dígrafo \texttt{D\_original}, a raiz \texttt{r0}, e a lista \texttt{A\_zero} de arcos justos. A implementação começa criando um dígrafo auxiliar \texttt{Arb} (linha 2) onde cada arco de \texttt{A\_zero} recebe um peso igual ao seu índice na lista (linhas 3-4), estabelecendo uma ordem de exploração. Inicializa-se o conjunto \texttt{V} de vértices visitados contendo apenas a raiz (linha 5) e uma fila de prioridade vazia \texttt{q} (linha 6).

Todos os arcos que saem da raiz em \texttt{Arb} são adicionados à fila de prioridade (linhas 7-8), usando o peso (índice) como critério de ordenação. Cria-se então o dígrafo \texttt{A} que conterá a arborescência resultante (linha 9).

O loop principal (linhas 10-17) extrai arcos da fila de prioridade em ordem crescente de índice. Para cada arco \((u,v)\) extraído (linha 11), verifica-se se o vértice destino \(v\) já foi visitado (linha 12); em caso positivo, o arco é ignorado via \texttt{continue} (linha 13). Caso contrário, adiciona-se o arco \((u,v)\) à arborescência \texttt{A} com o peso original de \texttt{D\_original} (linha 14), marca-se \(v\) como visitado (linha 15), e todos os arcos que saem de \(v\) em \texttt{Arb} são adicionados à fila de prioridade para exploração futura (linhas 16-17).

A função devolve o dígrafo \texttt{A} representando a r-arborescência de custo mínimo. A complexidade é \(O(m \log m)\), onde \(m = |A_0|\), devido às operações de inserção e remoção na fila de prioridade. Esta versão é mais eficiente que a anterior quando \(|A_0|\) é grande, pois evita percorrer todos os arcos em cada iteração.

\begin{tcolorbox}[
		enhanced, breakable,
		colframe=blue!60!black, colback=blue!2,
		colbacktitle=blue!15, coltitle=black,
		title={Fase 2 (versão BFS): Construção da arborescência com fila de prioridade},
		boxed title style={sharp corners, boxrule=0.6pt},
		sharp corners, boxrule=0.6pt
	]
	\emph{Constrói a r-arborescência usando busca em largura guiada por fila de prioridade, explorando arcos de $A_0$ em ordem e evitando buscas lineares repetidas. Complexidade $O(m \log m)$.}
	\tcblower
	\begin{lstlisting}[language=Python]
def phase2_v2(D_original, r0, A_zero):
    Arb = nx.DiGraph()
    for i, (u, v) in enumerate(A_zero):
        Arb.add_edge(u, v, w=i)
    V = {r0}
    q = []
    for u, v, data in Arb.out_edges(r0, data=True):
        heapq.heappush(q, (data["w"], u, v))
    A = nx.DiGraph()
    while q:
        _, u, v = heapq.heappop(q)
        if v in V:
            continue
        A.add_edge(u, v, w=D_original[u][v]["w"])
        V.add(v)
        for x, y, data in Arb.out_edges(v, data=True):
            heapq.heappush(q, (data["w"], x, y))
    return A
\end{lstlisting}
\end{tcolorbox}

As duas versões da Fase~2 implementadas acima correspondem diretamente ao Passo~4 da descrição teórica do algoritmo de András Frank:

\begin{tcolorbox}[
		enhanced, breakable,
		colframe=purple!60!black, colback=purple!5,
		colbacktitle=purple!15, coltitle=black,
		title={Correspondência entre Teoria e Implementação — Fase 2},
		boxed title style={sharp corners, boxrule=0.6pt},
		sharp corners, boxrule=0.6pt
	]
	\textbf{Passo 4 — Construção da arborescência (caso acíclico):}
	\begin{itemize}
		\item \textbf{Descrição teórica:} Se \((V,A_0)\) forma uma r-arborescência, devolva \(A_0\). Por otimalidade dos potenciais duais, trata-se de uma r-arborescência de custo mínimo.
		\item \textbf{Implementação:} Ambas as versões de \texttt{phase2} constroem uma r-arborescência a partir do conjunto \(A_0\) de arcos justos obtido na Fase~1. A corretude baseia-se no fato de que todos os arcos em \(A_0\) têm custo reduzido zero, e a Fase~1 garante que existe uma r-arborescência formada exclusivamente por arcos justos.
		      \begin{itemize}
			      \item \textbf{Versão 1 (\texttt{phase2}):} Construção incremental por exploração exaustiva. Em cada uma das \(n-1\) iterações, percorre todos os arcos de \(A_0\) procurando um arco \((u,v)\) tal que \(u\) já pertence à arborescência parcial e \(v\) ainda não. Essa abordagem simples corresponde diretamente à ideia teórica de construir a arborescência adicionando um vértice por vez, conectando-o à estrutura existente através de um arco justo. Complexidade: \(O(nm)\).

			      \item \textbf{Versão 2 (\texttt{phase2\_v2}):} Construção por busca em largura guiada por fila de prioridade. Cria um dígrafo auxiliar onde arcos são indexados, usa a fila de prioridade para explorar arcos sistematicamente a partir da raiz, evitando buscas lineares repetidas. Essa versão otimizada mantém a mesma correção teórica — construir uma r-arborescência usando apenas arcos de \(A_0\) — mas melhora a eficiência prática. Complexidade: \(O(m \log m)\).
		      \end{itemize}
	\end{itemize}

	\vspace{2mm}
	\textbf{Nota sobre Passos 5-7 (contração/recursão/expansão):}
	\begin{itemize}
		\item A descrição teórica do algoritmo de András Frank inclui os Passos~5-7 para tratar o caso onde \(A_0\) contém ciclos dirigidos, exigindo contração, resolução recursiva e reexpansão, de forma análoga ao algoritmo de Chu--Liu--Edmonds.
		\item Na implementação apresentada, optamos por uma abordagem não-recursiva baseada em componentes fortemente conexas. A Fase~1 já garante que \(A_0\) formará uma r-arborescência ao término das iterações de elevação de potenciais, eliminando a necessidade de tratar ciclos explicitamente na Fase~2. Essa simplificação é possível porque a elevação de potenciais progressivamente "quebra" todos os ciclos ao criar novos arcos justos que conectam diferentes componentes, até que reste apenas uma única componente fortemente conexa contendo todos os vértices.
		\item Portanto, quando a Fase~2 é executada, o conjunto \(A_0\) já está livre de ciclos e forma uma r-arborescência, correspondendo diretamente ao caso tratado pelo Passo~4 da descrição teórica. A verificação prévia \texttt{has\_arborescence(D, r0)} (realizada pela função principal) confirma essa propriedade antes de invocar a Fase~2.
	\end{itemize}
\end{tcolorbox}

\subsection{Verificação de otimalidade dual}

Esta função verifica se a condição de otimalidade dual é satisfeita para a r-arborescência construída. Segundo a teoria de programação linear dual aplicada ao problema de arborescência de custo mínimo, uma solução é ótima se e somente se cada conjunto \(X \subseteq V \setminus \{r\}\) que teve seu potencial elevado durante a Fase~1 possui exatamente um arco entrando na arborescência final.

Recebe como entrada a arborescência \texttt{Arb} e a lista \texttt{Dual\_list} contendo pares \((X, \Delta(X))\) onde \(X\) é um conjunto de vértices cujos potenciais foram elevados e \(\Delta(X) > 0\) é o valor da elevação. A implementação itera sobre cada par \((X, z)\) na lista dual (linha 2), e para cada conjunto \(X\), percorre todos os arcos \((u,v)\) da arborescência (linha 3). Inicializa um contador \texttt{count} em zero (linha 4) e verifica se o arco cruza a fronteira de \(X\), isto é, se \(u \notin X\) e \(v \in X\) (linha 5). Quando essa condição é satisfeita, incrementa o contador (linha 6) e imediatamente verifica se já há mais de um arco entrando em \(X\) (linha 7). Caso positivo, a condição de otimalidade dual é violada e a função devolve \texttt{False} (linha 8).

A função devolve \texttt{True} se todos os conjuntos em \texttt{Dual\_list} possuem exatamente um arco entrando na arborescência, confirmando que a solução satisfaz as condições de folga complementar da programação linear dual. A complexidade é \(O(km)\), onde \(k = |\texttt{Dual\_list}|\) e \(m = |A|\), pois para cada conjunto dual verifica-se todos os arcos da arborescência.

Esta verificação é fundamental para garantir a correção do algoritmo: a Fase~1 constrói uma solução dual viável (potenciais \(y(v)\)), a Fase~2 constrói uma solução primal viável (arborescência), e esta função confirma que ambas satisfazem as condições de folga complementar, implicando otimalidade pelo teorema da dualidade forte.

\begin{tcolorbox}[
		enhanced, breakable,
		colframe=blue!60!black, colback=blue!2,
		colbacktitle=blue!15, coltitle=black,
		title={Verificação de otimalidade dual},
		boxed title style={sharp corners, boxrule=0.6pt},
		sharp corners, boxrule=0.6pt
	]
	\emph{Verifica se a condição de otimalidade dual é satisfeita, confirmando que cada conjunto dual possui exatamente um arco entrando na arborescência.}
	\tcblower
	\begin{lstlisting}[language=Python]
def check_dual_optimality_condition(Arb, Dual_list):
    for X, z in Dual_list:
        count = 0
        for u, v in Arb.edges():
            if u not in X and v in X:
                count += 1
                if count > 1:
                    return False
    return True
\end{lstlisting}
\end{tcolorbox}

\subsection{O algoritmo completo de András Frank}

Finalmente, apresentamos a função principal que implementa o algoritmo de András Frank para encontrar uma r-arborescência de custo mínimo em um dígrafo com pesos. A função integra as fases de construção dos potenciais duais, obtenção dos arcos justos, construção da arborescência e verificação de otimalidade dual.

\begin{tcolorbox}[
		enhanced, breakable,
		colframe=blue!60!black, colback=blue!2,
		colbacktitle=blue!15, coltitle=black,
		title={Verificação de otimalidade dual},
		boxed title style={sharp corners, boxrule=0.6pt},
		sharp corners, boxrule=0.6pt
	]
	\emph{Implementa o algoritmo completo de András Frank, integrando as fases de construção dos potenciais duais, obtenção dos arcos justos, construção da arborescência e verificação de otimalidade dual.}
	\tcblower
	\begin{lstlisting}[language=Python]
def andras_frank_algorithm(D):
    A_zero, Dual_list = phase1(D,"r0")
    arborescence_frank = phase2(D, "r0", A_zero)
    arborescence_frank_v2 = phase2_v2(D, "r0", A_zero)
    dual_frank = check_dual_optimality_condition(
        arborescence_frank, Dual_list)
    dual_frank_v2 = check_dual_optimality_condition(
        arborescence_frank_v2, Dual_list)
    return arborescence_frank, arborescence_frank_v2, dual_frank, dual_frank_v2
\end{lstlisting}
\end{tcolorbox}

\subsection*{Exemplo de execução do algoritmo}

Aqui ilustraremos cada fase do algoritmo de András Frank: inicialização dos potenciais, iterações de elevação, construção de \(A_0\), e formação da arborescência final. Utilizaremos o dígrafo \(D_{32}\) estendido com a raiz \(r_0\), mostrado na Figura~\ref{fig:frank-exemplo-inicial}.

\begin{figure}[H]
	\centering
	\begin{tikzpicture}[>=Stealth, scale=0.9, every node/.style={scale=0.9}]
		\tikzset{
			vtx/.style={circle, draw, fill=blue!8, thick, inner sep=1.2pt, minimum size=6.5mm},
			vtxRoot/.style={circle, draw, fill=red!15, thick, inner sep=1.2pt, minimum size=6.5mm},
			arc/.style={->, thick, gray!60},
			cost/.style={font=\scriptsize, fill=white, inner sep=1pt, rounded corners=1pt}
		}

		\node[font=\bfseries, blue!80!black] at (3, 4.5) {Dígrafo inicial $D$ com raiz $r_0$};

		% Raiz
		\node[vtxRoot] (r0) at (3, 3) {$r_0$};

		% Vértices camada 1
		\node[vtx] (u1) at (0, 1.5) {$u_1$};
		\node[vtx] (u2) at (3, 1.5) {$u_2$};
		\node[vtx] (u3) at (6, 1.5) {$u_3$};

		% Vértices camada 2
		\node[vtx] (v1) at (1, 0) {$v_1$};
		\node[vtx] (v2) at (3, 0) {$v_2$};
		\node[vtx] (v3) at (5, 0) {$v_3$};

		% Arcos da raiz
		\draw[arc] (r0) -- node[cost, left, pos=0.4] {1} (u1);
		\draw[arc] (r0) -- node[cost, right] {2} (u2);
		\draw[arc] (r0) -- node[cost, right, pos=0.4] {1} (u3);

		% Arcos camada 1
		\draw[arc] (u1) to[bend left=15] node[cost, above] {1} (u2);
		\draw[arc] (u2) to[bend left=15] node[cost, above] {3} (u3);

		% Arcos entrando camada 2
		\draw[arc] (u1) -- node[cost, left] {3} (v1);
		\draw[arc] (u2) -- node[cost, right] {2} (v2);
		\draw[arc] (u3) -- node[cost, right] {4} (v3);
		\draw[arc] (u1) to[bend right=20] node[cost, below, pos=0.3] {5} (v2);

		% Arcos internos camada 2
		\draw[arc] (v1) -- node[cost, above] {1} (v2);
		\draw[arc] (v2) -- node[cost, above] {2} (v3);

		% Arco retorno
		\draw[arc] (v1) to[bend right=30] node[cost, left, pos=0.3] {2} (u1);

		\node[font=\small, gray!70!black] at (3, -1.2) {Pesos originais $c(u,v)$};
	\end{tikzpicture}
	\caption{Dígrafo inicial \(D\) com raiz \(r_0\) e 6 vértices adicionais. O dígrafo contém múltiplos arcos com pesos variados. A Fase~1 do algoritmo iniciará com potenciais \(y(v) = 0\) para todos os vértices, correspondendo aos custos reduzidos iniciais \(c_y(u,v) = c(u,v)\).}
	\label{fig:frank-exemplo-inicial}
\end{figure}

\paragraph*{Fase 1 — Iteração 1:} O algoritmo inicia com todos os potenciais \(y(v) = 0\) e \(D_0\) vazio (sem arcos justos). Calcula-se as componentes fortemente conexas de \(D_0\): cada vértice forma sua própria componente. As fontes são todos os vértices exceto \(r_0\). Para o conjunto \(X = \{u_1, u_2, u_3\}\), identifica-se os arcos entrando: \((r_0, u_1)\) com peso 1, \((r_0, u_2)\) com peso 2, \((r_0, u_3)\) com peso 1. O mínimo é \(\Delta(X) = 1\). Subtraindo esse valor, os arcos \((r_0, u_1)\) e \((r_0, u_3)\) tornam-se justos (peso 0) e são adicionados a \(A_0\).

\begin{figure}[H]
	\centering
	\begin{tikzpicture}[>=Stealth, scale=0.85, every node/.style={scale=0.85}]
		\tikzset{
			vtx/.style={circle, draw, fill=blue!8, thick, inner sep=1.2pt, minimum size=6.5mm},
			vtxRoot/.style={circle, draw, fill=red!15, thick, inner sep=1.2pt, minimum size=6.5mm},
			vtxX/.style={circle, draw, fill=orange!20, thick, inner sep=1.2pt, minimum size=6.5mm},
			arc/.style={->, thick, gray!60},
			arcTight/.style={->, ultra thick, blue!70!black},
			cost/.style={font=\scriptsize, fill=white, inner sep=1pt},
			costTight/.style={font=\scriptsize\bfseries, fill=blue!10, inner sep=2pt, draw=blue!70!black, rounded corners=1pt}
		}

		% Antes
		\begin{scope}
			\node[font=\bfseries, blue!80!black] at (2, 4.5) {Antes: $\Delta(X_1) = 1$};

			\node[vtxRoot] (r0) at (2, 3) {$r_0$};
			\node[vtxX] (u1) at (0, 1.5) {$u_1$};
			\node[vtxX] (u2) at (2, 1.5) {$u_2$};
			\node[vtxX] (u3) at (4, 1.5) {$u_3$};

			\draw[arc] (r0) -- node[cost, left, pos=0.4] {1} (u1);
			\draw[arc] (r0) -- node[cost, right] {2} (u2);
			\draw[arc] (r0) -- node[cost, right, pos=0.4] {1} (u3);

			\node[font=\small, orange!80!black] at (2, 0.5) {$X_1 = \{u_1, u_2, u_3\}$};
		\end{scope}

		% Depois
		\begin{scope}[xshift=7cm]
			\node[font=\bfseries, blue!80!black] at (2, 4.5) {Depois: arcos justos};

			\node[vtxRoot] (r0b) at (2, 3) {$r_0$};
			\node[vtx] (u1b) at (0, 1.5) {$u_1$};
			\node[vtx] (u2b) at (2, 1.5) {$u_2$};
			\node[vtx] (u3b) at (4, 1.5) {$u_3$};

			\draw[arcTight] (r0b) -- node[costTight, left, pos=0.4] {0} (u1b);
			\draw[arc] (r0b) -- node[cost, right] {1} (u2b);
			\draw[arcTight] (r0b) -- node[costTight, right, pos=0.4] {0} (u3b);

			\node[font=\small, blue!70!black] at (2, 0.5) {2 arcos adicionados a $A_0$};
		\end{scope}
	\end{tikzpicture}
	\caption{Fase~1, Iteração 1: Elevação de potenciais para \(X_1 = \{u_1, u_2, u_3\}\). À esquerda, o conjunto \(X_1\) (em laranja) sem arcos justos entrando. Calcula-se \(\Delta(X_1) = \min\{1, 2, 1\} = 1\). À direita, após subtrair \(\Delta(X_1)\), os arcos \((r_0, u_1)\) e \((r_0, u_3)\) tornam-se \textcolor{blue!70!black}{justos} (custo reduzido 0) e são adicionados a \(A_0\).}
	\label{fig:frank-exemplo-iter1}
\end{figure}

\paragraph*{Fase 1 — Iteração 2:} Com \(A_0 = \{(r_0, u_1), (r_0, u_3)\}\), as componentes fortemente conexas são: \(\{r_0, u_1, u_3\}\) e \(\{u_2\}\), \(\{v_1\}\), \(\{v_2\}\), \(\{v_3\}\). A fonte (excluindo a raiz) é \(\{u_2\}\). Para \(X = \{u_2\}\), o único arco entrando é \((r_0, u_2)\) com peso atual 1 (após iteração anterior). Portanto \(\Delta(X) = 1\), e após subtração, \((r_0, u_2)\) torna-se justo. Também na mesma iteração, para \(X = \{v_1, v_2, v_3\}\), calcula-se \(\Delta(X) = \min\{3, 2, 4, 5\} = 2\), e os arcos são atualizados conforme ilustrado.

\begin{figure}[H]
	\centering
	\begin{tikzpicture}[>=Stealth, scale=0.75, every node/.style={scale=0.75}]
		\tikzset{
			vtx/.style={circle, draw, fill=blue!8, thick, inner sep=1.2pt, minimum size=6mm},
			vtxRoot/.style={circle, draw, fill=red!15, thick, inner sep=1.2pt, minimum size=6mm},
			vtxX/.style={circle, draw, fill=orange!20, thick, inner sep=1.2pt, minimum size=6mm},
			arc/.style={->, thick, gray!60},
			arcTight/.style={->, ultra thick, blue!70!black},
			cost/.style={font=\scriptsize, fill=white, inner sep=1pt},
			costTight/.style={font=\scriptsize\bfseries, fill=blue!10, inner sep=2pt, draw=blue!70!black, rounded corners=1pt}
		}

		\node[font=\bfseries, blue!80!black] at (3, 5.5) {Iteração 2: $\Delta(\{v_1, v_2, v_3\}) = 2$};

		\node[vtxRoot] (r0) at (3, 4) {$r_0$};
		\node[vtx] (u1) at (0.5, 2.5) {$u_1$};
		\node[vtx] (u2) at (3, 2.5) {$u_2$};
		\node[vtx] (u3) at (5.5, 2.5) {$u_3$};
		\node[vtxX] (v1) at (1.5, 0.5) {$v_1$};
		\node[vtxX] (v2) at (3, 0.5) {$v_2$};
		\node[vtxX] (v3) at (4.5, 0.5) {$v_3$};

		% Arcos já justos
		\draw[arcTight] (r0) -- node[costTight, left, pos=0.3] {0} (u1);
		\draw[arcTight] (r0) -- node[costTight, right, pos=0.3] {0} (u3);
		\draw[arcTight] (u1) to[bend left=12] node[costTight, above] {0} (u2);

		% Arcos entrando em X (antes de ficarem justos)
		\draw[arc] (u1) -- node[cost, left, fill=orange!10] {3→1} (v1);
		\draw[arc] (u2) -- node[cost, right, fill=orange!10] {2→0} (v2);
		\draw[arc] (u3) -- node[cost, right] {4→2} (v3);
		\draw[arc] (u1) to[bend right=15] node[cost, below, pos=0.35] {5→3} (v2);

		\node[font=\small, orange!80!black] at (3, -0.5) {Após subtração: $(u_2, v_2)$ torna-se justo};
	\end{tikzpicture}
	\caption{Fase~1, Iteração 2: Para o conjunto \(X = \{v_1, v_2, v_3\}\) (em laranja), calcula-se \(\Delta(X) = \min\{3, 2, 4, 5\} = 2\). Após subtrair esse valor dos arcos entrando em \(X\), o arco \((u_2, v_2)\) atinge custo reduzido 0 e é adicionado a \(A_0\). Note que \((u_1, u_2)\) também se tornou justo na mesma iteração ao processar \(X = \{u_2\}\).}
	\label{fig:frank-exemplo-iter2}
\end{figure}

\paragraph*{Fase 1 — Estado final:} Após mais iterações elevando potenciais para os conjuntos restantes, o dígrafo \(D_0\) de arcos justos fica completo, contendo uma r-arborescência. A Figura~\ref{fig:frank-exemplo-A0-final} mostra o conjunto final \(A_0\) com 6 arcos de custo reduzido zero.

\begin{figure}[H]
	\centering
	\begin{tikzpicture}[>=Stealth, scale=0.85, every node/.style={scale=0.85}]
		\tikzset{
			vtx/.style={circle, draw, fill=blue!8, thick, inner sep=1.2pt, minimum size=6.5mm},
			vtxRoot/.style={circle, draw, fill=red!15, thick, inner sep=1.2pt, minimum size=6.5mm},
			arcTight/.style={->, ultra thick, blue!70!black},
			costTight/.style={font=\scriptsize\bfseries, fill=blue!10, inner sep=2pt, draw=blue!70!black, rounded corners=1pt}
		}

		\node[font=\bfseries, blue!80!black] at (3, 4.5) {$A_0$ final: arcos justos após Fase~1};

		\node[vtxRoot] (r0) at (3, 3) {$r_0$};
		\node[vtx] (u1) at (0.5, 1.5) {$u_1$};
		\node[vtx] (u2) at (3, 1.5) {$u_2$};
		\node[vtx] (u3) at (5.5, 1.5) {$u_3$};
		\node[vtx] (v1) at (1.5, 0) {$v_1$};
		\node[vtx] (v2) at (3, 0) {$v_2$};
		\node[vtx] (v3) at (4.5, 0) {$v_3$};

		\draw[arcTight] (r0) -- node[costTight, left, pos=0.3] {0} (u1);
		\draw[arcTight] (r0) -- node[costTight, right, pos=0.3] {0} (u3);
		\draw[arcTight] (u1) to[bend left=12] node[costTight, above] {0} (u2);
		\draw[arcTight] (u1) -- node[costTight, left] {0} (v1);
		\draw[arcTight] (u2) -- node[costTight, right] {0} (v2);
		\draw[arcTight] (v2) -- node[costTight, above] {0} (v3);

		\node[font=\small, blue!70!black] at (3, -1.2) {$|A_0| = 6$ arcos com custo reduzido $c_y = 0$};
	\end{tikzpicture}
	\caption{Conjunto final \(A_0\) de arcos justos após conclusão da Fase~1. Todos os 6 arcos \textcolor{blue!70!black}{destacados} possuem custo reduzido zero: \((r_0, u_1)\), \((r_0, u_3)\), \((u_1, u_2)\), \((u_1, v_1)\), \((u_2, v_2)\), \((v_2, v_3)\). O dígrafo formado por esses arcos é acíclico e alcança todos os vértices a partir de \(r_0\), satisfazendo as condições para prosseguir à Fase~2.}
	\label{fig:frank-exemplo-A0-final}
\end{figure}

\paragraph*{Fase 2 — Construção da arborescência:} Com \(A_0\) completo e acíclico, a Fase~2 constrói incrementalmente a arborescência final. Inicia-se com \(\texttt{Arb} = \{r_0\}\) e em cada iteração adiciona-se um arco \((u,v) \in A_0\) tal que \(u \in \texttt{Arb}\) e \(v \notin \texttt{Arb}\). A Figura~\ref{fig:frank-exemplo-arborescencia-final} mostra a arborescência resultante.

\begin{figure}[H]
	\centering
	\begin{tikzpicture}[>=Stealth, scale=0.85, every node/.style={scale=0.85}]
		\tikzset{
			vtx/.style={circle, draw, fill=green!15, thick, inner sep=1.2pt, minimum size=6.5mm},
			vtxRoot/.style={circle, draw, fill=red!15, thick, inner sep=1.2pt, minimum size=6.5mm},
			arcArb/.style={->, ultra thick, green!70!black},
			costArb/.style={font=\scriptsize\bfseries, fill=green!10, inner sep=2pt, draw=green!70!black, rounded corners=1pt}
		}

		\node[font=\bfseries, blue!80!black] at (3, 4.5) {Arborescência final (Fase~2)};

		\node[vtxRoot] (r0) at (3, 3) {$r_0$};
		\node[vtx] (u1) at (0.5, 1.5) {$u_1$};
		\node[vtx] (u2) at (3, 1.5) {$u_2$};
		\node[vtx] (u3) at (5.5, 1.5) {$u_3$};
		\node[vtx] (v1) at (1.5, 0) {$v_1$};
		\node[vtx] (v2) at (3, 0) {$v_2$};
		\node[vtx] (v3) at (4.5, 0) {$v_3$};

		\draw[arcArb] (r0) -- node[costArb, left, pos=0.3] {1} (u1);
		\draw[arcArb] (r0) -- node[costArb, right, pos=0.3] {1} (u3);
		\draw[arcArb] (u1) to[bend left=12] node[costArb, above] {1} (u2);
		\draw[arcArb] (u1) -- node[costArb, left] {3} (v1);
		\draw[arcArb] (u2) -- node[costArb, right] {2} (v2);
		\draw[arcArb] (v2) -- node[costArb, above] {2} (v3);

		\node[font=\small, green!70!black] at (3, -1.2) {Custo total: $1+1+1+3+2+2 = 10$};
	\end{tikzpicture}
	\caption{Arborescência de custo mínimo final obtida pela Fase~2. Os pesos mostrados são os \emph{originais} de \(D\), restaurados após a construção. Cada vértice não-raiz possui exatamente um arco de entrada, não há ciclos, e todos os vértices são alcançáveis a partir de \(r_0\). A verificação de otimalidade dual confirma que cada conjunto \(X\) que teve potenciais elevados possui exatamente um arco da arborescência cruzando sua fronteira, garantindo que a solução é ótima.}
	\label{fig:frank-exemplo-arborescencia-final}
\end{figure}

\paragraph*{Verificação de otimalidade dual:} A função \texttt{check\_dual\_optimality\_condition} confirma que para cada par \((X, \Delta(X))\) em \texttt{Dual\_list} (conjuntos cujos potenciais foram elevados com \(\Delta(X) > 0\)), existe exatamente um arco da arborescência final cruzando a fronteira de \(X\). Essa condição, juntamente com os arcos justos, garante que as condições de folga complementar da programação linear dual são satisfeitas, implicando que a arborescência encontrada é de custo mínimo global.

\subsection{Correspondência entre teoria e implementação}

A implementação em Python do algoritmo de András Frank segue fielmente a descrição teórica primal-dual apresentada anteriormente. A tabela abaixo estabelece o paralelo direto entre os passos teóricos e sua realização no código:

\begin{table}[H]
	\centering
	\scriptsize
	\renewcommand{\arraystretch}{1.3}
	\begin{tabular}{>{\raggedright\arraybackslash}p{0.44\textwidth}|>{\raggedright\arraybackslash}p{0.48\textwidth}}
		\toprule
		\textbf{\normalsize Descrição Teórica} & \textbf{\normalsize Implementação Python} \\
		\midrule
		\rowcolor{blue!5}
		\textbf{Passo 1:} Inicialização

		Defina \(y(v) := 0\) para todo \(v \in V\).

		Inicialize \(A_0 := \emptyset\).

		Construa dígrafo vazio \(D_0\) (arcos justos).
		                                       &
		\textbf{Função \texttt{phase1} — Linhas 2--5:}

		\texttt{D\_copy = D\_original.copy()}

		\texttt{A\_zero = []}

		\texttt{D\_zero = build\_D\_zero(D\_copy)}

		\vspace{1mm}
		Potenciais \(y(v) = 0\) implícitos, custos \(c_y = c\).
		\\
		\midrule
		\rowcolor{green!5}
		\textbf{Passo 2:} Elevação de potenciais

		Enquanto \(\exists X \subseteq V \setminus \{r\}\) sem arco justo:

		\quad Calcule \(\Delta(X) := \min\{c_y(u,v) : u \notin X, v \in X\}\)

		\quad Atualize \(y(v) := y(v) + \Delta(X)\), \(\forall v \in X\)

		\quad Adicione arcos com \(c_y = 0\) a \(A_0\)
		                                       &
		\textbf{Loop principal — Linhas 7--22:}

		\texttt{C = nx.condensation(D\_zero)}

		\texttt{sources = [x for x in C.nodes()]}

		\texttt{\quad if C.in\_degree(x) == 0]}

		Para fonte \texttt{u} (exceto raiz):

		\texttt{\quad X = C.nodes[u]["members"]}

		\texttt{\quad arcs = get\_arcs\_entering\_X(D, X)}

		\texttt{\quad min\_w = get\_minimum\_weight\_cut(arcs)}

		\texttt{\quad update\_weights\_in\_X(D, arcs, min\_w,}

		\texttt{\quad\quad A\_zero, D\_zero)}
		\\
		\midrule
		\rowcolor{orange!5}
		\textbf{Passo 2(a):} Identificar arcos entrando

		Determine \(\{(u,v) \in A : u \notin X, v \in X\}\)
		                                       &
		\textbf{Função \texttt{get\_arcs\_entering\_X}:}

		\texttt{return [(u, v, data)}

		\texttt{\quad for u, v, data in D.edges(data=True)}

		\texttt{\quad if u not in X and v in X]}
		\\
		\midrule
		\rowcolor{purple!5}
		\textbf{Passo 2(b):} Calcular \(\Delta(X)\)

		\(\Delta(X) := \min\{c_y(u,v) : u \notin X, v \in X\}\)
		                                       &
		\textbf{Função \texttt{get\_minimum\_weight\_cut}:}

		\texttt{return min(data["w"]}

		\texttt{\quad for \_, \_, data in arcs)}
		\\
		\midrule
		\rowcolor{yellow!5}
		\textbf{Passo 2(c):} Atualizar pesos

		Para \((u,v)\) entrando em \(X\):

		\quad \(c_y(u,v) := c_y(u,v) - \Delta(X)\)

		\quad Se \(c_y(u,v) = 0\), adicione a \(A_0\)
		                                       &
		\textbf{Função \texttt{update\_weights\_in\_X}:}

		\texttt{for u, v, \_ in arcs:}

		\texttt{\quad D[u][v]["w"] -= min\_weight}

		\texttt{\quad if D[u][v]["w"] == 0:}

		\texttt{\quad\quad A\_zero.append((u, v))}

		\texttt{\quad\quad D\_zero.add\_edge(u, v)}
		\\
		\midrule
		\rowcolor{cyan!5}
		\textbf{Passo 3:} Verificar término

		Se \(D_0\) contém r-arborescência, encerre.
		                                       &
		\textbf{Condição — Linhas 11--12:}

		\texttt{if len(sources) == 1: break}

		\vspace{1mm}
		Uma fonte \(\Rightarrow\) r-arborescência acíclica.
		\\
		\midrule
		\rowcolor{red!5}
		\textbf{Passo 4:} Construir arborescência

		Construa \(F\) a partir de \(A_0\), conectando vértices incrementalmente.
		                                       &
		\textbf{Função \texttt{phase2} (incremental):}

		\texttt{Arb = nx.DiGraph(); Arb.add\_node(r0)}

		\texttt{for \_ in range(n - 1):}

		\texttt{\quad for u, v in A\_zero:}

		\texttt{\quad\quad if u in Arb and v not in Arb:}

		\texttt{\quad\quad\quad Arb.add\_edge(u, v, **data)}

		\texttt{\quad\quad\quad break}

		\vspace{1mm}
		Complexidade: \(O(nm)\).
		\\
		\midrule
		\rowcolor{red!5}
		\textbf{Passo 4:} Versão otimizada

		Mesma ideia, com fila de prioridade.
		                                       &
		\textbf{Função \texttt{phase2\_v2} (BFS):}

		\texttt{Arb = nx.DiGraph()}

		\texttt{for i, (u, v) in enumerate(A\_zero):}

		\texttt{\quad Arb.add\_edge(u, v, w=i)}

		\texttt{q = [] \# fila de prioridade}

		\texttt{while q:}

		\texttt{\quad \_, u, v = heapq.heappop(q)}

		\texttt{\quad if v in V: continue}

		\texttt{\quad A.add\_edge(u, v, w=D[u][v]["w"])}

		\vspace{1mm}
		Complexidade: \(O(m \log m)\).
		\\
		\midrule
		\rowcolor{pink!5}
		\textbf{Otimalidade dual}

		Para cada \(X\) elevado (\(\Delta(X) > 0\)), exatamente um arco de \(F\) cruza \(\delta^-(X)\).
		                                       &
		\textbf{Função \texttt{check\_dual\_optimality}:}

		\texttt{for X, z in Dual\_list:}

		\texttt{\quad count = 0}

		\texttt{\quad for u, v in Arb.edges():}

		\texttt{\quad\quad if u not in X and v in X:}

		\texttt{\quad\quad\quad count += 1}

		\texttt{\quad\quad\quad if count > 1: return False}

		\texttt{return True}
		\\
		\bottomrule
	\end{tabular}
	\caption{Correspondência entre a descrição teórica do algoritmo de András Frank e sua implementação em Python. Cores: inicialização (azul), elevação de potenciais (verde/laranja/roxo/amarelo), verificação (ciano), construção (vermelho) e validação dual (rosa).}
	\label{tab:frank-teoria-implementacao}
\end{table}

Esta correspondência demonstra que a implementação traduz fielmente a abordagem primal-dual em código executável. As funções auxiliares (\texttt{get\_arcs\_entering\_X}, \texttt{get\_minimum\_weight\_cut}, \texttt{update\_weights\_in\_X}, \texttt{phase1}, \texttt{phase2}, \texttt{phase2\_v2},
\texttt{check\_dual\_}-
\texttt{optimality\_condition} encapsulam exatamente as operações descritas na teoria, preservando as propriedades de correção e as garantias de otimalidade do algoritmo original. A utilização de componentes fortemente conexas (\texttt{nx.condensation}) para identificar conjuntos sem arcos justos entrando é uma implementação eficiente da verificação teórica, evitando enumeração explícita de todos os subconjuntos de vértices.

\chapter{Algoritmo de András Frank}

Neste capítulo, apresentaremos o algoritmo de András Frank, que também determina
uma arborescência de custo mínimo em um digrafo ponderado.
O algoritmo baseia-se em uma abordagem primal--dual:
a elevação gulosa de potenciais duais\footnote{\textbf{Potencial dual} é uma variável associada a cada vértice, que representa um "desconto" aplicado ao custo dos arcos que entram nesse vértice e servem para ajustar os custos de modo que certas restrições fiquem "justas", - satisfeitas com igualdade, ou seja, o valor da variável atinge exatamente o limite imposto pela restrição.} e (ii) a construção de uma solução gulosa
a partir dos arcos de custo reduzido zero.
Diferentemente do algoritmo de Chu--Liu--Edmonds, que opera diretamente sobre os custos,
o método de Frank manipula potenciais duais associados aos vértices,
gerando arcos \emph{justos} (de custo reduzido zero) que formam a arborescência ótima.
O propósito deste capítulo é fornecer uma descrição precisa tanto do algoritmo
quanto da implementação desenvolvida neste trabalho.

\section{O algoritmo}
O algoritmo de András Frank também recebe uma tripla \((D,c,r)\), em que \(D=(V,A)\) é um digrafo,
\(c\colon A\to\mathbb{R}\) é uma função custo e \(r\in V\) é a raiz,
sob a hipótese de que \(D\) admite ao menos uma \(r\)-arborescência e devolve uma \(r\)-arborescência \(c\)-mínima de \(D\).

Assim como no capítulo anterior, adotamos a terminologia de \textbf{\(r\)-digrafo ponderado}
para uma tripla \((D,c,r)\) em que $(D, c)$ é um digrafo ponderado,
$r$ é um vértice de $D$, $\delta^-(r) = \varnothing$
e $D$ possui uma $r$-arborescência.

Vamos desenvolver as ideias do algoritmo utilizando o mesmo dígrafo que apresentamos
no capítulo anterior.

\subsection*{Abordagem Gulosa--Dual}

Considere novamente o dígrafo \(D\) figura 26 a seguir, com custos nos arcos note que é o mesmo digrafo da figura 1 que utilizamos para ilustrar o algoritmo de Chu--Liu--Edmonds. E portanto, suponha que T é uma arborescência de custo mínimo, com seus arcos destacados em azul.

\begin{figure}[h]
	\centering
	\begin{tikzpicture}[>=Stealth]
		% estilos
		\tikzset{
			vtx/.style={circle, draw,fill=blue!8, thick, inner sep=1.2pt, minimum size=6mm},
			chosen/.style={->, very thick, draw=blue},
			other/.style={->, semithick, draw=black!35},
			costB/.style={font=\scriptsize, fill=white, inner sep=1pt, text=blue},
			costG/.style={font=\scriptsize, fill=white, inner sep=1pt, text=black!55}
		}

		% layout (mais espaçado)
		\node[vtx] (r) at (0,0) {$r$};
		\node[vtx] (a) at (3,1.6) {$a$};
		\node[vtx] (b) at (3,-1.6) {$b$};
		\node[vtx] (c) at (6,2.2) {$c$};
		\node[vtx] (d) at (6,0.0) {$d$};
		\node[vtx] (e) at (6,-2.2) {$e$};

		% --- arcos NÃO escolhidos (cinza), agora com curvaturas mais limpas ---
		\draw[other] (b) to[bend left=12] node[costG, above] {4} (a);   % b->a
		\draw[other] (a) to[bend left=14] node[costG, below] {5} (b);   % a->b

		% r->c: externo por cima, suave (pontos de controle explícitos)
		\draw[other]
		(r) .. controls (0.8,2.4) and (5.2,3.0) ..
		node[costG, yshift=4pt, pos=0.53] {3} (c);

		% a->d (referência)
		\draw[other] (a) -- node[costG, right] {3} (d);

		% c->e: leve curvatura pela direita (sem "barriga")
		\draw[other]
		(c) .. controls (7.6,2.6) and (7.6,-2.6) ..
		node[costG, xshift=8pt, pos=0.52] {5} (e);

		% --- arcos ESCOLHIDOS (AZUL) — formam a r-arborescência T ---
		\draw[chosen] (r) -- node[costB, above] {1} (a);   % r->a
		\draw[chosen] (r) -- node[costB, below] {2} (b);   % r->b
		\draw[chosen] (a) -- node[costB, above] {1} (c);   % a->c
		\draw[chosen] (b) -- node[costB, above] {1} (d);   % b->d
		\draw[chosen] (b) -- node[costB, below] {2} (e);   % b->e
	\end{tikzpicture}

	\caption{O dígrafo \(D\) com custos nos arcos. Os arcos em \textcolor{blue}{azul}
		formam a \(r\)-arborescência de custo mínimo \(T\) (custo total = 7).}
\end{figure}

O algoritmo de Frank utiliza \textbf{potenciais} \(y: V \to \mathbb{R}\) associados aos vértices
para definir custos reduzidos nos arcos. Para cada arco \(a = (u,v) \in A\), o
\textbf{custo \(y\)-reduzido} é dado por
\[
	c_y(u,v) := c(u,v) - y(v),
\]
onde \(v\) é o vértice de destino do arco. Note que, diferentemente do procedimento de redução do capítulo anterior (onde subtraíamos o mínimo custo de entrada em cada vértice), aqui o potencial \(y(v)\) é uma variável que será ajustada pelo algoritmo através da seguinte noção.

Um arco \(a = (u,v)\) é dito \textbf{justo} (ou \textbf{tight}) se \(c_y(u,v) = 0\), isto é, se o custo original coincide com o potencial do vértice de destino: \(c(u,v) = y(v)\). A ideia central do algoritmo de Frank consiste em duas etapas: (i)~elevar os potenciais \(y(v)\) para cada \(v \neq r\) até que cada vértice possua ao menos um arco justo entrando nele; (ii)~construir uma arborescência ótima utilizando exclusivamente arcos justos.

\begin{tcolorbox}[
		enhanced, breakable,
		colback=blue!5!white, colframe=blue!70!black,
		colbacktitle=blue!85!black, coltitle=white,
		fonttitle=\bfseries,
		title={Arcos justos: interpretação e cálculo},
		sharp corners, boxrule=0.8pt
	]
	Um arco \((u,v)\) é \textbf{justo} quando seu custo reduzido é zero, ou seja, quando o custo original \(c(u,v)\) coincide exatamente com o potencial \(y(v)\) do vértice de destino. Formalmente:
	\[
		c_y(u,v) = c(u,v) - y(v) = 0 \quad\Longleftrightarrow\quad c(u,v) = y(v).
	\]

	Em outras palavras, o potencial \(y(v)\) funciona como um ``desconto'' aplicado aos arcos que chegam em \(v\). Quando esse desconto é suficiente para anular o custo reduzido, o arco torna-se justo e candidato a participar da solução ótima.

	\medskip
	\textbf{Exemplos numéricos (referência: figura~26):}
	\begin{itemize}\setlength{\itemsep}{3pt}
		\item Arco \((r,a)\): \(c(r,a) = 1\), \(y(a) = 1 \implies c_y(r,a) = 1-1 = 0\). \quad\textcolor{blue!70!black}{\textbf{Justo.}}
		\item Arco \((b,a)\): \(c(b,a) = 4\), \(y(a) = 1 \implies c_y(b,a) = 4-1 = 3\). \quad\textcolor{red!70!black}{\textbf{Não justo.}}
		\item Arco \((a,c)\): \(c(a,c) = 1\), \(y(c) = 1 \implies c_y(a,c) = 1-1 = 0\). \quad\textcolor{blue!70!black}{\textbf{Justo.}}
	\end{itemize}

	\textbf{Propriedade fundamental:} o potencial \(y(v)\) afeta \emph{exclusivamente} os arcos que entram em \(v\). Por exemplo, \(y(c)\) influencia apenas arcos da forma \((\cdot, c)\), como \((r,c)\) e \((a,c)\), não afetando arcos que saem de \(c\).

	\medskip
	O algoritmo eleva progressivamente os potenciais até que cada vértice \(v \neq r\) possua ao menos um arco justo de entrada. No exemplo da figura~26, os arcos justos após a elevação correspondem precisamente aos arcos da arborescência ótima.
\end{tcolorbox}

% Figura 27: Exemplo visual com múltiplos arcos e vértice raiz
\begin{figure}[H]
	\centering
	\begin{tikzpicture}[>=Stealth]
		% estilos
		\tikzset{
			vtx/.style={circle, draw, fill=blue!8, thick, inner sep=1.2pt, minimum size=8mm},
			arc/.style={->, very thick, draw=blue},
			arcTight/.style={->, very thick, draw=red!70!black},
			cost/.style={font=\scriptsize, fill=white, inner sep=1pt, text=blue},
			pot/.style={font=\small, text=red!70!black},
			label/.style={font=\small, text=black!80}
		}

		% vértices
		\node[vtx] (r) at (0,0) {$r$};
		\node[vtx] (u) at (3,1.5) {$u$};
		\node[vtx] (v) at (3,-1.5) {$v$};

		% arcos
		\draw[arcTight] (r) -- node[cost, above left] {$c(r,u)=2$} (u);
		\draw[arc] (r) -- node[cost, below left] {$c(r,v)=4$} (v);
		\draw[arcTight] (u) -- node[cost, right] {$c(u,v)=3$} (v);

		% potenciais
		\node[pot] at (3,2.5) {$y(u)=2$};
		\node[pot] at (3,-2.5) {$y(v)=3$};

		% destaques
		\draw[red!70, thick, dashed] (u) circle [radius=0.7];
		\draw[red!70, thick, dashed] (v) circle [radius=0.7];

		% custos reduzidos
		\node[label] at (-0.0,1.5) {$c_y(r,u)=2-2=0$};
		\node[label] at (-0.0,-1.5) {$c_y(r,v)=4-3=1$};
		\node[label] at (5.0,-0.5) {$c_y(u,v)=3-3=0$};
	\end{tikzpicture}
	\caption{O vértice $r$ é a raiz. Os potenciais são $y(u)=2$ e $y(v)=3$. O arco $(r,u)$ tem custo reduzido $0$ (justo), $(r,v)$ tem custo reduzido $1$ (não justo), $(u,v)$ tem custo reduzido $0$ (justo). Os potenciais só afetam os arcos que chegam em cada vértice.}
	\label{fig:potencial-destino}
\end{figure}

\subsection*{Elevação de Potenciais}

Inicialmente, definimos \(y(v) = 0\) para todo \(v \in V\).
Para cada vértice \(v \neq r\), calculamos
\[
	\Delta(v) := \min\{\, c(a) : a \in \delta^{-}(v) \,\} - y(v).
\]
O valor \(\Delta(v)\) representa quanto precisamos elevar \(y(v)\) para que exista
ao menos um arco de custo reduzido zero entrando em \(v\).

Vamos ilustrar esse processo no dígrafo \(D\).

\begin{figure}[H]
	\centering
	\begin{tikzpicture}[>=Stealth]
		% estilos
		\tikzset{
			vtx/.style={circle, draw,fill=blue!8, thick, inner sep=1.2pt, minimum size=6mm},
			arc/.style={->, semithick, draw=black!35},
			costG/.style={font=\scriptsize, fill=white, inner sep=1pt, text=black!55}
		}

		% layout
		\node[vtx] (r) at (0,0) {$r$};
		\node[vtx] (a) at (3,1.6) {$a$};
		\node[vtx] (b) at (3,-1.6) {$b$};
		\node[vtx] (c) at (6,2.2) {$c$};
		\node[vtx] (d) at (6,0.0) {$d$};
		\node[vtx] (e) at (6,-2.2) {$e$};

		% arcos com custos originais
		\draw[arc] (r) -- node[costG, above] {1} (a);
		\draw[arc] (r) -- node[costG, below] {2} (b);
		\draw[arc] (r) .. controls (0.8,2.4) and (5.2,3.0) .. node[costG, yshift=4pt, pos=0.53] {3} (c);
		\draw[arc] (b) to[bend left=12] node[costG, above] {4} (a);
		\draw[arc] (a) to[bend left=14] node[costG, below] {5} (b);
		\draw[arc] (a) -- node[costG, above] {1} (c);
		\draw[arc] (a) -- node[costG, right] {3} (d);
		\draw[arc] (b) -- node[costG, above] {1} (d);
		\draw[arc] (b) -- node[costG, below] {2} (e);
		\draw[arc] (c) .. controls (7.6,2.6) and (7.6,-2.6) .. node[costG, xshift=8pt, pos=0.52] {5} (e);

		% potenciais iniciais
		\node[above=6mm of a, font=\small] {$y(a) = 0$};
		\node[below=2mm of b, font=\small] {$y(b) = 0$};
		\node[above=2mm of c, font=\small] {$y(c) = 0$};
		\node[below=2mm of d, font=\small] {$y(d) = 0$};
		\node[below=2mm of e, font=\small] {$y(e) = 0$};
	\end{tikzpicture}
	\caption{Dígrafo \(D\) com potenciais iniciais \(y(v) = 0\) para todo \(v \in V\).}
\end{figure}

Calculamos \(\Delta(v)\) para cada vértice \(v \neq r\):
\begin{align*}
	\Delta(a) & = \min\{1, 4\} - 0 = 1, \\
	\Delta(b) & = \min\{2, 5\} - 0 = 2, \\
	\Delta(c) & = \min\{3, 1\} - 0 = 1, \\
	\Delta(d) & = \min\{3, 1\} - 0 = 1, \\
	\Delta(e) & = \min\{2, 5\} - 0 = 2.
\end{align*}

Elevamos cada potencial:
\begin{align*}
	y(a) & \leftarrow y(a) + \Delta(a) = 0 + 1 = 1, \\
	y(b) & \leftarrow y(b) + \Delta(b) = 0 + 2 = 2, \\
	y(c) & \leftarrow y(c) + \Delta(c) = 0 + 1 = 1, \\
	y(d) & \leftarrow y(d) + \Delta(d) = 0 + 1 = 1, \\
	y(e) & \leftarrow y(e) + \Delta(e) = 0 + 2 = 2.
\end{align*}

Após a elevação, os custos reduzidos são dados por \(c_y(u,v) = c(u,v) - y(v)\) para cada arco \((u,v)\):

\begin{figure}[H]
	\centering
	\begin{tikzpicture}[>=Stealth]
		% estilos
		\tikzset{
			vtx/.style={circle, draw,fill=blue!8, thick, inner sep=1.2pt, minimum size=6mm},
			tight/.style={->, very thick, draw=blue},
			loose/.style={->, semithick, draw=black!35},
			costB/.style={font=\scriptsize, fill=white, inner sep=1pt, text=blue},
			costG/.style={font=\scriptsize, fill=white, inner sep=1pt, text=black!55}
		}

		% layout
		\node[vtx] (r) at (0,0) {$r$};
		\node[vtx] (a) at (3,1.6) {$a$};
		\node[vtx] (b) at (3,-1.6) {$b$};
		\node[vtx] (c) at (6,2.2) {$c$};
		\node[vtx] (d) at (6,0.0) {$d$};
		\node[vtx] (e) at (6,-2.2) {$e$};

		% arcos justos (custo reduzido = 0)
		\draw[tight] (r) -- node[costB, above] {0} (a);   % 1 - 1 = 0
		\draw[tight] (r) -- node[costB, below] {0} (b);   % 2 - 2 = 0
		\draw[tight] (a) -- node[costB, above] {0} (c);   % 1 - 1 = 0
		\draw[tight] (b) -- node[costB, above] {0} (d);   % 1 - 1 = 0
		\draw[tight] (b) -- node[costB, below] {0} (e);   % 2 - 2 = 0

		% arcos não justos
		\draw[loose] (r) .. controls (0.8,2.4) and (5.2,3.0) .. node[costG, yshift=4pt, pos=0.53] {2} (c);  % 3 - 1 = 2
		\draw[loose] (b) to[bend left=12] node[costG, above] {3} (a);   % 4 - 1 = 3
		\draw[loose] (a) to[bend left=14] node[costG, below] {3} (b);   % 5 - 2 = 3
		\draw[loose] (a) -- node[costG, right] {2} (d);   % 3 - 1 = 2
		\draw[loose] (c) .. controls (7.6,2.6) and (7.6,-2.6) .. node[costG, xshift=8pt, pos=0.52] {3} (e);  % 5 - 2 = 3

		% potenciais finais
		\node[above=6mm of a, font=\small] {$y(a) = 1$};
		\node[below=2mm of b, font=\small] {$y(b) = 2$};
		\node[above=2mm of c, font=\small] {$y(c) = 1$};
		\node[below=2mm of d, font=\small] {$y(d) = 1$};
		\node[below=2mm of e, font=\small] {$y(e) = 2$};
	\end{tikzpicture}
	\caption{Dígrafo após elevação de potenciais. Arcos em \textcolor{blue}{azul} são
		justos (\(c_y = 0\)); arcos em cinza têm custo reduzido positivo.}
\end{figure}

Observe que cada vértice \(v \neq r\) possui agora ao menos um arco justo entrando nele.
Mais ainda: neste exemplo, os arcos justos formam uma \(r\)-arborescência! Precisamos justificar por que a arborescência formada pelos arcos justos é ótima.

\subsection*{Construção da Arborescência}

Se os arcos justos formam uma \(r\)-arborescência, então encontramos a solução ótima.
Caso contrário, os arcos justos contêm ciclos, e precisamos tratá-los adequadamente.

No nosso exemplo, os arcos justos são:
\[
	H = \{\, (r,a), (r,b), (a,c), (b,d), (b,e) \,\}.
\]
Esses arcos formam a \(r\)-arborescência \(T\) de custo mínimo.

Para verificar a otimalidade, note que uma propriedade fundamental dos potenciais duais é:

\begin{proposicao}
	Para toda função \(y: V \to \mathbb{R}\) com \(y(r) = 0\),
	uma \(r\)-arborescência \(T\) é \(c\)-mínima em \(D\) se, e somente se,
	todos os arcos de \(T\) são justos (i.e., \(c_y(a) = 0\) para todo \(a \in T\))
	e \(c_y(a) \geq 0\) para todo \(a \in A\).
\end{proposicao}

\begin{proof}[Prova.]
	Seja \(F\) uma \(r\)-arborescência qualquer. Para cada \(v \in V \setminus \{r\}\),
	seja \(a_v\) o único arco de \(F\) que entra em \(v\). Então
	\begin{align*}
		c(F)
		 & = \sum_{v \in V \setminus \{r\}} c(a_v)                                          \\
		 & = \sum_{v \in V \setminus \{r\}} \bigl(c_y(a_v) + y(v)\bigr)                     \\
		 & = \sum_{v \in V \setminus \{r\}} c_y(a_v) + \sum_{v \in V \setminus \{r\}} y(v).
	\end{align*}
	Como a soma \(\sum_{v \in V \setminus \{r\}} y(v)\) é constante para todas as
	\(r\)-arborescências, minimizar \(c(F)\) equivale a minimizar \(\sum_{v} c_y(a_v)\).
	Se todos os arcos de \(T\) são justos e \(c_y(a) \geq 0\) para todo \(a\),
	então \(\sum_{v} c_y(a_v^T) = 0 \leq \sum_{v} c_y(a_v^F)\) para qualquer \(F\),
	provando que \(T\) é ótima.
\end{proof}

No nosso exemplo, todos os arcos de \(T\) são justos e \(c_y(a) \geq 0\) para todo
arco \(a \in A\), portanto \(T\) é a \(r\)-arborescência de custo mínimo. Evidentemente, nem sempre será o caso que os arcos justos formam uma arborescência e precisaremos tratar ciclos.

\subsection*{Tratamento de Ciclos}

Em geral, após a elevação de potenciais, os arcos justos podem conter ciclos.
Nesse caso, o algoritmo de Frank procede de forma similar ao de Chu--Liu--Edmonds:
contrai cada ciclo justo em um supervértice, resolve o problema recursivamente
no dígrafo contraído, e depois reexpande a solução.

Vamos ilustrar esse processo com um exemplo modificado. Suponha que adicionamos o arco
\((c, a)\) com custo 1 ao dígrafo original:

\begin{figure}[H]
	\centering
	\begin{tikzpicture}[>=Stealth]
		% estilos
		\tikzset{
			vtx/.style={circle, draw,fill=blue!8, thick, inner sep=1.2pt, minimum size=6mm},
			arc/.style={->, semithick, draw=black!35},
			costG/.style={font=\scriptsize, fill=white, inner sep=1pt, text=black!55}
		}

		% layout
		\node[vtx] (r) at (0,0) {$r$};
		\node[vtx] (a) at (3,1.6) {$a$};
		\node[vtx] (b) at (3,-1.6) {$b$};
		\node[vtx] (c) at (6,2.2) {$c$};
		\node[vtx] (d) at (6,0.0) {$d$};
		\node[vtx] (e) at (6,-2.2) {$e$};

		% arcos originais
		\draw[arc] (r) -- node[costG, above] {1} (a);
		\draw[arc] (r) -- node[costG, below] {2} (b);
		\draw[arc] (r) .. controls (0.8,2.4) and (5.2,3.0) .. node[costG, yshift=4pt, pos=0.53] {3} (c);
		\draw[arc] (b) to[bend left=12] node[costG, above] {4} (a);
		\draw[arc] (a) to[bend left=14] node[costG, below] {5} (b);
		\draw[arc] (a) -- node[costG, above] {1} (c);
		\draw[arc] (a) -- node[costG, right] {3} (d);
		\draw[arc] (b) -- node[costG, above] {1} (d);
		\draw[arc] (b) -- node[costG, below] {2} (e);
		\draw[arc] (c) .. controls (7.6,2.6) and (7.6,-2.6) .. node[costG, xshift=8pt, pos=0.52] {5} (e);

		% NOVO arco (c, a)
		\draw[arc, draw=red!60] (c) to[bend left=20] node[costG, left, text=red!60] {1} (a);
	\end{tikzpicture}
	\caption{Dígrafo modificado com o arco \((c,a)\) de custo 1 (em vermelho).}
\end{figure}

Após a elevação de potenciais com \(y(a) = 1, y(c) = 1\), o arco \((c,a)\) torna-se
justo: \(c_y(c,a) = 1 - 1 = 0\).

\begin{figure}[H]
	\centering
	\begin{tikzpicture}[>=Stealth]
		% estilos
		\tikzset{
			vtx/.style={circle, draw,fill=blue!8, thick, inner sep=1.2pt, minimum size=6mm},
			tight/.style={->, very thick, draw=blue},
			cycle/.style={->, ultra thick, draw=red!70},
			loose/.style={->, semithick, draw=black!35},
			costB/.style={font=\scriptsize, fill=white, inner sep=1pt, text=blue},
			costR/.style={font=\scriptsize, fill=white, inner sep=1pt, text=red!70},
			costG/.style={font=\scriptsize, fill=white, inner sep=1pt, text=black!55}
		}

		% layout
		\node[vtx] (r) at (0,0) {$r$};
		\node[vtx] (a) at (3,1.6) {$a$};
		\node[vtx] (b) at (3,-1.6) {$b$};
		\node[vtx] (c) at (6,2.2) {$c$};
		\node[vtx] (d) at (6,0.0) {$d$};
		\node[vtx] (e) at (6,-2.2) {$e$};

		% arcos justos que formam ciclo
		\draw[cycle] (a) -- node[costR, above] {0} (c);
		\draw[cycle] (c) to[bend left=20] node[costR, left] {0} (a);

		% outros arcos justos
		\draw[tight] (r) -- node[costB, above] {0} (a);
		\draw[tight] (r) -- node[costB, below] {0} (b);
		\draw[tight] (b) -- node[costB, above] {0} (d);
		\draw[tight] (b) -- node[costB, below] {0} (e);

		% arcos não justos
		\draw[loose] (r) .. controls (0.8,2.4) and (5.2,3.0) .. node[costG, yshift=4pt, pos=0.53] {2} (c);
		\draw[loose] (b) to[bend left=12] node[costG, above] {3} (a);
		\draw[loose] (a) to[bend left=14] node[costG, below] {3} (b);
		\draw[loose] (a) -- node[costG, right] {2} (d);   % 3 - 1 = 2
		\draw[loose] (c) .. controls (7.6,2.6) and (7.6,-2.6) .. node[costG, xshift=8pt, pos=0.52] {3} (e);  % 5 - 2 = 3
	\end{tikzpicture}
	\caption{Arcos justos após elevação. O ciclo \(C = (a, c, a)\) é destacado em
		\textcolor{red!70}{vermelho}.}
\end{figure}

Nesse caso, contraímos o ciclo \(C = (a, c)\) em um supervértice \(x_C\), ajustamos
os custos dos arcos incidentes, e resolvemos o problema no dígrafo contraído.
A reexpansão é feita removendo um arco do ciclo para manter a estrutura de arborescência,
de forma análoga ao algoritmo de Chu--Liu--Edmonds.

\section{Descrição do algoritmo}

O algoritmo de András Frank difere substancialmente do algoritmo de Chu--Liu--Edmonds em sua abordagem: enquanto Chu--Liu--Edmonds opera diretamente sobre os custos dos arcos, o método de Frank utiliza uma técnica gulosa--dual baseada em \emph{potenciais} associados aos vértices. A ideia central é elevar progressivamente os potenciais \(y(v)\) para cada vértice \(v\neq r\), de modo que os custos \emph{reduzidos} \(c_y(u,v) := c(u,v) - y(v)\) revelem quais arcos devem participar da solução ótima. Um arco \((u,v)\) é dito \emph{justo} quando seu custo reduzido é zero, isto é, quando \(c(u,v) = y(v)\). O algoritmo garante que toda r-arborescência ótima pode ser formada apenas por arcos justos.

A seguir apresentamos uma descrição formal do algoritmo de András Frank e detalhes de implementação serão discutidos na próxima seção.

\begin{algobox}{András Frank}{andras-frank}
	Entrada: dígrafo \(D=(V,A)\), custos \(c:A\to\mathbb{R}_{\ge 0}\), raiz \(r\).\footnote{Se algum \(v\neq r\) não possui arco de entrada, não existe r-arborescência.}

	\textbf{Fase 1: Elevação de potenciais e construção de \(A_0\)}
	\begin{enumerate}\setlength{\itemsep}{2pt}
		\item \textbf{Inicialização:} defina \(y(v) := 0\) para todo \(v \in V\).
		\item \textbf{Iteração:} enquanto existir conjunto \(X \subseteq V \setminus \{r\}\) sem arco justo entrando:
		      \begin{itemize}\setlength{\itemsep}{2pt}
			      \item Calcule \(\Delta(X) := \min\{c(u,v) - y(v) : u \notin X, v \in X\}\).
			      \item Para cada \(v \in X\), atualize \(y(v) := y(v) + \Delta(X)\).
		      \end{itemize}
		\item Defina \(A_0 := \{a \in A : c_y(a) = 0\}\), o conjunto de arcos justos.
	\end{enumerate}

	\textbf{Fase 2: Construção da arborescência}
	\begin{enumerate}\setlength{\itemsep}{2pt}\setcounter{enumi}{3}
		\item Se \((V,A_0)\) forma uma r-arborescência, devolva \(A_0\). Por otimalidade dos potenciais duais, trata-se de uma r-arborescência de custo mínimo.
		\item Caso contrário, identifique um ciclo dirigido \(C\) em \(A_0\) (com \(r\notin C\)). \textbf{Contração:} contraia \(C\) em um supervértice \(x_C\) e defina custos reduzidos \(c'\) por
		      \begin{align*}
			      c'(u,x_C) & := c_y(u,w) = c(u,w) - y(w) &  & \text{para } u\notin C,\ w\in C, \\
			      c'(x_C,v) & := c_y(w,v) = c(w,v) - y(v) &  & \text{para } w\in C,\ v\notin C,
		      \end{align*}
		      descartando laços em \(x_C\) e permitindo paralelos. Denote o dígrafo contraído por \(D'=(V',A')\).
		\item \textbf{Recursão:} compute uma r-arborescência ótima \(T'\) de \(D'\) com custos \(c'\).
		\item \textbf{Expansão:} seja \((u,x_C)\in T'\) o único arco que entra em \(x_C\). No dígrafo original, ele corresponde a \((u,w)\) com \(w\in C\). Forme
		      \[
			      T := \bigl(T'\setminus\{\text{arcos incidentes a } x_C\}\bigr)\ \cup\ \{(u,w)\}\ \cup\ \bigl((A_0\cap A(C))\setminus\{a_w\}\bigr),
		      \]
		      onde \(a_w\) é o arco de \(C\) que entra em \(w\). Então \(T\) é uma r-arborescência ótima de \(D\).
	\end{enumerate}
\end{algobox}

\subsection{Corretude}

A corretude do algoritmo de András Frank baseia-se na teoria de dualidade em programação linear aplicada ao problema de arborescência de custo mínimo:

\begin{enumerate}\setlength{\itemsep}{2pt}
	\item \emph{Potenciais duais:} os valores \(y(v)\) funcionam como variáveis duais. Para qualquer r-arborescência \(T\) e qualquer escolha de potenciais \(y\), vale
	      \[
		      c(T) = \sum_{a\in T} c(a) = \sum_{a\in T} c_y(a) + \sum_{v\neq r} y(v),
	      \]
	      pois há exatamente um arco de \(T\) entrando em cada \(v\neq r\). O termo \(\sum_{v\neq r} y(v)\) é constante após a elevação completa dos potenciais; assim, minimizar \(c(T)\) equivale a minimizar \(\sum_{a\in T} c_y(a)\).

	\item \emph{Condições de otimalidade:} uma r-arborescência \(T\) é ótima se e somente se:
	      \begin{itemize}\setlength{\itemsep}{2pt}
		      \item Todos os arcos de \(T\) são justos: \(c_y(a) = 0\) para todo \(a \in T\).
		      \item Todos os arcos do dígrafo têm custo reduzido não-negativo: \(c_y(a) \geq 0\) para todo \(a \in A\).
	      \end{itemize}
	      Sob essas condições, \(\sum_{a\in T} c_y(a) = 0 \leq \sum_{a\in T'} c_y(a)\) para qualquer outra r-arborescência \(T'\), provando otimalidade.

	\item \emph{Elevação de potenciais:} o algoritmo eleva os potenciais de forma gulosa, garantindo que ao final cada \(v\neq r\) tenha ao menos um arco justo entrando. Isso é feito identificando componentes fortemente conexas em \(A\_0\): conjuntos \(X\) que ainda não possuem arcos justos entrando têm seus potenciais elevados pelo mínimo necessário \(\Delta(X)\).

	\item \emph{Caso acíclico:} se \(A_0\) forma uma r-arborescência, todos os arcos são justos e, pelas condições de otimalidade, a solução é ótima.

	\item \emph{Caso com ciclo (contração/expansão):} se \(A\_0\) contém um ciclo dirigido \(C\), todos os seus arcos são justos (custo reduzido zero). A contração preserva a estrutura de custos reduzidos: no dígrafo contraído, os novos custos \(c'\) já incorporam os potenciais, e uma solução ótima em \(D'\) mapeia para uma solução ótima em \(D\) pela reversibilidade da operação.
\end{enumerate}

Em termos intuitivos, os potenciais \(y(v)\) funcionam como "descontos" aplicados aos arcos que entram em cada vértice, revelando quais arcos são "justos" (custo líquido zero) e, portanto, candidatos à solução ótima. Ciclos de arcos justos podem ser contraídos sem perder otimalidade, simplificando o problema recursivamente.

\subsection{Complexidade}

A implementação, baseada em componentes fortemente conexas, detecta em cada iteração, quais conjuntos \(X\) necessitam elevação de potenciais. Calcular componentes fortemente conexas custa \(O(n+m)\) usando algoritmos como Tarjan ou Kosaraju. Para cada componente (exceto a raiz), eleva-se o potencial calculando \(\Delta(X)\) em \(O(m)\), atualizando os custos reduzidos.

No pior caso, cada iteração reduz o número de componentes em pelo menos uma unidade, resultando em \(O(n)\) iterações. Cada iteração processa todos os arcos para atualizar custos reduzidos e recalcular componentes, resultando em \(O(nm)\) no total para a Fase~1. A Fase~2 constrói a arborescência percorrendo \(A_0\) uma vez, custando \(O(n)\).

O uso de memória é \(O(n+m)\), incluindo as estruturas para armazenar o dígrafo, potenciais e componentes. A implementação a seguir adota a versão \(O(nm)\) por simplicidade e está disponível no repositório do projeto (\url{https://github.com/lorenypsum/GraphVisualizer}).

\section{Implementação em Python}

Esta seção descreve a implementação do algoritmo de András Frank em Python, estruturada para refletir com precisão as duas fases formais discutidas anteriormente. A Fase~1 realiza a elevação de potenciais e identifica os arcos justos, enquanto a Fase~2 constrói a arborescência de custo mínimo a partir desses arcos. Utilizamos a biblioteca NetworkX para manipulação de digrafos, aproveitando suas funcionalidades para representar grafos, calcular componentes fortemente conexas e gerenciar atributos de arcos.

A entrada consiste em um dígrafo orientado \(D = (V, A)\), com custos dos arcos registrados no atributo \texttt{"w"}, e uma raiz \(r \in V\). As hipóteses adotadas são: (i) o dígrafo é conexo a partir de \(r\), isto é, todo vértice \(v \neq r\) é alcançável a partir da raiz; (ii) para todo subconjunto \(X \subseteq V \setminus \{r\}\), existe ao menos um arco entrando em \(X\); e (iii) todos os custos são não negativos.

A saída é um subdigrafo \(T\) de \(D\) com \(|A_T| = |V| - 1\) arcos, tal que cada vértice \(v \neq r\) possui grau de entrada igual a 1, todos os vértices são alcançáveis a partir de \(r\), e o custo total \(\sum_{a \in A_T} c(a)\) é mínimo.

A estrutura do código é modular: funções auxiliares tratam cada etapa do algoritmo — cálculo de componentes fortemente conexas, elevação de potenciais, construção do subdigrafo \(A_0\) e construção da arborescência final. Todas operam sobre objetos \texttt{nx.DiGraph} e são coordenadas por uma função principal que gerencia o fluxo das duas fases. As subseções seguintes detalham cada função auxiliar, abordando lógica, parâmetros, saídas e complexidade.

\subsection{Construção do dígrafo \texorpdfstring{\(D_0\)}{D_0} inicial}

Começamos escrevendo uma função que constrói o dígrafo inicial \(D_0\) que será utilizado na Fase~1 do algoritmo. O dígrafo \(D_0\) é inicializado como um grafo vazio contendo apenas os vértices do dígrafo original, sem arcos. Essa estrutura será gradualmente populada com arcos de custo reduzido zero à medida que os potenciais são elevados.

Recebe como entrada um dígrafo \texttt{D} (objeto \texttt{nx.DiGraph}). A implementação cria um novo dígrafo vazio \texttt{D\_zero} (linha 2) e adiciona todos os vértices de \texttt{D} a \texttt{D\_zero} (linhas 3-4), preservando a estrutura de vértices sem incluir arcos inicialmente.

A função devolve o dígrafo \texttt{D\_zero} contendo todos os vértices de \texttt{D} mas nenhum arco. O dígrafo original \texttt{D} não é modificado. A complexidade é \(O(n)\), onde \(n = |V|\), pois itera sobre todos os vértices uma única vez.

\begin{tcolorbox}[
		enhanced, breakable,
		colframe=blue!60!black, colback=blue!2,
		colbacktitle=blue!15, coltitle=black,
		title={Construção do dígrafo $D_0$ inicial},
		boxed title style={sharp corners, boxrule=0.6pt},
		sharp corners, boxrule=0.6pt
	]
	\emph{Constrói um dígrafo vazio contendo apenas os vértices de D, sem arcos. Este dígrafo será populado com arcos justos durante a elevação de potenciais.}
	\tcblower
	\begin{lstlisting}[language=Python]
def build_D_zero(D):
    D_zero = nx.DiGraph()
    for v in D.nodes():
        D_zero.add_node(v)
    return D_zero
\end{lstlisting}
\end{tcolorbox}

\subsection{Identificação de arcos entrando em conjunto \texorpdfstring{\(X\)}{X}}

Esta função auxiliar identifica todos os arcos que entram em um conjunto \(X \subseteq V\), isto é, arcos \((u,v)\) tais que \(u \notin X\) e \(v \in X\). Essa operação é fundamental para calcular o mínimo custo de entrada em \(X\) durante a elevação de potenciais.

Recebe como entrada um dígrafo \texttt{D} e um conjunto de vértices \texttt{X}. A implementação cria uma lista vazia \texttt{arcs} (linha 2) e itera sobre todos os arcos do dígrafo com seus dados (linha 3), incluindo o peso. Para cada arco \((u,v,\text{data})\), verifica se \(u \notin X\) e \(v \in X\) (linha 4), adicionando à lista apenas os arcos que cruzam a fronteira de \(X\) (linha 5).

A função devolve uma lista de tuplas \((u, v, \text{data})\) representando os arcos que entram em \(X\), onde \texttt{data} contém o atributo \texttt{"w"} com o peso do arco. A complexidade é \(O(m)\), onde \(m = |A|\), pois examina cada arco uma vez.

\begin{tcolorbox}[
		enhanced, breakable,
		colframe=blue!60!black, colback=blue!2,
		colbacktitle=blue!15, coltitle=black,
		title={Identificação de arcos entrando em conjunto X},
		boxed title style={sharp corners, boxrule=0.6pt},
		sharp corners, boxrule=0.6pt
	]
	\emph{Identifica todos os arcos $(u,v)$ do dígrafo D tais que $u \notin X$ e $v \in X$, devolvendo uma lista com as tuplas $(u, v, data)$ onde data contém o peso do arco.}
	\tcblower
	\begin{lstlisting}[language=Python]
def get_arcs_entering_X(D, X):
    arcs = []
    for u, v, data in D.edges(data=True):
        if u not in X and v in X:
            arcs.append((u, v, data))
    return arcs
\end{lstlisting}
\end{tcolorbox}

A figura a seguir ilustra o funcionamento da função \texttt{get\_arcs\_entering\_X} em um digrafo que vamos denotar por \(D_{32}\). O dígrafo possui uma raiz \(r_0\) conectada aos vértices \(u_1, u_2, u_3\). Os vértices em laranja pertencem ao conjunto \(X = \{v_1, v_2, v_3\}\), e a função identifica apenas os arcos em vermelho, que saem de vértices fora de \(X\) e entram em vértices dentro de \(X\). Arcos da raiz, arcos internos a \(X\), externos a \(X\), ou saindo de \(X\) não são retornados.

\begin{figure}[H]
	\centering
	\begin{tikzpicture}[>=Stealth]
		% estilos
		\tikzset{
			vtx/.style={circle, draw, fill=blue!8, thick, inner sep=1.2pt, minimum size=7mm},
			vtxRoot/.style={circle, draw, fill=red!15, thick, inner sep=1.2pt, minimum size=7mm},
			vtxX/.style={circle, draw=orange!80!black, fill=orange!15, thick, inner sep=1.2pt, minimum size=7mm},
			arc/.style={->, >=Stealth, thick, draw=black!40},
			arcEntering/.style={->, >=Stealth, very thick, draw=red!70!black},
			cost/.style={font=\scriptsize, fill=white, inner sep=1pt}
		}

		% Raiz
		\node[vtxRoot] (r0) at (2, 4) {$r_0$};

		% Vértices fora de X
		\node[vtx] (u1) at (0, 2) {$u_1$};
		\node[vtx] (u2) at (2, 2) {$u_2$};
		\node[vtx] (u3) at (4, 2) {$u_3$};

		% Vértices dentro de X (destacados)
		\node[vtxX] (v1) at (0.5, -0.5) {$v_1$};
		\node[vtxX] (v2) at (2, -0.5) {$v_2$};
		\node[vtxX] (v3) at (3.5, -0.5) {$v_3$};

		% Arcos da raiz (não retornados, pois r0 não está em consideração para X)
		\draw[arc] (r0) -- node[cost, left, pos=0.4] {1} (u1);
		\draw[arc] (r0) -- node[cost, right] {2} (u2);
		\draw[arc] (r0) -- node[cost, right, pos=0.4] {1} (u3);

		% Arcos externos a X (não retornados)
		\draw[arc] (u1) to[bend left=15] node[cost, above] {1} (u2);
		\draw[arc] (u2) to[bend left=15] node[cost, above] {3} (u3);

		% Arcos entrando em X (destacados em vermelho)
		\draw[arcEntering] (u1) -- node[cost, left] {3} (v1);
		\draw[arcEntering] (u2) -- node[cost, right] {2} (v2);
		\draw[arcEntering] (u3) -- node[cost, right] {4} (v3);
		\draw[arcEntering] (u1) to[bend right=20] node[cost, below, pos=0.3] {5} (v2);

		% Arcos internos a X (não retornados)
		\draw[arc] (v1) -- node[cost, above] {1} (v2);
		\draw[arc] (v2) -- node[cost, above] {2} (v3);

		% Arcos saindo de X (não retornados)
		\draw[arc] (v1) to[bend right=30] node[cost, left, pos=0.3] {2} (u1);

		% Região X destacada
		\draw[orange!60!black, thick, dashed, rounded corners=8pt]
		(-0.3, 0.3) rectangle (4.3, -1.3);
		\node[orange!80!black, font=\bfseries] at (5, 0.3) {$X$};
	\end{tikzpicture}
	\caption{Ilustração da função \texttt{get\_arcs\_entering\_X} em \(D_{32}\). A raiz \(r_0\) (em vermelho claro) conecta-se aos vértices \(u_1, u_2, u_3\). Os vértices em \textcolor{orange!80!black}{laranja} pertencem ao conjunto \(X = \{v_1, v_2, v_3\}\). A função identifica apenas os arcos \textcolor{red!70!black}{em vermelho}: aqueles que saem de vértices fora de \(X\) e entram em vértices dentro de \(X\). Arcos da raiz, arcos internos a \(X\), externos a \(X\), ou saindo de \(X\) não são retornados.}
\end{figure}

\subsection{Cálculo do peso mínimo de corte}

Esta função calcula o peso mínimo entre todos os arcos fornecidos, correspondendo ao valor \(\Delta(X)\) necessário para elevar os potenciais dos vértices em \(X\).

Recebe como entrada uma lista \texttt{arcos} de tuplas \((u, v, \text{data})\). A implementação usa a função \texttt{min} com uma compreensão de gerador (linha 2) que extrai o atributo \texttt{"w"} de cada tupla em \texttt{data}.

A função devolve o peso mínimo encontrado entre todos os arcos da lista. A complexidade é \(O(k)\), onde \(k\) é o número de arcos na lista, pois examina cada arco uma vez para encontrar o mínimo.

\begin{tcolorbox}[
		enhanced, breakable,
		colframe=blue!60!black, colback=blue!2,
		colbacktitle=blue!15, coltitle=black,
		title={Cálculo do peso mínimo de corte},
		boxed title style={sharp corners, boxrule=0.6pt},
		sharp corners, boxrule=0.6pt
	]
	\emph{Calcula o peso mínimo entre todos os arcos fornecidos, correspondendo ao valor $\Delta(X)$ usado na elevação de potenciais.}
	\tcblower
	\begin{lstlisting}[language=Python]
def get_minimum_weight_cut(arcs):
    return min(data["w"] for _, _, data in arcs)
\end{lstlisting}
\end{tcolorbox}

A seguir temos uma ilustração do funcionamento da função \texttt{get\_minimum\_weight\_cut} em \(D_{32}\). Considerando os arcos em vermelho que entram em \(X\) (identificados pela função anterior), esta função calcula o peso mínimo entre eles. O arco em verde possui o menor peso (2), correspondendo ao valor \(\Delta(X) = 2\) que será devolvido pela função.

\begin{figure}[H]
	\centering
	\begin{tikzpicture}[>=Stealth]
		% estilos
		\tikzset{
			vtx/.style={circle, draw, fill=blue!8, thick, inner sep=1.2pt, minimum size=7mm},
			vtxRoot/.style={circle, draw, fill=red!15, thick, inner sep=1.2pt, minimum size=7mm},
			vtxX/.style={circle, draw=orange!80!black, fill=orange!15, thick, inner sep=1.2pt, minimum size=7mm},
			arc/.style={->, >=Stealth, thick, draw=black!40},
			arcEntering/.style={->, >=Stealth, very thick, draw=red!70!black},
			arcMin/.style={->, >=Stealth, ultra thick, draw=green!60!black},
			cost/.style={font=\scriptsize, fill=white, inner sep=1pt},
			costMin/.style={font=\scriptsize\bfseries, fill=green!10, inner sep=2pt, draw=green!60!black, rounded corners=1pt}
		}

		% Raiz
		\node[vtxRoot] (r0) at (2, 4) {$r_0$};

		% Vértices fora de X
		\node[vtx] (u1) at (0, 2) {$u_1$};
		\node[vtx] (u2) at (2, 2) {$u_2$};
		\node[vtx] (u3) at (4, 2) {$u_3$};

		% Vértices dentro de X (destacados)
		\node[vtxX] (v1) at (0.5, -0.5) {$v_1$};
		\node[vtxX] (v2) at (2, -0.5) {$v_2$};
		\node[vtxX] (v3) at (3.5, -0.5) {$v_3$};

		% Arcos da raiz
		\draw[arc] (r0) -- node[cost, left, pos=0.4] {1} (u1);
		\draw[arc] (r0) -- node[cost, right] {2} (u2);
		\draw[arc] (r0) -- node[cost, right, pos=0.4] {1} (u3);

		% Arcos externos a X
		\draw[arc] (u1) to[bend left=15] node[cost, above] {1} (u2);
		\draw[arc] (u2) to[bend left=15] node[cost, above] {3} (u3);

		% Arcos entrando em X (em vermelho, mas o mínimo em verde)
		\draw[arcEntering] (u1) -- node[cost, left] {3} (v1);
		\draw[arcMin] (u2) -- node[costMin, right] {2} (v2);
		\draw[arcEntering] (u3) -- node[cost, right] {4} (v3);
		\draw[arcEntering] (u1) to[bend right=20] node[cost, below, pos=0.3] {5} (v2);

		% Arcos internos a X
		\draw[arc] (v1) -- node[cost, above] {1} (v2);
		\draw[arc] (v2) -- node[cost, above] {2} (v3);

		% Arcos saindo de X
		\draw[arc] (v1) to[bend right=30] node[cost, left, pos=0.3] {2} (u1);

		% Região X destacada
		\draw[orange!60!black, thick, dashed, rounded corners=8pt]
		(-0.3, 0.3) rectangle (4.3, -1.3);
		\node[orange!80!black, font=\bfseries] at (5, 0.3) {$X$};

		% Anotação do resultado
		\node[font=\bfseries, green!60!black] at (2, -2.2) {$\Delta(X) = \min\{3, 2, 4, 5\} = 2$};
	\end{tikzpicture}
	\caption{Ilustração da função \texttt{get\_minimum\_weight\_cut} em \(D_{32}\). Considerando os arcos \textcolor{red!70!black}{em vermelho} que entram em \(X\) (identificados pela função anterior), esta função calcula o peso mínimo entre eles. O arco \textcolor{green!60!black}{em verde} possui o menor peso (2), correspondendo ao valor \(\Delta(X) = 2\).}
\end{figure}

\subsection{Atualização de pesos em \texorpdfstring{\(X\)}{X}}

Esta função auxiliar atualiza os pesos dos arcos que entram em um conjunto \(X\), subtraindo o valor \(\Delta(X)\) de cada peso. Arcos que atingem peso zero são adicionados a \(A_0\) e a \(D_0\).

Recebe como entrada um dígrafo \texttt{D}, lista de \texttt{arcs} entrando em \(X\), o valor \texttt{min\_weight} a ser subtraído, uma lista \texttt{A\_zero} para armazenar arcos de peso zero, e o dígrafo \texttt{D\_zero} para adicionar os arcos justos.

A implementação itera sobre cada arco \((u,v,\_)\) da lista (linha 2), subtrai \texttt{min\_weight} do peso armazenado em \texttt{D[u][v]["w"]} (linha 3), e verifica se o peso resultante é zero (linha 4). Caso sim, adiciona-se \((u,v)\) à lista \texttt{A\_zero} (linha 5) e ao dígrafo \texttt{D\_zero} (linha 6).

A função não devolve valor, pois modifica diretamente as estruturas passadas como parâmetros: o dígrafo \texttt{D} tem seus pesos atualizados, \texttt{A\_zero} acumula arcos justos, e \texttt{D\_zero} é populado com esses arcos. A complexidade é \(O(k)\), onde \(k\) é o número de arcos em \texttt{arcs}.

\begin{tcolorbox}[
		enhanced, breakable,
		colframe=blue!60!black, colback=blue!2,
		colbacktitle=blue!15, coltitle=black,
		title={Atualização de pesos em X},
		boxed title style={sharp corners, boxrule=0.6pt},
		sharp corners, boxrule=0.6pt
	]
	\emph{Atualiza os pesos dos arcos que entram em X, subtraindo o valor mínimo. Arcos que atingem peso zero são registrados em $A_0$ e adicionados a $D_0$.}
	\tcblower
	\begin{lstlisting}[language=Python]
def update_weights_in_X(D, arcs, min_weight, A_zero, D_zero):
    for u, v, _ in arcs:
        D[u][v]["w"] -= min_weight
        if D[u][v]["w"] == 0:
            A_zero.append((u, v))
            D_zero.add_edge(u, v)
\end{lstlisting}
\end{tcolorbox}

A seguir ilustramos o funcionamento da função \texttt{update\_weights\_in\_X} em \(D_{32}\). A figura mostra o dígrafo antes e depois da atualização dos pesos. No estado inicial (esquerda), temos \(\Delta(X) = 2\). A função subtrai esse valor de todos os arcos que entram em \(X\). No estado final (direita), os arcos que atingiram peso zero (destacados em azul) são adicionados a \(A_0\) e \(D_0\).

\begin{figure}[H]
	\centering
	\begin{tikzpicture}[>=Stealth]
		% estilos
		\tikzset{
			vtx/.style={circle, draw, fill=blue!8, thick, inner sep=1.2pt, minimum size=6.5mm},
			vtxRoot/.style={circle, draw, fill=red!15, thick, inner sep=1.2pt, minimum size=6.5mm},
			vtxX/.style={circle, draw=orange!80!black, fill=orange!15, thick, inner sep=1.2pt, minimum size=6.5mm},
			arc/.style={->, >=Stealth, thick, draw=black!40},
			arcEntering/.style={->, >=Stealth, very thick, draw=red!70!black},
			arcTight/.style={->, >=Stealth, ultra thick, draw=blue!70!black},
			cost/.style={font=\scriptsize, fill=white, inner sep=1pt},
			costTight/.style={font=\scriptsize\bfseries, fill=blue!10, inner sep=2pt, draw=blue!70!black, rounded corners=1pt}
		}

		% ===== ANTES (esquerda) =====
		\begin{scope}
			\node[font=\bfseries, blue!80!black] at (2, 4.2) {Antes: pesos originais};

			% Raiz
			\node[vtxRoot] (r0) at (2, 3) {$r_0$};

			% Vértices fora de X
			\node[vtx] (u1) at (0, 1.5) {$u_1$};
			\node[vtx] (u2) at (2, 1.5) {$u_2$};
			\node[vtx] (u3) at (4, 1.5) {$u_3$};

			% Vértices dentro de X
			\node[vtxX] (v1) at (0.5, -0.5) {$v_1$};
			\node[vtxX] (v2) at (2, -0.5) {$v_2$};
			\node[vtxX] (v3) at (3.5, -0.5) {$v_3$};

			% Arcos da raiz
			\draw[arc] (r0) -- node[cost, left, pos=0.4] {1} (u1);
			\draw[arc] (r0) -- node[cost, right] {2} (u2);
			\draw[arc] (r0) -- node[cost, right, pos=0.4] {1} (u3);

			% Arcos externos a X
			\draw[arc] (u1) to[bend left=15] node[cost, above] {1} (u2);
			\draw[arc] (u2) to[bend left=15] node[cost, above] {3} (u3);

			% Arcos entrando em X
			\draw[arcEntering] (u1) -- node[cost, left] {3} (v1);
			\draw[arcEntering] (u2) -- node[cost, right] {2} (v2);
			\draw[arcEntering] (u3) -- node[cost, right] {4} (v3);
			\draw[arcEntering] (u1) to[bend right=20] node[cost, below, pos=0.3] {5} (v2);

			% Arcos internos a X
			\draw[arc] (v1) -- node[cost, above] {1} (v2);
			\draw[arc] (v2) -- node[cost, above] {2} (v3);

			% Arcos saindo de X
			\draw[arc] (v1) to[bend right=30] node[cost, left, pos=0.3] {2} (u1);

			% Região X
			\draw[orange!60!black, thick, dashed, rounded corners=8pt]
			(-0.3, 0.3) rectangle (4.3, -1.3);
			\node[orange!80!black, font=\bfseries\small] at (4.8, 0.1) {$X$};

			\node[font=\small, green!60!black] at (2, -2.2) {$\Delta(X) = 2$};
		\end{scope}

		% Seta indicando transformação
		\draw[->, ultra thick, blue!60!black] (5.3, 0.5) -- node[above, font=\small] {subtrai $\Delta(X)$} (6.7, 0.5);

		% ===== DEPOIS (direita) =====
		\begin{scope}[xshift=7.5cm]
			\node[font=\bfseries, blue!80!black] at (2, 4.2) {Depois: pesos atualizados};

			% Raiz
			\node[vtxRoot] (r0b) at (2, 3) {$r_0$};

			% Vértices fora de X
			\node[vtx] (u1b) at (0, 1.5) {$u_1$};
			\node[vtx] (u2b) at (2, 1.5) {$u_2$};
			\node[vtx] (u3b) at (4, 1.5) {$u_3$};

			% Vértices dentro de X
			\node[vtxX] (v1b) at (0.5, -0.5) {$v_1$};
			\node[vtxX] (v2b) at (2, -0.5) {$v_2$};
			\node[vtxX] (v3b) at (3.5, -0.5) {$v_3$};

			% Arcos da raiz (inalterados)
			\draw[arc] (r0b) -- node[cost, left, pos=0.4] {1} (u1b);
			\draw[arc] (r0b) -- node[cost, right] {2} (u2b);
			\draw[arc] (r0b) -- node[cost, right, pos=0.4] {1} (u3b);

			% Arcos externos a X (inalterados)
			\draw[arc] (u1b) to[bend left=15] node[cost, above] {1} (u2b);
			\draw[arc] (u2b) to[bend left=15] node[cost, above] {3} (u3b);

			% Arcos entrando em X (atualizados)
			\draw[arcEntering] (u1b) -- node[cost, left] {1} (v1b);
			\draw[arcTight] (u2b) -- node[costTight, right] {0} (v2b);
			\draw[arcEntering] (u3b) -- node[cost, right] {2} (v3b);
			\draw[arcEntering] (u1b) to[bend right=20] node[cost, below, pos=0.3] {3} (v2b);

			% Arcos internos a X (inalterados)
			\draw[arc] (v1b) -- node[cost, above] {1} (v2b);
			\draw[arc] (v2b) -- node[cost, above] {2} (v3b);

			% Arcos saindo de X (inalterados)
			\draw[arc] (v1b) to[bend right=30] node[cost, left, pos=0.3] {2} (u1b);

			% Região X
			\draw[orange!60!black, thick, dashed, rounded corners=8pt]
			(-0.3, 0.3) rectangle (4.3, -1.3);
			\node[orange!80!black, font=\bfseries\small] at (4.8, 0.1) {$X$};

			\node[font=\small, blue!70!black] at (2, -2.2) {Arco \textcolor{blue!70!black}{justo} $\to A_0$};
		\end{scope}
	\end{tikzpicture}
	\caption{Ilustração da função \texttt{update\_weights\_in\_X} em \(D_{32}\). À esquerda, o dígrafo antes da atualização, com os arcos \textcolor{red!70!black}{em vermelho} entrando em \(X\) e \(\Delta(X) = 2\). À direita, após subtrair \(\Delta(X)\) de cada arco entrando em \(X\): o peso $(u_1, v_1)$ reduz de 3 para 1, $(u_2, v_2)$ de 2 para \textcolor{blue!70!black}{\textbf{0}} (torna-se justo), $(u_3, v_3)$ de 4 para 2, e $(u_1, v_2)$ de 5 para 3. O arco justo é adicionado a \(A_0\) e \(D_0\). Note que os arcos da raiz e arcos internos/externos a \(X\) permanecem inalterados.}
\end{figure}

\subsection{Verificação de arborescência}

Esta função verifica se um dígrafo \(D\) contém uma r-arborescência com raiz \(r_0\). Utiliza busca em profundidade (DFS) a partir da raiz para verificar se todos os vértices são alcançáveis.

Recebe como entrada um dígrafo \texttt{D} e a raiz \texttt{r0}. A implementação constrói uma árvore DFS a partir de \texttt{r0} usando \texttt{nx.dfs\_tree} (linha 2), que devolve um subdigrafo contendo apenas os vértices alcançáveis a partir da raiz seguindo arcos. Em seguida, compara o número de vértices da árvore DFS com o número total de vértices de \texttt{D} (linha 3).

A função devolve \texttt{True} se todos os vértices são alcançáveis (indicando presença de r-arborescência), \texttt{False} caso contrário. A complexidade é \(O(n+m)\), onde \(n=|V|\) e \(m=|A|\), devido à busca em profundidade. Precisamos dessa verificação para garantir que a Fase~1 produza um dígrafo \(D_0\) que contenha uma r-arborescência antes de prosseguir para a Fase~2.

\begin{tcolorbox}[
		enhanced, breakable,
		colframe=blue!60!black, colback=blue!2,
		colbacktitle=blue!15, coltitle=black,
		title={Verificação de arborescência},
		boxed title style={sharp corners, boxrule=0.6pt},
		sharp corners, boxrule=0.6pt
	]
	\emph{Verifica se o dígrafo D contém uma r-arborescência com raiz r0, usando busca em profundidade para testar alcançabilidade de todos os vértices.}
	\tcblower
	\begin{lstlisting}[language=Python]
def has_arborescence(D, r0):
    tree = nx.dfs_tree(D, r0)
    return tree.number_of_nodes() == D.number_of_nodes()
\end{lstlisting}
\end{tcolorbox}

\subsection{Fase 1: Elevação de potenciais e construção de \texorpdfstring{\(A_0\)}{A_0}}

A seguir apresentamos a função principal da Fase~1, responsável por elevar os potenciais dos vértices iterativamente até que cada conjunto de vértices possua ao menos um arco justo entrando. O processo utiliza componentes fortemente conexas para identificar quais conjuntos necessitam elevação.

Recebe como entrada um dígrafo \texttt{D\_original} e a raiz \texttt{r0}. A implementação cria uma cópia do dígrafo original (linha 2) para preservar a entrada, inicializa estruturas auxiliares \texttt{A\_zero} (lista de arcos justos), \texttt{Dual\_list} (lista de pares \((X, \Delta(X))\) para fins de validação dual), e \texttt{D\_zero} (dígrafo de arcos justos) (linhas 3-5). Um contador de iterações é inicializado na linha 6.

O loop principal (linhas 7-22) itera enquanto houver conjuntos sem arcos justos entrando. Em cada iteração, incrementa-se o contador (linha 8), calcula-se as componentes fortemente conexas de \(D_0\) usando \texttt{nx.condensation} (linha 9), que devolve um grafo acíclico dirigido (DAG, do inglês \emph{directed acyclic graph}) onde cada vértice representa uma componente e contém o atributo \texttt{"members"} com os vértices originais. Utilizamos componentes fortemente conexas porque elas identificam naturalmente os conjuntos maximais de vértices que ainda não possuem arcos justos entrando, evitando a necessidade de rastrear manualmente quais conjuntos já foram processados.

Em seguida, identificam-se as fontes (componentes sem arcos entrando) no grafo de condensação (linha 10). Se há apenas uma fonte, significa que todos os vértices estão em uma única componente alcançável pela raiz através de arcos justos, garantindo que \(D_0\) contém uma r-arborescência e encerrando o loop (linhas 11-12).

Para cada fonte \(u\) no grafo de condensação (linha 13), obtém-se o conjunto \(X\) de vértices da componente (linha 14). Se \(r_0 \in X\), a fonte é ignorada (linhas 15-16), pois a componente contendo a raiz não necessita elevação. Caso contrário, identificam-se os arcos entrando em \(X\) usando \texttt{get\_arcs\_entering\_X} (linha 17), calcula-se o peso mínimo \(\Delta(X)\) usando \texttt{get\_minimum\_weight\_cut} (linha 18), e atualiza-se os pesos com \texttt{update\_weights\_in\_X}, registrando novos arcos justos (linha 19). A elevação simultânea de potenciais para todos os vértices de \(X\) mantém a propriedade de que arcos internos a \(X\) permanecem com o mesmo custo reduzido relativo, preservando a correção do algoritmo. Finalmente, adiciona-se \((X, \Delta(X))\) à lista dual se \(\Delta(X) > 0\) (linhas 20-21), permitindo verificação posterior das condições de otimalidade dual.

A função devolve \texttt{A\_zero} (lista de arcos justos) e \texttt{Dual\_list} (pares \((X, \Delta(X))\) para validação). A complexidade é \(O(nm)\) no pior caso, com \(O(n)\) iterações, cada uma custando \(O(m)\) para calcular componentes e atualizar pesos.

\begin{tcolorbox}[
		enhanced, breakable,
		colframe=blue!60!black, colback=blue!2,
		colbacktitle=blue!15, coltitle=black,
		title={Fase 1: Elevação de potenciais e construção de $A_0$},
		boxed title style={sharp corners, boxrule=0.6pt},
		sharp corners, boxrule=0.6pt
	]
	\emph{Eleva iterativamente os potenciais dos vértices até que cada conjunto possua ao menos um arco justo entrando. Devolve a lista $A_0$ de arcos justos e a lista de pares $(X, \Delta(X))$ para validação dual.}
	\tcblower
	\begin{lstlisting}[language=Python, basicstyle=\ttfamily\fontsize{8}{9}\selectfont]
def phase1(D_original, r0):
    D_copy = D_original.copy()
    A_zero = []
    Dual_list = []
    D_zero = build_D_zero(D_copy)
    iteration = 0
    while True:
        iteration += 1
        C = nx.condensation(D_zero)
        sources = [x for x in C.nodes() if C.in_degree(x) == 0]
        if len(sources) == 1:
            break
        for u in sources:
            X = C.nodes[u]["members"]
            if r0 in X:
                continue
            arcs = get_arcs_entering_X(D_copy, X)
            min_weight = get_minimum_weight_cut(arcs)
            update_weights_in_X(D_copy, arcs, min_weight, A_zero, D_zero)
            if min_weight != 0:
                Dual_list.append((X, min_weight)) 
    return A_zero, Dual_list
\end{lstlisting}
\end{tcolorbox}

A seguir ilustramos a execução da função \texttt{phase1} no dígrafo \(D_{32}\) estendido com a raiz \(r_0\). A figura mostra três estados principais: o dígrafo original \(D\) com seus pesos, o dígrafo \(D_0\) inicial (vazio, sem arcos), e o dígrafo \(D_0\) final após as iterações de elevação de potenciais, contendo apenas os arcos justos (em azul) que formam uma r-arborescência.

\begin{figure}[H]
	\centering
	\begin{tikzpicture}[>=Stealth, scale=0.85, every node/.style={scale=0.85}]
		% Estilos
		\tikzset{
			vtx/.style={circle, draw, fill=blue!8, thick, inner sep=1.2pt, minimum size=6.5mm},
			vtxRoot/.style={circle, draw, fill=red!15, thick, inner sep=1.2pt, minimum size=6.5mm},
			vtxX/.style={circle, draw, fill=orange!20, thick, inner sep=1.2pt, minimum size=6.5mm},
			arc/.style={->, thick, gray!60},
			arcTight/.style={->, ultra thick, blue!70!black},
			cost/.style={font=\scriptsize, fill=white, inner sep=1pt, rounded corners=1pt},
			costTight/.style={font=\scriptsize\bfseries, fill=blue!10, inner sep=2pt, draw=blue!70!black, rounded corners=1pt}
		}

		% ===== DÍGRAFO ORIGINAL D (esquerda) =====
		\begin{scope}
			\node[font=\bfseries, blue!80!black] at (2, 4.2) {Dígrafo original $D$};

			% Raiz
			\node[vtxRoot] (r0) at (2, 3) {$r_0$};

			% Vértices fora de X
			\node[vtx] (u1) at (0, 1.5) {$u_1$};
			\node[vtx] (u2) at (2, 1.5) {$u_2$};
			\node[vtx] (u3) at (4, 1.5) {$u_3$};

			% Vértices dentro de X
			\node[vtxX] (v1) at (0.5, -0.5) {$v_1$};
			\node[vtxX] (v2) at (2, -0.5) {$v_2$};
			\node[vtxX] (v3) at (3.5, -0.5) {$v_3$};

			% Arcos da raiz
			\draw[arc] (r0) -- node[cost, left, pos=0.4] {1} (u1);
			\draw[arc] (r0) -- node[cost, right] {2} (u2);
			\draw[arc] (r0) -- node[cost, right, pos=0.4] {1} (u3);

			% Arcos externos a X
			\draw[arc] (u1) to[bend left=15] node[cost, above] {1} (u2);
			\draw[arc] (u2) to[bend left=15] node[cost, above] {3} (u3);

			% Arcos entrando em X
			\draw[arc] (u1) -- node[cost, left] {3} (v1);
			\draw[arc] (u2) -- node[cost, right] {2} (v2);
			\draw[arc] (u3) -- node[cost, right] {4} (v3);
			\draw[arc] (u1) to[bend right=20] node[cost, below, pos=0.3] {5} (v2);

			% Arcos internos a X
			\draw[arc] (v1) -- node[cost, above] {1} (v2);
			\draw[arc] (v2) -- node[cost, above] {2} (v3);

			% Arcos saindo de X
			\draw[arc] (v1) to[bend right=30] node[cost, left, pos=0.3] {2} (u1);

			\node[font=\small, gray!70!black] at (2, -2) {Pesos originais};
		\end{scope}

		% ===== D_0 INICIAL (centro) =====
		\begin{scope}[xshift=6cm]
			\node[font=\bfseries, blue!80!black] at (2, 4.2) {$D_0$ inicial (linha 5)};

			% Vértices apenas, sem arcos
			\node[vtxRoot] (r0i) at (2, 3) {$r_0$};
			\node[vtx] (u1i) at (0, 1.5) {$u_1$};
			\node[vtx] (u2i) at (2, 1.5) {$u_2$};
			\node[vtx] (u3i) at (4, 1.5) {$u_3$};
			\node[vtxX] (v1i) at (0.5, -0.5) {$v_1$};
			\node[vtxX] (v2i) at (2, -0.5) {$v_2$};
			\node[vtxX] (v3i) at (3.5, -0.5) {$v_3$};

			\node[font=\small, gray!70!black] at (2, -2) {Sem arcos (vazio)};
		\end{scope}

		% ===== D_0 FINAL (direita) =====
		\begin{scope}[xshift=12cm]
			\node[font=\bfseries, blue!80!black] at (2, 4.2) {$D_0$ final (após loop)};

			% Vértices
			\node[vtxRoot] (r0f) at (2, 3) {$r_0$};
			\node[vtx] (u1f) at (0, 1.5) {$u_1$};
			\node[vtx] (u2f) at (2, 1.5) {$u_2$};
			\node[vtx] (u3f) at (4, 1.5) {$u_3$};
			\node[vtxX] (v1f) at (0.5, -0.5) {$v_1$};
			\node[vtxX] (v2f) at (2, -0.5) {$v_2$};
			\node[vtxX] (v3f) at (3.5, -0.5) {$v_3$};

			% Arcos justos (peso 0)
			\draw[arcTight] (r0f) -- node[costTight, left, pos=0.4] {0} (u1f);
			\draw[arcTight] (r0f) -- node[costTight, right, pos=0.4] {0} (u3f);
			\draw[arcTight] (u1f) to[bend left=15] node[costTight, above] {0} (u2f);
			\draw[arcTight] (u1f) -- node[costTight, left] {0} (v1f);
			\draw[arcTight] (u2f) -- node[costTight, right] {0} (v2f);
			\draw[arcTight] (v2f) -- node[costTight, above] {0} (v3f);

			\node[font=\small, blue!70!black] at (2, -2) {Arcos justos em $A_0$};
		\end{scope}

		% Setas indicando progressão
		\draw[->, ultra thick, gray!50] (4.5, 1) -- node[above, font=\small] {elevação} (5.5, 1);
		\draw[->, ultra thick, gray!50] (10.5, 1) -- node[above, font=\small] {iterações} (11.5, 1);
	\end{tikzpicture}
	\caption{Execução da função \texttt{phase1} em \(D_{32}\). À esquerda, o dígrafo original \(D\) com pesos dos arcos. Ao centro, \(D_0\) inicial após linha 5 (\texttt{build\_D\_zero}), contendo apenas vértices sem arcos. À direita, \(D_0\) final após o loop de elevação de potenciais, contendo apenas os arcos \textcolor{blue!70!black}{justos} (custo reduzido zero) que formam uma r-arborescência. Durante as iterações (linhas 7-22), os potenciais são elevados para cada componente sem arcos justos entrando, até que todos os vértices sejam alcançáveis a partir de \(r_0\) através de arcos em \(D_0\).}
\end{figure}

A seguir ilustramos a execução da função \texttt{phase2} em \(D_{32}\), mostrando como a r-arborescência é construída incrementalmente a partir do conjunto \(A_0\) de arcos justos obtidos da Fase~1. A figura apresenta quatro estados: o conjunto inicial \(A_0\) com 6 arcos justos, e três estados intermediários da arborescência \texttt{Arb} sendo construída, destacando em verde os arcos adicionados em cada iteração.

\begin{figure}[H]
	\centering
	\begin{tikzpicture}[>=Stealth, scale=0.72, every node/.style={scale=0.72}]
		% Estilos
		\tikzset{
			vtx/.style={circle, draw, fill=blue!8, thick, inner sep=1.2pt, minimum size=6.5mm},
			vtxRoot/.style={circle, draw, fill=red!15, thick, inner sep=1.2pt, minimum size=6.5mm},
			vtxIn/.style={circle, draw, fill=green!15, thick, inner sep=1.2pt, minimum size=6.5mm},
			vtxOut/.style={circle, draw, fill=gray!10, thick, inner sep=1.2pt, minimum size=6.5mm},
			arc/.style={->, thick, blue!70!black},
			arcNew/.style={->, ultra thick, green!70!black},
			cost/.style={font=\scriptsize, fill=white, inner sep=1pt}
		}

		% ===== A_0 (esquerda superior) =====
		\begin{scope}
			\node[font=\bfseries, blue!80!black] at (2, 4.5) {$A_0$ (arcos justos)};

			\node[vtxRoot] (r0) at (2, 3.2) {$r_0$};
			\node[vtx] (u1) at (0, 1.7) {$u_1$};
			\node[vtx] (u2) at (2, 1.7) {$u_2$};
			\node[vtx] (u3) at (4, 1.7) {$u_3$};
			\node[vtx] (v1) at (0.5, 0) {$v_1$};
			\node[vtx] (v2) at (2, 0) {$v_2$};
			\node[vtx] (v3) at (3.5, 0) {$v_3$};

			\draw[arc] (r0) -- node[cost, left, pos=0.3] {0} (u1);
			\draw[arc] (r0) -- node[cost, right, pos=0.3] {0} (u3);
			\draw[arc] (u1) to[bend left=12] node[cost, above, pos=0.4] {0} (u2);
			\draw[arc] (u1) -- node[cost, left] {0} (v1);
			\draw[arc] (u2) -- node[cost, right] {0} (v2);
			\draw[arc] (v2) -- node[cost, above] {0} (v3);

			\node[font=\small, gray!70!black] at (2, -1) {6 arcos com custo 0};
		\end{scope}

		% ===== Iteração 1 (direita superior) =====
		\begin{scope}[xshift=6.5cm]
			\node[font=\bfseries, blue!80!black] at (2, 4.5) {Iteração 1: $|$Arb$| = 1$};

			\node[vtxRoot] (r01) at (2, 3.2) {$r_0$};
			\node[vtxIn] (u11) at (0, 1.7) {$u_1$};
			\node[vtxOut] (u21) at (2, 1.7) {$u_2$};
			\node[vtxOut] (u31) at (4, 1.7) {$u_3$};
			\node[vtxOut] (v11) at (0.5, 0) {$v_1$};
			\node[vtxOut] (v21) at (2, 0) {$v_2$};
			\node[vtxOut] (v31) at (3.5, 0) {$v_3$};

			\draw[arcNew] (r01) -- node[cost, left, pos=0.3, fill=green!10] {1} (u11);

			\node[font=\small, green!70!black] at (2, -1) {Adiciona $(r_0, u_1)$};
		\end{scope}

		% ===== Iteração 2 (esquerda inferior) =====
		\begin{scope}[yshift=-6.2cm]
			\node[font=\bfseries, blue!80!black] at (2, 4.5) {Iteração 2: $|$Arb$| = 2$};

			\node[vtxRoot] (r02) at (2, 3.2) {$r_0$};
			\node[vtxIn] (u12) at (0, 1.7) {$u_1$};
			\node[vtxOut] (u22) at (2, 1.7) {$u_2$};
			\node[vtxIn] (u32) at (4, 1.7) {$u_3$};
			\node[vtxOut] (v12) at (0.5, 0) {$v_1$};
			\node[vtxOut] (v22) at (2, 0) {$v_2$};
			\node[vtxOut] (v32) at (3.5, 0) {$v_3$};

			\draw[arc] (r02) -- node[cost, left, pos=0.3] {1} (u12);
			\draw[arcNew] (r02) -- node[cost, right, pos=0.3, fill=green!10] {1} (u32);

			\node[font=\small, green!70!black] at (2, -1) {Adiciona $(r_0, u_3)$};
		\end{scope}

		% ===== Iteração 4 (direita inferior) =====
		\begin{scope}[xshift=6.5cm, yshift=-6.2cm]
			\node[font=\bfseries, blue!80!black] at (2, 4.5) {Iteração 4: $|$Arb$| = 4$};

			\node[vtxRoot] (r04) at (2, 3.2) {$r_0$};
			\node[vtxIn] (u14) at (0, 1.7) {$u_1$};
			\node[vtxIn] (u24) at (2, 1.7) {$u_2$};
			\node[vtxIn] (u34) at (4, 1.7) {$u_3$};
			\node[vtxIn] (v14) at (0.5, 0) {$v_1$};
			\node[vtxOut] (v24) at (2, 0) {$v_2$};
			\node[vtxOut] (v34) at (3.5, 0) {$v_3$};

			\draw[arc] (r04) -- node[cost, left, pos=0.3] {1} (u14);
			\draw[arc] (r04) -- node[cost, right, pos=0.3] {1} (u34);
			\draw[arc] (u14) to[bend left=12] node[cost, above, pos=0.4] {1} (u24);
			\draw[arcNew] (u14) -- node[cost, left, fill=green!10] {3} (v14);

			\node[font=\small, green!70!black] at (2, -1) {Adiciona $(u_1, v_1)$};
		\end{scope}

		% Setas de progressão
		\draw[->, ultra thick, gray!50] (4.8, 2) -- node[above, font=\small] {linha 6-10} (5.7, 2);
		\draw[->, ultra thick, gray!50] (2, -1.5) -- node[left, font=\small] {linha 6-10} (2, -2.4);
		\draw[->, ultra thick, gray!50] (4.8, -4.2) -- node[above, font=\small] {linha 6-10} (5.7, -4.2);
	\end{tikzpicture}
	\caption{Execução da função \texttt{phase2} em \(D_{32}\). Superior esquerdo: conjunto \(A_0\) com 6 arcos justos (custo reduzido zero) obtidos da Fase~1. Superior direito: Iteração 1 — adiciona arco $(r_0, u_1)$ pois $r_0 \in$ Arb e $u_1 \notin$ Arb. Inferior esquerdo: Iteração 2 — adiciona $(r_0, u_3)$. Inferior direito: Iteração 4 — após adicionar $(u_1, u_2)$ na iteração 3, adiciona $(u_1, v_1)$. Vértices \textcolor{green!70!black}{verdes} foram recém-adicionados, vértices \textcolor{gray!70!black}{cinzas} ainda não pertencem a Arb. O processo continua até que todos os 6 arcos de \(A_0\) sejam incluídos, formando uma r-arborescência com pesos originais de \(D\).}
\end{figure}

As funções auxiliares implementadas nesta seção correspondem diretamente aos passos da Fase~1 do algoritmo de András Frank da seguinte forma:

\begin{tcolorbox}[
		enhanced, breakable,
		colframe=purple!60!black, colback=purple!5,
		colbacktitle=purple!15, coltitle=black,
		title={Correspondência entre Teoria e Implementação — Fase 1},
		boxed title style={sharp corners, boxrule=0.6pt},
		sharp corners, boxrule=0.6pt
	]
	\textbf{Passo 1 — Inicialização:}
	\begin{itemize}
		\item \textbf{Descrição teórica:} Defina \(y(v) := 0\) para todo \(v \in V\).
		\item \textbf{Implementação:} A função \texttt{build\_D\_zero(D)} inicializa o dígrafo \(D_0\) vazio, que será populado apenas com arcos justos. Implicitamente, os potenciais iniciam em zero, pois os pesos no dígrafo \(D\) representam os custos reduzidos \(c_y(u,v) = c(u,v) - y(v)\). Com \(y(v) = 0\), temos \(c_y = c\) no início.
	\end{itemize}

	\vspace{2mm}
	\textbf{Passo 2 — Iteração de elevação de potenciais:}
	\begin{itemize}
		\item \textbf{Descrição teórica:} Enquanto existir conjunto \(X \subseteq V \setminus \{r\}\) sem arco justo entrando:
		      \begin{itemize}
			      \item Calcule \(\Delta(X) := \min\{c(u,v) - y(v) : u \notin X, v \in X\}\).
			      \item Para cada \(v \in X\), atualize \(y(v) := y(v) + \Delta(X)\).
		      \end{itemize}
		\item \textbf{Implementação:} Esse processo de iteração é realizado pela composição sequencial de três funções auxiliares:
		      \begin{enumerate}
			      \item \texttt{get\_arcs\_entering\_X(D, X)}: Identifica o conjunto \(\{(u,v) : u \notin X, v \in X\}\), isto é, todos os arcos que cruzam a fronteira de \(X\). Essa função corresponde diretamente à definição do conjunto sobre o qual o mínimo é calculado na fórmula \(\Delta(X)\).

			      \item \texttt{get\_minimum\_weight\_cut(arcs)}: Calcula \(\min\{\text{data}["w"] : (u,v,\text{data}) \in \text{arcs}\}\), que é exatamente \(\Delta(X) = \min\{c_y(u,v) : u \notin X, v \in X\}\). Como os pesos no dígrafo já representam custos reduzidos (são atualizados a cada iteração), essa função devolve precisamente o valor teórico de \(\Delta(X)\).

			      \item \texttt{update\_weights\_in\_X(D, arcs, min\_weight, A\_zero, D\_zero)}: Implementa a atualização dos potenciais. Para cada arco \((u,v)\) entrando em \(X\), subtrai \texttt{min\_weight} de \texttt{D[u][v]["w"]}, efetivamente calculando o novo custo reduzido:
			            \[c_y'(u,v) = c_y(u,v) - \Delta(X) = \]
			            \[c(u,v) - y(v) - \Delta(X) = \]
			            \[c(u,v) - \bigl(y(v) + \Delta(X)\bigr) = c(u,v) - y'(v), \]
			            onde \(y'(v) = y(v) + \Delta(X)\) é o novo potencial. Arcos cujo custo reduzido atinge zero (\texttt{D[u][v]["w"] == 0}) são adicionados a \(A_0\) e \(D_0\), tornando-se \textbf{justos}.
		      \end{enumerate}
	\end{itemize}

	\vspace{2mm}
	\textbf{Passo 3 — Construção de \(A_0\):}
	\begin{itemize}
		\item \textbf{Descrição teórica:} Defina \(A_0 := \{a \in A : c_y(a) = 0\}\), o conjunto de arcos justos.
		\item \textbf{Implementação:} A construção de \(A_0\) ocorre de forma incremental durante as iterações do Passo~2. Cada chamada a \texttt{update\_weights\_in\_X} verifica quais arcos atingiram custo reduzido zero e os adiciona tanto à lista \texttt{A\_zero} quanto ao dígrafo \texttt{D\_zero}. Esse processo continua até que a função principal da Fase~1 (que será apresentada adiante) determine que cada componente fortemente conexa em \(D_0\) (exceto a raiz) possui ao menos um arco justo entrando, garantindo que \(A_0\) é suficiente para formar a base de uma \(r\)-arborescência.
		\item O resultado final é o conjunto completo \(A_0 = \{a \in A : c_y(a) = 0\}\), usado na Fase~2 para construir a arborescência ótima através de uma seleção gulosa de arcos justos.
	\end{itemize}
\end{tcolorbox}

\subsection{Fase 2: Construção da arborescência}
Esta é a função principal da Fase~2, responsável por construir a r-arborescência de custo mínimo a partir do conjunto \(A_0\) de arcos justos. A construção é incremental: inicia-se com a raiz e adiciona-se iterativamente arcos de \(A_0\) que conectam vértices já incluídos a novos vértices, garantindo que cada vértice não-raiz receba exatamente um arco de entrada.

Recebe como entrada um dígrafo \texttt{D\_original}, a raiz \texttt{r0}, e a lista \texttt{A\_zero} de arcos justos. A implementação cria um novo dígrafo vazio \texttt{Arb} (linha 2) e adiciona a raiz (linha 3).

O loop principal (linhas 5-12) itera \(n-1\) vezes, onde \(n = |V|\), pois uma r-arborescência tem exatamente \(|V|-1\) arcos. Em cada iteração:
\begin{enumerate}\setlength{\itemsep}{2pt}
	\item Percorre os arcos \((u,v)\) de \texttt{A\_zero} (linha 6).
	\item Verifica se \(u\) já está em \texttt{Arb} e \(v\) ainda não (linha 7).
	\item Se sim, obtém os dados do arco do dígrafo original (linha 8) e adiciona \((u,v)\) a \texttt{Arb} (linha 9).
	\item Interrompe o loop interno para reiniciar a busca, garantindo descoberta em largura (linha 10).
\end{enumerate}

A função devolve o dígrafo \texttt{Arb} representando a r-arborescência de custo mínimo. A complexidade é \(O(nm)\) no pior caso, pois cada uma das \(O(n)\) iterações pode percorrer todos os \(O(m)\) arcos de \texttt{A\_zero}.

\begin{tcolorbox}[
		enhanced, breakable,
		colframe=blue!60!black, colback=blue!2,
		colbacktitle=blue!15, coltitle=black,
		title={Fase 2: Construção da arborescência},
		boxed title style={sharp corners, boxrule=0.6pt},
		sharp corners, boxrule=0.6pt
	]
	\emph{Constrói incrementalmente a r-arborescência a partir de $A_0$, adicionando iterativamente arcos que conectam vértices já incluídos a novos vértices.}
	\tcblower
	\begin{lstlisting}[language=Python]
def phase2(D_original, r0, A_zero):
    Arb = nx.DiGraph()
    Arb.add_node(r0)
    n = len(D_original.nodes())
    for _ in range(n - 1):
        for u, v in A_zero:
            if u in Arb.nodes() and v not in Arb.nodes():
                edge_data = D_original.get_edge_data(u, v)
                Arb.add_edge(u, v, **edge_data)
                break
    return Arb
\end{lstlisting}
\end{tcolorbox}

Apresentamos também uma versão alternativa da Fase~2 que utiliza busca em largura (BFS) para construir a r-arborescência de forma mais eficiente. Diferentemente da versão anterior que itera \(n-1\) vezes sobre todos os arcos, esta implementação usa uma fila de prioridade para explorar os arcos em ordem, evitando buscas lineares repetidas.

Recebe como entrada um dígrafo \texttt{D\_original}, a raiz \texttt{r0}, e a lista \texttt{A\_zero} de arcos justos. A implementação começa criando um dígrafo auxiliar \texttt{Arb} (linha 2) onde cada arco de \texttt{A\_zero} recebe um peso igual ao seu índice na lista (linhas 3-4), estabelecendo uma ordem de exploração. Inicializa-se o conjunto \texttt{V} de vértices visitados contendo apenas a raiz (linha 5) e uma fila de prioridade vazia \texttt{q} (linha 6).

Todos os arcos que saem da raiz em \texttt{Arb} são adicionados à fila de prioridade (linhas 7-8), usando o peso (índice) como critério de ordenação. Cria-se então o dígrafo \texttt{A} que conterá a arborescência resultante (linha 9).

O loop principal (linhas 10-17) extrai arcos da fila de prioridade em ordem crescente de índice. Para cada arco \((u,v)\) extraído (linha 11), verifica-se se o vértice destino \(v\) já foi visitado (linha 12); em caso positivo, o arco é ignorado via \texttt{continue} (linha 13). Caso contrário, adiciona-se o arco \((u,v)\) à arborescência \texttt{A} com o peso original de \texttt{D\_original} (linha 14), marca-se \(v\) como visitado (linha 15), e todos os arcos que saem de \(v\) em \texttt{Arb} são adicionados à fila de prioridade para exploração futura (linhas 16-17).

A função devolve o dígrafo \texttt{A} representando a r-arborescência de custo mínimo. A complexidade é \(O(m \log m)\), onde \(m = |A_0|\), devido às operações de inserção e remoção na fila de prioridade. Esta versão é mais eficiente que a anterior quando \(|A_0|\) é grande, pois evita percorrer todos os arcos em cada iteração.

\begin{tcolorbox}[
		enhanced, breakable,
		colframe=blue!60!black, colback=blue!2,
		colbacktitle=blue!15, coltitle=black,
		title={Fase 2 (versão BFS): Construção da arborescência com fila de prioridade},
		boxed title style={sharp corners, boxrule=0.6pt},
		sharp corners, boxrule=0.6pt
	]
	\emph{Constrói a r-arborescência usando busca em largura guiada por fila de prioridade, explorando arcos de $A_0$ em ordem e evitando buscas lineares repetidas. Complexidade $O(m \log m)$.}
	\tcblower
	\begin{lstlisting}[language=Python]
def phase2_v2(D_original, r0, A_zero):
    Arb = nx.DiGraph()
    for i, (u, v) in enumerate(A_zero):
        Arb.add_edge(u, v, w=i)
    V = {r0}
    q = []
    for u, v, data in Arb.out_edges(r0, data=True):
        heapq.heappush(q, (data["w"], u, v))
    A = nx.DiGraph()
    while q:
        _, u, v = heapq.heappop(q)
        if v in V:
            continue
        A.add_edge(u, v, w=D_original[u][v]["w"])
        V.add(v)
        for x, y, data in Arb.out_edges(v, data=True):
            heapq.heappush(q, (data["w"], x, y))
    return A
\end{lstlisting}
\end{tcolorbox}

As duas versões da Fase~2 implementadas acima correspondem diretamente ao Passo~4 da descrição teórica do algoritmo de András Frank:

\begin{tcolorbox}[
		enhanced, breakable,
		colframe=purple!60!black, colback=purple!5,
		colbacktitle=purple!15, coltitle=black,
		title={Correspondência entre Teoria e Implementação — Fase 2},
		boxed title style={sharp corners, boxrule=0.6pt},
		sharp corners, boxrule=0.6pt
	]
	\textbf{Passo 4 — Construção da arborescência (caso acíclico):}
	\begin{itemize}
		\item \textbf{Descrição teórica:} Se \((V,A_0)\) forma uma r-arborescência, devolva \(A_0\). Por otimalidade dos potenciais duais, trata-se de uma r-arborescência de custo mínimo.
		\item \textbf{Implementação:} Ambas as versões de \texttt{phase2} constroem uma r-arborescência a partir do conjunto \(A_0\) de arcos justos obtido na Fase~1. A corretude baseia-se no fato de que todos os arcos em \(A_0\) têm custo reduzido zero, e a Fase~1 garante que existe uma r-arborescência formada exclusivamente por arcos justos.
		      \begin{itemize}
			      \item \textbf{Versão 1 (\texttt{phase2}):} Construção incremental por exploração exaustiva. Em cada uma das \(n-1\) iterações, percorre todos os arcos de \(A_0\) procurando um arco \((u,v)\) tal que \(u\) já pertence à arborescência parcial e \(v\) ainda não. Essa abordagem simples corresponde diretamente à ideia teórica de construir a arborescência adicionando um vértice por vez, conectando-o à estrutura existente através de um arco justo. Complexidade: \(O(nm)\).

			      \item \textbf{Versão 2 (\texttt{phase2\_v2}):} Construção por busca em largura guiada por fila de prioridade. Cria um dígrafo auxiliar onde arcos são indexados, usa a fila de prioridade para explorar arcos sistematicamente a partir da raiz, evitando buscas lineares repetidas. Essa versão otimizada mantém a mesma correção teórica — construir uma r-arborescência usando apenas arcos de \(A_0\) — mas melhora a eficiência prática. Complexidade: \(O(m \log m)\).
		      \end{itemize}
	\end{itemize}

	\vspace{2mm}
	\textbf{Nota sobre Passos 5-7 (contração/recursão/expansão):}
	\begin{itemize}
		\item A descrição teórica do algoritmo de András Frank inclui os Passos~5-7 para tratar o caso onde \(A_0\) contém ciclos dirigidos, exigindo contração, resolução recursiva e reexpansão, de forma análoga ao algoritmo de Chu--Liu--Edmonds.
		\item Na implementação apresentada, optamos por uma abordagem não-recursiva baseada em componentes fortemente conexas. A Fase~1 já garante que \(A_0\) formará uma r-arborescência ao término das iterações de elevação de potenciais, eliminando a necessidade de tratar ciclos explicitamente na Fase~2. Essa simplificação é possível porque a elevação de potenciais progressivamente "quebra" todos os ciclos ao criar novos arcos justos que conectam diferentes componentes, até que reste apenas uma única componente fortemente conexa contendo todos os vértices.
		\item Portanto, quando a Fase~2 é executada, o conjunto \(A_0\) já está livre de ciclos e forma uma r-arborescência, correspondendo diretamente ao caso tratado pelo Passo~4 da descrição teórica. A verificação prévia \texttt{has\_arborescence(D, r0)} (realizada pela função principal) confirma essa propriedade antes de invocar a Fase~2.
	\end{itemize}
\end{tcolorbox}

\subsection{Verificação de otimalidade dual}

Esta função verifica se a condição de otimalidade dual é satisfeita para a r-arborescência construída. Segundo a teoria de programação linear dual aplicada ao problema de arborescência de custo mínimo, uma solução é ótima se e somente se cada conjunto \(X \subseteq V \setminus \{r\}\) que teve seu potencial elevado durante a Fase~1 possui exatamente um arco entrando na arborescência final.

Recebe como entrada a arborescência \texttt{Arb} e a lista \texttt{Dual\_list} contendo pares \((X, \Delta(X))\) onde \(X\) é um conjunto de vértices cujos potenciais foram elevados e \(\Delta(X) > 0\) é o valor da elevação. A implementação itera sobre cada par \((X, z)\) na lista dual (linha 2), e para cada conjunto \(X\), percorre todos os arcos \((u,v)\) da arborescência (linha 3). Inicializa um contador \texttt{count} em zero (linha 4) e verifica se o arco cruza a fronteira de \(X\), isto é, se \(u \notin X\) e \(v \in X\) (linha 5). Quando essa condição é satisfeita, incrementa o contador (linha 6) e imediatamente verifica se já há mais de um arco entrando em \(X\) (linha 7). Caso positivo, a condição de otimalidade dual é violada e a função devolve \texttt{False} (linha 8).

A função devolve \texttt{True} se todos os conjuntos em \texttt{Dual\_list} possuem exatamente um arco entrando na arborescência, confirmando que a solução satisfaz as condições de folga complementar da programação linear dual. A complexidade é \(O(km)\), onde \(k = |\texttt{Dual\_list}|\) e \(m = |A|\), pois para cada conjunto dual verifica-se todos os arcos da arborescência.

Esta verificação é fundamental para garantir a correção do algoritmo: a Fase~1 constrói uma solução dual viável (potenciais \(y(v)\)), a Fase~2 constrói uma solução primal viável (arborescência), e esta função confirma que ambas satisfazem as condições de folga complementar, implicando otimalidade pelo teorema da dualidade forte.

\begin{tcolorbox}[
		enhanced, breakable,
		colframe=blue!60!black, colback=blue!2,
		colbacktitle=blue!15, coltitle=black,
		title={Verificação de otimalidade dual},
		boxed title style={sharp corners, boxrule=0.6pt},
		sharp corners, boxrule=0.6pt
	]
	\emph{Verifica se a condição de otimalidade dual é satisfeita, confirmando que cada conjunto dual possui exatamente um arco entrando na arborescência.}
	\tcblower
	\begin{lstlisting}[language=Python]
def check_dual_optimality_condition(Arb, Dual_list):
    for X, z in Dual_list:
        count = 0
        for u, v in Arb.edges():
            if u not in X and v in X:
                count += 1
                if count > 1:
                    return False
    return True
\end{lstlisting}
\end{tcolorbox}

\subsection{O algoritmo completo de András Frank}

Finalmente, apresentamos a função principal que implementa o algoritmo de András Frank para encontrar uma r-arborescência de custo mínimo em um dígrafo com pesos. A função integra as fases de construção dos potenciais duais, obtenção dos arcos justos, construção da arborescência e verificação de otimalidade dual.

\begin{tcolorbox}[
		enhanced, breakable,
		colframe=blue!60!black, colback=blue!2,
		colbacktitle=blue!15, coltitle=black,
		title={Verificação de otimalidade dual},
		boxed title style={sharp corners, boxrule=0.6pt},
		sharp corners, boxrule=0.6pt
	]
	\emph{Implementa o algoritmo completo de András Frank, integrando as fases de construção dos potenciais duais, obtenção dos arcos justos, construção da arborescência e verificação de otimalidade dual.}
	\tcblower
	\begin{lstlisting}[language=Python]
def andras_frank_algorithm(D):
    A_zero, Dual_list = phase1(D,"r0")
    arborescence_frank = phase2(D, "r0", A_zero)
    arborescence_frank_v2 = phase2_v2(D, "r0", A_zero)
    dual_frank = check_dual_optimality_condition(
        arborescence_frank, Dual_list)
    dual_frank_v2 = check_dual_optimality_condition(
        arborescence_frank_v2, Dual_list)
    return arborescence_frank, arborescence_frank_v2, dual_frank, dual_frank_v2
\end{lstlisting}
\end{tcolorbox}

\subsection*{Exemplo de execução do algoritmo}

Aqui ilustraremos cada fase do algoritmo de András Frank: inicialização dos potenciais, iterações de elevação, construção de \(A_0\), e formação da arborescência final. Utilizaremos o dígrafo \(D_{32}\) estendido com a raiz \(r_0\), mostrado na Figura~\ref{fig:frank-exemplo-inicial}.

\begin{figure}[H]
	\centering
	\begin{tikzpicture}[>=Stealth, scale=0.9, every node/.style={scale=0.9}]
		\tikzset{
			vtx/.style={circle, draw, fill=blue!8, thick, inner sep=1.2pt, minimum size=6.5mm},
			vtxRoot/.style={circle, draw, fill=red!15, thick, inner sep=1.2pt, minimum size=6.5mm},
			arc/.style={->, thick, gray!60},
			cost/.style={font=\scriptsize, fill=white, inner sep=1pt, rounded corners=1pt}
		}

		\node[font=\bfseries, blue!80!black] at (3, 4.5) {Dígrafo inicial $D$ com raiz $r_0$};

		% Raiz
		\node[vtxRoot] (r0) at (3, 3) {$r_0$};

		% Vértices camada 1
		\node[vtx] (u1) at (0, 1.5) {$u_1$};
		\node[vtx] (u2) at (3, 1.5) {$u_2$};
		\node[vtx] (u3) at (6, 1.5) {$u_3$};

		% Vértices camada 2
		\node[vtx] (v1) at (1, 0) {$v_1$};
		\node[vtx] (v2) at (3, 0) {$v_2$};
		\node[vtx] (v3) at (5, 0) {$v_3$};

		% Arcos da raiz
		\draw[arc] (r0) -- node[cost, left, pos=0.4] {1} (u1);
		\draw[arc] (r0) -- node[cost, right] {2} (u2);
		\draw[arc] (r0) -- node[cost, right, pos=0.4] {1} (u3);

		% Arcos camada 1
		\draw[arc] (u1) to[bend left=15] node[cost, above] {1} (u2);
		\draw[arc] (u2) to[bend left=15] node[cost, above] {3} (u3);

		% Arcos entrando camada 2
		\draw[arc] (u1) -- node[cost, left] {3} (v1);
		\draw[arc] (u2) -- node[cost, right] {2} (v2);
		\draw[arc] (u3) -- node[cost, right] {4} (v3);
		\draw[arc] (u1) to[bend right=20] node[cost, below, pos=0.3] {5} (v2);

		% Arcos internos camada 2
		\draw[arc] (v1) -- node[cost, above] {1} (v2);
		\draw[arc] (v2) -- node[cost, above] {2} (v3);

		% Arco retorno
		\draw[arc] (v1) to[bend right=30] node[cost, left, pos=0.3] {2} (u1);

		\node[font=\small, gray!70!black] at (3, -1.2) {Pesos originais $c(u,v)$};
	\end{tikzpicture}
	\caption{Dígrafo inicial \(D\) com raiz \(r_0\) e 6 vértices adicionais. O dígrafo contém múltiplos arcos com pesos variados. A Fase~1 do algoritmo iniciará com potenciais \(y(v) = 0\) para todos os vértices, correspondendo aos custos reduzidos iniciais \(c_y(u,v) = c(u,v)\).}
	\label{fig:frank-exemplo-inicial}
\end{figure}

\paragraph*{Fase 1 — Iteração 1:} O algoritmo inicia com todos os potenciais \(y(v) = 0\) e \(D_0\) vazio (sem arcos justos). Calcula-se as componentes fortemente conexas de \(D_0\): cada vértice forma sua própria componente. As fontes são todos os vértices exceto \(r_0\). Para o conjunto \(X = \{u_1, u_2, u_3\}\), identifica-se os arcos entrando: \((r_0, u_1)\) com peso 1, \((r_0, u_2)\) com peso 2, \((r_0, u_3)\) com peso 1. O mínimo é \(\Delta(X) = 1\). Subtraindo esse valor, os arcos \((r_0, u_1)\) e \((r_0, u_3)\) tornam-se justos (peso 0) e são adicionados a \(A_0\).

\begin{figure}[H]
	\centering
	\begin{tikzpicture}[>=Stealth, scale=0.85, every node/.style={scale=0.85}]
		\tikzset{
			vtx/.style={circle, draw, fill=blue!8, thick, inner sep=1.2pt, minimum size=6.5mm},
			vtxRoot/.style={circle, draw, fill=red!15, thick, inner sep=1.2pt, minimum size=6.5mm},
			vtxX/.style={circle, draw, fill=orange!20, thick, inner sep=1.2pt, minimum size=6.5mm},
			arc/.style={->, thick, gray!60},
			arcTight/.style={->, ultra thick, blue!70!black},
			cost/.style={font=\scriptsize, fill=white, inner sep=1pt},
			costTight/.style={font=\scriptsize\bfseries, fill=blue!10, inner sep=2pt, draw=blue!70!black, rounded corners=1pt}
		}

		% Antes
		\begin{scope}
			\node[font=\bfseries, blue!80!black] at (2, 4.5) {Antes: $\Delta(X_1) = 1$};

			\node[vtxRoot] (r0) at (2, 3) {$r_0$};
			\node[vtxX] (u1) at (0, 1.5) {$u_1$};
			\node[vtxX] (u2) at (2, 1.5) {$u_2$};
			\node[vtxX] (u3) at (4, 1.5) {$u_3$};

			\draw[arc] (r0) -- node[cost, left, pos=0.4] {1} (u1);
			\draw[arc] (r0) -- node[cost, right] {2} (u2);
			\draw[arc] (r0) -- node[cost, right, pos=0.4] {1} (u3);

			\node[font=\small, orange!80!black] at (2, 0.5) {$X_1 = \{u_1, u_2, u_3\}$};
		\end{scope}

		% Depois
		\begin{scope}[xshift=7cm]
			\node[font=\bfseries, blue!80!black] at (2, 4.5) {Depois: arcos justos};

			\node[vtxRoot] (r0b) at (2, 3) {$r_0$};
			\node[vtx] (u1b) at (0, 1.5) {$u_1$};
			\node[vtx] (u2b) at (2, 1.5) {$u_2$};
			\node[vtx] (u3b) at (4, 1.5) {$u_3$};

			\draw[arcTight] (r0b) -- node[costTight, left, pos=0.4] {0} (u1b);
			\draw[arc] (r0b) -- node[cost, right] {1} (u2b);
			\draw[arcTight] (r0b) -- node[costTight, right, pos=0.4] {0} (u3b);

			\node[font=\small, blue!70!black] at (2, 0.5) {2 arcos adicionados a $A_0$};
		\end{scope}
	\end{tikzpicture}
	\caption{Fase~1, Iteração 1: Elevação de potenciais para \(X_1 = \{u_1, u_2, u_3\}\). À esquerda, o conjunto \(X_1\) (em laranja) sem arcos justos entrando. Calcula-se \(\Delta(X_1) = \min\{1, 2, 1\} = 1\). À direita, após subtrair \(\Delta(X_1)\), os arcos \((r_0, u_1)\) e \((r_0, u_3)\) tornam-se \textcolor{blue!70!black}{justos} (custo reduzido 0) e são adicionados a \(A_0\).}
	\label{fig:frank-exemplo-iter1}
\end{figure}

\paragraph*{Fase 1 — Iteração 2:} Com \(A_0 = \{(r_0, u_1), (r_0, u_3)\}\), as componentes fortemente conexas são: \(\{r_0, u_1, u_3\}\) e \(\{u_2\}\), \(\{v_1\}\), \(\{v_2\}\), \(\{v_3\}\). A fonte (excluindo a raiz) é \(\{u_2\}\). Para \(X = \{u_2\}\), o único arco entrando é \((r_0, u_2)\) com peso atual 1 (após iteração anterior). Portanto \(\Delta(X) = 1\), e após subtração, \((r_0, u_2)\) torna-se justo. Também na mesma iteração, para \(X = \{v_1, v_2, v_3\}\), calcula-se \(\Delta(X) = \min\{3, 2, 4, 5\} = 2\), e os arcos são atualizados conforme ilustrado.

\begin{figure}[H]
	\centering
	\begin{tikzpicture}[>=Stealth, scale=0.75, every node/.style={scale=0.75}]
		\tikzset{
			vtx/.style={circle, draw, fill=blue!8, thick, inner sep=1.2pt, minimum size=6mm},
			vtxRoot/.style={circle, draw, fill=red!15, thick, inner sep=1.2pt, minimum size=6mm},
			vtxX/.style={circle, draw, fill=orange!20, thick, inner sep=1.2pt, minimum size=6mm},
			arc/.style={->, thick, gray!60},
			arcTight/.style={->, ultra thick, blue!70!black},
			cost/.style={font=\scriptsize, fill=white, inner sep=1pt},
			costTight/.style={font=\scriptsize\bfseries, fill=blue!10, inner sep=2pt, draw=blue!70!black, rounded corners=1pt}
		}

		\node[font=\bfseries, blue!80!black] at (3, 5.5) {Iteração 2: $\Delta(\{v_1, v_2, v_3\}) = 2$};

		\node[vtxRoot] (r0) at (3, 4) {$r_0$};
		\node[vtx] (u1) at (0.5, 2.5) {$u_1$};
		\node[vtx] (u2) at (3, 2.5) {$u_2$};
		\node[vtx] (u3) at (5.5, 2.5) {$u_3$};
		\node[vtxX] (v1) at (1.5, 0.5) {$v_1$};
		\node[vtxX] (v2) at (3, 0.5) {$v_2$};
		\node[vtxX] (v3) at (4.5, 0.5) {$v_3$};

		% Arcos já justos
		\draw[arcTight] (r0) -- node[costTight, left, pos=0.3] {0} (u1);
		\draw[arcTight] (r0) -- node[costTight, right, pos=0.3] {0} (u3);
		\draw[arcTight] (u1) to[bend left=12] node[costTight, above] {0} (u2);

		% Arcos entrando em X (antes de ficarem justos)
		\draw[arc] (u1) -- node[cost, left, fill=orange!10] {3→1} (v1);
		\draw[arc] (u2) -- node[cost, right, fill=orange!10] {2→0} (v2);
		\draw[arc] (u3) -- node[cost, right] {4→2} (v3);
		\draw[arc] (u1) to[bend right=15] node[cost, below, pos=0.35] {5→3} (v2);

		\node[font=\small, orange!80!black] at (3, -0.5) {Após subtração: $(u_2, v_2)$ torna-se justo};
	\end{tikzpicture}
	\caption{Fase~1, Iteração 2: Para o conjunto \(X = \{v_1, v_2, v_3\}\) (em laranja), calcula-se \(\Delta(X) = \min\{3, 2, 4, 5\} = 2\). Após subtrair esse valor dos arcos entrando em \(X\), o arco \((u_2, v_2)\) atinge custo reduzido 0 e é adicionado a \(A_0\). Note que \((u_1, u_2)\) também se tornou justo na mesma iteração ao processar \(X = \{u_2\}\).}
	\label{fig:frank-exemplo-iter2}
\end{figure}

\paragraph*{Fase 1 — Estado final:} Após mais iterações elevando potenciais para os conjuntos restantes, o dígrafo \(D_0\) de arcos justos fica completo, contendo uma r-arborescência. A Figura~\ref{fig:frank-exemplo-A0-final} mostra o conjunto final \(A_0\) com 6 arcos de custo reduzido zero.

\begin{figure}[H]
	\centering
	\begin{tikzpicture}[>=Stealth, scale=0.85, every node/.style={scale=0.85}]
		\tikzset{
			vtx/.style={circle, draw, fill=blue!8, thick, inner sep=1.2pt, minimum size=6.5mm},
			vtxRoot/.style={circle, draw, fill=red!15, thick, inner sep=1.2pt, minimum size=6.5mm},
			arcTight/.style={->, ultra thick, blue!70!black},
			costTight/.style={font=\scriptsize\bfseries, fill=blue!10, inner sep=2pt, draw=blue!70!black, rounded corners=1pt}
		}

		\node[font=\bfseries, blue!80!black] at (3, 4.5) {$A_0$ final: arcos justos após Fase~1};

		\node[vtxRoot] (r0) at (3, 3) {$r_0$};
		\node[vtx] (u1) at (0.5, 1.5) {$u_1$};
		\node[vtx] (u2) at (3, 1.5) {$u_2$};
		\node[vtx] (u3) at (5.5, 1.5) {$u_3$};
		\node[vtx] (v1) at (1.5, 0) {$v_1$};
		\node[vtx] (v2) at (3, 0) {$v_2$};
		\node[vtx] (v3) at (4.5, 0) {$v_3$};

		\draw[arcTight] (r0) -- node[costTight, left, pos=0.3] {0} (u1);
		\draw[arcTight] (r0) -- node[costTight, right, pos=0.3] {0} (u3);
		\draw[arcTight] (u1) to[bend left=12] node[costTight, above] {0} (u2);
		\draw[arcTight] (u1) -- node[costTight, left] {0} (v1);
		\draw[arcTight] (u2) -- node[costTight, right] {0} (v2);
		\draw[arcTight] (v2) -- node[costTight, above] {0} (v3);

		\node[font=\small, blue!70!black] at (3, -1.2) {$|A_0| = 6$ arcos com custo reduzido $c_y = 0$};
	\end{tikzpicture}
	\caption{Conjunto final \(A_0\) de arcos justos após conclusão da Fase~1. Todos os 6 arcos \textcolor{blue!70!black}{destacados} possuem custo reduzido zero: \((r_0, u_1)\), \((r_0, u_3)\), \((u_1, u_2)\), \((u_1, v_1)\), \((u_2, v_2)\), \((v_2, v_3)\). O dígrafo formado por esses arcos é acíclico e alcança todos os vértices a partir de \(r_0\), satisfazendo as condições para prosseguir à Fase~2.}
	\label{fig:frank-exemplo-A0-final}
\end{figure}

\paragraph*{Fase 2 — Construção da arborescência:} Com \(A_0\) completo e acíclico, a Fase~2 constrói incrementalmente a arborescência final. Inicia-se com \(\texttt{Arb} = \{r_0\}\) e em cada iteração adiciona-se um arco \((u,v) \in A_0\) tal que \(u \in \texttt{Arb}\) e \(v \notin \texttt{Arb}\). A Figura~\ref{fig:frank-exemplo-arborescencia-final} mostra a arborescência resultante.

\begin{figure}[H]
	\centering
	\begin{tikzpicture}[>=Stealth, scale=0.85, every node/.style={scale=0.85}]
		\tikzset{
			vtx/.style={circle, draw, fill=green!15, thick, inner sep=1.2pt, minimum size=6.5mm},
			vtxRoot/.style={circle, draw, fill=red!15, thick, inner sep=1.2pt, minimum size=6.5mm},
			arcArb/.style={->, ultra thick, green!70!black},
			costArb/.style={font=\scriptsize\bfseries, fill=green!10, inner sep=2pt, draw=green!70!black, rounded corners=1pt}
		}

		\node[font=\bfseries, blue!80!black] at (3, 4.5) {Arborescência final (Fase~2)};

		\node[vtxRoot] (r0) at (3, 3) {$r_0$};
		\node[vtx] (u1) at (0.5, 1.5) {$u_1$};
		\node[vtx] (u2) at (3, 1.5) {$u_2$};
		\node[vtx] (u3) at (5.5, 1.5) {$u_3$};
		\node[vtx] (v1) at (1.5, 0) {$v_1$};
		\node[vtx] (v2) at (3, 0) {$v_2$};
		\node[vtx] (v3) at (4.5, 0) {$v_3$};

		\draw[arcArb] (r0) -- node[costArb, left, pos=0.3] {1} (u1);
		\draw[arcArb] (r0) -- node[costArb, right, pos=0.3] {1} (u3);
		\draw[arcArb] (u1) to[bend left=12] node[costArb, above] {1} (u2);
		\draw[arcArb] (u1) -- node[costArb, left] {3} (v1);
		\draw[arcArb] (u2) -- node[costArb, right] {2} (v2);
		\draw[arcArb] (v2) -- node[costArb, above] {2} (v3);

		\node[font=\small, green!70!black] at (3, -1.2) {Custo total: $1+1+1+3+2+2 = 10$};
	\end{tikzpicture}
	\caption{Arborescência de custo mínimo final obtida pela Fase~2. Os pesos mostrados são os \emph{originais} de \(D\), restaurados após a construção. Cada vértice não-raiz possui exatamente um arco de entrada, não há ciclos, e todos os vértices são alcançáveis a partir de \(r_0\). A verificação de otimalidade dual confirma que cada conjunto \(X\) que teve potenciais elevados possui exatamente um arco da arborescência cruzando sua fronteira, garantindo que a solução é ótima.}
	\label{fig:frank-exemplo-arborescencia-final}
\end{figure}

\paragraph*{Verificação de otimalidade dual:} A função \texttt{check\_dual\_optimality\_condition} confirma que para cada par \((X, \Delta(X))\) em \texttt{Dual\_list} (conjuntos cujos potenciais foram elevados com \(\Delta(X) > 0\)), existe exatamente um arco da arborescência final cruzando a fronteira de \(X\). Essa condição, juntamente com os arcos justos, garante que as condições de folga complementar da programação linear dual são satisfeitas, implicando que a arborescência encontrada é de custo mínimo global.

\subsection{Correspondência entre teoria e implementação}

A implementação em Python do algoritmo de András Frank segue fielmente a descrição teórica primal-dual apresentada anteriormente. A tabela abaixo estabelece o paralelo direto entre os passos teóricos e sua realização no código:

\begin{table}[H]
	\centering
	\scriptsize
	\renewcommand{\arraystretch}{1.3}
	\begin{tabular}{>{\raggedright\arraybackslash}p{0.44\textwidth}|>{\raggedright\arraybackslash}p{0.48\textwidth}}
		\toprule
		\textbf{\normalsize Descrição Teórica} & \textbf{\normalsize Implementação Python} \\
		\midrule
		\rowcolor{blue!5}
		\textbf{Passo 1:} Inicialização

		Defina \(y(v) := 0\) para todo \(v \in V\).

		Inicialize \(A_0 := \emptyset\).

		Construa dígrafo vazio \(D_0\) (arcos justos).
		                                       &
		\textbf{Função \texttt{phase1} — Linhas 2--5:}

		\texttt{D\_copy = D\_original.copy()}

		\texttt{A\_zero = []}

		\texttt{D\_zero = build\_D\_zero(D\_copy)}

		\vspace{1mm}
		Potenciais \(y(v) = 0\) implícitos, custos \(c_y = c\).
		\\
		\midrule
		\rowcolor{green!5}
		\textbf{Passo 2:} Elevação de potenciais

		Enquanto \(\exists X \subseteq V \setminus \{r\}\) sem arco justo:

		\quad Calcule \(\Delta(X) := \min\{c_y(u,v) : u \notin X, v \in X\}\)

		\quad Atualize \(y(v) := y(v) + \Delta(X)\), \(\forall v \in X\)

		\quad Adicione arcos com \(c_y = 0\) a \(A_0\)
		                                       &
		\textbf{Loop principal — Linhas 7--22:}

		\texttt{C = nx.condensation(D\_zero)}

		\texttt{sources = [x for x in C.nodes()]}

		\texttt{\quad if C.in\_degree(x) == 0]}

		Para fonte \texttt{u} (exceto raiz):

		\texttt{\quad X = C.nodes[u]["members"]}

		\texttt{\quad arcs = get\_arcs\_entering\_X(D, X)}

		\texttt{\quad min\_w = get\_minimum\_weight\_cut(arcs)}

		\texttt{\quad update\_weights\_in\_X(D, arcs, min\_w,}

		\texttt{\quad\quad A\_zero, D\_zero)}
		\\
		\midrule
		\rowcolor{orange!5}
		\textbf{Passo 2(a):} Identificar arcos entrando

		Determine \(\{(u,v) \in A : u \notin X, v \in X\}\)
		                                       &
		\textbf{Função \texttt{get\_arcs\_entering\_X}:}

		\texttt{return [(u, v, data)}

		\texttt{\quad for u, v, data in D.edges(data=True)}

		\texttt{\quad if u not in X and v in X]}
		\\
		\midrule
		\rowcolor{purple!5}
		\textbf{Passo 2(b):} Calcular \(\Delta(X)\)

		\(\Delta(X) := \min\{c_y(u,v) : u \notin X, v \in X\}\)
		                                       &
		\textbf{Função \texttt{get\_minimum\_weight\_cut}:}

		\texttt{return min(data["w"]}

		\texttt{\quad for \_, \_, data in arcs)}
		\\
		\midrule
		\rowcolor{yellow!5}
		\textbf{Passo 2(c):} Atualizar pesos

		Para \((u,v)\) entrando em \(X\):

		\quad \(c_y(u,v) := c_y(u,v) - \Delta(X)\)

		\quad Se \(c_y(u,v) = 0\), adicione a \(A_0\)
		                                       &
		\textbf{Função \texttt{update\_weights\_in\_X}:}

		\texttt{for u, v, \_ in arcs:}

		\texttt{\quad D[u][v]["w"] -= min\_weight}

		\texttt{\quad if D[u][v]["w"] == 0:}

		\texttt{\quad\quad A\_zero.append((u, v))}

		\texttt{\quad\quad D\_zero.add\_edge(u, v)}
		\\
		\midrule
		\rowcolor{cyan!5}
		\textbf{Passo 3:} Verificar término

		Se \(D_0\) contém r-arborescência, encerre.
		                                       &
		\textbf{Condição — Linhas 11--12:}

		\texttt{if len(sources) == 1: break}

		\vspace{1mm}
		Uma fonte \(\Rightarrow\) r-arborescência acíclica.
		\\
		\midrule
		\rowcolor{red!5}
		\textbf{Passo 4:} Construir arborescência

		Construa \(F\) a partir de \(A_0\), conectando vértices incrementalmente.
		                                       &
		\textbf{Função \texttt{phase2} (incremental):}

		\texttt{Arb = nx.DiGraph(); Arb.add\_node(r0)}

		\texttt{for \_ in range(n - 1):}

		\texttt{\quad for u, v in A\_zero:}

		\texttt{\quad\quad if u in Arb and v not in Arb:}

		\texttt{\quad\quad\quad Arb.add\_edge(u, v, **data)}

		\texttt{\quad\quad\quad break}

		\vspace{1mm}
		Complexidade: \(O(nm)\).
		\\
		\midrule
		\rowcolor{red!5}
		\textbf{Passo 4:} Versão otimizada

		Mesma ideia, com fila de prioridade.
		                                       &
		\textbf{Função \texttt{phase2\_v2} (BFS):}

		\texttt{Arb = nx.DiGraph()}

		\texttt{for i, (u, v) in enumerate(A\_zero):}

		\texttt{\quad Arb.add\_edge(u, v, w=i)}

		\texttt{q = [] \# fila de prioridade}

		\texttt{while q:}

		\texttt{\quad \_, u, v = heapq.heappop(q)}

		\texttt{\quad if v in V: continue}

		\texttt{\quad A.add\_edge(u, v, w=D[u][v]["w"])}

		\vspace{1mm}
		Complexidade: \(O(m \log m)\).
		\\
		\midrule
		\rowcolor{pink!5}
		\textbf{Otimalidade dual}

		Para cada \(X\) elevado (\(\Delta(X) > 0\)), exatamente um arco de \(F\) cruza \(\delta^-(X)\).
		                                       &
		\textbf{Função \texttt{check\_dual\_optimality}:}

		\texttt{for X, z in Dual\_list:}

		\texttt{\quad count = 0}

		\texttt{\quad for u, v in Arb.edges():}

		\texttt{\quad\quad if u not in X and v in X:}

		\texttt{\quad\quad\quad count += 1}

		\texttt{\quad\quad\quad if count > 1: return False}

		\texttt{return True}
		\\
		\bottomrule
	\end{tabular}
	\caption{Correspondência entre a descrição teórica do algoritmo de András Frank e sua implementação em Python. Cores: inicialização (azul), elevação de potenciais (verde/laranja/roxo/amarelo), verificação (ciano), construção (vermelho) e validação dual (rosa).}
	\label{tab:frank-teoria-implementacao}
\end{table}

Esta correspondência demonstra que a implementação traduz fielmente a abordagem primal-dual em código executável. As funções auxiliares (\texttt{get\_arcs\_entering\_X}, \texttt{get\_minimum\_weight\_cut}, \texttt{update\_weights\_in\_X}, \texttt{phase1}, \texttt{phase2}, \texttt{phase2\_v2},
\texttt{check\_dual\_}-
\texttt{optimality\_condition} encapsulam exatamente as operações descritas na teoria, preservando as propriedades de correção e as garantias de otimalidade do algoritmo original. A utilização de componentes fortemente conexas (\texttt{nx.condensation}) para identificar conjuntos sem arcos justos entrando é uma implementação eficiente da verificação teórica, evitando enumeração explícita de todos os subconjuntos de vértices.

\chapter{Algoritmo de András Frank}

Neste capítulo, apresentaremos o algoritmo de András Frank, que também determina
uma arborescência de custo mínimo em um digrafo ponderado.
O algoritmo baseia-se em uma abordagem primal--dual:
a elevação gulosa de potenciais duais\footnote{\textbf{Potencial dual} é uma variável associada a cada vértice, que representa um "desconto" aplicado ao custo dos arcos que entram nesse vértice e servem para ajustar os custos de modo que certas restrições fiquem "justas", - satisfeitas com igualdade, ou seja, o valor da variável atinge exatamente o limite imposto pela restrição.} e (ii) a construção de uma solução gulosa
a partir dos arcos de custo reduzido zero.
Diferentemente do algoritmo de Chu--Liu--Edmonds, que opera diretamente sobre os custos,
o método de Frank manipula potenciais duais associados aos vértices,
gerando arcos \emph{justos} (de custo reduzido zero) que formam a arborescência ótima.
O propósito deste capítulo é fornecer uma descrição precisa tanto do algoritmo
quanto da implementação desenvolvida neste trabalho.

\section{O algoritmo}
O algoritmo de András Frank também recebe uma tripla \((D,c,r)\), em que \(D=(V,A)\) é um digrafo,
\(c\colon A\to\mathbb{R}\) é uma função custo e \(r\in V\) é a raiz,
sob a hipótese de que \(D\) admite ao menos uma \(r\)-arborescência e devolve uma \(r\)-arborescência \(c\)-mínima de \(D\).

Assim como no capítulo anterior, adotamos a terminologia de \textbf{\(r\)-digrafo ponderado}
para uma tripla \((D,c,r)\) em que $(D, c)$ é um digrafo ponderado,
$r$ é um vértice de $D$, $\delta^-(r) = \varnothing$
e $D$ possui uma $r$-arborescência.

Vamos desenvolver as ideias do algoritmo utilizando o mesmo dígrafo que apresentamos
no capítulo anterior.

\subsection*{Abordagem Gulosa--Dual}

Considere novamente o dígrafo \(D\) figura 26 a seguir, com custos nos arcos note que é o mesmo digrafo da figura 1 que utilizamos para ilustrar o algoritmo de Chu--Liu--Edmonds. E portanto, suponha que T é uma arborescência de custo mínimo, com seus arcos destacados em azul.

\begin{figure}[h]
	\centering
	\begin{tikzpicture}[>=Stealth]
		% estilos
		\tikzset{
			vtx/.style={circle, draw,fill=blue!8, thick, inner sep=1.2pt, minimum size=6mm},
			chosen/.style={->, very thick, draw=blue},
			other/.style={->, semithick, draw=black!35},
			costB/.style={font=\scriptsize, fill=white, inner sep=1pt, text=blue},
			costG/.style={font=\scriptsize, fill=white, inner sep=1pt, text=black!55}
		}

		% layout (mais espaçado)
		\node[vtx] (r) at (0,0) {$r$};
		\node[vtx] (a) at (3,1.6) {$a$};
		\node[vtx] (b) at (3,-1.6) {$b$};
		\node[vtx] (c) at (6,2.2) {$c$};
		\node[vtx] (d) at (6,0.0) {$d$};
		\node[vtx] (e) at (6,-2.2) {$e$};

		% --- arcos NÃO escolhidos (cinza), agora com curvaturas mais limpas ---
		\draw[other] (b) to[bend left=12] node[costG, above] {4} (a);   % b->a
		\draw[other] (a) to[bend left=14] node[costG, below] {5} (b);   % a->b

		% r->c: externo por cima, suave (pontos de controle explícitos)
		\draw[other]
		(r) .. controls (0.8,2.4) and (5.2,3.0) ..
		node[costG, yshift=4pt, pos=0.53] {3} (c);

		% a->d (referência)
		\draw[other] (a) -- node[costG, right] {3} (d);

		% c->e: leve curvatura pela direita (sem "barriga")
		\draw[other]
		(c) .. controls (7.6,2.6) and (7.6,-2.6) ..
		node[costG, xshift=8pt, pos=0.52] {5} (e);

		% --- arcos ESCOLHIDOS (AZUL) — formam a r-arborescência T ---
		\draw[chosen] (r) -- node[costB, above] {1} (a);   % r->a
		\draw[chosen] (r) -- node[costB, below] {2} (b);   % r->b
		\draw[chosen] (a) -- node[costB, above] {1} (c);   % a->c
		\draw[chosen] (b) -- node[costB, above] {1} (d);   % b->d
		\draw[chosen] (b) -- node[costB, below] {2} (e);   % b->e
	\end{tikzpicture}

	\caption{O dígrafo \(D\) com custos nos arcos. Os arcos em \textcolor{blue}{azul}
		formam a \(r\)-arborescência de custo mínimo \(T\) (custo total = 7).}
\end{figure}

O algoritmo de Frank utiliza \textbf{potenciais} \(y: V \to \mathbb{R}\) associados aos vértices
para definir custos reduzidos nos arcos. Para cada arco \(a = (u,v) \in A\), o
\textbf{custo \(y\)-reduzido} é dado por
\[
	c_y(u,v) := c(u,v) - y(v),
\]
onde \(v\) é o vértice de destino do arco. Note que, diferentemente do procedimento de redução do capítulo anterior (onde subtraíamos o mínimo custo de entrada em cada vértice), aqui o potencial \(y(v)\) é uma variável que será ajustada pelo algoritmo através da seguinte noção.

Um arco \(a = (u,v)\) é dito \textbf{justo} (ou \textbf{tight}) se \(c_y(u,v) = 0\), isto é, se o custo original coincide com o potencial do vértice de destino: \(c(u,v) = y(v)\). A ideia central do algoritmo de Frank consiste em duas etapas: (i)~elevar os potenciais \(y(v)\) para cada \(v \neq r\) até que cada vértice possua ao menos um arco justo entrando nele; (ii)~construir uma arborescência ótima utilizando exclusivamente arcos justos.

\begin{tcolorbox}[
		enhanced, breakable,
		colback=blue!5!white, colframe=blue!70!black,
		colbacktitle=blue!85!black, coltitle=white,
		fonttitle=\bfseries,
		title={Arcos justos: interpretação e cálculo},
		sharp corners, boxrule=0.8pt
	]
	Um arco \((u,v)\) é \textbf{justo} quando seu custo reduzido é zero, ou seja, quando o custo original \(c(u,v)\) coincide exatamente com o potencial \(y(v)\) do vértice de destino. Formalmente:
	\[
		c_y(u,v) = c(u,v) - y(v) = 0 \quad\Longleftrightarrow\quad c(u,v) = y(v).
	\]

	Em outras palavras, o potencial \(y(v)\) funciona como um ``desconto'' aplicado aos arcos que chegam em \(v\). Quando esse desconto é suficiente para anular o custo reduzido, o arco torna-se justo e candidato a participar da solução ótima.

	\medskip
	\textbf{Exemplos numéricos (referência: figura~26):}
	\begin{itemize}\setlength{\itemsep}{3pt}
		\item Arco \((r,a)\): \(c(r,a) = 1\), \(y(a) = 1 \implies c_y(r,a) = 1-1 = 0\). \quad\textcolor{blue!70!black}{\textbf{Justo.}}
		\item Arco \((b,a)\): \(c(b,a) = 4\), \(y(a) = 1 \implies c_y(b,a) = 4-1 = 3\). \quad\textcolor{red!70!black}{\textbf{Não justo.}}
		\item Arco \((a,c)\): \(c(a,c) = 1\), \(y(c) = 1 \implies c_y(a,c) = 1-1 = 0\). \quad\textcolor{blue!70!black}{\textbf{Justo.}}
	\end{itemize}

	\textbf{Propriedade fundamental:} o potencial \(y(v)\) afeta \emph{exclusivamente} os arcos que entram em \(v\). Por exemplo, \(y(c)\) influencia apenas arcos da forma \((\cdot, c)\), como \((r,c)\) e \((a,c)\), não afetando arcos que saem de \(c\).

	\medskip
	O algoritmo eleva progressivamente os potenciais até que cada vértice \(v \neq r\) possua ao menos um arco justo de entrada. No exemplo da figura~26, os arcos justos após a elevação correspondem precisamente aos arcos da arborescência ótima.
\end{tcolorbox}

% Figura 27: Exemplo visual com múltiplos arcos e vértice raiz
\begin{figure}[H]
	\centering
	\begin{tikzpicture}[>=Stealth]
		% estilos
		\tikzset{
			vtx/.style={circle, draw, fill=blue!8, thick, inner sep=1.2pt, minimum size=8mm},
			arc/.style={->, very thick, draw=blue},
			arcTight/.style={->, very thick, draw=red!70!black},
			cost/.style={font=\scriptsize, fill=white, inner sep=1pt, text=blue},
			pot/.style={font=\small, text=red!70!black},
			label/.style={font=\small, text=black!80}
		}

		% vértices
		\node[vtx] (r) at (0,0) {$r$};
		\node[vtx] (u) at (3,1.5) {$u$};
		\node[vtx] (v) at (3,-1.5) {$v$};

		% arcos
		\draw[arcTight] (r) -- node[cost, above left] {$c(r,u)=2$} (u);
		\draw[arc] (r) -- node[cost, below left] {$c(r,v)=4$} (v);
		\draw[arcTight] (u) -- node[cost, right] {$c(u,v)=3$} (v);

		% potenciais
		\node[pot] at (3,2.5) {$y(u)=2$};
		\node[pot] at (3,-2.5) {$y(v)=3$};

		% destaques
		\draw[red!70, thick, dashed] (u) circle [radius=0.7];
		\draw[red!70, thick, dashed] (v) circle [radius=0.7];

		% custos reduzidos
		\node[label] at (-0.0,1.5) {$c_y(r,u)=2-2=0$};
		\node[label] at (-0.0,-1.5) {$c_y(r,v)=4-3=1$};
		\node[label] at (5.0,-0.5) {$c_y(u,v)=3-3=0$};
	\end{tikzpicture}
	\caption{O vértice $r$ é a raiz. Os potenciais são $y(u)=2$ e $y(v)=3$. O arco $(r,u)$ tem custo reduzido $0$ (justo), $(r,v)$ tem custo reduzido $1$ (não justo), $(u,v)$ tem custo reduzido $0$ (justo). Os potenciais só afetam os arcos que chegam em cada vértice.}
	\label{fig:potencial-destino}
\end{figure}

\subsection*{Elevação de Potenciais}

Inicialmente, definimos \(y(v) = 0\) para todo \(v \in V\).
Para cada vértice \(v \neq r\), calculamos
\[
	\Delta(v) := \min\{\, c(a) : a \in \delta^{-}(v) \,\} - y(v).
\]
O valor \(\Delta(v)\) representa quanto precisamos elevar \(y(v)\) para que exista
ao menos um arco de custo reduzido zero entrando em \(v\).

Vamos ilustrar esse processo no dígrafo \(D\).

\begin{figure}[H]
	\centering
	\begin{tikzpicture}[>=Stealth]
		% estilos
		\tikzset{
			vtx/.style={circle, draw,fill=blue!8, thick, inner sep=1.2pt, minimum size=6mm},
			arc/.style={->, semithick, draw=black!35},
			costG/.style={font=\scriptsize, fill=white, inner sep=1pt, text=black!55}
		}

		% layout
		\node[vtx] (r) at (0,0) {$r$};
		\node[vtx] (a) at (3,1.6) {$a$};
		\node[vtx] (b) at (3,-1.6) {$b$};
		\node[vtx] (c) at (6,2.2) {$c$};
		\node[vtx] (d) at (6,0.0) {$d$};
		\node[vtx] (e) at (6,-2.2) {$e$};

		% arcos com custos originais
		\draw[arc] (r) -- node[costG, above] {1} (a);
		\draw[arc] (r) -- node[costG, below] {2} (b);
		\draw[arc] (r) .. controls (0.8,2.4) and (5.2,3.0) .. node[costG, yshift=4pt, pos=0.53] {3} (c);
		\draw[arc] (b) to[bend left=12] node[costG, above] {4} (a);
		\draw[arc] (a) to[bend left=14] node[costG, below] {5} (b);
		\draw[arc] (a) -- node[costG, above] {1} (c);
		\draw[arc] (a) -- node[costG, right] {3} (d);
		\draw[arc] (b) -- node[costG, above] {1} (d);
		\draw[arc] (b) -- node[costG, below] {2} (e);
		\draw[arc] (c) .. controls (7.6,2.6) and (7.6,-2.6) .. node[costG, xshift=8pt, pos=0.52] {5} (e);

		% potenciais iniciais
		\node[above=6mm of a, font=\small] {$y(a) = 0$};
		\node[below=2mm of b, font=\small] {$y(b) = 0$};
		\node[above=2mm of c, font=\small] {$y(c) = 0$};
		\node[below=2mm of d, font=\small] {$y(d) = 0$};
		\node[below=2mm of e, font=\small] {$y(e) = 0$};
	\end{tikzpicture}
	\caption{Dígrafo \(D\) com potenciais iniciais \(y(v) = 0\) para todo \(v \in V\).}
\end{figure}

Calculamos \(\Delta(v)\) para cada vértice \(v \neq r\):
\begin{align*}
	\Delta(a) & = \min\{1, 4\} - 0 = 1, \\
	\Delta(b) & = \min\{2, 5\} - 0 = 2, \\
	\Delta(c) & = \min\{3, 1\} - 0 = 1, \\
	\Delta(d) & = \min\{3, 1\} - 0 = 1, \\
	\Delta(e) & = \min\{2, 5\} - 0 = 2.
\end{align*}

Elevamos cada potencial:
\begin{align*}
	y(a) & \leftarrow y(a) + \Delta(a) = 0 + 1 = 1, \\
	y(b) & \leftarrow y(b) + \Delta(b) = 0 + 2 = 2, \\
	y(c) & \leftarrow y(c) + \Delta(c) = 0 + 1 = 1, \\
	y(d) & \leftarrow y(d) + \Delta(d) = 0 + 1 = 1, \\
	y(e) & \leftarrow y(e) + \Delta(e) = 0 + 2 = 2.
\end{align*}

Após a elevação, os custos reduzidos são dados por \(c_y(u,v) = c(u,v) - y(v)\) para cada arco \((u,v)\):

\begin{figure}[H]
	\centering
	\begin{tikzpicture}[>=Stealth]
		% estilos
		\tikzset{
			vtx/.style={circle, draw,fill=blue!8, thick, inner sep=1.2pt, minimum size=6mm},
			tight/.style={->, very thick, draw=blue},
			loose/.style={->, semithick, draw=black!35},
			costB/.style={font=\scriptsize, fill=white, inner sep=1pt, text=blue},
			costG/.style={font=\scriptsize, fill=white, inner sep=1pt, text=black!55}
		}

		% layout
		\node[vtx] (r) at (0,0) {$r$};
		\node[vtx] (a) at (3,1.6) {$a$};
		\node[vtx] (b) at (3,-1.6) {$b$};
		\node[vtx] (c) at (6,2.2) {$c$};
		\node[vtx] (d) at (6,0.0) {$d$};
		\node[vtx] (e) at (6,-2.2) {$e$};

		% arcos justos (custo reduzido = 0)
		\draw[tight] (r) -- node[costB, above] {0} (a);   % 1 - 1 = 0
		\draw[tight] (r) -- node[costB, below] {0} (b);   % 2 - 2 = 0
		\draw[tight] (a) -- node[costB, above] {0} (c);   % 1 - 1 = 0
		\draw[tight] (b) -- node[costB, above] {0} (d);   % 1 - 1 = 0
		\draw[tight] (b) -- node[costB, below] {0} (e);   % 2 - 2 = 0

		% arcos não justos
		\draw[loose] (r) .. controls (0.8,2.4) and (5.2,3.0) .. node[costG, yshift=4pt, pos=0.53] {2} (c);  % 3 - 1 = 2
		\draw[loose] (b) to[bend left=12] node[costG, above] {3} (a);   % 4 - 1 = 3
		\draw[loose] (a) to[bend left=14] node[costG, below] {3} (b);   % 5 - 2 = 3
		\draw[loose] (a) -- node[costG, right] {2} (d);   % 3 - 1 = 2
		\draw[loose] (c) .. controls (7.6,2.6) and (7.6,-2.6) .. node[costG, xshift=8pt, pos=0.52] {3} (e);  % 5 - 2 = 3

		% potenciais finais
		\node[above=6mm of a, font=\small] {$y(a) = 1$};
		\node[below=2mm of b, font=\small] {$y(b) = 2$};
		\node[above=2mm of c, font=\small] {$y(c) = 1$};
		\node[below=2mm of d, font=\small] {$y(d) = 1$};
		\node[below=2mm of e, font=\small] {$y(e) = 2$};
	\end{tikzpicture}
	\caption{Dígrafo após elevação de potenciais. Arcos em \textcolor{blue}{azul} são
		justos (\(c_y = 0\)); arcos em cinza têm custo reduzido positivo.}
\end{figure}

Observe que cada vértice \(v \neq r\) possui agora ao menos um arco justo entrando nele.
Mais ainda: neste exemplo, os arcos justos formam uma \(r\)-arborescência! Precisamos justificar por que a arborescência formada pelos arcos justos é ótima.

\subsection*{Construção da Arborescência}

Se os arcos justos formam uma \(r\)-arborescência, então encontramos a solução ótima.
Caso contrário, os arcos justos contêm ciclos, e precisamos tratá-los adequadamente.

No nosso exemplo, os arcos justos são:
\[
	H = \{\, (r,a), (r,b), (a,c), (b,d), (b,e) \,\}.
\]
Esses arcos formam a \(r\)-arborescência \(T\) de custo mínimo.

Para verificar a otimalidade, note que uma propriedade fundamental dos potenciais duais é:

\begin{proposicao}
	Para toda função \(y: V \to \mathbb{R}\) com \(y(r) = 0\),
	uma \(r\)-arborescência \(T\) é \(c\)-mínima em \(D\) se, e somente se,
	todos os arcos de \(T\) são justos (i.e., \(c_y(a) = 0\) para todo \(a \in T\))
	e \(c_y(a) \geq 0\) para todo \(a \in A\).
\end{proposicao}

\begin{proof}[Prova.]
	Seja \(F\) uma \(r\)-arborescência qualquer. Para cada \(v \in V \setminus \{r\}\),
	seja \(a_v\) o único arco de \(F\) que entra em \(v\). Então
	\begin{align*}
		c(F)
		 & = \sum_{v \in V \setminus \{r\}} c(a_v)                                          \\
		 & = \sum_{v \in V \setminus \{r\}} \bigl(c_y(a_v) + y(v)\bigr)                     \\
		 & = \sum_{v \in V \setminus \{r\}} c_y(a_v) + \sum_{v \in V \setminus \{r\}} y(v).
	\end{align*}
	Como a soma \(\sum_{v \in V \setminus \{r\}} y(v)\) é constante para todas as
	\(r\)-arborescências, minimizar \(c(F)\) equivale a minimizar \(\sum_{v} c_y(a_v)\).
	Se todos os arcos de \(T\) são justos e \(c_y(a) \geq 0\) para todo \(a\),
	então \(\sum_{v} c_y(a_v^T) = 0 \leq \sum_{v} c_y(a_v^F)\) para qualquer \(F\),
	provando que \(T\) é ótima.
\end{proof}

No nosso exemplo, todos os arcos de \(T\) são justos e \(c_y(a) \geq 0\) para todo
arco \(a \in A\), portanto \(T\) é a \(r\)-arborescência de custo mínimo. Evidentemente, nem sempre será o caso que os arcos justos formam uma arborescência e precisaremos tratar ciclos.

\subsection*{Tratamento de Ciclos}

Em geral, após a elevação de potenciais, os arcos justos podem conter ciclos.
Nesse caso, o algoritmo de Frank procede de forma similar ao de Chu--Liu--Edmonds:
contrai cada ciclo justo em um supervértice, resolve o problema recursivamente
no dígrafo contraído, e depois reexpande a solução.

Vamos ilustrar esse processo com um exemplo modificado. Suponha que adicionamos o arco
\((c, a)\) com custo 1 ao dígrafo original:

\begin{figure}[H]
	\centering
	\begin{tikzpicture}[>=Stealth]
		% estilos
		\tikzset{
			vtx/.style={circle, draw,fill=blue!8, thick, inner sep=1.2pt, minimum size=6mm},
			arc/.style={->, semithick, draw=black!35},
			costG/.style={font=\scriptsize, fill=white, inner sep=1pt, text=black!55}
		}

		% layout
		\node[vtx] (r) at (0,0) {$r$};
		\node[vtx] (a) at (3,1.6) {$a$};
		\node[vtx] (b) at (3,-1.6) {$b$};
		\node[vtx] (c) at (6,2.2) {$c$};
		\node[vtx] (d) at (6,0.0) {$d$};
		\node[vtx] (e) at (6,-2.2) {$e$};

		% arcos originais
		\draw[arc] (r) -- node[costG, above] {1} (a);
		\draw[arc] (r) -- node[costG, below] {2} (b);
		\draw[arc] (r) .. controls (0.8,2.4) and (5.2,3.0) .. node[costG, yshift=4pt, pos=0.53] {3} (c);
		\draw[arc] (b) to[bend left=12] node[costG, above] {4} (a);
		\draw[arc] (a) to[bend left=14] node[costG, below] {5} (b);
		\draw[arc] (a) -- node[costG, above] {1} (c);
		\draw[arc] (a) -- node[costG, right] {3} (d);
		\draw[arc] (b) -- node[costG, above] {1} (d);
		\draw[arc] (b) -- node[costG, below] {2} (e);
		\draw[arc] (c) .. controls (7.6,2.6) and (7.6,-2.6) .. node[costG, xshift=8pt, pos=0.52] {5} (e);

		% NOVO arco (c, a)
		\draw[arc, draw=red!60] (c) to[bend left=20] node[costG, left, text=red!60] {1} (a);
	\end{tikzpicture}
	\caption{Dígrafo modificado com o arco \((c,a)\) de custo 1 (em vermelho).}
\end{figure}

Após a elevação de potenciais com \(y(a) = 1, y(c) = 1\), o arco \((c,a)\) torna-se
justo: \(c_y(c,a) = 1 - 1 = 0\).

\begin{figure}[H]
	\centering
	\begin{tikzpicture}[>=Stealth]
		% estilos
		\tikzset{
			vtx/.style={circle, draw,fill=blue!8, thick, inner sep=1.2pt, minimum size=6mm},
			tight/.style={->, very thick, draw=blue},
			cycle/.style={->, ultra thick, draw=red!70},
			loose/.style={->, semithick, draw=black!35},
			costB/.style={font=\scriptsize, fill=white, inner sep=1pt, text=blue},
			costR/.style={font=\scriptsize, fill=white, inner sep=1pt, text=red!70},
			costG/.style={font=\scriptsize, fill=white, inner sep=1pt, text=black!55}
		}

		% layout
		\node[vtx] (r) at (0,0) {$r$};
		\node[vtx] (a) at (3,1.6) {$a$};
		\node[vtx] (b) at (3,-1.6) {$b$};
		\node[vtx] (c) at (6,2.2) {$c$};
		\node[vtx] (d) at (6,0.0) {$d$};
		\node[vtx] (e) at (6,-2.2) {$e$};

		% arcos justos que formam ciclo
		\draw[cycle] (a) -- node[costR, above] {0} (c);
		\draw[cycle] (c) to[bend left=20] node[costR, left] {0} (a);

		% outros arcos justos
		\draw[tight] (r) -- node[costB, above] {0} (a);
		\draw[tight] (r) -- node[costB, below] {0} (b);
		\draw[tight] (b) -- node[costB, above] {0} (d);
		\draw[tight] (b) -- node[costB, below] {0} (e);

		% arcos não justos
		\draw[loose] (r) .. controls (0.8,2.4) and (5.2,3.0) .. node[costG, yshift=4pt, pos=0.53] {2} (c);
		\draw[loose] (b) to[bend left=12] node[costG, above] {3} (a);
		\draw[loose] (a) to[bend left=14] node[costG, below] {3} (b);
		\draw[loose] (a) -- node[costG, right] {2} (d);   % 3 - 1 = 2
		\draw[loose] (c) .. controls (7.6,2.6) and (7.6,-2.6) .. node[costG, xshift=8pt, pos=0.52] {3} (e);  % 5 - 2 = 3
	\end{tikzpicture}
	\caption{Arcos justos após elevação. O ciclo \(C = (a, c, a)\) é destacado em
		\textcolor{red!70}{vermelho}.}
\end{figure}

Nesse caso, contraímos o ciclo \(C = (a, c)\) em um supervértice \(x_C\), ajustamos
os custos dos arcos incidentes, e resolvemos o problema no dígrafo contraído.
A reexpansão é feita removendo um arco do ciclo para manter a estrutura de arborescência,
de forma análoga ao algoritmo de Chu--Liu--Edmonds.

\section{Descrição do algoritmo}

O algoritmo de András Frank difere substancialmente do algoritmo de Chu--Liu--Edmonds em sua abordagem: enquanto Chu--Liu--Edmonds opera diretamente sobre os custos dos arcos, o método de Frank utiliza uma técnica gulosa--dual baseada em \emph{potenciais} associados aos vértices. A ideia central é elevar progressivamente os potenciais \(y(v)\) para cada vértice \(v\neq r\), de modo que os custos \emph{reduzidos} \(c_y(u,v) := c(u,v) - y(v)\) revelem quais arcos devem participar da solução ótima. Um arco \((u,v)\) é dito \emph{justo} quando seu custo reduzido é zero, isto é, quando \(c(u,v) = y(v)\). O algoritmo garante que toda r-arborescência ótima pode ser formada apenas por arcos justos.

A seguir apresentamos uma descrição formal do algoritmo de András Frank e detalhes de implementação serão discutidos na próxima seção.

\begin{algobox}{András Frank}{andras-frank}
	Entrada: dígrafo \(D=(V,A)\), custos \(c:A\to\mathbb{R}_{\ge 0}\), raiz \(r\).\footnote{Se algum \(v\neq r\) não possui arco de entrada, não existe r-arborescência.}

	\textbf{Fase 1: Elevação de potenciais e construção de \(A_0\)}
	\begin{enumerate}\setlength{\itemsep}{2pt}
		\item \textbf{Inicialização:} defina \(y(v) := 0\) para todo \(v \in V\).
		\item \textbf{Iteração:} enquanto existir conjunto \(X \subseteq V \setminus \{r\}\) sem arco justo entrando:
		      \begin{itemize}\setlength{\itemsep}{2pt}
			      \item Calcule \(\Delta(X) := \min\{c(u,v) - y(v) : u \notin X, v \in X\}\).
			      \item Para cada \(v \in X\), atualize \(y(v) := y(v) + \Delta(X)\).
		      \end{itemize}
		\item Defina \(A_0 := \{a \in A : c_y(a) = 0\}\), o conjunto de arcos justos.
	\end{enumerate}

	\textbf{Fase 2: Construção da arborescência}
	\begin{enumerate}\setlength{\itemsep}{2pt}\setcounter{enumi}{3}
		\item Se \((V,A_0)\) forma uma r-arborescência, devolva \(A_0\). Por otimalidade dos potenciais duais, trata-se de uma r-arborescência de custo mínimo.
		\item Caso contrário, identifique um ciclo dirigido \(C\) em \(A_0\) (com \(r\notin C\)). \textbf{Contração:} contraia \(C\) em um supervértice \(x_C\) e defina custos reduzidos \(c'\) por
		      \begin{align*}
			      c'(u,x_C) & := c_y(u,w) = c(u,w) - y(w) &  & \text{para } u\notin C,\ w\in C, \\
			      c'(x_C,v) & := c_y(w,v) = c(w,v) - y(v) &  & \text{para } w\in C,\ v\notin C,
		      \end{align*}
		      descartando laços em \(x_C\) e permitindo paralelos. Denote o dígrafo contraído por \(D'=(V',A')\).
		\item \textbf{Recursão:} compute uma r-arborescência ótima \(T'\) de \(D'\) com custos \(c'\).
		\item \textbf{Expansão:} seja \((u,x_C)\in T'\) o único arco que entra em \(x_C\). No dígrafo original, ele corresponde a \((u,w)\) com \(w\in C\). Forme
		      \[
			      T := \bigl(T'\setminus\{\text{arcos incidentes a } x_C\}\bigr)\ \cup\ \{(u,w)\}\ \cup\ \bigl((A_0\cap A(C))\setminus\{a_w\}\bigr),
		      \]
		      onde \(a_w\) é o arco de \(C\) que entra em \(w\). Então \(T\) é uma r-arborescência ótima de \(D\).
	\end{enumerate}
\end{algobox}

\subsection{Corretude}

A corretude do algoritmo de András Frank baseia-se na teoria de dualidade em programação linear aplicada ao problema de arborescência de custo mínimo:

\begin{enumerate}\setlength{\itemsep}{2pt}
	\item \emph{Potenciais duais:} os valores \(y(v)\) funcionam como variáveis duais. Para qualquer r-arborescência \(T\) e qualquer escolha de potenciais \(y\), vale
	      \[
		      c(T) = \sum_{a\in T} c(a) = \sum_{a\in T} c_y(a) + \sum_{v\neq r} y(v),
	      \]
	      pois há exatamente um arco de \(T\) entrando em cada \(v\neq r\). O termo \(\sum_{v\neq r} y(v)\) é constante após a elevação completa dos potenciais; assim, minimizar \(c(T)\) equivale a minimizar \(\sum_{a\in T} c_y(a)\).

	\item \emph{Condições de otimalidade:} uma r-arborescência \(T\) é ótima se e somente se:
	      \begin{itemize}\setlength{\itemsep}{2pt}
		      \item Todos os arcos de \(T\) são justos: \(c_y(a) = 0\) para todo \(a \in T\).
		      \item Todos os arcos do dígrafo têm custo reduzido não-negativo: \(c_y(a) \geq 0\) para todo \(a \in A\).
	      \end{itemize}
	      Sob essas condições, \(\sum_{a\in T} c_y(a) = 0 \leq \sum_{a\in T'} c_y(a)\) para qualquer outra r-arborescência \(T'\), provando otimalidade.

	\item \emph{Elevação de potenciais:} o algoritmo eleva os potenciais de forma gulosa, garantindo que ao final cada \(v\neq r\) tenha ao menos um arco justo entrando. Isso é feito identificando componentes fortemente conexas em \(A\_0\): conjuntos \(X\) que ainda não possuem arcos justos entrando têm seus potenciais elevados pelo mínimo necessário \(\Delta(X)\).

	\item \emph{Caso acíclico:} se \(A_0\) forma uma r-arborescência, todos os arcos são justos e, pelas condições de otimalidade, a solução é ótima.

	\item \emph{Caso com ciclo (contração/expansão):} se \(A\_0\) contém um ciclo dirigido \(C\), todos os seus arcos são justos (custo reduzido zero). A contração preserva a estrutura de custos reduzidos: no dígrafo contraído, os novos custos \(c'\) já incorporam os potenciais, e uma solução ótima em \(D'\) mapeia para uma solução ótima em \(D\) pela reversibilidade da operação.
\end{enumerate}

Em termos intuitivos, os potenciais \(y(v)\) funcionam como "descontos" aplicados aos arcos que entram em cada vértice, revelando quais arcos são "justos" (custo líquido zero) e, portanto, candidatos à solução ótima. Ciclos de arcos justos podem ser contraídos sem perder otimalidade, simplificando o problema recursivamente.

\subsection{Complexidade}

A implementação, baseada em componentes fortemente conexas, detecta em cada iteração, quais conjuntos \(X\) necessitam elevação de potenciais. Calcular componentes fortemente conexas custa \(O(n+m)\) usando algoritmos como Tarjan ou Kosaraju. Para cada componente (exceto a raiz), eleva-se o potencial calculando \(\Delta(X)\) em \(O(m)\), atualizando os custos reduzidos.

No pior caso, cada iteração reduz o número de componentes em pelo menos uma unidade, resultando em \(O(n)\) iterações. Cada iteração processa todos os arcos para atualizar custos reduzidos e recalcular componentes, resultando em \(O(nm)\) no total para a Fase~1. A Fase~2 constrói a arborescência percorrendo \(A_0\) uma vez, custando \(O(n)\).

O uso de memória é \(O(n+m)\), incluindo as estruturas para armazenar o dígrafo, potenciais e componentes. A implementação a seguir adota a versão \(O(nm)\) por simplicidade e está disponível no repositório do projeto (\url{https://github.com/lorenypsum/GraphVisualizer}).

\section{Implementação em Python}

Esta seção descreve a implementação do algoritmo de András Frank em Python, estruturada para refletir com precisão as duas fases formais discutidas anteriormente. A Fase~1 realiza a elevação de potenciais e identifica os arcos justos, enquanto a Fase~2 constrói a arborescência de custo mínimo a partir desses arcos. Utilizamos a biblioteca NetworkX para manipulação de digrafos, aproveitando suas funcionalidades para representar grafos, calcular componentes fortemente conexas e gerenciar atributos de arcos.

A entrada consiste em um dígrafo orientado \(D = (V, A)\), com custos dos arcos registrados no atributo \texttt{"w"}, e uma raiz \(r \in V\). As hipóteses adotadas são: (i) o dígrafo é conexo a partir de \(r\), isto é, todo vértice \(v \neq r\) é alcançável a partir da raiz; (ii) para todo subconjunto \(X \subseteq V \setminus \{r\}\), existe ao menos um arco entrando em \(X\); e (iii) todos os custos são não negativos.

A saída é um subdigrafo \(T\) de \(D\) com \(|A_T| = |V| - 1\) arcos, tal que cada vértice \(v \neq r\) possui grau de entrada igual a 1, todos os vértices são alcançáveis a partir de \(r\), e o custo total \(\sum_{a \in A_T} c(a)\) é mínimo.

A estrutura do código é modular: funções auxiliares tratam cada etapa do algoritmo — cálculo de componentes fortemente conexas, elevação de potenciais, construção do subdigrafo \(A_0\) e construção da arborescência final. Todas operam sobre objetos \texttt{nx.DiGraph} e são coordenadas por uma função principal que gerencia o fluxo das duas fases. As subseções seguintes detalham cada função auxiliar, abordando lógica, parâmetros, saídas e complexidade.

\subsection{Construção do dígrafo \texorpdfstring{\(D_0\)}{D_0} inicial}

Começamos escrevendo uma função que constrói o dígrafo inicial \(D_0\) que será utilizado na Fase~1 do algoritmo. O dígrafo \(D_0\) é inicializado como um grafo vazio contendo apenas os vértices do dígrafo original, sem arcos. Essa estrutura será gradualmente populada com arcos de custo reduzido zero à medida que os potenciais são elevados.

Recebe como entrada um dígrafo \texttt{D} (objeto \texttt{nx.DiGraph}). A implementação cria um novo dígrafo vazio \texttt{D\_zero} (linha 2) e adiciona todos os vértices de \texttt{D} a \texttt{D\_zero} (linhas 3-4), preservando a estrutura de vértices sem incluir arcos inicialmente.

A função devolve o dígrafo \texttt{D\_zero} contendo todos os vértices de \texttt{D} mas nenhum arco. O dígrafo original \texttt{D} não é modificado. A complexidade é \(O(n)\), onde \(n = |V|\), pois itera sobre todos os vértices uma única vez.

\begin{tcolorbox}[
		enhanced, breakable,
		colframe=blue!60!black, colback=blue!2,
		colbacktitle=blue!15, coltitle=black,
		title={Construção do dígrafo $D_0$ inicial},
		boxed title style={sharp corners, boxrule=0.6pt},
		sharp corners, boxrule=0.6pt
	]
	\emph{Constrói um dígrafo vazio contendo apenas os vértices de D, sem arcos. Este dígrafo será populado com arcos justos durante a elevação de potenciais.}
	\tcblower
	\begin{lstlisting}[language=Python]
def build_D_zero(D):
    D_zero = nx.DiGraph()
    for v in D.nodes():
        D_zero.add_node(v)
    return D_zero
\end{lstlisting}
\end{tcolorbox}

\subsection{Identificação de arcos entrando em conjunto \texorpdfstring{\(X\)}{X}}

Esta função auxiliar identifica todos os arcos que entram em um conjunto \(X \subseteq V\), isto é, arcos \((u,v)\) tais que \(u \notin X\) e \(v \in X\). Essa operação é fundamental para calcular o mínimo custo de entrada em \(X\) durante a elevação de potenciais.

Recebe como entrada um dígrafo \texttt{D} e um conjunto de vértices \texttt{X}. A implementação cria uma lista vazia \texttt{arcs} (linha 2) e itera sobre todos os arcos do dígrafo com seus dados (linha 3), incluindo o peso. Para cada arco \((u,v,\text{data})\), verifica se \(u \notin X\) e \(v \in X\) (linha 4), adicionando à lista apenas os arcos que cruzam a fronteira de \(X\) (linha 5).

A função devolve uma lista de tuplas \((u, v, \text{data})\) representando os arcos que entram em \(X\), onde \texttt{data} contém o atributo \texttt{"w"} com o peso do arco. A complexidade é \(O(m)\), onde \(m = |A|\), pois examina cada arco uma vez.

\begin{tcolorbox}[
		enhanced, breakable,
		colframe=blue!60!black, colback=blue!2,
		colbacktitle=blue!15, coltitle=black,
		title={Identificação de arcos entrando em conjunto X},
		boxed title style={sharp corners, boxrule=0.6pt},
		sharp corners, boxrule=0.6pt
	]
	\emph{Identifica todos os arcos $(u,v)$ do dígrafo D tais que $u \notin X$ e $v \in X$, devolvendo uma lista com as tuplas $(u, v, data)$ onde data contém o peso do arco.}
	\tcblower
	\begin{lstlisting}[language=Python]
def get_arcs_entering_X(D, X):
    arcs = []
    for u, v, data in D.edges(data=True):
        if u not in X and v in X:
            arcs.append((u, v, data))
    return arcs
\end{lstlisting}
\end{tcolorbox}

A figura a seguir ilustra o funcionamento da função \texttt{get\_arcs\_entering\_X} em um digrafo que vamos denotar por \(D_{32}\). O dígrafo possui uma raiz \(r_0\) conectada aos vértices \(u_1, u_2, u_3\). Os vértices em laranja pertencem ao conjunto \(X = \{v_1, v_2, v_3\}\), e a função identifica apenas os arcos em vermelho, que saem de vértices fora de \(X\) e entram em vértices dentro de \(X\). Arcos da raiz, arcos internos a \(X\), externos a \(X\), ou saindo de \(X\) não são retornados.

\begin{figure}[H]
	\centering
	\begin{tikzpicture}[>=Stealth]
		% estilos
		\tikzset{
			vtx/.style={circle, draw, fill=blue!8, thick, inner sep=1.2pt, minimum size=7mm},
			vtxRoot/.style={circle, draw, fill=red!15, thick, inner sep=1.2pt, minimum size=7mm},
			vtxX/.style={circle, draw=orange!80!black, fill=orange!15, thick, inner sep=1.2pt, minimum size=7mm},
			arc/.style={->, >=Stealth, thick, draw=black!40},
			arcEntering/.style={->, >=Stealth, very thick, draw=red!70!black},
			cost/.style={font=\scriptsize, fill=white, inner sep=1pt}
		}

		% Raiz
		\node[vtxRoot] (r0) at (2, 4) {$r_0$};

		% Vértices fora de X
		\node[vtx] (u1) at (0, 2) {$u_1$};
		\node[vtx] (u2) at (2, 2) {$u_2$};
		\node[vtx] (u3) at (4, 2) {$u_3$};

		% Vértices dentro de X (destacados)
		\node[vtxX] (v1) at (0.5, -0.5) {$v_1$};
		\node[vtxX] (v2) at (2, -0.5) {$v_2$};
		\node[vtxX] (v3) at (3.5, -0.5) {$v_3$};

		% Arcos da raiz (não retornados, pois r0 não está em consideração para X)
		\draw[arc] (r0) -- node[cost, left, pos=0.4] {1} (u1);
		\draw[arc] (r0) -- node[cost, right] {2} (u2);
		\draw[arc] (r0) -- node[cost, right, pos=0.4] {1} (u3);

		% Arcos externos a X (não retornados)
		\draw[arc] (u1) to[bend left=15] node[cost, above] {1} (u2);
		\draw[arc] (u2) to[bend left=15] node[cost, above] {3} (u3);

		% Arcos entrando em X (destacados em vermelho)
		\draw[arcEntering] (u1) -- node[cost, left] {3} (v1);
		\draw[arcEntering] (u2) -- node[cost, right] {2} (v2);
		\draw[arcEntering] (u3) -- node[cost, right] {4} (v3);
		\draw[arcEntering] (u1) to[bend right=20] node[cost, below, pos=0.3] {5} (v2);

		% Arcos internos a X (não retornados)
		\draw[arc] (v1) -- node[cost, above] {1} (v2);
		\draw[arc] (v2) -- node[cost, above] {2} (v3);

		% Arcos saindo de X (não retornados)
		\draw[arc] (v1) to[bend right=30] node[cost, left, pos=0.3] {2} (u1);

		% Região X destacada
		\draw[orange!60!black, thick, dashed, rounded corners=8pt]
		(-0.3, 0.3) rectangle (4.3, -1.3);
		\node[orange!80!black, font=\bfseries] at (5, 0.3) {$X$};
	\end{tikzpicture}
	\caption{Ilustração da função \texttt{get\_arcs\_entering\_X} em \(D_{32}\). A raiz \(r_0\) (em vermelho claro) conecta-se aos vértices \(u_1, u_2, u_3\). Os vértices em \textcolor{orange!80!black}{laranja} pertencem ao conjunto \(X = \{v_1, v_2, v_3\}\). A função identifica apenas os arcos \textcolor{red!70!black}{em vermelho}: aqueles que saem de vértices fora de \(X\) e entram em vértices dentro de \(X\). Arcos da raiz, arcos internos a \(X\), externos a \(X\), ou saindo de \(X\) não são retornados.}
\end{figure}

\subsection{Cálculo do peso mínimo de corte}

Esta função calcula o peso mínimo entre todos os arcos fornecidos, correspondendo ao valor \(\Delta(X)\) necessário para elevar os potenciais dos vértices em \(X\).

Recebe como entrada uma lista \texttt{arcos} de tuplas \((u, v, \text{data})\). A implementação usa a função \texttt{min} com uma compreensão de gerador (linha 2) que extrai o atributo \texttt{"w"} de cada tupla em \texttt{data}.

A função devolve o peso mínimo encontrado entre todos os arcos da lista. A complexidade é \(O(k)\), onde \(k\) é o número de arcos na lista, pois examina cada arco uma vez para encontrar o mínimo.

\begin{tcolorbox}[
		enhanced, breakable,
		colframe=blue!60!black, colback=blue!2,
		colbacktitle=blue!15, coltitle=black,
		title={Cálculo do peso mínimo de corte},
		boxed title style={sharp corners, boxrule=0.6pt},
		sharp corners, boxrule=0.6pt
	]
	\emph{Calcula o peso mínimo entre todos os arcos fornecidos, correspondendo ao valor $\Delta(X)$ usado na elevação de potenciais.}
	\tcblower
	\begin{lstlisting}[language=Python]
def get_minimum_weight_cut(arcs):
    return min(data["w"] for _, _, data in arcs)
\end{lstlisting}
\end{tcolorbox}

A seguir temos uma ilustração do funcionamento da função \texttt{get\_minimum\_weight\_cut} em \(D_{32}\). Considerando os arcos em vermelho que entram em \(X\) (identificados pela função anterior), esta função calcula o peso mínimo entre eles. O arco em verde possui o menor peso (2), correspondendo ao valor \(\Delta(X) = 2\) que será devolvido pela função.

\begin{figure}[H]
	\centering
	\begin{tikzpicture}[>=Stealth]
		% estilos
		\tikzset{
			vtx/.style={circle, draw, fill=blue!8, thick, inner sep=1.2pt, minimum size=7mm},
			vtxRoot/.style={circle, draw, fill=red!15, thick, inner sep=1.2pt, minimum size=7mm},
			vtxX/.style={circle, draw=orange!80!black, fill=orange!15, thick, inner sep=1.2pt, minimum size=7mm},
			arc/.style={->, >=Stealth, thick, draw=black!40},
			arcEntering/.style={->, >=Stealth, very thick, draw=red!70!black},
			arcMin/.style={->, >=Stealth, ultra thick, draw=green!60!black},
			cost/.style={font=\scriptsize, fill=white, inner sep=1pt},
			costMin/.style={font=\scriptsize\bfseries, fill=green!10, inner sep=2pt, draw=green!60!black, rounded corners=1pt}
		}

		% Raiz
		\node[vtxRoot] (r0) at (2, 4) {$r_0$};

		% Vértices fora de X
		\node[vtx] (u1) at (0, 2) {$u_1$};
		\node[vtx] (u2) at (2, 2) {$u_2$};
		\node[vtx] (u3) at (4, 2) {$u_3$};

		% Vértices dentro de X (destacados)
		\node[vtxX] (v1) at (0.5, -0.5) {$v_1$};
		\node[vtxX] (v2) at (2, -0.5) {$v_2$};
		\node[vtxX] (v3) at (3.5, -0.5) {$v_3$};

		% Arcos da raiz
		\draw[arc] (r0) -- node[cost, left, pos=0.4] {1} (u1);
		\draw[arc] (r0) -- node[cost, right] {2} (u2);
		\draw[arc] (r0) -- node[cost, right, pos=0.4] {1} (u3);

		% Arcos externos a X
		\draw[arc] (u1) to[bend left=15] node[cost, above] {1} (u2);
		\draw[arc] (u2) to[bend left=15] node[cost, above] {3} (u3);

		% Arcos entrando em X (em vermelho, mas o mínimo em verde)
		\draw[arcEntering] (u1) -- node[cost, left] {3} (v1);
		\draw[arcMin] (u2) -- node[costMin, right] {2} (v2);
		\draw[arcEntering] (u3) -- node[cost, right] {4} (v3);
		\draw[arcEntering] (u1) to[bend right=20] node[cost, below, pos=0.3] {5} (v2);

		% Arcos internos a X
		\draw[arc] (v1) -- node[cost, above] {1} (v2);
		\draw[arc] (v2) -- node[cost, above] {2} (v3);

		% Arcos saindo de X
		\draw[arc] (v1) to[bend right=30] node[cost, left, pos=0.3] {2} (u1);

		% Região X destacada
		\draw[orange!60!black, thick, dashed, rounded corners=8pt]
		(-0.3, 0.3) rectangle (4.3, -1.3);
		\node[orange!80!black, font=\bfseries] at (5, 0.3) {$X$};

		% Anotação do resultado
		\node[font=\bfseries, green!60!black] at (2, -2.2) {$\Delta(X) = \min\{3, 2, 4, 5\} = 2$};
	\end{tikzpicture}
	\caption{Ilustração da função \texttt{get\_minimum\_weight\_cut} em \(D_{32}\). Considerando os arcos \textcolor{red!70!black}{em vermelho} que entram em \(X\) (identificados pela função anterior), esta função calcula o peso mínimo entre eles. O arco \textcolor{green!60!black}{em verde} possui o menor peso (2), correspondendo ao valor \(\Delta(X) = 2\).}
\end{figure}

\subsection{Atualização de pesos em \texorpdfstring{\(X\)}{X}}

Esta função auxiliar atualiza os pesos dos arcos que entram em um conjunto \(X\), subtraindo o valor \(\Delta(X)\) de cada peso. Arcos que atingem peso zero são adicionados a \(A_0\) e a \(D_0\).

Recebe como entrada um dígrafo \texttt{D}, lista de \texttt{arcs} entrando em \(X\), o valor \texttt{min\_weight} a ser subtraído, uma lista \texttt{A\_zero} para armazenar arcos de peso zero, e o dígrafo \texttt{D\_zero} para adicionar os arcos justos.

A implementação itera sobre cada arco \((u,v,\_)\) da lista (linha 2), subtrai \texttt{min\_weight} do peso armazenado em \texttt{D[u][v]["w"]} (linha 3), e verifica se o peso resultante é zero (linha 4). Caso sim, adiciona-se \((u,v)\) à lista \texttt{A\_zero} (linha 5) e ao dígrafo \texttt{D\_zero} (linha 6).

A função não devolve valor, pois modifica diretamente as estruturas passadas como parâmetros: o dígrafo \texttt{D} tem seus pesos atualizados, \texttt{A\_zero} acumula arcos justos, e \texttt{D\_zero} é populado com esses arcos. A complexidade é \(O(k)\), onde \(k\) é o número de arcos em \texttt{arcs}.

\begin{tcolorbox}[
		enhanced, breakable,
		colframe=blue!60!black, colback=blue!2,
		colbacktitle=blue!15, coltitle=black,
		title={Atualização de pesos em X},
		boxed title style={sharp corners, boxrule=0.6pt},
		sharp corners, boxrule=0.6pt
	]
	\emph{Atualiza os pesos dos arcos que entram em X, subtraindo o valor mínimo. Arcos que atingem peso zero são registrados em $A_0$ e adicionados a $D_0$.}
	\tcblower
	\begin{lstlisting}[language=Python]
def update_weights_in_X(D, arcs, min_weight, A_zero, D_zero):
    for u, v, _ in arcs:
        D[u][v]["w"] -= min_weight
        if D[u][v]["w"] == 0:
            A_zero.append((u, v))
            D_zero.add_edge(u, v)
\end{lstlisting}
\end{tcolorbox}

A seguir ilustramos o funcionamento da função \texttt{update\_weights\_in\_X} em \(D_{32}\). A figura mostra o dígrafo antes e depois da atualização dos pesos. No estado inicial (esquerda), temos \(\Delta(X) = 2\). A função subtrai esse valor de todos os arcos que entram em \(X\). No estado final (direita), os arcos que atingiram peso zero (destacados em azul) são adicionados a \(A_0\) e \(D_0\).

\begin{figure}[H]
	\centering
	\begin{tikzpicture}[>=Stealth]
		% estilos
		\tikzset{
			vtx/.style={circle, draw, fill=blue!8, thick, inner sep=1.2pt, minimum size=6.5mm},
			vtxRoot/.style={circle, draw, fill=red!15, thick, inner sep=1.2pt, minimum size=6.5mm},
			vtxX/.style={circle, draw=orange!80!black, fill=orange!15, thick, inner sep=1.2pt, minimum size=6.5mm},
			arc/.style={->, >=Stealth, thick, draw=black!40},
			arcEntering/.style={->, >=Stealth, very thick, draw=red!70!black},
			arcTight/.style={->, >=Stealth, ultra thick, draw=blue!70!black},
			cost/.style={font=\scriptsize, fill=white, inner sep=1pt},
			costTight/.style={font=\scriptsize\bfseries, fill=blue!10, inner sep=2pt, draw=blue!70!black, rounded corners=1pt}
		}

		% ===== ANTES (esquerda) =====
		\begin{scope}
			\node[font=\bfseries, blue!80!black] at (2, 4.2) {Antes: pesos originais};

			% Raiz
			\node[vtxRoot] (r0) at (2, 3) {$r_0$};

			% Vértices fora de X
			\node[vtx] (u1) at (0, 1.5) {$u_1$};
			\node[vtx] (u2) at (2, 1.5) {$u_2$};
			\node[vtx] (u3) at (4, 1.5) {$u_3$};

			% Vértices dentro de X
			\node[vtxX] (v1) at (0.5, -0.5) {$v_1$};
			\node[vtxX] (v2) at (2, -0.5) {$v_2$};
			\node[vtxX] (v3) at (3.5, -0.5) {$v_3$};

			% Arcos da raiz
			\draw[arc] (r0) -- node[cost, left, pos=0.4] {1} (u1);
			\draw[arc] (r0) -- node[cost, right] {2} (u2);
			\draw[arc] (r0) -- node[cost, right, pos=0.4] {1} (u3);

			% Arcos externos a X
			\draw[arc] (u1) to[bend left=15] node[cost, above] {1} (u2);
			\draw[arc] (u2) to[bend left=15] node[cost, above] {3} (u3);

			% Arcos entrando em X
			\draw[arcEntering] (u1) -- node[cost, left] {3} (v1);
			\draw[arcEntering] (u2) -- node[cost, right] {2} (v2);
			\draw[arcEntering] (u3) -- node[cost, right] {4} (v3);
			\draw[arcEntering] (u1) to[bend right=20] node[cost, below, pos=0.3] {5} (v2);

			% Arcos internos a X
			\draw[arc] (v1) -- node[cost, above] {1} (v2);
			\draw[arc] (v2) -- node[cost, above] {2} (v3);

			% Arcos saindo de X
			\draw[arc] (v1) to[bend right=30] node[cost, left, pos=0.3] {2} (u1);

			% Região X
			\draw[orange!60!black, thick, dashed, rounded corners=8pt]
			(-0.3, 0.3) rectangle (4.3, -1.3);
			\node[orange!80!black, font=\bfseries\small] at (4.8, 0.1) {$X$};

			\node[font=\small, green!60!black] at (2, -2.2) {$\Delta(X) = 2$};
		\end{scope}

		% Seta indicando transformação
		\draw[->, ultra thick, blue!60!black] (5.3, 0.5) -- node[above, font=\small] {subtrai $\Delta(X)$} (6.7, 0.5);

		% ===== DEPOIS (direita) =====
		\begin{scope}[xshift=7.5cm]
			\node[font=\bfseries, blue!80!black] at (2, 4.2) {Depois: pesos atualizados};

			% Raiz
			\node[vtxRoot] (r0b) at (2, 3) {$r_0$};

			% Vértices fora de X
			\node[vtx] (u1b) at (0, 1.5) {$u_1$};
			\node[vtx] (u2b) at (2, 1.5) {$u_2$};
			\node[vtx] (u3b) at (4, 1.5) {$u_3$};

			% Vértices dentro de X
			\node[vtxX] (v1b) at (0.5, -0.5) {$v_1$};
			\node[vtxX] (v2b) at (2, -0.5) {$v_2$};
			\node[vtxX] (v3b) at (3.5, -0.5) {$v_3$};

			% Arcos da raiz (inalterados)
			\draw[arc] (r0b) -- node[cost, left, pos=0.4] {1} (u1b);
			\draw[arc] (r0b) -- node[cost, right] {2} (u2b);
			\draw[arc] (r0b) -- node[cost, right, pos=0.4] {1} (u3b);

			% Arcos externos a X (inalterados)
			\draw[arc] (u1b) to[bend left=15] node[cost, above] {1} (u2b);
			\draw[arc] (u2b) to[bend left=15] node[cost, above] {3} (u3b);

			% Arcos entrando em X (atualizados)
			\draw[arcEntering] (u1b) -- node[cost, left] {1} (v1b);
			\draw[arcTight] (u2b) -- node[costTight, right] {0} (v2b);
			\draw[arcEntering] (u3b) -- node[cost, right] {2} (v3b);
			\draw[arcEntering] (u1b) to[bend right=20] node[cost, below, pos=0.3] {3} (v2b);

			% Arcos internos a X (inalterados)
			\draw[arc] (v1b) -- node[cost, above] {1} (v2b);
			\draw[arc] (v2b) -- node[cost, above] {2} (v3b);

			% Arcos saindo de X (inalterados)
			\draw[arc] (v1b) to[bend right=30] node[cost, left, pos=0.3] {2} (u1b);

			% Região X
			\draw[orange!60!black, thick, dashed, rounded corners=8pt]
			(-0.3, 0.3) rectangle (4.3, -1.3);
			\node[orange!80!black, font=\bfseries\small] at (4.8, 0.1) {$X$};

			\node[font=\small, blue!70!black] at (2, -2.2) {Arco \textcolor{blue!70!black}{justo} $\to A_0$};
		\end{scope}
	\end{tikzpicture}
	\caption{Ilustração da função \texttt{update\_weights\_in\_X} em \(D_{32}\). À esquerda, o dígrafo antes da atualização, com os arcos \textcolor{red!70!black}{em vermelho} entrando em \(X\) e \(\Delta(X) = 2\). À direita, após subtrair \(\Delta(X)\) de cada arco entrando em \(X\): o peso $(u_1, v_1)$ reduz de 3 para 1, $(u_2, v_2)$ de 2 para \textcolor{blue!70!black}{\textbf{0}} (torna-se justo), $(u_3, v_3)$ de 4 para 2, e $(u_1, v_2)$ de 5 para 3. O arco justo é adicionado a \(A_0\) e \(D_0\). Note que os arcos da raiz e arcos internos/externos a \(X\) permanecem inalterados.}
\end{figure}

\subsection{Verificação de arborescência}

Esta função verifica se um dígrafo \(D\) contém uma r-arborescência com raiz \(r_0\). Utiliza busca em profundidade (DFS) a partir da raiz para verificar se todos os vértices são alcançáveis.

Recebe como entrada um dígrafo \texttt{D} e a raiz \texttt{r0}. A implementação constrói uma árvore DFS a partir de \texttt{r0} usando \texttt{nx.dfs\_tree} (linha 2), que devolve um subdigrafo contendo apenas os vértices alcançáveis a partir da raiz seguindo arcos. Em seguida, compara o número de vértices da árvore DFS com o número total de vértices de \texttt{D} (linha 3).

A função devolve \texttt{True} se todos os vértices são alcançáveis (indicando presença de r-arborescência), \texttt{False} caso contrário. A complexidade é \(O(n+m)\), onde \(n=|V|\) e \(m=|A|\), devido à busca em profundidade. Precisamos dessa verificação para garantir que a Fase~1 produza um dígrafo \(D_0\) que contenha uma r-arborescência antes de prosseguir para a Fase~2.

\begin{tcolorbox}[
		enhanced, breakable,
		colframe=blue!60!black, colback=blue!2,
		colbacktitle=blue!15, coltitle=black,
		title={Verificação de arborescência},
		boxed title style={sharp corners, boxrule=0.6pt},
		sharp corners, boxrule=0.6pt
	]
	\emph{Verifica se o dígrafo D contém uma r-arborescência com raiz r0, usando busca em profundidade para testar alcançabilidade de todos os vértices.}
	\tcblower
	\begin{lstlisting}[language=Python]
def has_arborescence(D, r0):
    tree = nx.dfs_tree(D, r0)
    return tree.number_of_nodes() == D.number_of_nodes()
\end{lstlisting}
\end{tcolorbox}

\subsection{Fase 1: Elevação de potenciais e construção de \texorpdfstring{\(A_0\)}{A_0}}

A seguir apresentamos a função principal da Fase~1, responsável por elevar os potenciais dos vértices iterativamente até que cada conjunto de vértices possua ao menos um arco justo entrando. O processo utiliza componentes fortemente conexas para identificar quais conjuntos necessitam elevação.

Recebe como entrada um dígrafo \texttt{D\_original} e a raiz \texttt{r0}. A implementação cria uma cópia do dígrafo original (linha 2) para preservar a entrada, inicializa estruturas auxiliares \texttt{A\_zero} (lista de arcos justos), \texttt{Dual\_list} (lista de pares \((X, \Delta(X))\) para fins de validação dual), e \texttt{D\_zero} (dígrafo de arcos justos) (linhas 3-5). Um contador de iterações é inicializado na linha 6.

O loop principal (linhas 7-22) itera enquanto houver conjuntos sem arcos justos entrando. Em cada iteração, incrementa-se o contador (linha 8), calcula-se as componentes fortemente conexas de \(D_0\) usando \texttt{nx.condensation} (linha 9), que devolve um grafo acíclico dirigido (DAG, do inglês \emph{directed acyclic graph}) onde cada vértice representa uma componente e contém o atributo \texttt{"members"} com os vértices originais. Utilizamos componentes fortemente conexas porque elas identificam naturalmente os conjuntos maximais de vértices que ainda não possuem arcos justos entrando, evitando a necessidade de rastrear manualmente quais conjuntos já foram processados.

Em seguida, identificam-se as fontes (componentes sem arcos entrando) no grafo de condensação (linha 10). Se há apenas uma fonte, significa que todos os vértices estão em uma única componente alcançável pela raiz através de arcos justos, garantindo que \(D_0\) contém uma r-arborescência e encerrando o loop (linhas 11-12).

Para cada fonte \(u\) no grafo de condensação (linha 13), obtém-se o conjunto \(X\) de vértices da componente (linha 14). Se \(r_0 \in X\), a fonte é ignorada (linhas 15-16), pois a componente contendo a raiz não necessita elevação. Caso contrário, identificam-se os arcos entrando em \(X\) usando \texttt{get\_arcs\_entering\_X} (linha 17), calcula-se o peso mínimo \(\Delta(X)\) usando \texttt{get\_minimum\_weight\_cut} (linha 18), e atualiza-se os pesos com \texttt{update\_weights\_in\_X}, registrando novos arcos justos (linha 19). A elevação simultânea de potenciais para todos os vértices de \(X\) mantém a propriedade de que arcos internos a \(X\) permanecem com o mesmo custo reduzido relativo, preservando a correção do algoritmo. Finalmente, adiciona-se \((X, \Delta(X))\) à lista dual se \(\Delta(X) > 0\) (linhas 20-21), permitindo verificação posterior das condições de otimalidade dual.

A função devolve \texttt{A\_zero} (lista de arcos justos) e \texttt{Dual\_list} (pares \((X, \Delta(X))\) para validação). A complexidade é \(O(nm)\) no pior caso, com \(O(n)\) iterações, cada uma custando \(O(m)\) para calcular componentes e atualizar pesos.

\begin{tcolorbox}[
		enhanced, breakable,
		colframe=blue!60!black, colback=blue!2,
		colbacktitle=blue!15, coltitle=black,
		title={Fase 1: Elevação de potenciais e construção de $A_0$},
		boxed title style={sharp corners, boxrule=0.6pt},
		sharp corners, boxrule=0.6pt
	]
	\emph{Eleva iterativamente os potenciais dos vértices até que cada conjunto possua ao menos um arco justo entrando. Devolve a lista $A_0$ de arcos justos e a lista de pares $(X, \Delta(X))$ para validação dual.}
	\tcblower
	\begin{lstlisting}[language=Python, basicstyle=\ttfamily\fontsize{8}{9}\selectfont]
def phase1(D_original, r0):
    D_copy = D_original.copy()
    A_zero = []
    Dual_list = []
    D_zero = build_D_zero(D_copy)
    iteration = 0
    while True:
        iteration += 1
        C = nx.condensation(D_zero)
        sources = [x for x in C.nodes() if C.in_degree(x) == 0]
        if len(sources) == 1:
            break
        for u in sources:
            X = C.nodes[u]["members"]
            if r0 in X:
                continue
            arcs = get_arcs_entering_X(D_copy, X)
            min_weight = get_minimum_weight_cut(arcs)
            update_weights_in_X(D_copy, arcs, min_weight, A_zero, D_zero)
            if min_weight != 0:
                Dual_list.append((X, min_weight)) 
    return A_zero, Dual_list
\end{lstlisting}
\end{tcolorbox}

A seguir ilustramos a execução da função \texttt{phase1} no dígrafo \(D_{32}\) estendido com a raiz \(r_0\). A figura mostra três estados principais: o dígrafo original \(D\) com seus pesos, o dígrafo \(D_0\) inicial (vazio, sem arcos), e o dígrafo \(D_0\) final após as iterações de elevação de potenciais, contendo apenas os arcos justos (em azul) que formam uma r-arborescência.

\begin{figure}[H]
	\centering
	\begin{tikzpicture}[>=Stealth, scale=0.85, every node/.style={scale=0.85}]
		% Estilos
		\tikzset{
			vtx/.style={circle, draw, fill=blue!8, thick, inner sep=1.2pt, minimum size=6.5mm},
			vtxRoot/.style={circle, draw, fill=red!15, thick, inner sep=1.2pt, minimum size=6.5mm},
			vtxX/.style={circle, draw, fill=orange!20, thick, inner sep=1.2pt, minimum size=6.5mm},
			arc/.style={->, thick, gray!60},
			arcTight/.style={->, ultra thick, blue!70!black},
			cost/.style={font=\scriptsize, fill=white, inner sep=1pt, rounded corners=1pt},
			costTight/.style={font=\scriptsize\bfseries, fill=blue!10, inner sep=2pt, draw=blue!70!black, rounded corners=1pt}
		}

		% ===== DÍGRAFO ORIGINAL D (esquerda) =====
		\begin{scope}
			\node[font=\bfseries, blue!80!black] at (2, 4.2) {Dígrafo original $D$};

			% Raiz
			\node[vtxRoot] (r0) at (2, 3) {$r_0$};

			% Vértices fora de X
			\node[vtx] (u1) at (0, 1.5) {$u_1$};
			\node[vtx] (u2) at (2, 1.5) {$u_2$};
			\node[vtx] (u3) at (4, 1.5) {$u_3$};

			% Vértices dentro de X
			\node[vtxX] (v1) at (0.5, -0.5) {$v_1$};
			\node[vtxX] (v2) at (2, -0.5) {$v_2$};
			\node[vtxX] (v3) at (3.5, -0.5) {$v_3$};

			% Arcos da raiz
			\draw[arc] (r0) -- node[cost, left, pos=0.4] {1} (u1);
			\draw[arc] (r0) -- node[cost, right] {2} (u2);
			\draw[arc] (r0) -- node[cost, right, pos=0.4] {1} (u3);

			% Arcos externos a X
			\draw[arc] (u1) to[bend left=15] node[cost, above] {1} (u2);
			\draw[arc] (u2) to[bend left=15] node[cost, above] {3} (u3);

			% Arcos entrando em X
			\draw[arc] (u1) -- node[cost, left] {3} (v1);
			\draw[arc] (u2) -- node[cost, right] {2} (v2);
			\draw[arc] (u3) -- node[cost, right] {4} (v3);
			\draw[arc] (u1) to[bend right=20] node[cost, below, pos=0.3] {5} (v2);

			% Arcos internos a X
			\draw[arc] (v1) -- node[cost, above] {1} (v2);
			\draw[arc] (v2) -- node[cost, above] {2} (v3);

			% Arcos saindo de X
			\draw[arc] (v1) to[bend right=30] node[cost, left, pos=0.3] {2} (u1);

			\node[font=\small, gray!70!black] at (2, -2) {Pesos originais};
		\end{scope}

		% ===== D_0 INICIAL (centro) =====
		\begin{scope}[xshift=6cm]
			\node[font=\bfseries, blue!80!black] at (2, 4.2) {$D_0$ inicial (linha 5)};

			% Vértices apenas, sem arcos
			\node[vtxRoot] (r0i) at (2, 3) {$r_0$};
			\node[vtx] (u1i) at (0, 1.5) {$u_1$};
			\node[vtx] (u2i) at (2, 1.5) {$u_2$};
			\node[vtx] (u3i) at (4, 1.5) {$u_3$};
			\node[vtxX] (v1i) at (0.5, -0.5) {$v_1$};
			\node[vtxX] (v2i) at (2, -0.5) {$v_2$};
			\node[vtxX] (v3i) at (3.5, -0.5) {$v_3$};

			\node[font=\small, gray!70!black] at (2, -2) {Sem arcos (vazio)};
		\end{scope}

		% ===== D_0 FINAL (direita) =====
		\begin{scope}[xshift=12cm]
			\node[font=\bfseries, blue!80!black] at (2, 4.2) {$D_0$ final (após loop)};

			% Vértices
			\node[vtxRoot] (r0f) at (2, 3) {$r_0$};
			\node[vtx] (u1f) at (0, 1.5) {$u_1$};
			\node[vtx] (u2f) at (2, 1.5) {$u_2$};
			\node[vtx] (u3f) at (4, 1.5) {$u_3$};
			\node[vtxX] (v1f) at (0.5, -0.5) {$v_1$};
			\node[vtxX] (v2f) at (2, -0.5) {$v_2$};
			\node[vtxX] (v3f) at (3.5, -0.5) {$v_3$};

			% Arcos justos (peso 0)
			\draw[arcTight] (r0f) -- node[costTight, left, pos=0.4] {0} (u1f);
			\draw[arcTight] (r0f) -- node[costTight, right, pos=0.4] {0} (u3f);
			\draw[arcTight] (u1f) to[bend left=15] node[costTight, above] {0} (u2f);
			\draw[arcTight] (u1f) -- node[costTight, left] {0} (v1f);
			\draw[arcTight] (u2f) -- node[costTight, right] {0} (v2f);
			\draw[arcTight] (v2f) -- node[costTight, above] {0} (v3f);

			\node[font=\small, blue!70!black] at (2, -2) {Arcos justos em $A_0$};
		\end{scope}

		% Setas indicando progressão
		\draw[->, ultra thick, gray!50] (4.5, 1) -- node[above, font=\small] {elevação} (5.5, 1);
		\draw[->, ultra thick, gray!50] (10.5, 1) -- node[above, font=\small] {iterações} (11.5, 1);
	\end{tikzpicture}
	\caption{Execução da função \texttt{phase1} em \(D_{32}\). À esquerda, o dígrafo original \(D\) com pesos dos arcos. Ao centro, \(D_0\) inicial após linha 5 (\texttt{build\_D\_zero}), contendo apenas vértices sem arcos. À direita, \(D_0\) final após o loop de elevação de potenciais, contendo apenas os arcos \textcolor{blue!70!black}{justos} (custo reduzido zero) que formam uma r-arborescência. Durante as iterações (linhas 7-22), os potenciais são elevados para cada componente sem arcos justos entrando, até que todos os vértices sejam alcançáveis a partir de \(r_0\) através de arcos em \(D_0\).}
\end{figure}

A seguir ilustramos a execução da função \texttt{phase2} em \(D_{32}\), mostrando como a r-arborescência é construída incrementalmente a partir do conjunto \(A_0\) de arcos justos obtidos da Fase~1. A figura apresenta quatro estados: o conjunto inicial \(A_0\) com 6 arcos justos, e três estados intermediários da arborescência \texttt{Arb} sendo construída, destacando em verde os arcos adicionados em cada iteração.

\begin{figure}[H]
	\centering
	\begin{tikzpicture}[>=Stealth, scale=0.72, every node/.style={scale=0.72}]
		% Estilos
		\tikzset{
			vtx/.style={circle, draw, fill=blue!8, thick, inner sep=1.2pt, minimum size=6.5mm},
			vtxRoot/.style={circle, draw, fill=red!15, thick, inner sep=1.2pt, minimum size=6.5mm},
			vtxIn/.style={circle, draw, fill=green!15, thick, inner sep=1.2pt, minimum size=6.5mm},
			vtxOut/.style={circle, draw, fill=gray!10, thick, inner sep=1.2pt, minimum size=6.5mm},
			arc/.style={->, thick, blue!70!black},
			arcNew/.style={->, ultra thick, green!70!black},
			cost/.style={font=\scriptsize, fill=white, inner sep=1pt}
		}

		% ===== A_0 (esquerda superior) =====
		\begin{scope}
			\node[font=\bfseries, blue!80!black] at (2, 4.5) {$A_0$ (arcos justos)};

			\node[vtxRoot] (r0) at (2, 3.2) {$r_0$};
			\node[vtx] (u1) at (0, 1.7) {$u_1$};
			\node[vtx] (u2) at (2, 1.7) {$u_2$};
			\node[vtx] (u3) at (4, 1.7) {$u_3$};
			\node[vtx] (v1) at (0.5, 0) {$v_1$};
			\node[vtx] (v2) at (2, 0) {$v_2$};
			\node[vtx] (v3) at (3.5, 0) {$v_3$};

			\draw[arc] (r0) -- node[cost, left, pos=0.3] {0} (u1);
			\draw[arc] (r0) -- node[cost, right, pos=0.3] {0} (u3);
			\draw[arc] (u1) to[bend left=12] node[cost, above, pos=0.4] {0} (u2);
			\draw[arc] (u1) -- node[cost, left] {0} (v1);
			\draw[arc] (u2) -- node[cost, right] {0} (v2);
			\draw[arc] (v2) -- node[cost, above] {0} (v3);

			\node[font=\small, gray!70!black] at (2, -1) {6 arcos com custo 0};
		\end{scope}

		% ===== Iteração 1 (direita superior) =====
		\begin{scope}[xshift=6.5cm]
			\node[font=\bfseries, blue!80!black] at (2, 4.5) {Iteração 1: $|$Arb$| = 1$};

			\node[vtxRoot] (r01) at (2, 3.2) {$r_0$};
			\node[vtxIn] (u11) at (0, 1.7) {$u_1$};
			\node[vtxOut] (u21) at (2, 1.7) {$u_2$};
			\node[vtxOut] (u31) at (4, 1.7) {$u_3$};
			\node[vtxOut] (v11) at (0.5, 0) {$v_1$};
			\node[vtxOut] (v21) at (2, 0) {$v_2$};
			\node[vtxOut] (v31) at (3.5, 0) {$v_3$};

			\draw[arcNew] (r01) -- node[cost, left, pos=0.3, fill=green!10] {1} (u11);

			\node[font=\small, green!70!black] at (2, -1) {Adiciona $(r_0, u_1)$};
		\end{scope}

		% ===== Iteração 2 (esquerda inferior) =====
		\begin{scope}[yshift=-6.2cm]
			\node[font=\bfseries, blue!80!black] at (2, 4.5) {Iteração 2: $|$Arb$| = 2$};

			\node[vtxRoot] (r02) at (2, 3.2) {$r_0$};
			\node[vtxIn] (u12) at (0, 1.7) {$u_1$};
			\node[vtxOut] (u22) at (2, 1.7) {$u_2$};
			\node[vtxIn] (u32) at (4, 1.7) {$u_3$};
			\node[vtxOut] (v12) at (0.5, 0) {$v_1$};
			\node[vtxOut] (v22) at (2, 0) {$v_2$};
			\node[vtxOut] (v32) at (3.5, 0) {$v_3$};

			\draw[arc] (r02) -- node[cost, left, pos=0.3] {1} (u12);
			\draw[arcNew] (r02) -- node[cost, right, pos=0.3, fill=green!10] {1} (u32);

			\node[font=\small, green!70!black] at (2, -1) {Adiciona $(r_0, u_3)$};
		\end{scope}

		% ===== Iteração 4 (direita inferior) =====
		\begin{scope}[xshift=6.5cm, yshift=-6.2cm]
			\node[font=\bfseries, blue!80!black] at (2, 4.5) {Iteração 4: $|$Arb$| = 4$};

			\node[vtxRoot] (r04) at (2, 3.2) {$r_0$};
			\node[vtxIn] (u14) at (0, 1.7) {$u_1$};
			\node[vtxIn] (u24) at (2, 1.7) {$u_2$};
			\node[vtxIn] (u34) at (4, 1.7) {$u_3$};
			\node[vtxIn] (v14) at (0.5, 0) {$v_1$};
			\node[vtxOut] (v24) at (2, 0) {$v_2$};
			\node[vtxOut] (v34) at (3.5, 0) {$v_3$};

			\draw[arc] (r04) -- node[cost, left, pos=0.3] {1} (u14);
			\draw[arc] (r04) -- node[cost, right, pos=0.3] {1} (u34);
			\draw[arc] (u14) to[bend left=12] node[cost, above, pos=0.4] {1} (u24);
			\draw[arcNew] (u14) -- node[cost, left, fill=green!10] {3} (v14);

			\node[font=\small, green!70!black] at (2, -1) {Adiciona $(u_1, v_1)$};
		\end{scope}

		% Setas de progressão
		\draw[->, ultra thick, gray!50] (4.8, 2) -- node[above, font=\small] {linha 6-10} (5.7, 2);
		\draw[->, ultra thick, gray!50] (2, -1.5) -- node[left, font=\small] {linha 6-10} (2, -2.4);
		\draw[->, ultra thick, gray!50] (4.8, -4.2) -- node[above, font=\small] {linha 6-10} (5.7, -4.2);
	\end{tikzpicture}
	\caption{Execução da função \texttt{phase2} em \(D_{32}\). Superior esquerdo: conjunto \(A_0\) com 6 arcos justos (custo reduzido zero) obtidos da Fase~1. Superior direito: Iteração 1 — adiciona arco $(r_0, u_1)$ pois $r_0 \in$ Arb e $u_1 \notin$ Arb. Inferior esquerdo: Iteração 2 — adiciona $(r_0, u_3)$. Inferior direito: Iteração 4 — após adicionar $(u_1, u_2)$ na iteração 3, adiciona $(u_1, v_1)$. Vértices \textcolor{green!70!black}{verdes} foram recém-adicionados, vértices \textcolor{gray!70!black}{cinzas} ainda não pertencem a Arb. O processo continua até que todos os 6 arcos de \(A_0\) sejam incluídos, formando uma r-arborescência com pesos originais de \(D\).}
\end{figure}

As funções auxiliares implementadas nesta seção correspondem diretamente aos passos da Fase~1 do algoritmo de András Frank da seguinte forma:

\begin{tcolorbox}[
		enhanced, breakable,
		colframe=purple!60!black, colback=purple!5,
		colbacktitle=purple!15, coltitle=black,
		title={Correspondência entre Teoria e Implementação — Fase 1},
		boxed title style={sharp corners, boxrule=0.6pt},
		sharp corners, boxrule=0.6pt
	]
	\textbf{Passo 1 — Inicialização:}
	\begin{itemize}
		\item \textbf{Descrição teórica:} Defina \(y(v) := 0\) para todo \(v \in V\).
		\item \textbf{Implementação:} A função \texttt{build\_D\_zero(D)} inicializa o dígrafo \(D_0\) vazio, que será populado apenas com arcos justos. Implicitamente, os potenciais iniciam em zero, pois os pesos no dígrafo \(D\) representam os custos reduzidos \(c_y(u,v) = c(u,v) - y(v)\). Com \(y(v) = 0\), temos \(c_y = c\) no início.
	\end{itemize}

	\vspace{2mm}
	\textbf{Passo 2 — Iteração de elevação de potenciais:}
	\begin{itemize}
		\item \textbf{Descrição teórica:} Enquanto existir conjunto \(X \subseteq V \setminus \{r\}\) sem arco justo entrando:
		      \begin{itemize}
			      \item Calcule \(\Delta(X) := \min\{c(u,v) - y(v) : u \notin X, v \in X\}\).
			      \item Para cada \(v \in X\), atualize \(y(v) := y(v) + \Delta(X)\).
		      \end{itemize}
		\item \textbf{Implementação:} Esse processo de iteração é realizado pela composição sequencial de três funções auxiliares:
		      \begin{enumerate}
			      \item \texttt{get\_arcs\_entering\_X(D, X)}: Identifica o conjunto \(\{(u,v) : u \notin X, v \in X\}\), isto é, todos os arcos que cruzam a fronteira de \(X\). Essa função corresponde diretamente à definição do conjunto sobre o qual o mínimo é calculado na fórmula \(\Delta(X)\).

			      \item \texttt{get\_minimum\_weight\_cut(arcs)}: Calcula \(\min\{\text{data}["w"] : (u,v,\text{data}) \in \text{arcs}\}\), que é exatamente \(\Delta(X) = \min\{c_y(u,v) : u \notin X, v \in X\}\). Como os pesos no dígrafo já representam custos reduzidos (são atualizados a cada iteração), essa função devolve precisamente o valor teórico de \(\Delta(X)\).

			      \item \texttt{update\_weights\_in\_X(D, arcs, min\_weight, A\_zero, D\_zero)}: Implementa a atualização dos potenciais. Para cada arco \((u,v)\) entrando em \(X\), subtrai \texttt{min\_weight} de \texttt{D[u][v]["w"]}, efetivamente calculando o novo custo reduzido:
			            \[c_y'(u,v) = c_y(u,v) - \Delta(X) = \]
			            \[c(u,v) - y(v) - \Delta(X) = \]
			            \[c(u,v) - \bigl(y(v) + \Delta(X)\bigr) = c(u,v) - y'(v), \]
			            onde \(y'(v) = y(v) + \Delta(X)\) é o novo potencial. Arcos cujo custo reduzido atinge zero (\texttt{D[u][v]["w"] == 0}) são adicionados a \(A_0\) e \(D_0\), tornando-se \textbf{justos}.
		      \end{enumerate}
	\end{itemize}

	\vspace{2mm}
	\textbf{Passo 3 — Construção de \(A_0\):}
	\begin{itemize}
		\item \textbf{Descrição teórica:} Defina \(A_0 := \{a \in A : c_y(a) = 0\}\), o conjunto de arcos justos.
		\item \textbf{Implementação:} A construção de \(A_0\) ocorre de forma incremental durante as iterações do Passo~2. Cada chamada a \texttt{update\_weights\_in\_X} verifica quais arcos atingiram custo reduzido zero e os adiciona tanto à lista \texttt{A\_zero} quanto ao dígrafo \texttt{D\_zero}. Esse processo continua até que a função principal da Fase~1 (que será apresentada adiante) determine que cada componente fortemente conexa em \(D_0\) (exceto a raiz) possui ao menos um arco justo entrando, garantindo que \(A_0\) é suficiente para formar a base de uma \(r\)-arborescência.
		\item O resultado final é o conjunto completo \(A_0 = \{a \in A : c_y(a) = 0\}\), usado na Fase~2 para construir a arborescência ótima através de uma seleção gulosa de arcos justos.
	\end{itemize}
\end{tcolorbox}

\subsection{Fase 2: Construção da arborescência}
Esta é a função principal da Fase~2, responsável por construir a r-arborescência de custo mínimo a partir do conjunto \(A_0\) de arcos justos. A construção é incremental: inicia-se com a raiz e adiciona-se iterativamente arcos de \(A_0\) que conectam vértices já incluídos a novos vértices, garantindo que cada vértice não-raiz receba exatamente um arco de entrada.

Recebe como entrada um dígrafo \texttt{D\_original}, a raiz \texttt{r0}, e a lista \texttt{A\_zero} de arcos justos. A implementação cria um novo dígrafo vazio \texttt{Arb} (linha 2) e adiciona a raiz (linha 3).

O loop principal (linhas 5-12) itera \(n-1\) vezes, onde \(n = |V|\), pois uma r-arborescência tem exatamente \(|V|-1\) arcos. Em cada iteração:
\begin{enumerate}\setlength{\itemsep}{2pt}
	\item Percorre os arcos \((u,v)\) de \texttt{A\_zero} (linha 6).
	\item Verifica se \(u\) já está em \texttt{Arb} e \(v\) ainda não (linha 7).
	\item Se sim, obtém os dados do arco do dígrafo original (linha 8) e adiciona \((u,v)\) a \texttt{Arb} (linha 9).
	\item Interrompe o loop interno para reiniciar a busca, garantindo descoberta em largura (linha 10).
\end{enumerate}

A função devolve o dígrafo \texttt{Arb} representando a r-arborescência de custo mínimo. A complexidade é \(O(nm)\) no pior caso, pois cada uma das \(O(n)\) iterações pode percorrer todos os \(O(m)\) arcos de \texttt{A\_zero}.

\begin{tcolorbox}[
		enhanced, breakable,
		colframe=blue!60!black, colback=blue!2,
		colbacktitle=blue!15, coltitle=black,
		title={Fase 2: Construção da arborescência},
		boxed title style={sharp corners, boxrule=0.6pt},
		sharp corners, boxrule=0.6pt
	]
	\emph{Constrói incrementalmente a r-arborescência a partir de $A_0$, adicionando iterativamente arcos que conectam vértices já incluídos a novos vértices.}
	\tcblower
	\begin{lstlisting}[language=Python]
def phase2(D_original, r0, A_zero):
    Arb = nx.DiGraph()
    Arb.add_node(r0)
    n = len(D_original.nodes())
    for _ in range(n - 1):
        for u, v in A_zero:
            if u in Arb.nodes() and v not in Arb.nodes():
                edge_data = D_original.get_edge_data(u, v)
                Arb.add_edge(u, v, **edge_data)
                break
    return Arb
\end{lstlisting}
\end{tcolorbox}

Apresentamos também uma versão alternativa da Fase~2 que utiliza busca em largura (BFS) para construir a r-arborescência de forma mais eficiente. Diferentemente da versão anterior que itera \(n-1\) vezes sobre todos os arcos, esta implementação usa uma fila de prioridade para explorar os arcos em ordem, evitando buscas lineares repetidas.

Recebe como entrada um dígrafo \texttt{D\_original}, a raiz \texttt{r0}, e a lista \texttt{A\_zero} de arcos justos. A implementação começa criando um dígrafo auxiliar \texttt{Arb} (linha 2) onde cada arco de \texttt{A\_zero} recebe um peso igual ao seu índice na lista (linhas 3-4), estabelecendo uma ordem de exploração. Inicializa-se o conjunto \texttt{V} de vértices visitados contendo apenas a raiz (linha 5) e uma fila de prioridade vazia \texttt{q} (linha 6).

Todos os arcos que saem da raiz em \texttt{Arb} são adicionados à fila de prioridade (linhas 7-8), usando o peso (índice) como critério de ordenação. Cria-se então o dígrafo \texttt{A} que conterá a arborescência resultante (linha 9).

O loop principal (linhas 10-17) extrai arcos da fila de prioridade em ordem crescente de índice. Para cada arco \((u,v)\) extraído (linha 11), verifica-se se o vértice destino \(v\) já foi visitado (linha 12); em caso positivo, o arco é ignorado via \texttt{continue} (linha 13). Caso contrário, adiciona-se o arco \((u,v)\) à arborescência \texttt{A} com o peso original de \texttt{D\_original} (linha 14), marca-se \(v\) como visitado (linha 15), e todos os arcos que saem de \(v\) em \texttt{Arb} são adicionados à fila de prioridade para exploração futura (linhas 16-17).

A função devolve o dígrafo \texttt{A} representando a r-arborescência de custo mínimo. A complexidade é \(O(m \log m)\), onde \(m = |A_0|\), devido às operações de inserção e remoção na fila de prioridade. Esta versão é mais eficiente que a anterior quando \(|A_0|\) é grande, pois evita percorrer todos os arcos em cada iteração.

\begin{tcolorbox}[
		enhanced, breakable,
		colframe=blue!60!black, colback=blue!2,
		colbacktitle=blue!15, coltitle=black,
		title={Fase 2 (versão BFS): Construção da arborescência com fila de prioridade},
		boxed title style={sharp corners, boxrule=0.6pt},
		sharp corners, boxrule=0.6pt
	]
	\emph{Constrói a r-arborescência usando busca em largura guiada por fila de prioridade, explorando arcos de $A_0$ em ordem e evitando buscas lineares repetidas. Complexidade $O(m \log m)$.}
	\tcblower
	\begin{lstlisting}[language=Python]
def phase2_v2(D_original, r0, A_zero):
    Arb = nx.DiGraph()
    for i, (u, v) in enumerate(A_zero):
        Arb.add_edge(u, v, w=i)
    V = {r0}
    q = []
    for u, v, data in Arb.out_edges(r0, data=True):
        heapq.heappush(q, (data["w"], u, v))
    A = nx.DiGraph()
    while q:
        _, u, v = heapq.heappop(q)
        if v in V:
            continue
        A.add_edge(u, v, w=D_original[u][v]["w"])
        V.add(v)
        for x, y, data in Arb.out_edges(v, data=True):
            heapq.heappush(q, (data["w"], x, y))
    return A
\end{lstlisting}
\end{tcolorbox}

As duas versões da Fase~2 implementadas acima correspondem diretamente ao Passo~4 da descrição teórica do algoritmo de András Frank:

\begin{tcolorbox}[
		enhanced, breakable,
		colframe=purple!60!black, colback=purple!5,
		colbacktitle=purple!15, coltitle=black,
		title={Correspondência entre Teoria e Implementação — Fase 2},
		boxed title style={sharp corners, boxrule=0.6pt},
		sharp corners, boxrule=0.6pt
	]
	\textbf{Passo 4 — Construção da arborescência (caso acíclico):}
	\begin{itemize}
		\item \textbf{Descrição teórica:} Se \((V,A_0)\) forma uma r-arborescência, devolva \(A_0\). Por otimalidade dos potenciais duais, trata-se de uma r-arborescência de custo mínimo.
		\item \textbf{Implementação:} Ambas as versões de \texttt{phase2} constroem uma r-arborescência a partir do conjunto \(A_0\) de arcos justos obtido na Fase~1. A corretude baseia-se no fato de que todos os arcos em \(A_0\) têm custo reduzido zero, e a Fase~1 garante que existe uma r-arborescência formada exclusivamente por arcos justos.
		      \begin{itemize}
			      \item \textbf{Versão 1 (\texttt{phase2}):} Construção incremental por exploração exaustiva. Em cada uma das \(n-1\) iterações, percorre todos os arcos de \(A_0\) procurando um arco \((u,v)\) tal que \(u\) já pertence à arborescência parcial e \(v\) ainda não. Essa abordagem simples corresponde diretamente à ideia teórica de construir a arborescência adicionando um vértice por vez, conectando-o à estrutura existente através de um arco justo. Complexidade: \(O(nm)\).

			      \item \textbf{Versão 2 (\texttt{phase2\_v2}):} Construção por busca em largura guiada por fila de prioridade. Cria um dígrafo auxiliar onde arcos são indexados, usa a fila de prioridade para explorar arcos sistematicamente a partir da raiz, evitando buscas lineares repetidas. Essa versão otimizada mantém a mesma correção teórica — construir uma r-arborescência usando apenas arcos de \(A_0\) — mas melhora a eficiência prática. Complexidade: \(O(m \log m)\).
		      \end{itemize}
	\end{itemize}

	\vspace{2mm}
	\textbf{Nota sobre Passos 5-7 (contração/recursão/expansão):}
	\begin{itemize}
		\item A descrição teórica do algoritmo de András Frank inclui os Passos~5-7 para tratar o caso onde \(A_0\) contém ciclos dirigidos, exigindo contração, resolução recursiva e reexpansão, de forma análoga ao algoritmo de Chu--Liu--Edmonds.
		\item Na implementação apresentada, optamos por uma abordagem não-recursiva baseada em componentes fortemente conexas. A Fase~1 já garante que \(A_0\) formará uma r-arborescência ao término das iterações de elevação de potenciais, eliminando a necessidade de tratar ciclos explicitamente na Fase~2. Essa simplificação é possível porque a elevação de potenciais progressivamente "quebra" todos os ciclos ao criar novos arcos justos que conectam diferentes componentes, até que reste apenas uma única componente fortemente conexa contendo todos os vértices.
		\item Portanto, quando a Fase~2 é executada, o conjunto \(A_0\) já está livre de ciclos e forma uma r-arborescência, correspondendo diretamente ao caso tratado pelo Passo~4 da descrição teórica. A verificação prévia \texttt{has\_arborescence(D, r0)} (realizada pela função principal) confirma essa propriedade antes de invocar a Fase~2.
	\end{itemize}
\end{tcolorbox}

\subsection{Verificação de otimalidade dual}

Esta função verifica se a condição de otimalidade dual é satisfeita para a r-arborescência construída. Segundo a teoria de programação linear dual aplicada ao problema de arborescência de custo mínimo, uma solução é ótima se e somente se cada conjunto \(X \subseteq V \setminus \{r\}\) que teve seu potencial elevado durante a Fase~1 possui exatamente um arco entrando na arborescência final.

Recebe como entrada a arborescência \texttt{Arb} e a lista \texttt{Dual\_list} contendo pares \((X, \Delta(X))\) onde \(X\) é um conjunto de vértices cujos potenciais foram elevados e \(\Delta(X) > 0\) é o valor da elevação. A implementação itera sobre cada par \((X, z)\) na lista dual (linha 2), e para cada conjunto \(X\), percorre todos os arcos \((u,v)\) da arborescência (linha 3). Inicializa um contador \texttt{count} em zero (linha 4) e verifica se o arco cruza a fronteira de \(X\), isto é, se \(u \notin X\) e \(v \in X\) (linha 5). Quando essa condição é satisfeita, incrementa o contador (linha 6) e imediatamente verifica se já há mais de um arco entrando em \(X\) (linha 7). Caso positivo, a condição de otimalidade dual é violada e a função devolve \texttt{False} (linha 8).

A função devolve \texttt{True} se todos os conjuntos em \texttt{Dual\_list} possuem exatamente um arco entrando na arborescência, confirmando que a solução satisfaz as condições de folga complementar da programação linear dual. A complexidade é \(O(km)\), onde \(k = |\texttt{Dual\_list}|\) e \(m = |A|\), pois para cada conjunto dual verifica-se todos os arcos da arborescência.

Esta verificação é fundamental para garantir a correção do algoritmo: a Fase~1 constrói uma solução dual viável (potenciais \(y(v)\)), a Fase~2 constrói uma solução primal viável (arborescência), e esta função confirma que ambas satisfazem as condições de folga complementar, implicando otimalidade pelo teorema da dualidade forte.

\begin{tcolorbox}[
		enhanced, breakable,
		colframe=blue!60!black, colback=blue!2,
		colbacktitle=blue!15, coltitle=black,
		title={Verificação de otimalidade dual},
		boxed title style={sharp corners, boxrule=0.6pt},
		sharp corners, boxrule=0.6pt
	]
	\emph{Verifica se a condição de otimalidade dual é satisfeita, confirmando que cada conjunto dual possui exatamente um arco entrando na arborescência.}
	\tcblower
	\begin{lstlisting}[language=Python]
def check_dual_optimality_condition(Arb, Dual_list):
    for X, z in Dual_list:
        count = 0
        for u, v in Arb.edges():
            if u not in X and v in X:
                count += 1
                if count > 1:
                    return False
    return True
\end{lstlisting}
\end{tcolorbox}

\subsection{O algoritmo completo de András Frank}

Finalmente, apresentamos a função principal que implementa o algoritmo de András Frank para encontrar uma r-arborescência de custo mínimo em um dígrafo com pesos. A função integra as fases de construção dos potenciais duais, obtenção dos arcos justos, construção da arborescência e verificação de otimalidade dual.

\begin{tcolorbox}[
		enhanced, breakable,
		colframe=blue!60!black, colback=blue!2,
		colbacktitle=blue!15, coltitle=black,
		title={Verificação de otimalidade dual},
		boxed title style={sharp corners, boxrule=0.6pt},
		sharp corners, boxrule=0.6pt
	]
	\emph{Implementa o algoritmo completo de András Frank, integrando as fases de construção dos potenciais duais, obtenção dos arcos justos, construção da arborescência e verificação de otimalidade dual.}
	\tcblower
	\begin{lstlisting}[language=Python]
def andras_frank_algorithm(D):
    A_zero, Dual_list = phase1(D,"r0")
    arborescence_frank = phase2(D, "r0", A_zero)
    arborescence_frank_v2 = phase2_v2(D, "r0", A_zero)
    dual_frank = check_dual_optimality_condition(
        arborescence_frank, Dual_list)
    dual_frank_v2 = check_dual_optimality_condition(
        arborescence_frank_v2, Dual_list)
    return arborescence_frank, arborescence_frank_v2, dual_frank, dual_frank_v2
\end{lstlisting}
\end{tcolorbox}

\subsection*{Exemplo de execução do algoritmo}

Aqui ilustraremos cada fase do algoritmo de András Frank: inicialização dos potenciais, iterações de elevação, construção de \(A_0\), e formação da arborescência final. Utilizaremos o dígrafo \(D_{32}\) estendido com a raiz \(r_0\), mostrado na Figura~\ref{fig:frank-exemplo-inicial}.

\begin{figure}[H]
	\centering
	\begin{tikzpicture}[>=Stealth, scale=0.9, every node/.style={scale=0.9}]
		\tikzset{
			vtx/.style={circle, draw, fill=blue!8, thick, inner sep=1.2pt, minimum size=6.5mm},
			vtxRoot/.style={circle, draw, fill=red!15, thick, inner sep=1.2pt, minimum size=6.5mm},
			arc/.style={->, thick, gray!60},
			cost/.style={font=\scriptsize, fill=white, inner sep=1pt, rounded corners=1pt}
		}

		\node[font=\bfseries, blue!80!black] at (3, 4.5) {Dígrafo inicial $D$ com raiz $r_0$};

		% Raiz
		\node[vtxRoot] (r0) at (3, 3) {$r_0$};

		% Vértices camada 1
		\node[vtx] (u1) at (0, 1.5) {$u_1$};
		\node[vtx] (u2) at (3, 1.5) {$u_2$};
		\node[vtx] (u3) at (6, 1.5) {$u_3$};

		% Vértices camada 2
		\node[vtx] (v1) at (1, 0) {$v_1$};
		\node[vtx] (v2) at (3, 0) {$v_2$};
		\node[vtx] (v3) at (5, 0) {$v_3$};

		% Arcos da raiz
		\draw[arc] (r0) -- node[cost, left, pos=0.4] {1} (u1);
		\draw[arc] (r0) -- node[cost, right] {2} (u2);
		\draw[arc] (r0) -- node[cost, right, pos=0.4] {1} (u3);

		% Arcos camada 1
		\draw[arc] (u1) to[bend left=15] node[cost, above] {1} (u2);
		\draw[arc] (u2) to[bend left=15] node[cost, above] {3} (u3);

		% Arcos entrando camada 2
		\draw[arc] (u1) -- node[cost, left] {3} (v1);
		\draw[arc] (u2) -- node[cost, right] {2} (v2);
		\draw[arc] (u3) -- node[cost, right] {4} (v3);
		\draw[arc] (u1) to[bend right=20] node[cost, below, pos=0.3] {5} (v2);

		% Arcos internos camada 2
		\draw[arc] (v1) -- node[cost, above] {1} (v2);
		\draw[arc] (v2) -- node[cost, above] {2} (v3);

		% Arco retorno
		\draw[arc] (v1) to[bend right=30] node[cost, left, pos=0.3] {2} (u1);

		\node[font=\small, gray!70!black] at (3, -1.2) {Pesos originais $c(u,v)$};
	\end{tikzpicture}
	\caption{Dígrafo inicial \(D\) com raiz \(r_0\) e 6 vértices adicionais. O dígrafo contém múltiplos arcos com pesos variados. A Fase~1 do algoritmo iniciará com potenciais \(y(v) = 0\) para todos os vértices, correspondendo aos custos reduzidos iniciais \(c_y(u,v) = c(u,v)\).}
	\label{fig:frank-exemplo-inicial}
\end{figure}

\paragraph*{Fase 1 — Iteração 1:} O algoritmo inicia com todos os potenciais \(y(v) = 0\) e \(D_0\) vazio (sem arcos justos). Calcula-se as componentes fortemente conexas de \(D_0\): cada vértice forma sua própria componente. As fontes são todos os vértices exceto \(r_0\). Para o conjunto \(X = \{u_1, u_2, u_3\}\), identifica-se os arcos entrando: \((r_0, u_1)\) com peso 1, \((r_0, u_2)\) com peso 2, \((r_0, u_3)\) com peso 1. O mínimo é \(\Delta(X) = 1\). Subtraindo esse valor, os arcos \((r_0, u_1)\) e \((r_0, u_3)\) tornam-se justos (peso 0) e são adicionados a \(A_0\).

\begin{figure}[H]
	\centering
	\begin{tikzpicture}[>=Stealth, scale=0.85, every node/.style={scale=0.85}]
		\tikzset{
			vtx/.style={circle, draw, fill=blue!8, thick, inner sep=1.2pt, minimum size=6.5mm},
			vtxRoot/.style={circle, draw, fill=red!15, thick, inner sep=1.2pt, minimum size=6.5mm},
			vtxX/.style={circle, draw, fill=orange!20, thick, inner sep=1.2pt, minimum size=6.5mm},
			arc/.style={->, thick, gray!60},
			arcTight/.style={->, ultra thick, blue!70!black},
			cost/.style={font=\scriptsize, fill=white, inner sep=1pt},
			costTight/.style={font=\scriptsize\bfseries, fill=blue!10, inner sep=2pt, draw=blue!70!black, rounded corners=1pt}
		}

		% Antes
		\begin{scope}
			\node[font=\bfseries, blue!80!black] at (2, 4.5) {Antes: $\Delta(X_1) = 1$};

			\node[vtxRoot] (r0) at (2, 3) {$r_0$};
			\node[vtxX] (u1) at (0, 1.5) {$u_1$};
			\node[vtxX] (u2) at (2, 1.5) {$u_2$};
			\node[vtxX] (u3) at (4, 1.5) {$u_3$};

			\draw[arc] (r0) -- node[cost, left, pos=0.4] {1} (u1);
			\draw[arc] (r0) -- node[cost, right] {2} (u2);
			\draw[arc] (r0) -- node[cost, right, pos=0.4] {1} (u3);

			\node[font=\small, orange!80!black] at (2, 0.5) {$X_1 = \{u_1, u_2, u_3\}$};
		\end{scope}

		% Depois
		\begin{scope}[xshift=7cm]
			\node[font=\bfseries, blue!80!black] at (2, 4.5) {Depois: arcos justos};

			\node[vtxRoot] (r0b) at (2, 3) {$r_0$};
			\node[vtx] (u1b) at (0, 1.5) {$u_1$};
			\node[vtx] (u2b) at (2, 1.5) {$u_2$};
			\node[vtx] (u3b) at (4, 1.5) {$u_3$};

			\draw[arcTight] (r0b) -- node[costTight, left, pos=0.4] {0} (u1b);
			\draw[arc] (r0b) -- node[cost, right] {1} (u2b);
			\draw[arcTight] (r0b) -- node[costTight, right, pos=0.4] {0} (u3b);

			\node[font=\small, blue!70!black] at (2, 0.5) {2 arcos adicionados a $A_0$};
		\end{scope}
	\end{tikzpicture}
	\caption{Fase~1, Iteração 1: Elevação de potenciais para \(X_1 = \{u_1, u_2, u_3\}\). À esquerda, o conjunto \(X_1\) (em laranja) sem arcos justos entrando. Calcula-se \(\Delta(X_1) = \min\{1, 2, 1\} = 1\). À direita, após subtrair \(\Delta(X_1)\), os arcos \((r_0, u_1)\) e \((r_0, u_3)\) tornam-se \textcolor{blue!70!black}{justos} (custo reduzido 0) e são adicionados a \(A_0\).}
	\label{fig:frank-exemplo-iter1}
\end{figure}

\paragraph*{Fase 1 — Iteração 2:} Com \(A_0 = \{(r_0, u_1), (r_0, u_3)\}\), as componentes fortemente conexas são: \(\{r_0, u_1, u_3\}\) e \(\{u_2\}\), \(\{v_1\}\), \(\{v_2\}\), \(\{v_3\}\). A fonte (excluindo a raiz) é \(\{u_2\}\). Para \(X = \{u_2\}\), o único arco entrando é \((r_0, u_2)\) com peso atual 1 (após iteração anterior). Portanto \(\Delta(X) = 1\), e após subtração, \((r_0, u_2)\) torna-se justo. Também na mesma iteração, para \(X = \{v_1, v_2, v_3\}\), calcula-se \(\Delta(X) = \min\{3, 2, 4, 5\} = 2\), e os arcos são atualizados conforme ilustrado.

\begin{figure}[H]
	\centering
	\begin{tikzpicture}[>=Stealth, scale=0.75, every node/.style={scale=0.75}]
		\tikzset{
			vtx/.style={circle, draw, fill=blue!8, thick, inner sep=1.2pt, minimum size=6mm},
			vtxRoot/.style={circle, draw, fill=red!15, thick, inner sep=1.2pt, minimum size=6mm},
			vtxX/.style={circle, draw, fill=orange!20, thick, inner sep=1.2pt, minimum size=6mm},
			arc/.style={->, thick, gray!60},
			arcTight/.style={->, ultra thick, blue!70!black},
			cost/.style={font=\scriptsize, fill=white, inner sep=1pt},
			costTight/.style={font=\scriptsize\bfseries, fill=blue!10, inner sep=2pt, draw=blue!70!black, rounded corners=1pt}
		}

		\node[font=\bfseries, blue!80!black] at (3, 5.5) {Iteração 2: $\Delta(\{v_1, v_2, v_3\}) = 2$};

		\node[vtxRoot] (r0) at (3, 4) {$r_0$};
		\node[vtx] (u1) at (0.5, 2.5) {$u_1$};
		\node[vtx] (u2) at (3, 2.5) {$u_2$};
		\node[vtx] (u3) at (5.5, 2.5) {$u_3$};
		\node[vtxX] (v1) at (1.5, 0.5) {$v_1$};
		\node[vtxX] (v2) at (3, 0.5) {$v_2$};
		\node[vtxX] (v3) at (4.5, 0.5) {$v_3$};

		% Arcos já justos
		\draw[arcTight] (r0) -- node[costTight, left, pos=0.3] {0} (u1);
		\draw[arcTight] (r0) -- node[costTight, right, pos=0.3] {0} (u3);
		\draw[arcTight] (u1) to[bend left=12] node[costTight, above] {0} (u2);

		% Arcos entrando em X (antes de ficarem justos)
		\draw[arc] (u1) -- node[cost, left, fill=orange!10] {3→1} (v1);
		\draw[arc] (u2) -- node[cost, right, fill=orange!10] {2→0} (v2);
		\draw[arc] (u3) -- node[cost, right] {4→2} (v3);
		\draw[arc] (u1) to[bend right=15] node[cost, below, pos=0.35] {5→3} (v2);

		\node[font=\small, orange!80!black] at (3, -0.5) {Após subtração: $(u_2, v_2)$ torna-se justo};
	\end{tikzpicture}
	\caption{Fase~1, Iteração 2: Para o conjunto \(X = \{v_1, v_2, v_3\}\) (em laranja), calcula-se \(\Delta(X) = \min\{3, 2, 4, 5\} = 2\). Após subtrair esse valor dos arcos entrando em \(X\), o arco \((u_2, v_2)\) atinge custo reduzido 0 e é adicionado a \(A_0\). Note que \((u_1, u_2)\) também se tornou justo na mesma iteração ao processar \(X = \{u_2\}\).}
	\label{fig:frank-exemplo-iter2}
\end{figure}

\paragraph*{Fase 1 — Estado final:} Após mais iterações elevando potenciais para os conjuntos restantes, o dígrafo \(D_0\) de arcos justos fica completo, contendo uma r-arborescência. A Figura~\ref{fig:frank-exemplo-A0-final} mostra o conjunto final \(A_0\) com 6 arcos de custo reduzido zero.

\begin{figure}[H]
	\centering
	\begin{tikzpicture}[>=Stealth, scale=0.85, every node/.style={scale=0.85}]
		\tikzset{
			vtx/.style={circle, draw, fill=blue!8, thick, inner sep=1.2pt, minimum size=6.5mm},
			vtxRoot/.style={circle, draw, fill=red!15, thick, inner sep=1.2pt, minimum size=6.5mm},
			arcTight/.style={->, ultra thick, blue!70!black},
			costTight/.style={font=\scriptsize\bfseries, fill=blue!10, inner sep=2pt, draw=blue!70!black, rounded corners=1pt}
		}

		\node[font=\bfseries, blue!80!black] at (3, 4.5) {$A_0$ final: arcos justos após Fase~1};

		\node[vtxRoot] (r0) at (3, 3) {$r_0$};
		\node[vtx] (u1) at (0.5, 1.5) {$u_1$};
		\node[vtx] (u2) at (3, 1.5) {$u_2$};
		\node[vtx] (u3) at (5.5, 1.5) {$u_3$};
		\node[vtx] (v1) at (1.5, 0) {$v_1$};
		\node[vtx] (v2) at (3, 0) {$v_2$};
		\node[vtx] (v3) at (4.5, 0) {$v_3$};

		\draw[arcTight] (r0) -- node[costTight, left, pos=0.3] {0} (u1);
		\draw[arcTight] (r0) -- node[costTight, right, pos=0.3] {0} (u3);
		\draw[arcTight] (u1) to[bend left=12] node[costTight, above] {0} (u2);
		\draw[arcTight] (u1) -- node[costTight, left] {0} (v1);
		\draw[arcTight] (u2) -- node[costTight, right] {0} (v2);
		\draw[arcTight] (v2) -- node[costTight, above] {0} (v3);

		\node[font=\small, blue!70!black] at (3, -1.2) {$|A_0| = 6$ arcos com custo reduzido $c_y = 0$};
	\end{tikzpicture}
	\caption{Conjunto final \(A_0\) de arcos justos após conclusão da Fase~1. Todos os 6 arcos \textcolor{blue!70!black}{destacados} possuem custo reduzido zero: \((r_0, u_1)\), \((r_0, u_3)\), \((u_1, u_2)\), \((u_1, v_1)\), \((u_2, v_2)\), \((v_2, v_3)\). O dígrafo formado por esses arcos é acíclico e alcança todos os vértices a partir de \(r_0\), satisfazendo as condições para prosseguir à Fase~2.}
	\label{fig:frank-exemplo-A0-final}
\end{figure}

\paragraph*{Fase 2 — Construção da arborescência:} Com \(A_0\) completo e acíclico, a Fase~2 constrói incrementalmente a arborescência final. Inicia-se com \(\texttt{Arb} = \{r_0\}\) e em cada iteração adiciona-se um arco \((u,v) \in A_0\) tal que \(u \in \texttt{Arb}\) e \(v \notin \texttt{Arb}\). A Figura~\ref{fig:frank-exemplo-arborescencia-final} mostra a arborescência resultante.

\begin{figure}[H]
	\centering
	\begin{tikzpicture}[>=Stealth, scale=0.85, every node/.style={scale=0.85}]
		\tikzset{
			vtx/.style={circle, draw, fill=green!15, thick, inner sep=1.2pt, minimum size=6.5mm},
			vtxRoot/.style={circle, draw, fill=red!15, thick, inner sep=1.2pt, minimum size=6.5mm},
			arcArb/.style={->, ultra thick, green!70!black},
			costArb/.style={font=\scriptsize\bfseries, fill=green!10, inner sep=2pt, draw=green!70!black, rounded corners=1pt}
		}

		\node[font=\bfseries, blue!80!black] at (3, 4.5) {Arborescência final (Fase~2)};

		\node[vtxRoot] (r0) at (3, 3) {$r_0$};
		\node[vtx] (u1) at (0.5, 1.5) {$u_1$};
		\node[vtx] (u2) at (3, 1.5) {$u_2$};
		\node[vtx] (u3) at (5.5, 1.5) {$u_3$};
		\node[vtx] (v1) at (1.5, 0) {$v_1$};
		\node[vtx] (v2) at (3, 0) {$v_2$};
		\node[vtx] (v3) at (4.5, 0) {$v_3$};

		\draw[arcArb] (r0) -- node[costArb, left, pos=0.3] {1} (u1);
		\draw[arcArb] (r0) -- node[costArb, right, pos=0.3] {1} (u3);
		\draw[arcArb] (u1) to[bend left=12] node[costArb, above] {1} (u2);
		\draw[arcArb] (u1) -- node[costArb, left] {3} (v1);
		\draw[arcArb] (u2) -- node[costArb, right] {2} (v2);
		\draw[arcArb] (v2) -- node[costArb, above] {2} (v3);

		\node[font=\small, green!70!black] at (3, -1.2) {Custo total: $1+1+1+3+2+2 = 10$};
	\end{tikzpicture}
	\caption{Arborescência de custo mínimo final obtida pela Fase~2. Os pesos mostrados são os \emph{originais} de \(D\), restaurados após a construção. Cada vértice não-raiz possui exatamente um arco de entrada, não há ciclos, e todos os vértices são alcançáveis a partir de \(r_0\). A verificação de otimalidade dual confirma que cada conjunto \(X\) que teve potenciais elevados possui exatamente um arco da arborescência cruzando sua fronteira, garantindo que a solução é ótima.}
	\label{fig:frank-exemplo-arborescencia-final}
\end{figure}

\paragraph*{Verificação de otimalidade dual:} A função \texttt{check\_dual\_optimality\_condition} confirma que para cada par \((X, \Delta(X))\) em \texttt{Dual\_list} (conjuntos cujos potenciais foram elevados com \(\Delta(X) > 0\)), existe exatamente um arco da arborescência final cruzando a fronteira de \(X\). Essa condição, juntamente com os arcos justos, garante que as condições de folga complementar da programação linear dual são satisfeitas, implicando que a arborescência encontrada é de custo mínimo global.

\subsection{Correspondência entre teoria e implementação}

A implementação em Python do algoritmo de András Frank segue fielmente a descrição teórica primal-dual apresentada anteriormente. A tabela abaixo estabelece o paralelo direto entre os passos teóricos e sua realização no código:

\begin{table}[H]
	\centering
	\scriptsize
	\renewcommand{\arraystretch}{1.3}
	\begin{tabular}{>{\raggedright\arraybackslash}p{0.44\textwidth}|>{\raggedright\arraybackslash}p{0.48\textwidth}}
		\toprule
		\textbf{\normalsize Descrição Teórica} & \textbf{\normalsize Implementação Python} \\
		\midrule
		\rowcolor{blue!5}
		\textbf{Passo 1:} Inicialização

		Defina \(y(v) := 0\) para todo \(v \in V\).

		Inicialize \(A_0 := \emptyset\).

		Construa dígrafo vazio \(D_0\) (arcos justos).
		                                       &
		\textbf{Função \texttt{phase1} — Linhas 2--5:}

		\texttt{D\_copy = D\_original.copy()}

		\texttt{A\_zero = []}

		\texttt{D\_zero = build\_D\_zero(D\_copy)}

		\vspace{1mm}
		Potenciais \(y(v) = 0\) implícitos, custos \(c_y = c\).
		\\
		\midrule
		\rowcolor{green!5}
		\textbf{Passo 2:} Elevação de potenciais

		Enquanto \(\exists X \subseteq V \setminus \{r\}\) sem arco justo:

		\quad Calcule \(\Delta(X) := \min\{c_y(u,v) : u \notin X, v \in X\}\)

		\quad Atualize \(y(v) := y(v) + \Delta(X)\), \(\forall v \in X\)

		\quad Adicione arcos com \(c_y = 0\) a \(A_0\)
		                                       &
		\textbf{Loop principal — Linhas 7--22:}

		\texttt{C = nx.condensation(D\_zero)}

		\texttt{sources = [x for x in C.nodes()]}

		\texttt{\quad if C.in\_degree(x) == 0]}

		Para fonte \texttt{u} (exceto raiz):

		\texttt{\quad X = C.nodes[u]["members"]}

		\texttt{\quad arcs = get\_arcs\_entering\_X(D, X)}

		\texttt{\quad min\_w = get\_minimum\_weight\_cut(arcs)}

		\texttt{\quad update\_weights\_in\_X(D, arcs, min\_w,}

		\texttt{\quad\quad A\_zero, D\_zero)}
		\\
		\midrule
		\rowcolor{orange!5}
		\textbf{Passo 2(a):} Identificar arcos entrando

		Determine \(\{(u,v) \in A : u \notin X, v \in X\}\)
		                                       &
		\textbf{Função \texttt{get\_arcs\_entering\_X}:}

		\texttt{return [(u, v, data)}

		\texttt{\quad for u, v, data in D.edges(data=True)}

		\texttt{\quad if u not in X and v in X]}
		\\
		\midrule
		\rowcolor{purple!5}
		\textbf{Passo 2(b):} Calcular \(\Delta(X)\)

		\(\Delta(X) := \min\{c_y(u,v) : u \notin X, v \in X\}\)
		                                       &
		\textbf{Função \texttt{get\_minimum\_weight\_cut}:}

		\texttt{return min(data["w"]}

		\texttt{\quad for \_, \_, data in arcs)}
		\\
		\midrule
		\rowcolor{yellow!5}
		\textbf{Passo 2(c):} Atualizar pesos

		Para \((u,v)\) entrando em \(X\):

		\quad \(c_y(u,v) := c_y(u,v) - \Delta(X)\)

		\quad Se \(c_y(u,v) = 0\), adicione a \(A_0\)
		                                       &
		\textbf{Função \texttt{update\_weights\_in\_X}:}

		\texttt{for u, v, \_ in arcs:}

		\texttt{\quad D[u][v]["w"] -= min\_weight}

		\texttt{\quad if D[u][v]["w"] == 0:}

		\texttt{\quad\quad A\_zero.append((u, v))}

		\texttt{\quad\quad D\_zero.add\_edge(u, v)}
		\\
		\midrule
		\rowcolor{cyan!5}
		\textbf{Passo 3:} Verificar término

		Se \(D_0\) contém r-arborescência, encerre.
		                                       &
		\textbf{Condição — Linhas 11--12:}

		\texttt{if len(sources) == 1: break}

		\vspace{1mm}
		Uma fonte \(\Rightarrow\) r-arborescência acíclica.
		\\
		\midrule
		\rowcolor{red!5}
		\textbf{Passo 4:} Construir arborescência

		Construa \(F\) a partir de \(A_0\), conectando vértices incrementalmente.
		                                       &
		\textbf{Função \texttt{phase2} (incremental):}

		\texttt{Arb = nx.DiGraph(); Arb.add\_node(r0)}

		\texttt{for \_ in range(n - 1):}

		\texttt{\quad for u, v in A\_zero:}

		\texttt{\quad\quad if u in Arb and v not in Arb:}

		\texttt{\quad\quad\quad Arb.add\_edge(u, v, **data)}

		\texttt{\quad\quad\quad break}

		\vspace{1mm}
		Complexidade: \(O(nm)\).
		\\
		\midrule
		\rowcolor{red!5}
		\textbf{Passo 4:} Versão otimizada

		Mesma ideia, com fila de prioridade.
		                                       &
		\textbf{Função \texttt{phase2\_v2} (BFS):}

		\texttt{Arb = nx.DiGraph()}

		\texttt{for i, (u, v) in enumerate(A\_zero):}

		\texttt{\quad Arb.add\_edge(u, v, w=i)}

		\texttt{q = [] \# fila de prioridade}

		\texttt{while q:}

		\texttt{\quad \_, u, v = heapq.heappop(q)}

		\texttt{\quad if v in V: continue}

		\texttt{\quad A.add\_edge(u, v, w=D[u][v]["w"])}

		\vspace{1mm}
		Complexidade: \(O(m \log m)\).
		\\
		\midrule
		\rowcolor{pink!5}
		\textbf{Otimalidade dual}

		Para cada \(X\) elevado (\(\Delta(X) > 0\)), exatamente um arco de \(F\) cruza \(\delta^-(X)\).
		                                       &
		\textbf{Função \texttt{check\_dual\_optimality}:}

		\texttt{for X, z in Dual\_list:}

		\texttt{\quad count = 0}

		\texttt{\quad for u, v in Arb.edges():}

		\texttt{\quad\quad if u not in X and v in X:}

		\texttt{\quad\quad\quad count += 1}

		\texttt{\quad\quad\quad if count > 1: return False}

		\texttt{return True}
		\\
		\bottomrule
	\end{tabular}
	\caption{Correspondência entre a descrição teórica do algoritmo de András Frank e sua implementação em Python. Cores: inicialização (azul), elevação de potenciais (verde/laranja/roxo/amarelo), verificação (ciano), construção (vermelho) e validação dual (rosa).}
	\label{tab:frank-teoria-implementacao}
\end{table}

Esta correspondência demonstra que a implementação traduz fielmente a abordagem primal-dual em código executável. As funções auxiliares (\texttt{get\_arcs\_entering\_X}, \texttt{get\_minimum\_weight\_cut}, \texttt{update\_weights\_in\_X}, \texttt{phase1}, \texttt{phase2}, \texttt{phase2\_v2},
\texttt{check\_dual\_}-
\texttt{optimality\_condition} encapsulam exatamente as operações descritas na teoria, preservando as propriedades de correção e as garantias de otimalidade do algoritmo original. A utilização de componentes fortemente conexas (\texttt{nx.condensation}) para identificar conjuntos sem arcos justos entrando é uma implementação eficiente da verificação teórica, evitando enumeração explícita de todos os subconjuntos de vértices.

\chapter{Algoritmo de András Frank}

Neste capítulo, apresentaremos o algoritmo de András Frank, que também determina
uma arborescência de custo mínimo em um digrafo ponderado.
O algoritmo baseia-se em uma abordagem primal--dual:
a elevação gulosa de potenciais duais\footnote{\textbf{Potencial dual} é uma variável associada a cada vértice, que representa um "desconto" aplicado ao custo dos arcos que entram nesse vértice e servem para ajustar os custos de modo que certas restrições fiquem "justas", - satisfeitas com igualdade, ou seja, o valor da variável atinge exatamente o limite imposto pela restrição.} e (ii) a construção de uma solução gulosa
a partir dos arcos de custo reduzido zero.
Diferentemente do algoritmo de Chu--Liu--Edmonds, que opera diretamente sobre os custos,
o método de Frank manipula potenciais duais associados aos vértices,
gerando arcos \emph{justos} (de custo reduzido zero) que formam a arborescência ótima.
O propósito deste capítulo é fornecer uma descrição precisa tanto do algoritmo
quanto da implementação desenvolvida neste trabalho.

\section{O algoritmo}
O algoritmo de András Frank também recebe uma tripla \((D,c,r)\), em que \(D=(V,A)\) é um digrafo,
\(c\colon A\to\mathbb{R}\) é uma função custo e \(r\in V\) é a raiz,
sob a hipótese de que \(D\) admite ao menos uma \(r\)-arborescência e devolve uma \(r\)-arborescência \(c\)-mínima de \(D\).

Assim como no capítulo anterior, adotamos a terminologia de \textbf{\(r\)-digrafo ponderado}
para uma tripla \((D,c,r)\) em que $(D, c)$ é um digrafo ponderado,
$r$ é um vértice de $D$, $\delta^-(r) = \varnothing$
e $D$ possui uma $r$-arborescência.

Vamos desenvolver as ideias do algoritmo utilizando o mesmo dígrafo que apresentamos
no capítulo anterior.

\subsection*{Abordagem Gulosa--Dual}

Considere novamente o dígrafo \(D\) figura 26 a seguir, com custos nos arcos note que é o mesmo digrafo da figura 1 que utilizamos para ilustrar o algoritmo de Chu--Liu--Edmonds. E portanto, suponha que T é uma arborescência de custo mínimo, com seus arcos destacados em azul.

\begin{figure}[h]
	\centering
	\begin{tikzpicture}[>=Stealth]
		% estilos
		\tikzset{
			vtx/.style={circle, draw,fill=blue!8, thick, inner sep=1.2pt, minimum size=6mm},
			chosen/.style={->, very thick, draw=blue},
			other/.style={->, semithick, draw=black!35},
			costB/.style={font=\scriptsize, fill=white, inner sep=1pt, text=blue},
			costG/.style={font=\scriptsize, fill=white, inner sep=1pt, text=black!55}
		}

		% layout (mais espaçado)
		\node[vtx] (r) at (0,0) {$r$};
		\node[vtx] (a) at (3,1.6) {$a$};
		\node[vtx] (b) at (3,-1.6) {$b$};
		\node[vtx] (c) at (6,2.2) {$c$};
		\node[vtx] (d) at (6,0.0) {$d$};
		\node[vtx] (e) at (6,-2.2) {$e$};

		% --- arcos NÃO escolhidos (cinza), agora com curvaturas mais limpas ---
		\draw[other] (b) to[bend left=12] node[costG, above] {4} (a);   % b->a
		\draw[other] (a) to[bend left=14] node[costG, below] {5} (b);   % a->b

		% r->c: externo por cima, suave (pontos de controle explícitos)
		\draw[other]
		(r) .. controls (0.8,2.4) and (5.2,3.0) ..
		node[costG, yshift=4pt, pos=0.53] {3} (c);

		% a->d (referência)
		\draw[other] (a) -- node[costG, right] {3} (d);

		% c->e: leve curvatura pela direita (sem "barriga")
		\draw[other]
		(c) .. controls (7.6,2.6) and (7.6,-2.6) ..
		node[costG, xshift=8pt, pos=0.52] {5} (e);

		% --- arcos ESCOLHIDOS (AZUL) — formam a r-arborescência T ---
		\draw[chosen] (r) -- node[costB, above] {1} (a);   % r->a
		\draw[chosen] (r) -- node[costB, below] {2} (b);   % r->b
		\draw[chosen] (a) -- node[costB, above] {1} (c);   % a->c
		\draw[chosen] (b) -- node[costB, above] {1} (d);   % b->d
		\draw[chosen] (b) -- node[costB, below] {2} (e);   % b->e
	\end{tikzpicture}

	\caption{O dígrafo \(D\) com custos nos arcos. Os arcos em \textcolor{blue}{azul}
		formam a \(r\)-arborescência de custo mínimo \(T\) (custo total = 7).}
\end{figure}

O algoritmo de Frank utiliza \textbf{potenciais} \(y: V \to \mathbb{R}\) associados aos vértices
para definir custos reduzidos nos arcos. Para cada arco \(a = (u,v) \in A\), o
\textbf{custo \(y\)-reduzido} é dado por
\[
	c_y(u,v) := c(u,v) - y(v),
\]
onde \(v\) é o vértice de destino do arco. Note que, diferentemente do procedimento de redução do capítulo anterior (onde subtraíamos o mínimo custo de entrada em cada vértice), aqui o potencial \(y(v)\) é uma variável que será ajustada pelo algoritmo através da seguinte noção.

Um arco \(a = (u,v)\) é dito \textbf{justo} (ou \textbf{tight}) se \(c_y(u,v) = 0\), isto é, se o custo original coincide com o potencial do vértice de destino: \(c(u,v) = y(v)\). A ideia central do algoritmo de Frank consiste em duas etapas: (i)~elevar os potenciais \(y(v)\) para cada \(v \neq r\) até que cada vértice possua ao menos um arco justo entrando nele; (ii)~construir uma arborescência ótima utilizando exclusivamente arcos justos.

\begin{tcolorbox}[
		enhanced, breakable,
		colback=blue!5!white, colframe=blue!70!black,
		colbacktitle=blue!85!black, coltitle=white,
		fonttitle=\bfseries,
		title={Arcos justos: interpretação e cálculo},
		sharp corners, boxrule=0.8pt
	]
	Um arco \((u,v)\) é \textbf{justo} quando seu custo reduzido é zero, ou seja, quando o custo original \(c(u,v)\) coincide exatamente com o potencial \(y(v)\) do vértice de destino. Formalmente:
	\[
		c_y(u,v) = c(u,v) - y(v) = 0 \quad\Longleftrightarrow\quad c(u,v) = y(v).
	\]

	Em outras palavras, o potencial \(y(v)\) funciona como um ``desconto'' aplicado aos arcos que chegam em \(v\). Quando esse desconto é suficiente para anular o custo reduzido, o arco torna-se justo e candidato a participar da solução ótima.

	\medskip
	\textbf{Exemplos numéricos (referência: figura~26):}
	\begin{itemize}\setlength{\itemsep}{3pt}
		\item Arco \((r,a)\): \(c(r,a) = 1\), \(y(a) = 1 \implies c_y(r,a) = 1-1 = 0\). \quad\textcolor{blue!70!black}{\textbf{Justo.}}
		\item Arco \((b,a)\): \(c(b,a) = 4\), \(y(a) = 1 \implies c_y(b,a) = 4-1 = 3\). \quad\textcolor{red!70!black}{\textbf{Não justo.}}
		\item Arco \((a,c)\): \(c(a,c) = 1\), \(y(c) = 1 \implies c_y(a,c) = 1-1 = 0\). \quad\textcolor{blue!70!black}{\textbf{Justo.}}
	\end{itemize}

	\textbf{Propriedade fundamental:} o potencial \(y(v)\) afeta \emph{exclusivamente} os arcos que entram em \(v\). Por exemplo, \(y(c)\) influencia apenas arcos da forma \((\cdot, c)\), como \((r,c)\) e \((a,c)\), não afetando arcos que saem de \(c\).

	\medskip
	O algoritmo eleva progressivamente os potenciais até que cada vértice \(v \neq r\) possua ao menos um arco justo de entrada. No exemplo da figura~26, os arcos justos após a elevação correspondem precisamente aos arcos da arborescência ótima.
\end{tcolorbox}

% Figura 27: Exemplo visual com múltiplos arcos e vértice raiz
\begin{figure}[H]
	\centering
	\begin{tikzpicture}[>=Stealth]
		% estilos
		\tikzset{
			vtx/.style={circle, draw, fill=blue!8, thick, inner sep=1.2pt, minimum size=8mm},
			arc/.style={->, very thick, draw=blue},
			arcTight/.style={->, very thick, draw=red!70!black},
			cost/.style={font=\scriptsize, fill=white, inner sep=1pt, text=blue},
			pot/.style={font=\small, text=red!70!black},
			label/.style={font=\small, text=black!80}
		}

		% vértices
		\node[vtx] (r) at (0,0) {$r$};
		\node[vtx] (u) at (3,1.5) {$u$};
		\node[vtx] (v) at (3,-1.5) {$v$};

		% arcos
		\draw[arcTight] (r) -- node[cost, above left] {$c(r,u)=2$} (u);
		\draw[arc] (r) -- node[cost, below left] {$c(r,v)=4$} (v);
		\draw[arcTight] (u) -- node[cost, right] {$c(u,v)=3$} (v);

		% potenciais
		\node[pot] at (3,2.5) {$y(u)=2$};
		\node[pot] at (3,-2.5) {$y(v)=3$};

		% destaques
		\draw[red!70, thick, dashed] (u) circle [radius=0.7];
		\draw[red!70, thick, dashed] (v) circle [radius=0.7];

		% custos reduzidos
		\node[label] at (-0.0,1.5) {$c_y(r,u)=2-2=0$};
		\node[label] at (-0.0,-1.5) {$c_y(r,v)=4-3=1$};
		\node[label] at (5.0,-0.5) {$c_y(u,v)=3-3=0$};
	\end{tikzpicture}
	\caption{O vértice $r$ é a raiz. Os potenciais são $y(u)=2$ e $y(v)=3$. O arco $(r,u)$ tem custo reduzido $0$ (justo), $(r,v)$ tem custo reduzido $1$ (não justo), $(u,v)$ tem custo reduzido $0$ (justo). Os potenciais só afetam os arcos que chegam em cada vértice.}
	\label{fig:potencial-destino}
\end{figure}

\subsection*{Elevação de Potenciais}

Inicialmente, definimos \(y(v) = 0\) para todo \(v \in V\).
Para cada vértice \(v \neq r\), calculamos
\[
	\Delta(v) := \min\{\, c(a) : a \in \delta^{-}(v) \,\} - y(v).
\]
O valor \(\Delta(v)\) representa quanto precisamos elevar \(y(v)\) para que exista
ao menos um arco de custo reduzido zero entrando em \(v\).

Vamos ilustrar esse processo no dígrafo \(D\).

\begin{figure}[H]
	\centering
	\begin{tikzpicture}[>=Stealth]
		% estilos
		\tikzset{
			vtx/.style={circle, draw,fill=blue!8, thick, inner sep=1.2pt, minimum size=6mm},
			arc/.style={->, semithick, draw=black!35},
			costG/.style={font=\scriptsize, fill=white, inner sep=1pt, text=black!55}
		}

		% layout
		\node[vtx] (r) at (0,0) {$r$};
		\node[vtx] (a) at (3,1.6) {$a$};
		\node[vtx] (b) at (3,-1.6) {$b$};
		\node[vtx] (c) at (6,2.2) {$c$};
		\node[vtx] (d) at (6,0.0) {$d$};
		\node[vtx] (e) at (6,-2.2) {$e$};

		% arcos com custos originais
		\draw[arc] (r) -- node[costG, above] {1} (a);
		\draw[arc] (r) -- node[costG, below] {2} (b);
		\draw[arc] (r) .. controls (0.8,2.4) and (5.2,3.0) .. node[costG, yshift=4pt, pos=0.53] {3} (c);
		\draw[arc] (b) to[bend left=12] node[costG, above] {4} (a);
		\draw[arc] (a) to[bend left=14] node[costG, below] {5} (b);
		\draw[arc] (a) -- node[costG, above] {1} (c);
		\draw[arc] (a) -- node[costG, right] {3} (d);
		\draw[arc] (b) -- node[costG, above] {1} (d);
		\draw[arc] (b) -- node[costG, below] {2} (e);
		\draw[arc] (c) .. controls (7.6,2.6) and (7.6,-2.6) .. node[costG, xshift=8pt, pos=0.52] {5} (e);

		% potenciais iniciais
		\node[above=6mm of a, font=\small] {$y(a) = 0$};
		\node[below=2mm of b, font=\small] {$y(b) = 0$};
		\node[above=2mm of c, font=\small] {$y(c) = 0$};
		\node[below=2mm of d, font=\small] {$y(d) = 0$};
		\node[below=2mm of e, font=\small] {$y(e) = 0$};
	\end{tikzpicture}
	\caption{Dígrafo \(D\) com potenciais iniciais \(y(v) = 0\) para todo \(v \in V\).}
\end{figure}

Calculamos \(\Delta(v)\) para cada vértice \(v \neq r\):
\begin{align*}
	\Delta(a) & = \min\{1, 4\} - 0 = 1, \\
	\Delta(b) & = \min\{2, 5\} - 0 = 2, \\
	\Delta(c) & = \min\{3, 1\} - 0 = 1, \\
	\Delta(d) & = \min\{3, 1\} - 0 = 1, \\
	\Delta(e) & = \min\{2, 5\} - 0 = 2.
\end{align*}

Elevamos cada potencial:
\begin{align*}
	y(a) & \leftarrow y(a) + \Delta(a) = 0 + 1 = 1, \\
	y(b) & \leftarrow y(b) + \Delta(b) = 0 + 2 = 2, \\
	y(c) & \leftarrow y(c) + \Delta(c) = 0 + 1 = 1, \\
	y(d) & \leftarrow y(d) + \Delta(d) = 0 + 1 = 1, \\
	y(e) & \leftarrow y(e) + \Delta(e) = 0 + 2 = 2.
\end{align*}

Após a elevação, os custos reduzidos são dados por \(c_y(u,v) = c(u,v) - y(v)\) para cada arco \((u,v)\):

\begin{figure}[H]
	\centering
	\begin{tikzpicture}[>=Stealth]
		% estilos
		\tikzset{
			vtx/.style={circle, draw,fill=blue!8, thick, inner sep=1.2pt, minimum size=6mm},
			tight/.style={->, very thick, draw=blue},
			loose/.style={->, semithick, draw=black!35},
			costB/.style={font=\scriptsize, fill=white, inner sep=1pt, text=blue},
			costG/.style={font=\scriptsize, fill=white, inner sep=1pt, text=black!55}
		}

		% layout
		\node[vtx] (r) at (0,0) {$r$};
		\node[vtx] (a) at (3,1.6) {$a$};
		\node[vtx] (b) at (3,-1.6) {$b$};
		\node[vtx] (c) at (6,2.2) {$c$};
		\node[vtx] (d) at (6,0.0) {$d$};
		\node[vtx] (e) at (6,-2.2) {$e$};

		% arcos justos (custo reduzido = 0)
		\draw[tight] (r) -- node[costB, above] {0} (a);   % 1 - 1 = 0
		\draw[tight] (r) -- node[costB, below] {0} (b);   % 2 - 2 = 0
		\draw[tight] (a) -- node[costB, above] {0} (c);   % 1 - 1 = 0
		\draw[tight] (b) -- node[costB, above] {0} (d);   % 1 - 1 = 0
		\draw[tight] (b) -- node[costB, below] {0} (e);   % 2 - 2 = 0

		% arcos não justos
		\draw[loose] (r) .. controls (0.8,2.4) and (5.2,3.0) .. node[costG, yshift=4pt, pos=0.53] {2} (c);  % 3 - 1 = 2
		\draw[loose] (b) to[bend left=12] node[costG, above] {3} (a);   % 4 - 1 = 3
		\draw[loose] (a) to[bend left=14] node[costG, below] {3} (b);   % 5 - 2 = 3
		\draw[loose] (a) -- node[costG, right] {2} (d);   % 3 - 1 = 2
		\draw[loose] (c) .. controls (7.6,2.6) and (7.6,-2.6) .. node[costG, xshift=8pt, pos=0.52] {3} (e);  % 5 - 2 = 3

		% potenciais finais
		\node[above=6mm of a, font=\small] {$y(a) = 1$};
		\node[below=2mm of b, font=\small] {$y(b) = 2$};
		\node[above=2mm of c, font=\small] {$y(c) = 1$};
		\node[below=2mm of d, font=\small] {$y(d) = 1$};
		\node[below=2mm of e, font=\small] {$y(e) = 2$};
	\end{tikzpicture}
	\caption{Dígrafo após elevação de potenciais. Arcos em \textcolor{blue}{azul} são
		justos (\(c_y = 0\)); arcos em cinza têm custo reduzido positivo.}
\end{figure}

Observe que cada vértice \(v \neq r\) possui agora ao menos um arco justo entrando nele.
Mais ainda: neste exemplo, os arcos justos formam uma \(r\)-arborescência! Precisamos justificar por que a arborescência formada pelos arcos justos é ótima.

\subsection*{Construção da Arborescência}

Se os arcos justos formam uma \(r\)-arborescência, então encontramos a solução ótima.
Caso contrário, os arcos justos contêm ciclos, e precisamos tratá-los adequadamente.

No nosso exemplo, os arcos justos são:
\[
	H = \{\, (r,a), (r,b), (a,c), (b,d), (b,e) \,\}.
\]
Esses arcos formam a \(r\)-arborescência \(T\) de custo mínimo.

Para verificar a otimalidade, note que uma propriedade fundamental dos potenciais duais é:

\begin{proposicao}
	Para toda função \(y: V \to \mathbb{R}\) com \(y(r) = 0\),
	uma \(r\)-arborescência \(T\) é \(c\)-mínima em \(D\) se, e somente se,
	todos os arcos de \(T\) são justos (i.e., \(c_y(a) = 0\) para todo \(a \in T\))
	e \(c_y(a) \geq 0\) para todo \(a \in A\).
\end{proposicao}

\begin{proof}[Prova.]
	Seja \(F\) uma \(r\)-arborescência qualquer. Para cada \(v \in V \setminus \{r\}\),
	seja \(a_v\) o único arco de \(F\) que entra em \(v\). Então
	\begin{align*}
		c(F)
		 & = \sum_{v \in V \setminus \{r\}} c(a_v)                                          \\
		 & = \sum_{v \in V \setminus \{r\}} \bigl(c_y(a_v) + y(v)\bigr)                     \\
		 & = \sum_{v \in V \setminus \{r\}} c_y(a_v) + \sum_{v \in V \setminus \{r\}} y(v).
	\end{align*}
	Como a soma \(\sum_{v \in V \setminus \{r\}} y(v)\) é constante para todas as
	\(r\)-arborescências, minimizar \(c(F)\) equivale a minimizar \(\sum_{v} c_y(a_v)\).
	Se todos os arcos de \(T\) são justos e \(c_y(a) \geq 0\) para todo \(a\),
	então \(\sum_{v} c_y(a_v^T) = 0 \leq \sum_{v} c_y(a_v^F)\) para qualquer \(F\),
	provando que \(T\) é ótima.
\end{proof}

No nosso exemplo, todos os arcos de \(T\) são justos e \(c_y(a) \geq 0\) para todo
arco \(a \in A\), portanto \(T\) é a \(r\)-arborescência de custo mínimo. Evidentemente, nem sempre será o caso que os arcos justos formam uma arborescência e precisaremos tratar ciclos.

\subsection*{Tratamento de Ciclos}

Em geral, após a elevação de potenciais, os arcos justos podem conter ciclos.
Nesse caso, o algoritmo de Frank procede de forma similar ao de Chu--Liu--Edmonds:
contrai cada ciclo justo em um supervértice, resolve o problema recursivamente
no dígrafo contraído, e depois reexpande a solução.

Vamos ilustrar esse processo com um exemplo modificado. Suponha que adicionamos o arco
\((c, a)\) com custo 1 ao dígrafo original:

\begin{figure}[H]
	\centering
	\begin{tikzpicture}[>=Stealth]
		% estilos
		\tikzset{
			vtx/.style={circle, draw,fill=blue!8, thick, inner sep=1.2pt, minimum size=6mm},
			arc/.style={->, semithick, draw=black!35},
			costG/.style={font=\scriptsize, fill=white, inner sep=1pt, text=black!55}
		}

		% layout
		\node[vtx] (r) at (0,0) {$r$};
		\node[vtx] (a) at (3,1.6) {$a$};
		\node[vtx] (b) at (3,-1.6) {$b$};
		\node[vtx] (c) at (6,2.2) {$c$};
		\node[vtx] (d) at (6,0.0) {$d$};
		\node[vtx] (e) at (6,-2.2) {$e$};

		% arcos originais
		\draw[arc] (r) -- node[costG, above] {1} (a);
		\draw[arc] (r) -- node[costG, below] {2} (b);
		\draw[arc] (r) .. controls (0.8,2.4) and (5.2,3.0) .. node[costG, yshift=4pt, pos=0.53] {3} (c);
		\draw[arc] (b) to[bend left=12] node[costG, above] {4} (a);
		\draw[arc] (a) to[bend left=14] node[costG, below] {5} (b);
		\draw[arc] (a) -- node[costG, above] {1} (c);
		\draw[arc] (a) -- node[costG, right] {3} (d);
		\draw[arc] (b) -- node[costG, above] {1} (d);
		\draw[arc] (b) -- node[costG, below] {2} (e);
		\draw[arc] (c) .. controls (7.6,2.6) and (7.6,-2.6) .. node[costG, xshift=8pt, pos=0.52] {5} (e);

		% NOVO arco (c, a)
		\draw[arc, draw=red!60] (c) to[bend left=20] node[costG, left, text=red!60] {1} (a);
	\end{tikzpicture}
	\caption{Dígrafo modificado com o arco \((c,a)\) de custo 1 (em vermelho).}
\end{figure}

Após a elevação de potenciais com \(y(a) = 1, y(c) = 1\), o arco \((c,a)\) torna-se
justo: \(c_y(c,a) = 1 - 1 = 0\).

\begin{figure}[H]
	\centering
	\begin{tikzpicture}[>=Stealth]
		% estilos
		\tikzset{
			vtx/.style={circle, draw,fill=blue!8, thick, inner sep=1.2pt, minimum size=6mm},
			tight/.style={->, very thick, draw=blue},
			cycle/.style={->, ultra thick, draw=red!70},
			loose/.style={->, semithick, draw=black!35},
			costB/.style={font=\scriptsize, fill=white, inner sep=1pt, text=blue},
			costR/.style={font=\scriptsize, fill=white, inner sep=1pt, text=red!70},
			costG/.style={font=\scriptsize, fill=white, inner sep=1pt, text=black!55}
		}

		% layout
		\node[vtx] (r) at (0,0) {$r$};
		\node[vtx] (a) at (3,1.6) {$a$};
		\node[vtx] (b) at (3,-1.6) {$b$};
		\node[vtx] (c) at (6,2.2) {$c$};
		\node[vtx] (d) at (6,0.0) {$d$};
		\node[vtx] (e) at (6,-2.2) {$e$};

		% arcos justos que formam ciclo
		\draw[cycle] (a) -- node[costR, above] {0} (c);
		\draw[cycle] (c) to[bend left=20] node[costR, left] {0} (a);

		% outros arcos justos
		\draw[tight] (r) -- node[costB, above] {0} (a);
		\draw[tight] (r) -- node[costB, below] {0} (b);
		\draw[tight] (b) -- node[costB, above] {0} (d);
		\draw[tight] (b) -- node[costB, below] {0} (e);

		% arcos não justos
		\draw[loose] (r) .. controls (0.8,2.4) and (5.2,3.0) .. node[costG, yshift=4pt, pos=0.53] {2} (c);
		\draw[loose] (b) to[bend left=12] node[costG, above] {3} (a);
		\draw[loose] (a) to[bend left=14] node[costG, below] {3} (b);
		\draw[loose] (a) -- node[costG, right] {2} (d);   % 3 - 1 = 2
		\draw[loose] (c) .. controls (7.6,2.6) and (7.6,-2.6) .. node[costG, xshift=8pt, pos=0.52] {3} (e);  % 5 - 2 = 3
	\end{tikzpicture}
	\caption{Arcos justos após elevação. O ciclo \(C = (a, c, a)\) é destacado em
		\textcolor{red!70}{vermelho}.}
\end{figure}

Nesse caso, contraímos o ciclo \(C = (a, c)\) em um supervértice \(x_C\), ajustamos
os custos dos arcos incidentes, e resolvemos o problema no dígrafo contraído.
A reexpansão é feita removendo um arco do ciclo para manter a estrutura de arborescência,
de forma análoga ao algoritmo de Chu--Liu--Edmonds.

\section{Descrição do algoritmo}

O algoritmo de András Frank difere substancialmente do algoritmo de Chu--Liu--Edmonds em sua abordagem: enquanto Chu--Liu--Edmonds opera diretamente sobre os custos dos arcos, o método de Frank utiliza uma técnica gulosa--dual baseada em \emph{potenciais} associados aos vértices. A ideia central é elevar progressivamente os potenciais \(y(v)\) para cada vértice \(v\neq r\), de modo que os custos \emph{reduzidos} \(c_y(u,v) := c(u,v) - y(v)\) revelem quais arcos devem participar da solução ótima. Um arco \((u,v)\) é dito \emph{justo} quando seu custo reduzido é zero, isto é, quando \(c(u,v) = y(v)\). O algoritmo garante que toda r-arborescência ótima pode ser formada apenas por arcos justos.

A seguir apresentamos uma descrição formal do algoritmo de András Frank e detalhes de implementação serão discutidos na próxima seção.

\begin{algobox}{András Frank}{andras-frank}
	Entrada: dígrafo \(D=(V,A)\), custos \(c:A\to\mathbb{R}_{\ge 0}\), raiz \(r\).\footnote{Se algum \(v\neq r\) não possui arco de entrada, não existe r-arborescência.}

	\textbf{Fase 1: Elevação de potenciais e construção de \(A_0\)}
	\begin{enumerate}\setlength{\itemsep}{2pt}
		\item \textbf{Inicialização:} defina \(y(v) := 0\) para todo \(v \in V\).
		\item \textbf{Iteração:} enquanto existir conjunto \(X \subseteq V \setminus \{r\}\) sem arco justo entrando:
		      \begin{itemize}\setlength{\itemsep}{2pt}
			      \item Calcule \(\Delta(X) := \min\{c(u,v) - y(v) : u \notin X, v \in X\}\).
			      \item Para cada \(v \in X\), atualize \(y(v) := y(v) + \Delta(X)\).
		      \end{itemize}
		\item Defina \(A_0 := \{a \in A : c_y(a) = 0\}\), o conjunto de arcos justos.
	\end{enumerate}

	\textbf{Fase 2: Construção da arborescência}
	\begin{enumerate}\setlength{\itemsep}{2pt}\setcounter{enumi}{3}
		\item Se \((V,A_0)\) forma uma r-arborescência, devolva \(A_0\). Por otimalidade dos potenciais duais, trata-se de uma r-arborescência de custo mínimo.
		\item Caso contrário, identifique um ciclo dirigido \(C\) em \(A_0\) (com \(r\notin C\)). \textbf{Contração:} contraia \(C\) em um supervértice \(x_C\) e defina custos reduzidos \(c'\) por
		      \begin{align*}
			      c'(u,x_C) & := c_y(u,w) = c(u,w) - y(w) &  & \text{para } u\notin C,\ w\in C, \\
			      c'(x_C,v) & := c_y(w,v) = c(w,v) - y(v) &  & \text{para } w\in C,\ v\notin C,
		      \end{align*}
		      descartando laços em \(x_C\) e permitindo paralelos. Denote o dígrafo contraído por \(D'=(V',A')\).
		\item \textbf{Recursão:} compute uma r-arborescência ótima \(T'\) de \(D'\) com custos \(c'\).
		\item \textbf{Expansão:} seja \((u,x_C)\in T'\) o único arco que entra em \(x_C\). No dígrafo original, ele corresponde a \((u,w)\) com \(w\in C\). Forme
		      \[
			      T := \bigl(T'\setminus\{\text{arcos incidentes a } x_C\}\bigr)\ \cup\ \{(u,w)\}\ \cup\ \bigl((A_0\cap A(C))\setminus\{a_w\}\bigr),
		      \]
		      onde \(a_w\) é o arco de \(C\) que entra em \(w\). Então \(T\) é uma r-arborescência ótima de \(D\).
	\end{enumerate}
\end{algobox}

\subsection{Corretude}

A corretude do algoritmo de András Frank baseia-se na teoria de dualidade em programação linear aplicada ao problema de arborescência de custo mínimo:

\begin{enumerate}\setlength{\itemsep}{2pt}
	\item \emph{Potenciais duais:} os valores \(y(v)\) funcionam como variáveis duais. Para qualquer r-arborescência \(T\) e qualquer escolha de potenciais \(y\), vale
	      \[
		      c(T) = \sum_{a\in T} c(a) = \sum_{a\in T} c_y(a) + \sum_{v\neq r} y(v),
	      \]
	      pois há exatamente um arco de \(T\) entrando em cada \(v\neq r\). O termo \(\sum_{v\neq r} y(v)\) é constante após a elevação completa dos potenciais; assim, minimizar \(c(T)\) equivale a minimizar \(\sum_{a\in T} c_y(a)\).

	\item \emph{Condições de otimalidade:} uma r-arborescência \(T\) é ótima se e somente se:
	      \begin{itemize}\setlength{\itemsep}{2pt}
		      \item Todos os arcos de \(T\) são justos: \(c_y(a) = 0\) para todo \(a \in T\).
		      \item Todos os arcos do dígrafo têm custo reduzido não-negativo: \(c_y(a) \geq 0\) para todo \(a \in A\).
	      \end{itemize}
	      Sob essas condições, \(\sum_{a\in T} c_y(a) = 0 \leq \sum_{a\in T'} c_y(a)\) para qualquer outra r-arborescência \(T'\), provando otimalidade.

	\item \emph{Elevação de potenciais:} o algoritmo eleva os potenciais de forma gulosa, garantindo que ao final cada \(v\neq r\) tenha ao menos um arco justo entrando. Isso é feito identificando componentes fortemente conexas em \(A\_0\): conjuntos \(X\) que ainda não possuem arcos justos entrando têm seus potenciais elevados pelo mínimo necessário \(\Delta(X)\).

	\item \emph{Caso acíclico:} se \(A_0\) forma uma r-arborescência, todos os arcos são justos e, pelas condições de otimalidade, a solução é ótima.

	\item \emph{Caso com ciclo (contração/expansão):} se \(A\_0\) contém um ciclo dirigido \(C\), todos os seus arcos são justos (custo reduzido zero). A contração preserva a estrutura de custos reduzidos: no dígrafo contraído, os novos custos \(c'\) já incorporam os potenciais, e uma solução ótima em \(D'\) mapeia para uma solução ótima em \(D\) pela reversibilidade da operação.
\end{enumerate}

Em termos intuitivos, os potenciais \(y(v)\) funcionam como "descontos" aplicados aos arcos que entram em cada vértice, revelando quais arcos são "justos" (custo líquido zero) e, portanto, candidatos à solução ótima. Ciclos de arcos justos podem ser contraídos sem perder otimalidade, simplificando o problema recursivamente.

\subsection{Complexidade}

A implementação, baseada em componentes fortemente conexas, detecta em cada iteração, quais conjuntos \(X\) necessitam elevação de potenciais. Calcular componentes fortemente conexas custa \(O(n+m)\) usando algoritmos como Tarjan ou Kosaraju. Para cada componente (exceto a raiz), eleva-se o potencial calculando \(\Delta(X)\) em \(O(m)\), atualizando os custos reduzidos.

No pior caso, cada iteração reduz o número de componentes em pelo menos uma unidade, resultando em \(O(n)\) iterações. Cada iteração processa todos os arcos para atualizar custos reduzidos e recalcular componentes, resultando em \(O(nm)\) no total para a Fase~1. A Fase~2 constrói a arborescência percorrendo \(A_0\) uma vez, custando \(O(n)\).

O uso de memória é \(O(n+m)\), incluindo as estruturas para armazenar o dígrafo, potenciais e componentes. A implementação a seguir adota a versão \(O(nm)\) por simplicidade e está disponível no repositório do projeto (\url{https://github.com/lorenypsum/GraphVisualizer}).

\section{Implementação em Python}

Esta seção descreve a implementação do algoritmo de András Frank em Python, estruturada para refletir com precisão as duas fases formais discutidas anteriormente. A Fase~1 realiza a elevação de potenciais e identifica os arcos justos, enquanto a Fase~2 constrói a arborescência de custo mínimo a partir desses arcos. Utilizamos a biblioteca NetworkX para manipulação de digrafos, aproveitando suas funcionalidades para representar grafos, calcular componentes fortemente conexas e gerenciar atributos de arcos.

A entrada consiste em um dígrafo orientado \(D = (V, A)\), com custos dos arcos registrados no atributo \texttt{"w"}, e uma raiz \(r \in V\). As hipóteses adotadas são: (i) o dígrafo é conexo a partir de \(r\), isto é, todo vértice \(v \neq r\) é alcançável a partir da raiz; (ii) para todo subconjunto \(X \subseteq V \setminus \{r\}\), existe ao menos um arco entrando em \(X\); e (iii) todos os custos são não negativos.

A saída é um subdigrafo \(T\) de \(D\) com \(|A_T| = |V| - 1\) arcos, tal que cada vértice \(v \neq r\) possui grau de entrada igual a 1, todos os vértices são alcançáveis a partir de \(r\), e o custo total \(\sum_{a \in A_T} c(a)\) é mínimo.

A estrutura do código é modular: funções auxiliares tratam cada etapa do algoritmo — cálculo de componentes fortemente conexas, elevação de potenciais, construção do subdigrafo \(A_0\) e construção da arborescência final. Todas operam sobre objetos \texttt{nx.DiGraph} e são coordenadas por uma função principal que gerencia o fluxo das duas fases. As subseções seguintes detalham cada função auxiliar, abordando lógica, parâmetros, saídas e complexidade.

\subsection{Construção do dígrafo \texorpdfstring{\(D_0\)}{D_0} inicial}

Começamos escrevendo uma função que constrói o dígrafo inicial \(D_0\) que será utilizado na Fase~1 do algoritmo. O dígrafo \(D_0\) é inicializado como um grafo vazio contendo apenas os vértices do dígrafo original, sem arcos. Essa estrutura será gradualmente populada com arcos de custo reduzido zero à medida que os potenciais são elevados.

Recebe como entrada um dígrafo \texttt{D} (objeto \texttt{nx.DiGraph}). A implementação cria um novo dígrafo vazio \texttt{D\_zero} (linha 2) e adiciona todos os vértices de \texttt{D} a \texttt{D\_zero} (linhas 3-4), preservando a estrutura de vértices sem incluir arcos inicialmente.

A função devolve o dígrafo \texttt{D\_zero} contendo todos os vértices de \texttt{D} mas nenhum arco. O dígrafo original \texttt{D} não é modificado. A complexidade é \(O(n)\), onde \(n = |V|\), pois itera sobre todos os vértices uma única vez.

\begin{tcolorbox}[
		enhanced, breakable,
		colframe=blue!60!black, colback=blue!2,
		colbacktitle=blue!15, coltitle=black,
		title={Construção do dígrafo $D_0$ inicial},
		boxed title style={sharp corners, boxrule=0.6pt},
		sharp corners, boxrule=0.6pt
	]
	\emph{Constrói um dígrafo vazio contendo apenas os vértices de D, sem arcos. Este dígrafo será populado com arcos justos durante a elevação de potenciais.}
	\tcblower
	\begin{lstlisting}[language=Python]
def build_D_zero(D):
    D_zero = nx.DiGraph()
    for v in D.nodes():
        D_zero.add_node(v)
    return D_zero
\end{lstlisting}
\end{tcolorbox}

\subsection{Identificação de arcos entrando em conjunto \texorpdfstring{\(X\)}{X}}

Esta função auxiliar identifica todos os arcos que entram em um conjunto \(X \subseteq V\), isto é, arcos \((u,v)\) tais que \(u \notin X\) e \(v \in X\). Essa operação é fundamental para calcular o mínimo custo de entrada em \(X\) durante a elevação de potenciais.

Recebe como entrada um dígrafo \texttt{D} e um conjunto de vértices \texttt{X}. A implementação cria uma lista vazia \texttt{arcs} (linha 2) e itera sobre todos os arcos do dígrafo com seus dados (linha 3), incluindo o peso. Para cada arco \((u,v,\text{data})\), verifica se \(u \notin X\) e \(v \in X\) (linha 4), adicionando à lista apenas os arcos que cruzam a fronteira de \(X\) (linha 5).

A função devolve uma lista de tuplas \((u, v, \text{data})\) representando os arcos que entram em \(X\), onde \texttt{data} contém o atributo \texttt{"w"} com o peso do arco. A complexidade é \(O(m)\), onde \(m = |A|\), pois examina cada arco uma vez.

\begin{tcolorbox}[
		enhanced, breakable,
		colframe=blue!60!black, colback=blue!2,
		colbacktitle=blue!15, coltitle=black,
		title={Identificação de arcos entrando em conjunto X},
		boxed title style={sharp corners, boxrule=0.6pt},
		sharp corners, boxrule=0.6pt
	]
	\emph{Identifica todos os arcos $(u,v)$ do dígrafo D tais que $u \notin X$ e $v \in X$, devolvendo uma lista com as tuplas $(u, v, data)$ onde data contém o peso do arco.}
	\tcblower
	\begin{lstlisting}[language=Python]
def get_arcs_entering_X(D, X):
    arcs = []
    for u, v, data in D.edges(data=True):
        if u not in X and v in X:
            arcs.append((u, v, data))
    return arcs
\end{lstlisting}
\end{tcolorbox}

A figura a seguir ilustra o funcionamento da função \texttt{get\_arcs\_entering\_X} em um digrafo que vamos denotar por \(D_{32}\). O dígrafo possui uma raiz \(r_0\) conectada aos vértices \(u_1, u_2, u_3\). Os vértices em laranja pertencem ao conjunto \(X = \{v_1, v_2, v_3\}\), e a função identifica apenas os arcos em vermelho, que saem de vértices fora de \(X\) e entram em vértices dentro de \(X\). Arcos da raiz, arcos internos a \(X\), externos a \(X\), ou saindo de \(X\) não são retornados.

\begin{figure}[H]
	\centering
	\begin{tikzpicture}[>=Stealth]
		% estilos
		\tikzset{
			vtx/.style={circle, draw, fill=blue!8, thick, inner sep=1.2pt, minimum size=7mm},
			vtxRoot/.style={circle, draw, fill=red!15, thick, inner sep=1.2pt, minimum size=7mm},
			vtxX/.style={circle, draw=orange!80!black, fill=orange!15, thick, inner sep=1.2pt, minimum size=7mm},
			arc/.style={->, >=Stealth, thick, draw=black!40},
			arcEntering/.style={->, >=Stealth, very thick, draw=red!70!black},
			cost/.style={font=\scriptsize, fill=white, inner sep=1pt}
		}

		% Raiz
		\node[vtxRoot] (r0) at (2, 4) {$r_0$};

		% Vértices fora de X
		\node[vtx] (u1) at (0, 2) {$u_1$};
		\node[vtx] (u2) at (2, 2) {$u_2$};
		\node[vtx] (u3) at (4, 2) {$u_3$};

		% Vértices dentro de X (destacados)
		\node[vtxX] (v1) at (0.5, -0.5) {$v_1$};
		\node[vtxX] (v2) at (2, -0.5) {$v_2$};
		\node[vtxX] (v3) at (3.5, -0.5) {$v_3$};

		% Arcos da raiz (não retornados, pois r0 não está em consideração para X)
		\draw[arc] (r0) -- node[cost, left, pos=0.4] {1} (u1);
		\draw[arc] (r0) -- node[cost, right] {2} (u2);
		\draw[arc] (r0) -- node[cost, right, pos=0.4] {1} (u3);

		% Arcos externos a X (não retornados)
		\draw[arc] (u1) to[bend left=15] node[cost, above] {1} (u2);
		\draw[arc] (u2) to[bend left=15] node[cost, above] {3} (u3);

		% Arcos entrando em X (destacados em vermelho)
		\draw[arcEntering] (u1) -- node[cost, left] {3} (v1);
		\draw[arcEntering] (u2) -- node[cost, right] {2} (v2);
		\draw[arcEntering] (u3) -- node[cost, right] {4} (v3);
		\draw[arcEntering] (u1) to[bend right=20] node[cost, below, pos=0.3] {5} (v2);

		% Arcos internos a X (não retornados)
		\draw[arc] (v1) -- node[cost, above] {1} (v2);
		\draw[arc] (v2) -- node[cost, above] {2} (v3);

		% Arcos saindo de X (não retornados)
		\draw[arc] (v1) to[bend right=30] node[cost, left, pos=0.3] {2} (u1);

		% Região X destacada
		\draw[orange!60!black, thick, dashed, rounded corners=8pt]
		(-0.3, 0.3) rectangle (4.3, -1.3);
		\node[orange!80!black, font=\bfseries] at (5, 0.3) {$X$};
	\end{tikzpicture}
	\caption{Ilustração da função \texttt{get\_arcs\_entering\_X} em \(D_{32}\). A raiz \(r_0\) (em vermelho claro) conecta-se aos vértices \(u_1, u_2, u_3\). Os vértices em \textcolor{orange!80!black}{laranja} pertencem ao conjunto \(X = \{v_1, v_2, v_3\}\). A função identifica apenas os arcos \textcolor{red!70!black}{em vermelho}: aqueles que saem de vértices fora de \(X\) e entram em vértices dentro de \(X\). Arcos da raiz, arcos internos a \(X\), externos a \(X\), ou saindo de \(X\) não são retornados.}
\end{figure}

\subsection{Cálculo do peso mínimo de corte}

Esta função calcula o peso mínimo entre todos os arcos fornecidos, correspondendo ao valor \(\Delta(X)\) necessário para elevar os potenciais dos vértices em \(X\).

Recebe como entrada uma lista \texttt{arcos} de tuplas \((u, v, \text{data})\). A implementação usa a função \texttt{min} com uma compreensão de gerador (linha 2) que extrai o atributo \texttt{"w"} de cada tupla em \texttt{data}.

A função devolve o peso mínimo encontrado entre todos os arcos da lista. A complexidade é \(O(k)\), onde \(k\) é o número de arcos na lista, pois examina cada arco uma vez para encontrar o mínimo.

\begin{tcolorbox}[
		enhanced, breakable,
		colframe=blue!60!black, colback=blue!2,
		colbacktitle=blue!15, coltitle=black,
		title={Cálculo do peso mínimo de corte},
		boxed title style={sharp corners, boxrule=0.6pt},
		sharp corners, boxrule=0.6pt
	]
	\emph{Calcula o peso mínimo entre todos os arcos fornecidos, correspondendo ao valor $\Delta(X)$ usado na elevação de potenciais.}
	\tcblower
	\begin{lstlisting}[language=Python]
def get_minimum_weight_cut(arcs):
    return min(data["w"] for _, _, data in arcs)
\end{lstlisting}
\end{tcolorbox}

A seguir temos uma ilustração do funcionamento da função \texttt{get\_minimum\_weight\_cut} em \(D_{32}\). Considerando os arcos em vermelho que entram em \(X\) (identificados pela função anterior), esta função calcula o peso mínimo entre eles. O arco em verde possui o menor peso (2), correspondendo ao valor \(\Delta(X) = 2\) que será devolvido pela função.

\begin{figure}[H]
	\centering
	\begin{tikzpicture}[>=Stealth]
		% estilos
		\tikzset{
			vtx/.style={circle, draw, fill=blue!8, thick, inner sep=1.2pt, minimum size=7mm},
			vtxRoot/.style={circle, draw, fill=red!15, thick, inner sep=1.2pt, minimum size=7mm},
			vtxX/.style={circle, draw=orange!80!black, fill=orange!15, thick, inner sep=1.2pt, minimum size=7mm},
			arc/.style={->, >=Stealth, thick, draw=black!40},
			arcEntering/.style={->, >=Stealth, very thick, draw=red!70!black},
			arcMin/.style={->, >=Stealth, ultra thick, draw=green!60!black},
			cost/.style={font=\scriptsize, fill=white, inner sep=1pt},
			costMin/.style={font=\scriptsize\bfseries, fill=green!10, inner sep=2pt, draw=green!60!black, rounded corners=1pt}
		}

		% Raiz
		\node[vtxRoot] (r0) at (2, 4) {$r_0$};

		% Vértices fora de X
		\node[vtx] (u1) at (0, 2) {$u_1$};
		\node[vtx] (u2) at (2, 2) {$u_2$};
		\node[vtx] (u3) at (4, 2) {$u_3$};

		% Vértices dentro de X (destacados)
		\node[vtxX] (v1) at (0.5, -0.5) {$v_1$};
		\node[vtxX] (v2) at (2, -0.5) {$v_2$};
		\node[vtxX] (v3) at (3.5, -0.5) {$v_3$};

		% Arcos da raiz
		\draw[arc] (r0) -- node[cost, left, pos=0.4] {1} (u1);
		\draw[arc] (r0) -- node[cost, right] {2} (u2);
		\draw[arc] (r0) -- node[cost, right, pos=0.4] {1} (u3);

		% Arcos externos a X
		\draw[arc] (u1) to[bend left=15] node[cost, above] {1} (u2);
		\draw[arc] (u2) to[bend left=15] node[cost, above] {3} (u3);

		% Arcos entrando em X (em vermelho, mas o mínimo em verde)
		\draw[arcEntering] (u1) -- node[cost, left] {3} (v1);
		\draw[arcMin] (u2) -- node[costMin, right] {2} (v2);
		\draw[arcEntering] (u3) -- node[cost, right] {4} (v3);
		\draw[arcEntering] (u1) to[bend right=20] node[cost, below, pos=0.3] {5} (v2);

		% Arcos internos a X
		\draw[arc] (v1) -- node[cost, above] {1} (v2);
		\draw[arc] (v2) -- node[cost, above] {2} (v3);

		% Arcos saindo de X
		\draw[arc] (v1) to[bend right=30] node[cost, left, pos=0.3] {2} (u1);

		% Região X destacada
		\draw[orange!60!black, thick, dashed, rounded corners=8pt]
		(-0.3, 0.3) rectangle (4.3, -1.3);
		\node[orange!80!black, font=\bfseries] at (5, 0.3) {$X$};

		% Anotação do resultado
		\node[font=\bfseries, green!60!black] at (2, -2.2) {$\Delta(X) = \min\{3, 2, 4, 5\} = 2$};
	\end{tikzpicture}
	\caption{Ilustração da função \texttt{get\_minimum\_weight\_cut} em \(D_{32}\). Considerando os arcos \textcolor{red!70!black}{em vermelho} que entram em \(X\) (identificados pela função anterior), esta função calcula o peso mínimo entre eles. O arco \textcolor{green!60!black}{em verde} possui o menor peso (2), correspondendo ao valor \(\Delta(X) = 2\).}
\end{figure}

\subsection{Atualização de pesos em \texorpdfstring{\(X\)}{X}}

Esta função auxiliar atualiza os pesos dos arcos que entram em um conjunto \(X\), subtraindo o valor \(\Delta(X)\) de cada peso. Arcos que atingem peso zero são adicionados a \(A_0\) e a \(D_0\).

Recebe como entrada um dígrafo \texttt{D}, lista de \texttt{arcs} entrando em \(X\), o valor \texttt{min\_weight} a ser subtraído, uma lista \texttt{A\_zero} para armazenar arcos de peso zero, e o dígrafo \texttt{D\_zero} para adicionar os arcos justos.

A implementação itera sobre cada arco \((u,v,\_)\) da lista (linha 2), subtrai \texttt{min\_weight} do peso armazenado em \texttt{D[u][v]["w"]} (linha 3), e verifica se o peso resultante é zero (linha 4). Caso sim, adiciona-se \((u,v)\) à lista \texttt{A\_zero} (linha 5) e ao dígrafo \texttt{D\_zero} (linha 6).

A função não devolve valor, pois modifica diretamente as estruturas passadas como parâmetros: o dígrafo \texttt{D} tem seus pesos atualizados, \texttt{A\_zero} acumula arcos justos, e \texttt{D\_zero} é populado com esses arcos. A complexidade é \(O(k)\), onde \(k\) é o número de arcos em \texttt{arcs}.

\begin{tcolorbox}[
		enhanced, breakable,
		colframe=blue!60!black, colback=blue!2,
		colbacktitle=blue!15, coltitle=black,
		title={Atualização de pesos em X},
		boxed title style={sharp corners, boxrule=0.6pt},
		sharp corners, boxrule=0.6pt
	]
	\emph{Atualiza os pesos dos arcos que entram em X, subtraindo o valor mínimo. Arcos que atingem peso zero são registrados em $A_0$ e adicionados a $D_0$.}
	\tcblower
	\begin{lstlisting}[language=Python]
def update_weights_in_X(D, arcs, min_weight, A_zero, D_zero):
    for u, v, _ in arcs:
        D[u][v]["w"] -= min_weight
        if D[u][v]["w"] == 0:
            A_zero.append((u, v))
            D_zero.add_edge(u, v)
\end{lstlisting}
\end{tcolorbox}

A seguir ilustramos o funcionamento da função \texttt{update\_weights\_in\_X} em \(D_{32}\). A figura mostra o dígrafo antes e depois da atualização dos pesos. No estado inicial (esquerda), temos \(\Delta(X) = 2\). A função subtrai esse valor de todos os arcos que entram em \(X\). No estado final (direita), os arcos que atingiram peso zero (destacados em azul) são adicionados a \(A_0\) e \(D_0\).

\begin{figure}[H]
	\centering
	\begin{tikzpicture}[>=Stealth]
		% estilos
		\tikzset{
			vtx/.style={circle, draw, fill=blue!8, thick, inner sep=1.2pt, minimum size=6.5mm},
			vtxRoot/.style={circle, draw, fill=red!15, thick, inner sep=1.2pt, minimum size=6.5mm},
			vtxX/.style={circle, draw=orange!80!black, fill=orange!15, thick, inner sep=1.2pt, minimum size=6.5mm},
			arc/.style={->, >=Stealth, thick, draw=black!40},
			arcEntering/.style={->, >=Stealth, very thick, draw=red!70!black},
			arcTight/.style={->, >=Stealth, ultra thick, draw=blue!70!black},
			cost/.style={font=\scriptsize, fill=white, inner sep=1pt},
			costTight/.style={font=\scriptsize\bfseries, fill=blue!10, inner sep=2pt, draw=blue!70!black, rounded corners=1pt}
		}

		% ===== ANTES (esquerda) =====
		\begin{scope}
			\node[font=\bfseries, blue!80!black] at (2, 4.2) {Antes: pesos originais};

			% Raiz
			\node[vtxRoot] (r0) at (2, 3) {$r_0$};

			% Vértices fora de X
			\node[vtx] (u1) at (0, 1.5) {$u_1$};
			\node[vtx] (u2) at (2, 1.5) {$u_2$};
			\node[vtx] (u3) at (4, 1.5) {$u_3$};

			% Vértices dentro de X
			\node[vtxX] (v1) at (0.5, -0.5) {$v_1$};
			\node[vtxX] (v2) at (2, -0.5) {$v_2$};
			\node[vtxX] (v3) at (3.5, -0.5) {$v_3$};

			% Arcos da raiz
			\draw[arc] (r0) -- node[cost, left, pos=0.4] {1} (u1);
			\draw[arc] (r0) -- node[cost, right] {2} (u2);
			\draw[arc] (r0) -- node[cost, right, pos=0.4] {1} (u3);

			% Arcos externos a X
			\draw[arc] (u1) to[bend left=15] node[cost, above] {1} (u2);
			\draw[arc] (u2) to[bend left=15] node[cost, above] {3} (u3);

			% Arcos entrando em X
			\draw[arcEntering] (u1) -- node[cost, left] {3} (v1);
			\draw[arcEntering] (u2) -- node[cost, right] {2} (v2);
			\draw[arcEntering] (u3) -- node[cost, right] {4} (v3);
			\draw[arcEntering] (u1) to[bend right=20] node[cost, below, pos=0.3] {5} (v2);

			% Arcos internos a X
			\draw[arc] (v1) -- node[cost, above] {1} (v2);
			\draw[arc] (v2) -- node[cost, above] {2} (v3);

			% Arcos saindo de X
			\draw[arc] (v1) to[bend right=30] node[cost, left, pos=0.3] {2} (u1);

			% Região X
			\draw[orange!60!black, thick, dashed, rounded corners=8pt]
			(-0.3, 0.3) rectangle (4.3, -1.3);
			\node[orange!80!black, font=\bfseries\small] at (4.8, 0.1) {$X$};

			\node[font=\small, green!60!black] at (2, -2.2) {$\Delta(X) = 2$};
		\end{scope}

		% Seta indicando transformação
		\draw[->, ultra thick, blue!60!black] (5.3, 0.5) -- node[above, font=\small] {subtrai $\Delta(X)$} (6.7, 0.5);

		% ===== DEPOIS (direita) =====
		\begin{scope}[xshift=7.5cm]
			\node[font=\bfseries, blue!80!black] at (2, 4.2) {Depois: pesos atualizados};

			% Raiz
			\node[vtxRoot] (r0b) at (2, 3) {$r_0$};

			% Vértices fora de X
			\node[vtx] (u1b) at (0, 1.5) {$u_1$};
			\node[vtx] (u2b) at (2, 1.5) {$u_2$};
			\node[vtx] (u3b) at (4, 1.5) {$u_3$};

			% Vértices dentro de X
			\node[vtxX] (v1b) at (0.5, -0.5) {$v_1$};
			\node[vtxX] (v2b) at (2, -0.5) {$v_2$};
			\node[vtxX] (v3b) at (3.5, -0.5) {$v_3$};

			% Arcos da raiz (inalterados)
			\draw[arc] (r0b) -- node[cost, left, pos=0.4] {1} (u1b);
			\draw[arc] (r0b) -- node[cost, right] {2} (u2b);
			\draw[arc] (r0b) -- node[cost, right, pos=0.4] {1} (u3b);

			% Arcos externos a X (inalterados)
			\draw[arc] (u1b) to[bend left=15] node[cost, above] {1} (u2b);
			\draw[arc] (u2b) to[bend left=15] node[cost, above] {3} (u3b);

			% Arcos entrando em X (atualizados)
			\draw[arcEntering] (u1b) -- node[cost, left] {1} (v1b);
			\draw[arcTight] (u2b) -- node[costTight, right] {0} (v2b);
			\draw[arcEntering] (u3b) -- node[cost, right] {2} (v3b);
			\draw[arcEntering] (u1b) to[bend right=20] node[cost, below, pos=0.3] {3} (v2b);

			% Arcos internos a X (inalterados)
			\draw[arc] (v1b) -- node[cost, above] {1} (v2b);
			\draw[arc] (v2b) -- node[cost, above] {2} (v3b);

			% Arcos saindo de X (inalterados)
			\draw[arc] (v1b) to[bend right=30] node[cost, left, pos=0.3] {2} (u1b);

			% Região X
			\draw[orange!60!black, thick, dashed, rounded corners=8pt]
			(-0.3, 0.3) rectangle (4.3, -1.3);
			\node[orange!80!black, font=\bfseries\small] at (4.8, 0.1) {$X$};

			\node[font=\small, blue!70!black] at (2, -2.2) {Arco \textcolor{blue!70!black}{justo} $\to A_0$};
		\end{scope}
	\end{tikzpicture}
	\caption{Ilustração da função \texttt{update\_weights\_in\_X} em \(D_{32}\). À esquerda, o dígrafo antes da atualização, com os arcos \textcolor{red!70!black}{em vermelho} entrando em \(X\) e \(\Delta(X) = 2\). À direita, após subtrair \(\Delta(X)\) de cada arco entrando em \(X\): o peso $(u_1, v_1)$ reduz de 3 para 1, $(u_2, v_2)$ de 2 para \textcolor{blue!70!black}{\textbf{0}} (torna-se justo), $(u_3, v_3)$ de 4 para 2, e $(u_1, v_2)$ de 5 para 3. O arco justo é adicionado a \(A_0\) e \(D_0\). Note que os arcos da raiz e arcos internos/externos a \(X\) permanecem inalterados.}
\end{figure}

\subsection{Verificação de arborescência}

Esta função verifica se um dígrafo \(D\) contém uma r-arborescência com raiz \(r_0\). Utiliza busca em profundidade (DFS) a partir da raiz para verificar se todos os vértices são alcançáveis.

Recebe como entrada um dígrafo \texttt{D} e a raiz \texttt{r0}. A implementação constrói uma árvore DFS a partir de \texttt{r0} usando \texttt{nx.dfs\_tree} (linha 2), que devolve um subdigrafo contendo apenas os vértices alcançáveis a partir da raiz seguindo arcos. Em seguida, compara o número de vértices da árvore DFS com o número total de vértices de \texttt{D} (linha 3).

A função devolve \texttt{True} se todos os vértices são alcançáveis (indicando presença de r-arborescência), \texttt{False} caso contrário. A complexidade é \(O(n+m)\), onde \(n=|V|\) e \(m=|A|\), devido à busca em profundidade. Precisamos dessa verificação para garantir que a Fase~1 produza um dígrafo \(D_0\) que contenha uma r-arborescência antes de prosseguir para a Fase~2.

\begin{tcolorbox}[
		enhanced, breakable,
		colframe=blue!60!black, colback=blue!2,
		colbacktitle=blue!15, coltitle=black,
		title={Verificação de arborescência},
		boxed title style={sharp corners, boxrule=0.6pt},
		sharp corners, boxrule=0.6pt
	]
	\emph{Verifica se o dígrafo D contém uma r-arborescência com raiz r0, usando busca em profundidade para testar alcançabilidade de todos os vértices.}
	\tcblower
	\begin{lstlisting}[language=Python]
def has_arborescence(D, r0):
    tree = nx.dfs_tree(D, r0)
    return tree.number_of_nodes() == D.number_of_nodes()
\end{lstlisting}
\end{tcolorbox}

\subsection{Fase 1: Elevação de potenciais e construção de \texorpdfstring{\(A_0\)}{A_0}}

A seguir apresentamos a função principal da Fase~1, responsável por elevar os potenciais dos vértices iterativamente até que cada conjunto de vértices possua ao menos um arco justo entrando. O processo utiliza componentes fortemente conexas para identificar quais conjuntos necessitam elevação.

Recebe como entrada um dígrafo \texttt{D\_original} e a raiz \texttt{r0}. A implementação cria uma cópia do dígrafo original (linha 2) para preservar a entrada, inicializa estruturas auxiliares \texttt{A\_zero} (lista de arcos justos), \texttt{Dual\_list} (lista de pares \((X, \Delta(X))\) para fins de validação dual), e \texttt{D\_zero} (dígrafo de arcos justos) (linhas 3-5). Um contador de iterações é inicializado na linha 6.

O loop principal (linhas 7-22) itera enquanto houver conjuntos sem arcos justos entrando. Em cada iteração, incrementa-se o contador (linha 8), calcula-se as componentes fortemente conexas de \(D_0\) usando \texttt{nx.condensation} (linha 9), que devolve um grafo acíclico dirigido (DAG, do inglês \emph{directed acyclic graph}) onde cada vértice representa uma componente e contém o atributo \texttt{"members"} com os vértices originais. Utilizamos componentes fortemente conexas porque elas identificam naturalmente os conjuntos maximais de vértices que ainda não possuem arcos justos entrando, evitando a necessidade de rastrear manualmente quais conjuntos já foram processados.

Em seguida, identificam-se as fontes (componentes sem arcos entrando) no grafo de condensação (linha 10). Se há apenas uma fonte, significa que todos os vértices estão em uma única componente alcançável pela raiz através de arcos justos, garantindo que \(D_0\) contém uma r-arborescência e encerrando o loop (linhas 11-12).

Para cada fonte \(u\) no grafo de condensação (linha 13), obtém-se o conjunto \(X\) de vértices da componente (linha 14). Se \(r_0 \in X\), a fonte é ignorada (linhas 15-16), pois a componente contendo a raiz não necessita elevação. Caso contrário, identificam-se os arcos entrando em \(X\) usando \texttt{get\_arcs\_entering\_X} (linha 17), calcula-se o peso mínimo \(\Delta(X)\) usando \texttt{get\_minimum\_weight\_cut} (linha 18), e atualiza-se os pesos com \texttt{update\_weights\_in\_X}, registrando novos arcos justos (linha 19). A elevação simultânea de potenciais para todos os vértices de \(X\) mantém a propriedade de que arcos internos a \(X\) permanecem com o mesmo custo reduzido relativo, preservando a correção do algoritmo. Finalmente, adiciona-se \((X, \Delta(X))\) à lista dual se \(\Delta(X) > 0\) (linhas 20-21), permitindo verificação posterior das condições de otimalidade dual.

A função devolve \texttt{A\_zero} (lista de arcos justos) e \texttt{Dual\_list} (pares \((X, \Delta(X))\) para validação). A complexidade é \(O(nm)\) no pior caso, com \(O(n)\) iterações, cada uma custando \(O(m)\) para calcular componentes e atualizar pesos.

\begin{tcolorbox}[
		enhanced, breakable,
		colframe=blue!60!black, colback=blue!2,
		colbacktitle=blue!15, coltitle=black,
		title={Fase 1: Elevação de potenciais e construção de $A_0$},
		boxed title style={sharp corners, boxrule=0.6pt},
		sharp corners, boxrule=0.6pt
	]
	\emph{Eleva iterativamente os potenciais dos vértices até que cada conjunto possua ao menos um arco justo entrando. Devolve a lista $A_0$ de arcos justos e a lista de pares $(X, \Delta(X))$ para validação dual.}
	\tcblower
	\begin{lstlisting}[language=Python, basicstyle=\ttfamily\fontsize{8}{9}\selectfont]
def phase1(D_original, r0):
    D_copy = D_original.copy()
    A_zero = []
    Dual_list = []
    D_zero = build_D_zero(D_copy)
    iteration = 0
    while True:
        iteration += 1
        C = nx.condensation(D_zero)
        sources = [x for x in C.nodes() if C.in_degree(x) == 0]
        if len(sources) == 1:
            break
        for u in sources:
            X = C.nodes[u]["members"]
            if r0 in X:
                continue
            arcs = get_arcs_entering_X(D_copy, X)
            min_weight = get_minimum_weight_cut(arcs)
            update_weights_in_X(D_copy, arcs, min_weight, A_zero, D_zero)
            if min_weight != 0:
                Dual_list.append((X, min_weight)) 
    return A_zero, Dual_list
\end{lstlisting}
\end{tcolorbox}

A seguir ilustramos a execução da função \texttt{phase1} no dígrafo \(D_{32}\) estendido com a raiz \(r_0\). A figura mostra três estados principais: o dígrafo original \(D\) com seus pesos, o dígrafo \(D_0\) inicial (vazio, sem arcos), e o dígrafo \(D_0\) final após as iterações de elevação de potenciais, contendo apenas os arcos justos (em azul) que formam uma r-arborescência.

\begin{figure}[H]
	\centering
	\begin{tikzpicture}[>=Stealth, scale=0.85, every node/.style={scale=0.85}]
		% Estilos
		\tikzset{
			vtx/.style={circle, draw, fill=blue!8, thick, inner sep=1.2pt, minimum size=6.5mm},
			vtxRoot/.style={circle, draw, fill=red!15, thick, inner sep=1.2pt, minimum size=6.5mm},
			vtxX/.style={circle, draw, fill=orange!20, thick, inner sep=1.2pt, minimum size=6.5mm},
			arc/.style={->, thick, gray!60},
			arcTight/.style={->, ultra thick, blue!70!black},
			cost/.style={font=\scriptsize, fill=white, inner sep=1pt, rounded corners=1pt},
			costTight/.style={font=\scriptsize\bfseries, fill=blue!10, inner sep=2pt, draw=blue!70!black, rounded corners=1pt}
		}

		% ===== DÍGRAFO ORIGINAL D (esquerda) =====
		\begin{scope}
			\node[font=\bfseries, blue!80!black] at (2, 4.2) {Dígrafo original $D$};

			% Raiz
			\node[vtxRoot] (r0) at (2, 3) {$r_0$};

			% Vértices fora de X
			\node[vtx] (u1) at (0, 1.5) {$u_1$};
			\node[vtx] (u2) at (2, 1.5) {$u_2$};
			\node[vtx] (u3) at (4, 1.5) {$u_3$};

			% Vértices dentro de X
			\node[vtxX] (v1) at (0.5, -0.5) {$v_1$};
			\node[vtxX] (v2) at (2, -0.5) {$v_2$};
			\node[vtxX] (v3) at (3.5, -0.5) {$v_3$};

			% Arcos da raiz
			\draw[arc] (r0) -- node[cost, left, pos=0.4] {1} (u1);
			\draw[arc] (r0) -- node[cost, right] {2} (u2);
			\draw[arc] (r0) -- node[cost, right, pos=0.4] {1} (u3);

			% Arcos externos a X
			\draw[arc] (u1) to[bend left=15] node[cost, above] {1} (u2);
			\draw[arc] (u2) to[bend left=15] node[cost, above] {3} (u3);

			% Arcos entrando em X
			\draw[arc] (u1) -- node[cost, left] {3} (v1);
			\draw[arc] (u2) -- node[cost, right] {2} (v2);
			\draw[arc] (u3) -- node[cost, right] {4} (v3);
			\draw[arc] (u1) to[bend right=20] node[cost, below, pos=0.3] {5} (v2);

			% Arcos internos a X
			\draw[arc] (v1) -- node[cost, above] {1} (v2);
			\draw[arc] (v2) -- node[cost, above] {2} (v3);

			% Arcos saindo de X
			\draw[arc] (v1) to[bend right=30] node[cost, left, pos=0.3] {2} (u1);

			\node[font=\small, gray!70!black] at (2, -2) {Pesos originais};
		\end{scope}

		% ===== D_0 INICIAL (centro) =====
		\begin{scope}[xshift=6cm]
			\node[font=\bfseries, blue!80!black] at (2, 4.2) {$D_0$ inicial (linha 5)};

			% Vértices apenas, sem arcos
			\node[vtxRoot] (r0i) at (2, 3) {$r_0$};
			\node[vtx] (u1i) at (0, 1.5) {$u_1$};
			\node[vtx] (u2i) at (2, 1.5) {$u_2$};
			\node[vtx] (u3i) at (4, 1.5) {$u_3$};
			\node[vtxX] (v1i) at (0.5, -0.5) {$v_1$};
			\node[vtxX] (v2i) at (2, -0.5) {$v_2$};
			\node[vtxX] (v3i) at (3.5, -0.5) {$v_3$};

			\node[font=\small, gray!70!black] at (2, -2) {Sem arcos (vazio)};
		\end{scope}

		% ===== D_0 FINAL (direita) =====
		\begin{scope}[xshift=12cm]
			\node[font=\bfseries, blue!80!black] at (2, 4.2) {$D_0$ final (após loop)};

			% Vértices
			\node[vtxRoot] (r0f) at (2, 3) {$r_0$};
			\node[vtx] (u1f) at (0, 1.5) {$u_1$};
			\node[vtx] (u2f) at (2, 1.5) {$u_2$};
			\node[vtx] (u3f) at (4, 1.5) {$u_3$};
			\node[vtxX] (v1f) at (0.5, -0.5) {$v_1$};
			\node[vtxX] (v2f) at (2, -0.5) {$v_2$};
			\node[vtxX] (v3f) at (3.5, -0.5) {$v_3$};

			% Arcos justos (peso 0)
			\draw[arcTight] (r0f) -- node[costTight, left, pos=0.4] {0} (u1f);
			\draw[arcTight] (r0f) -- node[costTight, right, pos=0.4] {0} (u3f);
			\draw[arcTight] (u1f) to[bend left=15] node[costTight, above] {0} (u2f);
			\draw[arcTight] (u1f) -- node[costTight, left] {0} (v1f);
			\draw[arcTight] (u2f) -- node[costTight, right] {0} (v2f);
			\draw[arcTight] (v2f) -- node[costTight, above] {0} (v3f);

			\node[font=\small, blue!70!black] at (2, -2) {Arcos justos em $A_0$};
		\end{scope}

		% Setas indicando progressão
		\draw[->, ultra thick, gray!50] (4.5, 1) -- node[above, font=\small] {elevação} (5.5, 1);
		\draw[->, ultra thick, gray!50] (10.5, 1) -- node[above, font=\small] {iterações} (11.5, 1);
	\end{tikzpicture}
	\caption{Execução da função \texttt{phase1} em \(D_{32}\). À esquerda, o dígrafo original \(D\) com pesos dos arcos. Ao centro, \(D_0\) inicial após linha 5 (\texttt{build\_D\_zero}), contendo apenas vértices sem arcos. À direita, \(D_0\) final após o loop de elevação de potenciais, contendo apenas os arcos \textcolor{blue!70!black}{justos} (custo reduzido zero) que formam uma r-arborescência. Durante as iterações (linhas 7-22), os potenciais são elevados para cada componente sem arcos justos entrando, até que todos os vértices sejam alcançáveis a partir de \(r_0\) através de arcos em \(D_0\).}
\end{figure}

A seguir ilustramos a execução da função \texttt{phase2} em \(D_{32}\), mostrando como a r-arborescência é construída incrementalmente a partir do conjunto \(A_0\) de arcos justos obtidos da Fase~1. A figura apresenta quatro estados: o conjunto inicial \(A_0\) com 6 arcos justos, e três estados intermediários da arborescência \texttt{Arb} sendo construída, destacando em verde os arcos adicionados em cada iteração.

\begin{figure}[H]
	\centering
	\begin{tikzpicture}[>=Stealth, scale=0.72, every node/.style={scale=0.72}]
		% Estilos
		\tikzset{
			vtx/.style={circle, draw, fill=blue!8, thick, inner sep=1.2pt, minimum size=6.5mm},
			vtxRoot/.style={circle, draw, fill=red!15, thick, inner sep=1.2pt, minimum size=6.5mm},
			vtxIn/.style={circle, draw, fill=green!15, thick, inner sep=1.2pt, minimum size=6.5mm},
			vtxOut/.style={circle, draw, fill=gray!10, thick, inner sep=1.2pt, minimum size=6.5mm},
			arc/.style={->, thick, blue!70!black},
			arcNew/.style={->, ultra thick, green!70!black},
			cost/.style={font=\scriptsize, fill=white, inner sep=1pt}
		}

		% ===== A_0 (esquerda superior) =====
		\begin{scope}
			\node[font=\bfseries, blue!80!black] at (2, 4.5) {$A_0$ (arcos justos)};

			\node[vtxRoot] (r0) at (2, 3.2) {$r_0$};
			\node[vtx] (u1) at (0, 1.7) {$u_1$};
			\node[vtx] (u2) at (2, 1.7) {$u_2$};
			\node[vtx] (u3) at (4, 1.7) {$u_3$};
			\node[vtx] (v1) at (0.5, 0) {$v_1$};
			\node[vtx] (v2) at (2, 0) {$v_2$};
			\node[vtx] (v3) at (3.5, 0) {$v_3$};

			\draw[arc] (r0) -- node[cost, left, pos=0.3] {0} (u1);
			\draw[arc] (r0) -- node[cost, right, pos=0.3] {0} (u3);
			\draw[arc] (u1) to[bend left=12] node[cost, above, pos=0.4] {0} (u2);
			\draw[arc] (u1) -- node[cost, left] {0} (v1);
			\draw[arc] (u2) -- node[cost, right] {0} (v2);
			\draw[arc] (v2) -- node[cost, above] {0} (v3);

			\node[font=\small, gray!70!black] at (2, -1) {6 arcos com custo 0};
		\end{scope}

		% ===== Iteração 1 (direita superior) =====
		\begin{scope}[xshift=6.5cm]
			\node[font=\bfseries, blue!80!black] at (2, 4.5) {Iteração 1: $|$Arb$| = 1$};

			\node[vtxRoot] (r01) at (2, 3.2) {$r_0$};
			\node[vtxIn] (u11) at (0, 1.7) {$u_1$};
			\node[vtxOut] (u21) at (2, 1.7) {$u_2$};
			\node[vtxOut] (u31) at (4, 1.7) {$u_3$};
			\node[vtxOut] (v11) at (0.5, 0) {$v_1$};
			\node[vtxOut] (v21) at (2, 0) {$v_2$};
			\node[vtxOut] (v31) at (3.5, 0) {$v_3$};

			\draw[arcNew] (r01) -- node[cost, left, pos=0.3, fill=green!10] {1} (u11);

			\node[font=\small, green!70!black] at (2, -1) {Adiciona $(r_0, u_1)$};
		\end{scope}

		% ===== Iteração 2 (esquerda inferior) =====
		\begin{scope}[yshift=-6.2cm]
			\node[font=\bfseries, blue!80!black] at (2, 4.5) {Iteração 2: $|$Arb$| = 2$};

			\node[vtxRoot] (r02) at (2, 3.2) {$r_0$};
			\node[vtxIn] (u12) at (0, 1.7) {$u_1$};
			\node[vtxOut] (u22) at (2, 1.7) {$u_2$};
			\node[vtxIn] (u32) at (4, 1.7) {$u_3$};
			\node[vtxOut] (v12) at (0.5, 0) {$v_1$};
			\node[vtxOut] (v22) at (2, 0) {$v_2$};
			\node[vtxOut] (v32) at (3.5, 0) {$v_3$};

			\draw[arc] (r02) -- node[cost, left, pos=0.3] {1} (u12);
			\draw[arcNew] (r02) -- node[cost, right, pos=0.3, fill=green!10] {1} (u32);

			\node[font=\small, green!70!black] at (2, -1) {Adiciona $(r_0, u_3)$};
		\end{scope}

		% ===== Iteração 4 (direita inferior) =====
		\begin{scope}[xshift=6.5cm, yshift=-6.2cm]
			\node[font=\bfseries, blue!80!black] at (2, 4.5) {Iteração 4: $|$Arb$| = 4$};

			\node[vtxRoot] (r04) at (2, 3.2) {$r_0$};
			\node[vtxIn] (u14) at (0, 1.7) {$u_1$};
			\node[vtxIn] (u24) at (2, 1.7) {$u_2$};
			\node[vtxIn] (u34) at (4, 1.7) {$u_3$};
			\node[vtxIn] (v14) at (0.5, 0) {$v_1$};
			\node[vtxOut] (v24) at (2, 0) {$v_2$};
			\node[vtxOut] (v34) at (3.5, 0) {$v_3$};

			\draw[arc] (r04) -- node[cost, left, pos=0.3] {1} (u14);
			\draw[arc] (r04) -- node[cost, right, pos=0.3] {1} (u34);
			\draw[arc] (u14) to[bend left=12] node[cost, above, pos=0.4] {1} (u24);
			\draw[arcNew] (u14) -- node[cost, left, fill=green!10] {3} (v14);

			\node[font=\small, green!70!black] at (2, -1) {Adiciona $(u_1, v_1)$};
		\end{scope}

		% Setas de progressão
		\draw[->, ultra thick, gray!50] (4.8, 2) -- node[above, font=\small] {linha 6-10} (5.7, 2);
		\draw[->, ultra thick, gray!50] (2, -1.5) -- node[left, font=\small] {linha 6-10} (2, -2.4);
		\draw[->, ultra thick, gray!50] (4.8, -4.2) -- node[above, font=\small] {linha 6-10} (5.7, -4.2);
	\end{tikzpicture}
	\caption{Execução da função \texttt{phase2} em \(D_{32}\). Superior esquerdo: conjunto \(A_0\) com 6 arcos justos (custo reduzido zero) obtidos da Fase~1. Superior direito: Iteração 1 — adiciona arco $(r_0, u_1)$ pois $r_0 \in$ Arb e $u_1 \notin$ Arb. Inferior esquerdo: Iteração 2 — adiciona $(r_0, u_3)$. Inferior direito: Iteração 4 — após adicionar $(u_1, u_2)$ na iteração 3, adiciona $(u_1, v_1)$. Vértices \textcolor{green!70!black}{verdes} foram recém-adicionados, vértices \textcolor{gray!70!black}{cinzas} ainda não pertencem a Arb. O processo continua até que todos os 6 arcos de \(A_0\) sejam incluídos, formando uma r-arborescência com pesos originais de \(D\).}
\end{figure}

As funções auxiliares implementadas nesta seção correspondem diretamente aos passos da Fase~1 do algoritmo de András Frank da seguinte forma:

\begin{tcolorbox}[
		enhanced, breakable,
		colframe=purple!60!black, colback=purple!5,
		colbacktitle=purple!15, coltitle=black,
		title={Correspondência entre Teoria e Implementação — Fase 1},
		boxed title style={sharp corners, boxrule=0.6pt},
		sharp corners, boxrule=0.6pt
	]
	\textbf{Passo 1 — Inicialização:}
	\begin{itemize}
		\item \textbf{Descrição teórica:} Defina \(y(v) := 0\) para todo \(v \in V\).
		\item \textbf{Implementação:} A função \texttt{build\_D\_zero(D)} inicializa o dígrafo \(D_0\) vazio, que será populado apenas com arcos justos. Implicitamente, os potenciais iniciam em zero, pois os pesos no dígrafo \(D\) representam os custos reduzidos \(c_y(u,v) = c(u,v) - y(v)\). Com \(y(v) = 0\), temos \(c_y = c\) no início.
	\end{itemize}

	\vspace{2mm}
	\textbf{Passo 2 — Iteração de elevação de potenciais:}
	\begin{itemize}
		\item \textbf{Descrição teórica:} Enquanto existir conjunto \(X \subseteq V \setminus \{r\}\) sem arco justo entrando:
		      \begin{itemize}
			      \item Calcule \(\Delta(X) := \min\{c(u,v) - y(v) : u \notin X, v \in X\}\).
			      \item Para cada \(v \in X\), atualize \(y(v) := y(v) + \Delta(X)\).
		      \end{itemize}
		\item \textbf{Implementação:} Esse processo de iteração é realizado pela composição sequencial de três funções auxiliares:
		      \begin{enumerate}
			      \item \texttt{get\_arcs\_entering\_X(D, X)}: Identifica o conjunto \(\{(u,v) : u \notin X, v \in X\}\), isto é, todos os arcos que cruzam a fronteira de \(X\). Essa função corresponde diretamente à definição do conjunto sobre o qual o mínimo é calculado na fórmula \(\Delta(X)\).

			      \item \texttt{get\_minimum\_weight\_cut(arcs)}: Calcula \(\min\{\text{data}["w"] : (u,v,\text{data}) \in \text{arcs}\}\), que é exatamente \(\Delta(X) = \min\{c_y(u,v) : u \notin X, v \in X\}\). Como os pesos no dígrafo já representam custos reduzidos (são atualizados a cada iteração), essa função devolve precisamente o valor teórico de \(\Delta(X)\).

			      \item \texttt{update\_weights\_in\_X(D, arcs, min\_weight, A\_zero, D\_zero)}: Implementa a atualização dos potenciais. Para cada arco \((u,v)\) entrando em \(X\), subtrai \texttt{min\_weight} de \texttt{D[u][v]["w"]}, efetivamente calculando o novo custo reduzido:
			            \[c_y'(u,v) = c_y(u,v) - \Delta(X) = \]
			            \[c(u,v) - y(v) - \Delta(X) = \]
			            \[c(u,v) - \bigl(y(v) + \Delta(X)\bigr) = c(u,v) - y'(v), \]
			            onde \(y'(v) = y(v) + \Delta(X)\) é o novo potencial. Arcos cujo custo reduzido atinge zero (\texttt{D[u][v]["w"] == 0}) são adicionados a \(A_0\) e \(D_0\), tornando-se \textbf{justos}.
		      \end{enumerate}
	\end{itemize}

	\vspace{2mm}
	\textbf{Passo 3 — Construção de \(A_0\):}
	\begin{itemize}
		\item \textbf{Descrição teórica:} Defina \(A_0 := \{a \in A : c_y(a) = 0\}\), o conjunto de arcos justos.
		\item \textbf{Implementação:} A construção de \(A_0\) ocorre de forma incremental durante as iterações do Passo~2. Cada chamada a \texttt{update\_weights\_in\_X} verifica quais arcos atingiram custo reduzido zero e os adiciona tanto à lista \texttt{A\_zero} quanto ao dígrafo \texttt{D\_zero}. Esse processo continua até que a função principal da Fase~1 (que será apresentada adiante) determine que cada componente fortemente conexa em \(D_0\) (exceto a raiz) possui ao menos um arco justo entrando, garantindo que \(A_0\) é suficiente para formar a base de uma \(r\)-arborescência.
		\item O resultado final é o conjunto completo \(A_0 = \{a \in A : c_y(a) = 0\}\), usado na Fase~2 para construir a arborescência ótima através de uma seleção gulosa de arcos justos.
	\end{itemize}
\end{tcolorbox}

\subsection{Fase 2: Construção da arborescência}
Esta é a função principal da Fase~2, responsável por construir a r-arborescência de custo mínimo a partir do conjunto \(A_0\) de arcos justos. A construção é incremental: inicia-se com a raiz e adiciona-se iterativamente arcos de \(A_0\) que conectam vértices já incluídos a novos vértices, garantindo que cada vértice não-raiz receba exatamente um arco de entrada.

Recebe como entrada um dígrafo \texttt{D\_original}, a raiz \texttt{r0}, e a lista \texttt{A\_zero} de arcos justos. A implementação cria um novo dígrafo vazio \texttt{Arb} (linha 2) e adiciona a raiz (linha 3).

O loop principal (linhas 5-12) itera \(n-1\) vezes, onde \(n = |V|\), pois uma r-arborescência tem exatamente \(|V|-1\) arcos. Em cada iteração:
\begin{enumerate}\setlength{\itemsep}{2pt}
	\item Percorre os arcos \((u,v)\) de \texttt{A\_zero} (linha 6).
	\item Verifica se \(u\) já está em \texttt{Arb} e \(v\) ainda não (linha 7).
	\item Se sim, obtém os dados do arco do dígrafo original (linha 8) e adiciona \((u,v)\) a \texttt{Arb} (linha 9).
	\item Interrompe o loop interno para reiniciar a busca, garantindo descoberta em largura (linha 10).
\end{enumerate}

A função devolve o dígrafo \texttt{Arb} representando a r-arborescência de custo mínimo. A complexidade é \(O(nm)\) no pior caso, pois cada uma das \(O(n)\) iterações pode percorrer todos os \(O(m)\) arcos de \texttt{A\_zero}.

\begin{tcolorbox}[
		enhanced, breakable,
		colframe=blue!60!black, colback=blue!2,
		colbacktitle=blue!15, coltitle=black,
		title={Fase 2: Construção da arborescência},
		boxed title style={sharp corners, boxrule=0.6pt},
		sharp corners, boxrule=0.6pt
	]
	\emph{Constrói incrementalmente a r-arborescência a partir de $A_0$, adicionando iterativamente arcos que conectam vértices já incluídos a novos vértices.}
	\tcblower
	\begin{lstlisting}[language=Python]
def phase2(D_original, r0, A_zero):
    Arb = nx.DiGraph()
    Arb.add_node(r0)
    n = len(D_original.nodes())
    for _ in range(n - 1):
        for u, v in A_zero:
            if u in Arb.nodes() and v not in Arb.nodes():
                edge_data = D_original.get_edge_data(u, v)
                Arb.add_edge(u, v, **edge_data)
                break
    return Arb
\end{lstlisting}
\end{tcolorbox}

Apresentamos também uma versão alternativa da Fase~2 que utiliza busca em largura (BFS) para construir a r-arborescência de forma mais eficiente. Diferentemente da versão anterior que itera \(n-1\) vezes sobre todos os arcos, esta implementação usa uma fila de prioridade para explorar os arcos em ordem, evitando buscas lineares repetidas.

Recebe como entrada um dígrafo \texttt{D\_original}, a raiz \texttt{r0}, e a lista \texttt{A\_zero} de arcos justos. A implementação começa criando um dígrafo auxiliar \texttt{Arb} (linha 2) onde cada arco de \texttt{A\_zero} recebe um peso igual ao seu índice na lista (linhas 3-4), estabelecendo uma ordem de exploração. Inicializa-se o conjunto \texttt{V} de vértices visitados contendo apenas a raiz (linha 5) e uma fila de prioridade vazia \texttt{q} (linha 6).

Todos os arcos que saem da raiz em \texttt{Arb} são adicionados à fila de prioridade (linhas 7-8), usando o peso (índice) como critério de ordenação. Cria-se então o dígrafo \texttt{A} que conterá a arborescência resultante (linha 9).

O loop principal (linhas 10-17) extrai arcos da fila de prioridade em ordem crescente de índice. Para cada arco \((u,v)\) extraído (linha 11), verifica-se se o vértice destino \(v\) já foi visitado (linha 12); em caso positivo, o arco é ignorado via \texttt{continue} (linha 13). Caso contrário, adiciona-se o arco \((u,v)\) à arborescência \texttt{A} com o peso original de \texttt{D\_original} (linha 14), marca-se \(v\) como visitado (linha 15), e todos os arcos que saem de \(v\) em \texttt{Arb} são adicionados à fila de prioridade para exploração futura (linhas 16-17).

A função devolve o dígrafo \texttt{A} representando a r-arborescência de custo mínimo. A complexidade é \(O(m \log m)\), onde \(m = |A_0|\), devido às operações de inserção e remoção na fila de prioridade. Esta versão é mais eficiente que a anterior quando \(|A_0|\) é grande, pois evita percorrer todos os arcos em cada iteração.

\begin{tcolorbox}[
		enhanced, breakable,
		colframe=blue!60!black, colback=blue!2,
		colbacktitle=blue!15, coltitle=black,
		title={Fase 2 (versão BFS): Construção da arborescência com fila de prioridade},
		boxed title style={sharp corners, boxrule=0.6pt},
		sharp corners, boxrule=0.6pt
	]
	\emph{Constrói a r-arborescência usando busca em largura guiada por fila de prioridade, explorando arcos de $A_0$ em ordem e evitando buscas lineares repetidas. Complexidade $O(m \log m)$.}
	\tcblower
	\begin{lstlisting}[language=Python]
def phase2_v2(D_original, r0, A_zero):
    Arb = nx.DiGraph()
    for i, (u, v) in enumerate(A_zero):
        Arb.add_edge(u, v, w=i)
    V = {r0}
    q = []
    for u, v, data in Arb.out_edges(r0, data=True):
        heapq.heappush(q, (data["w"], u, v))
    A = nx.DiGraph()
    while q:
        _, u, v = heapq.heappop(q)
        if v in V:
            continue
        A.add_edge(u, v, w=D_original[u][v]["w"])
        V.add(v)
        for x, y, data in Arb.out_edges(v, data=True):
            heapq.heappush(q, (data["w"], x, y))
    return A
\end{lstlisting}
\end{tcolorbox}

As duas versões da Fase~2 implementadas acima correspondem diretamente ao Passo~4 da descrição teórica do algoritmo de András Frank:

\begin{tcolorbox}[
		enhanced, breakable,
		colframe=purple!60!black, colback=purple!5,
		colbacktitle=purple!15, coltitle=black,
		title={Correspondência entre Teoria e Implementação — Fase 2},
		boxed title style={sharp corners, boxrule=0.6pt},
		sharp corners, boxrule=0.6pt
	]
	\textbf{Passo 4 — Construção da arborescência (caso acíclico):}
	\begin{itemize}
		\item \textbf{Descrição teórica:} Se \((V,A_0)\) forma uma r-arborescência, devolva \(A_0\). Por otimalidade dos potenciais duais, trata-se de uma r-arborescência de custo mínimo.
		\item \textbf{Implementação:} Ambas as versões de \texttt{phase2} constroem uma r-arborescência a partir do conjunto \(A_0\) de arcos justos obtido na Fase~1. A corretude baseia-se no fato de que todos os arcos em \(A_0\) têm custo reduzido zero, e a Fase~1 garante que existe uma r-arborescência formada exclusivamente por arcos justos.
		      \begin{itemize}
			      \item \textbf{Versão 1 (\texttt{phase2}):} Construção incremental por exploração exaustiva. Em cada uma das \(n-1\) iterações, percorre todos os arcos de \(A_0\) procurando um arco \((u,v)\) tal que \(u\) já pertence à arborescência parcial e \(v\) ainda não. Essa abordagem simples corresponde diretamente à ideia teórica de construir a arborescência adicionando um vértice por vez, conectando-o à estrutura existente através de um arco justo. Complexidade: \(O(nm)\).

			      \item \textbf{Versão 2 (\texttt{phase2\_v2}):} Construção por busca em largura guiada por fila de prioridade. Cria um dígrafo auxiliar onde arcos são indexados, usa a fila de prioridade para explorar arcos sistematicamente a partir da raiz, evitando buscas lineares repetidas. Essa versão otimizada mantém a mesma correção teórica — construir uma r-arborescência usando apenas arcos de \(A_0\) — mas melhora a eficiência prática. Complexidade: \(O(m \log m)\).
		      \end{itemize}
	\end{itemize}

	\vspace{2mm}
	\textbf{Nota sobre Passos 5-7 (contração/recursão/expansão):}
	\begin{itemize}
		\item A descrição teórica do algoritmo de András Frank inclui os Passos~5-7 para tratar o caso onde \(A_0\) contém ciclos dirigidos, exigindo contração, resolução recursiva e reexpansão, de forma análoga ao algoritmo de Chu--Liu--Edmonds.
		\item Na implementação apresentada, optamos por uma abordagem não-recursiva baseada em componentes fortemente conexas. A Fase~1 já garante que \(A_0\) formará uma r-arborescência ao término das iterações de elevação de potenciais, eliminando a necessidade de tratar ciclos explicitamente na Fase~2. Essa simplificação é possível porque a elevação de potenciais progressivamente "quebra" todos os ciclos ao criar novos arcos justos que conectam diferentes componentes, até que reste apenas uma única componente fortemente conexa contendo todos os vértices.
		\item Portanto, quando a Fase~2 é executada, o conjunto \(A_0\) já está livre de ciclos e forma uma r-arborescência, correspondendo diretamente ao caso tratado pelo Passo~4 da descrição teórica. A verificação prévia \texttt{has\_arborescence(D, r0)} (realizada pela função principal) confirma essa propriedade antes de invocar a Fase~2.
	\end{itemize}
\end{tcolorbox}

\subsection{Verificação de otimalidade dual}

Esta função verifica se a condição de otimalidade dual é satisfeita para a r-arborescência construída. Segundo a teoria de programação linear dual aplicada ao problema de arborescência de custo mínimo, uma solução é ótima se e somente se cada conjunto \(X \subseteq V \setminus \{r\}\) que teve seu potencial elevado durante a Fase~1 possui exatamente um arco entrando na arborescência final.

Recebe como entrada a arborescência \texttt{Arb} e a lista \texttt{Dual\_list} contendo pares \((X, \Delta(X))\) onde \(X\) é um conjunto de vértices cujos potenciais foram elevados e \(\Delta(X) > 0\) é o valor da elevação. A implementação itera sobre cada par \((X, z)\) na lista dual (linha 2), e para cada conjunto \(X\), percorre todos os arcos \((u,v)\) da arborescência (linha 3). Inicializa um contador \texttt{count} em zero (linha 4) e verifica se o arco cruza a fronteira de \(X\), isto é, se \(u \notin X\) e \(v \in X\) (linha 5). Quando essa condição é satisfeita, incrementa o contador (linha 6) e imediatamente verifica se já há mais de um arco entrando em \(X\) (linha 7). Caso positivo, a condição de otimalidade dual é violada e a função devolve \texttt{False} (linha 8).

A função devolve \texttt{True} se todos os conjuntos em \texttt{Dual\_list} possuem exatamente um arco entrando na arborescência, confirmando que a solução satisfaz as condições de folga complementar da programação linear dual. A complexidade é \(O(km)\), onde \(k = |\texttt{Dual\_list}|\) e \(m = |A|\), pois para cada conjunto dual verifica-se todos os arcos da arborescência.

Esta verificação é fundamental para garantir a correção do algoritmo: a Fase~1 constrói uma solução dual viável (potenciais \(y(v)\)), a Fase~2 constrói uma solução primal viável (arborescência), e esta função confirma que ambas satisfazem as condições de folga complementar, implicando otimalidade pelo teorema da dualidade forte.

\begin{tcolorbox}[
		enhanced, breakable,
		colframe=blue!60!black, colback=blue!2,
		colbacktitle=blue!15, coltitle=black,
		title={Verificação de otimalidade dual},
		boxed title style={sharp corners, boxrule=0.6pt},
		sharp corners, boxrule=0.6pt
	]
	\emph{Verifica se a condição de otimalidade dual é satisfeita, confirmando que cada conjunto dual possui exatamente um arco entrando na arborescência.}
	\tcblower
	\begin{lstlisting}[language=Python]
def check_dual_optimality_condition(Arb, Dual_list):
    for X, z in Dual_list:
        count = 0
        for u, v in Arb.edges():
            if u not in X and v in X:
                count += 1
                if count > 1:
                    return False
    return True
\end{lstlisting}
\end{tcolorbox}

\subsection{O algoritmo completo de András Frank}

Finalmente, apresentamos a função principal que implementa o algoritmo de András Frank para encontrar uma r-arborescência de custo mínimo em um dígrafo com pesos. A função integra as fases de construção dos potenciais duais, obtenção dos arcos justos, construção da arborescência e verificação de otimalidade dual.

\begin{tcolorbox}[
		enhanced, breakable,
		colframe=blue!60!black, colback=blue!2,
		colbacktitle=blue!15, coltitle=black,
		title={Verificação de otimalidade dual},
		boxed title style={sharp corners, boxrule=0.6pt},
		sharp corners, boxrule=0.6pt
	]
	\emph{Implementa o algoritmo completo de András Frank, integrando as fases de construção dos potenciais duais, obtenção dos arcos justos, construção da arborescência e verificação de otimalidade dual.}
	\tcblower
	\begin{lstlisting}[language=Python]
def andras_frank_algorithm(D):
    A_zero, Dual_list = phase1(D,"r0")
    arborescence_frank = phase2(D, "r0", A_zero)
    arborescence_frank_v2 = phase2_v2(D, "r0", A_zero)
    dual_frank = check_dual_optimality_condition(
        arborescence_frank, Dual_list)
    dual_frank_v2 = check_dual_optimality_condition(
        arborescence_frank_v2, Dual_list)
    return arborescence_frank, arborescence_frank_v2, dual_frank, dual_frank_v2
\end{lstlisting}
\end{tcolorbox}

\subsection*{Exemplo de execução do algoritmo}

Aqui ilustraremos cada fase do algoritmo de András Frank: inicialização dos potenciais, iterações de elevação, construção de \(A_0\), e formação da arborescência final. Utilizaremos o dígrafo \(D_{32}\) estendido com a raiz \(r_0\), mostrado na Figura~\ref{fig:frank-exemplo-inicial}.

\begin{figure}[H]
	\centering
	\begin{tikzpicture}[>=Stealth, scale=0.9, every node/.style={scale=0.9}]
		\tikzset{
			vtx/.style={circle, draw, fill=blue!8, thick, inner sep=1.2pt, minimum size=6.5mm},
			vtxRoot/.style={circle, draw, fill=red!15, thick, inner sep=1.2pt, minimum size=6.5mm},
			arc/.style={->, thick, gray!60},
			cost/.style={font=\scriptsize, fill=white, inner sep=1pt, rounded corners=1pt}
		}

		\node[font=\bfseries, blue!80!black] at (3, 4.5) {Dígrafo inicial $D$ com raiz $r_0$};

		% Raiz
		\node[vtxRoot] (r0) at (3, 3) {$r_0$};

		% Vértices camada 1
		\node[vtx] (u1) at (0, 1.5) {$u_1$};
		\node[vtx] (u2) at (3, 1.5) {$u_2$};
		\node[vtx] (u3) at (6, 1.5) {$u_3$};

		% Vértices camada 2
		\node[vtx] (v1) at (1, 0) {$v_1$};
		\node[vtx] (v2) at (3, 0) {$v_2$};
		\node[vtx] (v3) at (5, 0) {$v_3$};

		% Arcos da raiz
		\draw[arc] (r0) -- node[cost, left, pos=0.4] {1} (u1);
		\draw[arc] (r0) -- node[cost, right] {2} (u2);
		\draw[arc] (r0) -- node[cost, right, pos=0.4] {1} (u3);

		% Arcos camada 1
		\draw[arc] (u1) to[bend left=15] node[cost, above] {1} (u2);
		\draw[arc] (u2) to[bend left=15] node[cost, above] {3} (u3);

		% Arcos entrando camada 2
		\draw[arc] (u1) -- node[cost, left] {3} (v1);
		\draw[arc] (u2) -- node[cost, right] {2} (v2);
		\draw[arc] (u3) -- node[cost, right] {4} (v3);
		\draw[arc] (u1) to[bend right=20] node[cost, below, pos=0.3] {5} (v2);

		% Arcos internos camada 2
		\draw[arc] (v1) -- node[cost, above] {1} (v2);
		\draw[arc] (v2) -- node[cost, above] {2} (v3);

		% Arco retorno
		\draw[arc] (v1) to[bend right=30] node[cost, left, pos=0.3] {2} (u1);

		\node[font=\small, gray!70!black] at (3, -1.2) {Pesos originais $c(u,v)$};
	\end{tikzpicture}
	\caption{Dígrafo inicial \(D\) com raiz \(r_0\) e 6 vértices adicionais. O dígrafo contém múltiplos arcos com pesos variados. A Fase~1 do algoritmo iniciará com potenciais \(y(v) = 0\) para todos os vértices, correspondendo aos custos reduzidos iniciais \(c_y(u,v) = c(u,v)\).}
	\label{fig:frank-exemplo-inicial}
\end{figure}

\paragraph*{Fase 1 — Iteração 1:} O algoritmo inicia com todos os potenciais \(y(v) = 0\) e \(D_0\) vazio (sem arcos justos). Calcula-se as componentes fortemente conexas de \(D_0\): cada vértice forma sua própria componente. As fontes são todos os vértices exceto \(r_0\). Para o conjunto \(X = \{u_1, u_2, u_3\}\), identifica-se os arcos entrando: \((r_0, u_1)\) com peso 1, \((r_0, u_2)\) com peso 2, \((r_0, u_3)\) com peso 1. O mínimo é \(\Delta(X) = 1\). Subtraindo esse valor, os arcos \((r_0, u_1)\) e \((r_0, u_3)\) tornam-se justos (peso 0) e são adicionados a \(A_0\).

\begin{figure}[H]
	\centering
	\begin{tikzpicture}[>=Stealth, scale=0.85, every node/.style={scale=0.85}]
		\tikzset{
			vtx/.style={circle, draw, fill=blue!8, thick, inner sep=1.2pt, minimum size=6.5mm},
			vtxRoot/.style={circle, draw, fill=red!15, thick, inner sep=1.2pt, minimum size=6.5mm},
			vtxX/.style={circle, draw, fill=orange!20, thick, inner sep=1.2pt, minimum size=6.5mm},
			arc/.style={->, thick, gray!60},
			arcTight/.style={->, ultra thick, blue!70!black},
			cost/.style={font=\scriptsize, fill=white, inner sep=1pt},
			costTight/.style={font=\scriptsize\bfseries, fill=blue!10, inner sep=2pt, draw=blue!70!black, rounded corners=1pt}
		}

		% Antes
		\begin{scope}
			\node[font=\bfseries, blue!80!black] at (2, 4.5) {Antes: $\Delta(X_1) = 1$};

			\node[vtxRoot] (r0) at (2, 3) {$r_0$};
			\node[vtxX] (u1) at (0, 1.5) {$u_1$};
			\node[vtxX] (u2) at (2, 1.5) {$u_2$};
			\node[vtxX] (u3) at (4, 1.5) {$u_3$};

			\draw[arc] (r0) -- node[cost, left, pos=0.4] {1} (u1);
			\draw[arc] (r0) -- node[cost, right] {2} (u2);
			\draw[arc] (r0) -- node[cost, right, pos=0.4] {1} (u3);

			\node[font=\small, orange!80!black] at (2, 0.5) {$X_1 = \{u_1, u_2, u_3\}$};
		\end{scope}

		% Depois
		\begin{scope}[xshift=7cm]
			\node[font=\bfseries, blue!80!black] at (2, 4.5) {Depois: arcos justos};

			\node[vtxRoot] (r0b) at (2, 3) {$r_0$};
			\node[vtx] (u1b) at (0, 1.5) {$u_1$};
			\node[vtx] (u2b) at (2, 1.5) {$u_2$};
			\node[vtx] (u3b) at (4, 1.5) {$u_3$};

			\draw[arcTight] (r0b) -- node[costTight, left, pos=0.4] {0} (u1b);
			\draw[arc] (r0b) -- node[cost, right] {1} (u2b);
			\draw[arcTight] (r0b) -- node[costTight, right, pos=0.4] {0} (u3b);

			\node[font=\small, blue!70!black] at (2, 0.5) {2 arcos adicionados a $A_0$};
		\end{scope}
	\end{tikzpicture}
	\caption{Fase~1, Iteração 1: Elevação de potenciais para \(X_1 = \{u_1, u_2, u_3\}\). À esquerda, o conjunto \(X_1\) (em laranja) sem arcos justos entrando. Calcula-se \(\Delta(X_1) = \min\{1, 2, 1\} = 1\). À direita, após subtrair \(\Delta(X_1)\), os arcos \((r_0, u_1)\) e \((r_0, u_3)\) tornam-se \textcolor{blue!70!black}{justos} (custo reduzido 0) e são adicionados a \(A_0\).}
	\label{fig:frank-exemplo-iter1}
\end{figure}

\paragraph*{Fase 1 — Iteração 2:} Com \(A_0 = \{(r_0, u_1), (r_0, u_3)\}\), as componentes fortemente conexas são: \(\{r_0, u_1, u_3\}\) e \(\{u_2\}\), \(\{v_1\}\), \(\{v_2\}\), \(\{v_3\}\). A fonte (excluindo a raiz) é \(\{u_2\}\). Para \(X = \{u_2\}\), o único arco entrando é \((r_0, u_2)\) com peso atual 1 (após iteração anterior). Portanto \(\Delta(X) = 1\), e após subtração, \((r_0, u_2)\) torna-se justo. Também na mesma iteração, para \(X = \{v_1, v_2, v_3\}\), calcula-se \(\Delta(X) = \min\{3, 2, 4, 5\} = 2\), e os arcos são atualizados conforme ilustrado.

\begin{figure}[H]
	\centering
	\begin{tikzpicture}[>=Stealth, scale=0.75, every node/.style={scale=0.75}]
		\tikzset{
			vtx/.style={circle, draw, fill=blue!8, thick, inner sep=1.2pt, minimum size=6mm},
			vtxRoot/.style={circle, draw, fill=red!15, thick, inner sep=1.2pt, minimum size=6mm},
			vtxX/.style={circle, draw, fill=orange!20, thick, inner sep=1.2pt, minimum size=6mm},
			arc/.style={->, thick, gray!60},
			arcTight/.style={->, ultra thick, blue!70!black},
			cost/.style={font=\scriptsize, fill=white, inner sep=1pt},
			costTight/.style={font=\scriptsize\bfseries, fill=blue!10, inner sep=2pt, draw=blue!70!black, rounded corners=1pt}
		}

		\node[font=\bfseries, blue!80!black] at (3, 5.5) {Iteração 2: $\Delta(\{v_1, v_2, v_3\}) = 2$};

		\node[vtxRoot] (r0) at (3, 4) {$r_0$};
		\node[vtx] (u1) at (0.5, 2.5) {$u_1$};
		\node[vtx] (u2) at (3, 2.5) {$u_2$};
		\node[vtx] (u3) at (5.5, 2.5) {$u_3$};
		\node[vtxX] (v1) at (1.5, 0.5) {$v_1$};
		\node[vtxX] (v2) at (3, 0.5) {$v_2$};
		\node[vtxX] (v3) at (4.5, 0.5) {$v_3$};

		% Arcos já justos
		\draw[arcTight] (r0) -- node[costTight, left, pos=0.3] {0} (u1);
		\draw[arcTight] (r0) -- node[costTight, right, pos=0.3] {0} (u3);
		\draw[arcTight] (u1) to[bend left=12] node[costTight, above] {0} (u2);

		% Arcos entrando em X (antes de ficarem justos)
		\draw[arc] (u1) -- node[cost, left, fill=orange!10] {3→1} (v1);
		\draw[arc] (u2) -- node[cost, right, fill=orange!10] {2→0} (v2);
		\draw[arc] (u3) -- node[cost, right] {4→2} (v3);
		\draw[arc] (u1) to[bend right=15] node[cost, below, pos=0.35] {5→3} (v2);

		\node[font=\small, orange!80!black] at (3, -0.5) {Após subtração: $(u_2, v_2)$ torna-se justo};
	\end{tikzpicture}
	\caption{Fase~1, Iteração 2: Para o conjunto \(X = \{v_1, v_2, v_3\}\) (em laranja), calcula-se \(\Delta(X) = \min\{3, 2, 4, 5\} = 2\). Após subtrair esse valor dos arcos entrando em \(X\), o arco \((u_2, v_2)\) atinge custo reduzido 0 e é adicionado a \(A_0\). Note que \((u_1, u_2)\) também se tornou justo na mesma iteração ao processar \(X = \{u_2\}\).}
	\label{fig:frank-exemplo-iter2}
\end{figure}

\paragraph*{Fase 1 — Estado final:} Após mais iterações elevando potenciais para os conjuntos restantes, o dígrafo \(D_0\) de arcos justos fica completo, contendo uma r-arborescência. A Figura~\ref{fig:frank-exemplo-A0-final} mostra o conjunto final \(A_0\) com 6 arcos de custo reduzido zero.

\begin{figure}[H]
	\centering
	\begin{tikzpicture}[>=Stealth, scale=0.85, every node/.style={scale=0.85}]
		\tikzset{
			vtx/.style={circle, draw, fill=blue!8, thick, inner sep=1.2pt, minimum size=6.5mm},
			vtxRoot/.style={circle, draw, fill=red!15, thick, inner sep=1.2pt, minimum size=6.5mm},
			arcTight/.style={->, ultra thick, blue!70!black},
			costTight/.style={font=\scriptsize\bfseries, fill=blue!10, inner sep=2pt, draw=blue!70!black, rounded corners=1pt}
		}

		\node[font=\bfseries, blue!80!black] at (3, 4.5) {$A_0$ final: arcos justos após Fase~1};

		\node[vtxRoot] (r0) at (3, 3) {$r_0$};
		\node[vtx] (u1) at (0.5, 1.5) {$u_1$};
		\node[vtx] (u2) at (3, 1.5) {$u_2$};
		\node[vtx] (u3) at (5.5, 1.5) {$u_3$};
		\node[vtx] (v1) at (1.5, 0) {$v_1$};
		\node[vtx] (v2) at (3, 0) {$v_2$};
		\node[vtx] (v3) at (4.5, 0) {$v_3$};

		\draw[arcTight] (r0) -- node[costTight, left, pos=0.3] {0} (u1);
		\draw[arcTight] (r0) -- node[costTight, right, pos=0.3] {0} (u3);
		\draw[arcTight] (u1) to[bend left=12] node[costTight, above] {0} (u2);
		\draw[arcTight] (u1) -- node[costTight, left] {0} (v1);
		\draw[arcTight] (u2) -- node[costTight, right] {0} (v2);
		\draw[arcTight] (v2) -- node[costTight, above] {0} (v3);

		\node[font=\small, blue!70!black] at (3, -1.2) {$|A_0| = 6$ arcos com custo reduzido $c_y = 0$};
	\end{tikzpicture}
	\caption{Conjunto final \(A_0\) de arcos justos após conclusão da Fase~1. Todos os 6 arcos \textcolor{blue!70!black}{destacados} possuem custo reduzido zero: \((r_0, u_1)\), \((r_0, u_3)\), \((u_1, u_2)\), \((u_1, v_1)\), \((u_2, v_2)\), \((v_2, v_3)\). O dígrafo formado por esses arcos é acíclico e alcança todos os vértices a partir de \(r_0\), satisfazendo as condições para prosseguir à Fase~2.}
	\label{fig:frank-exemplo-A0-final}
\end{figure}

\paragraph*{Fase 2 — Construção da arborescência:} Com \(A_0\) completo e acíclico, a Fase~2 constrói incrementalmente a arborescência final. Inicia-se com \(\texttt{Arb} = \{r_0\}\) e em cada iteração adiciona-se um arco \((u,v) \in A_0\) tal que \(u \in \texttt{Arb}\) e \(v \notin \texttt{Arb}\). A Figura~\ref{fig:frank-exemplo-arborescencia-final} mostra a arborescência resultante.

\begin{figure}[H]
	\centering
	\begin{tikzpicture}[>=Stealth, scale=0.85, every node/.style={scale=0.85}]
		\tikzset{
			vtx/.style={circle, draw, fill=green!15, thick, inner sep=1.2pt, minimum size=6.5mm},
			vtxRoot/.style={circle, draw, fill=red!15, thick, inner sep=1.2pt, minimum size=6.5mm},
			arcArb/.style={->, ultra thick, green!70!black},
			costArb/.style={font=\scriptsize\bfseries, fill=green!10, inner sep=2pt, draw=green!70!black, rounded corners=1pt}
		}

		\node[font=\bfseries, blue!80!black] at (3, 4.5) {Arborescência final (Fase~2)};

		\node[vtxRoot] (r0) at (3, 3) {$r_0$};
		\node[vtx] (u1) at (0.5, 1.5) {$u_1$};
		\node[vtx] (u2) at (3, 1.5) {$u_2$};
		\node[vtx] (u3) at (5.5, 1.5) {$u_3$};
		\node[vtx] (v1) at (1.5, 0) {$v_1$};
		\node[vtx] (v2) at (3, 0) {$v_2$};
		\node[vtx] (v3) at (4.5, 0) {$v_3$};

		\draw[arcArb] (r0) -- node[costArb, left, pos=0.3] {1} (u1);
		\draw[arcArb] (r0) -- node[costArb, right, pos=0.3] {1} (u3);
		\draw[arcArb] (u1) to[bend left=12] node[costArb, above] {1} (u2);
		\draw[arcArb] (u1) -- node[costArb, left] {3} (v1);
		\draw[arcArb] (u2) -- node[costArb, right] {2} (v2);
		\draw[arcArb] (v2) -- node[costArb, above] {2} (v3);

		\node[font=\small, green!70!black] at (3, -1.2) {Custo total: $1+1+1+3+2+2 = 10$};
	\end{tikzpicture}
	\caption{Arborescência de custo mínimo final obtida pela Fase~2. Os pesos mostrados são os \emph{originais} de \(D\), restaurados após a construção. Cada vértice não-raiz possui exatamente um arco de entrada, não há ciclos, e todos os vértices são alcançáveis a partir de \(r_0\). A verificação de otimalidade dual confirma que cada conjunto \(X\) que teve potenciais elevados possui exatamente um arco da arborescência cruzando sua fronteira, garantindo que a solução é ótima.}
	\label{fig:frank-exemplo-arborescencia-final}
\end{figure}

\paragraph*{Verificação de otimalidade dual:} A função \texttt{check\_dual\_optimality\_condition} confirma que para cada par \((X, \Delta(X))\) em \texttt{Dual\_list} (conjuntos cujos potenciais foram elevados com \(\Delta(X) > 0\)), existe exatamente um arco da arborescência final cruzando a fronteira de \(X\). Essa condição, juntamente com os arcos justos, garante que as condições de folga complementar da programação linear dual são satisfeitas, implicando que a arborescência encontrada é de custo mínimo global.

\subsection{Correspondência entre teoria e implementação}

A implementação em Python do algoritmo de András Frank segue fielmente a descrição teórica primal-dual apresentada anteriormente. A tabela abaixo estabelece o paralelo direto entre os passos teóricos e sua realização no código:

\begin{table}[H]
	\centering
	\scriptsize
	\renewcommand{\arraystretch}{1.3}
	\begin{tabular}{>{\raggedright\arraybackslash}p{0.44\textwidth}|>{\raggedright\arraybackslash}p{0.48\textwidth}}
		\toprule
		\textbf{\normalsize Descrição Teórica} & \textbf{\normalsize Implementação Python} \\
		\midrule
		\rowcolor{blue!5}
		\textbf{Passo 1:} Inicialização

		Defina \(y(v) := 0\) para todo \(v \in V\).

		Inicialize \(A_0 := \emptyset\).

		Construa dígrafo vazio \(D_0\) (arcos justos).
		                                       &
		\textbf{Função \texttt{phase1} — Linhas 2--5:}

		\texttt{D\_copy = D\_original.copy()}

		\texttt{A\_zero = []}

		\texttt{D\_zero = build\_D\_zero(D\_copy)}

		\vspace{1mm}
		Potenciais \(y(v) = 0\) implícitos, custos \(c_y = c\).
		\\
		\midrule
		\rowcolor{green!5}
		\textbf{Passo 2:} Elevação de potenciais

		Enquanto \(\exists X \subseteq V \setminus \{r\}\) sem arco justo:

		\quad Calcule \(\Delta(X) := \min\{c_y(u,v) : u \notin X, v \in X\}\)

		\quad Atualize \(y(v) := y(v) + \Delta(X)\), \(\forall v \in X\)

		\quad Adicione arcos com \(c_y = 0\) a \(A_0\)
		                                       &
		\textbf{Loop principal — Linhas 7--22:}

		\texttt{C = nx.condensation(D\_zero)}

		\texttt{sources = [x for x in C.nodes()]}

		\texttt{\quad if C.in\_degree(x) == 0]}

		Para fonte \texttt{u} (exceto raiz):

		\texttt{\quad X = C.nodes[u]["members"]}

		\texttt{\quad arcs = get\_arcs\_entering\_X(D, X)}

		\texttt{\quad min\_w = get\_minimum\_weight\_cut(arcs)}

		\texttt{\quad update\_weights\_in\_X(D, arcs, min\_w,}

		\texttt{\quad\quad A\_zero, D\_zero)}
		\\
		\midrule
		\rowcolor{orange!5}
		\textbf{Passo 2(a):} Identificar arcos entrando

		Determine \(\{(u,v) \in A : u \notin X, v \in X\}\)
		                                       &
		\textbf{Função \texttt{get\_arcs\_entering\_X}:}

		\texttt{return [(u, v, data)}

		\texttt{\quad for u, v, data in D.edges(data=True)}

		\texttt{\quad if u not in X and v in X]}
		\\
		\midrule
		\rowcolor{purple!5}
		\textbf{Passo 2(b):} Calcular \(\Delta(X)\)

		\(\Delta(X) := \min\{c_y(u,v) : u \notin X, v \in X\}\)
		                                       &
		\textbf{Função \texttt{get\_minimum\_weight\_cut}:}

		\texttt{return min(data["w"]}

		\texttt{\quad for \_, \_, data in arcs)}
		\\
		\midrule
		\rowcolor{yellow!5}
		\textbf{Passo 2(c):} Atualizar pesos

		Para \((u,v)\) entrando em \(X\):

		\quad \(c_y(u,v) := c_y(u,v) - \Delta(X)\)

		\quad Se \(c_y(u,v) = 0\), adicione a \(A_0\)
		                                       &
		\textbf{Função \texttt{update\_weights\_in\_X}:}

		\texttt{for u, v, \_ in arcs:}

		\texttt{\quad D[u][v]["w"] -= min\_weight}

		\texttt{\quad if D[u][v]["w"] == 0:}

		\texttt{\quad\quad A\_zero.append((u, v))}

		\texttt{\quad\quad D\_zero.add\_edge(u, v)}
		\\
		\midrule
		\rowcolor{cyan!5}
		\textbf{Passo 3:} Verificar término

		Se \(D_0\) contém r-arborescência, encerre.
		                                       &
		\textbf{Condição — Linhas 11--12:}

		\texttt{if len(sources) == 1: break}

		\vspace{1mm}
		Uma fonte \(\Rightarrow\) r-arborescência acíclica.
		\\
		\midrule
		\rowcolor{red!5}
		\textbf{Passo 4:} Construir arborescência

		Construa \(F\) a partir de \(A_0\), conectando vértices incrementalmente.
		                                       &
		\textbf{Função \texttt{phase2} (incremental):}

		\texttt{Arb = nx.DiGraph(); Arb.add\_node(r0)}

		\texttt{for \_ in range(n - 1):}

		\texttt{\quad for u, v in A\_zero:}

		\texttt{\quad\quad if u in Arb and v not in Arb:}

		\texttt{\quad\quad\quad Arb.add\_edge(u, v, **data)}

		\texttt{\quad\quad\quad break}

		\vspace{1mm}
		Complexidade: \(O(nm)\).
		\\
		\midrule
		\rowcolor{red!5}
		\textbf{Passo 4:} Versão otimizada

		Mesma ideia, com fila de prioridade.
		                                       &
		\textbf{Função \texttt{phase2\_v2} (BFS):}

		\texttt{Arb = nx.DiGraph()}

		\texttt{for i, (u, v) in enumerate(A\_zero):}

		\texttt{\quad Arb.add\_edge(u, v, w=i)}

		\texttt{q = [] \# fila de prioridade}

		\texttt{while q:}

		\texttt{\quad \_, u, v = heapq.heappop(q)}

		\texttt{\quad if v in V: continue}

		\texttt{\quad A.add\_edge(u, v, w=D[u][v]["w"])}

		\vspace{1mm}
		Complexidade: \(O(m \log m)\).
		\\
		\midrule
		\rowcolor{pink!5}
		\textbf{Otimalidade dual}

		Para cada \(X\) elevado (\(\Delta(X) > 0\)), exatamente um arco de \(F\) cruza \(\delta^-(X)\).
		                                       &
		\textbf{Função \texttt{check\_dual\_optimality}:}

		\texttt{for X, z in Dual\_list:}

		\texttt{\quad count = 0}

		\texttt{\quad for u, v in Arb.edges():}

		\texttt{\quad\quad if u not in X and v in X:}

		\texttt{\quad\quad\quad count += 1}

		\texttt{\quad\quad\quad if count > 1: return False}

		\texttt{return True}
		\\
		\bottomrule
	\end{tabular}
	\caption{Correspondência entre a descrição teórica do algoritmo de András Frank e sua implementação em Python. Cores: inicialização (azul), elevação de potenciais (verde/laranja/roxo/amarelo), verificação (ciano), construção (vermelho) e validação dual (rosa).}
	\label{tab:frank-teoria-implementacao}
\end{table}

Esta correspondência demonstra que a implementação traduz fielmente a abordagem primal-dual em código executável. As funções auxiliares (\texttt{get\_arcs\_entering\_X}, \texttt{get\_minimum\_weight\_cut}, \texttt{update\_weights\_in\_X}, \texttt{phase1}, \texttt{phase2}, \texttt{phase2\_v2},
\texttt{check\_dual\_}-
\texttt{optimality\_condition} encapsulam exatamente as operações descritas na teoria, preservando as propriedades de correção e as garantias de otimalidade do algoritmo original. A utilização de componentes fortemente conexas (\texttt{nx.condensation}) para identificar conjuntos sem arcos justos entrando é uma implementação eficiente da verificação teórica, evitando enumeração explícita de todos os subconjuntos de vértices.


\chapter{Conclusão}
\label{cap:conclusao}

Este trabalho apresentou uma análise e implementação de algoritmos de busca de uma r-arborescência inversa de custo mínimo em grafos dirigidos.

\section{Contribuições}

As principais contribuições deste trabalho são...

\section{Trabalhos Futuros}

Como trabalhos futuros, sugere-se...


% ----------------------------------------------------------
% ELEMENTOS PÓS-TEXTUAIS (Referências, Glossário, Apêndices)
% ----------------------------------------------------------
\postextual

% Referências bibliográficas
%\nocite{*}
\bibliography{bibliografia}


% Glossário (Consulte o manual)
%\glossary
% Apêndices
%\include{posttextual/apendices}

% Anexos
%\begin{anexosenv}

	\partanexos

	\chapter{Atribuição de Tarefas}
	\label{anx:atribuicao}

	\section*{Orientador}
	Prof. Dr. Mário Leston Rey

	\section*{Lorena Silva Sampaio - 11201812025}
	\begin{itemize}
		\item Análise e implementação do Algoritmo de Chu-Liu-Edmonds
		\item Análise e implementação do Algoritmo de András Frank versão com heap
		\item Testes experimentais e análise de desempenho
	\end{itemize}

	\section*{Samira Haddad - 11201812350}
	\begin{itemize}
		\item Análise e implementação do Algoritmo de András Frank versão com lista
		\item Revisão bibliográfica sobre didática e IHC
		\item Desenvolvimento da interface da aplicação web interativa
	\end{itemize}

\end{anexosenv}


% Índice remissivo (Consultar manual)
\phantompart
\printindex

\end{document}
